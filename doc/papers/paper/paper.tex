\documentclass{sigplanconf}

\usepackage[utf8]{inputenc}
\usepackage{amsmath}
\usepackage{amssymb}
\usepackage{url}
\usepackage{color}
\usepackage{multirow}

\newcommand{\Value}{\mathbf{value}}
\newcommand{\Any}{\mathbf{any}}
\newcommand{\Nil}{\mathbf{nil}}
\newcommand{\Self}{\mathbf{self}}
\newcommand{\False}{\mathbf{false}}
\newcommand{\True}{\mathbf{true}}
\newcommand{\Boolean}{\mathbf{boolean}}
\newcommand{\Integer}{\mathbf{integer}}
\newcommand{\Number}{\mathbf{number}}
\newcommand{\String}{\mathbf{string}}
\newcommand{\Void}{\mathbf{void}}
\newcommand{\Const}{\mathbf{const}}

\newcommand{\mylabel}[1]{\; (\textsc{#1})}
\newcommand{\env}{\Gamma}
\newcommand{\penv}{\Pi}
\newcommand{\senv}{\Sigma}
\newcommand{\subtype}{<:}
\newcommand{\ret}{\rho}
\newcommand{\self}{\sigma}

\def\dstart{\hbox to \hsize{\vrule depth 4pt\hrulefill\vrule depth 4pt}}
\def\dend{\hbox to \hsize{\vrule height 4pt\hrulefill\vrule height 4pt}}

\begin{document}

\special{papersize=8.5in,11in}
\setlength{\pdfpageheight}{\paperheight}
\setlength{\pdfpagewidth}{\paperwidth}

\conferenceinfo{CONF 'yy}{Month d--d, 20yy, City, ST, Country}
\copyrightyear{20yy}
\copyrightdata{978-1-nnnn-nnnn-n/yy/mm}
\doi{nnnnnnn.nnnnnnn}

\titlebanner{banner above paper title}
\preprintfooter{short description of paper}

\title{Formalization of Typed Lua}

\authorinfo{André Murbach Maidl}
           {PUC-Rio}
           {amaidl@inf.puc-rio.br}
\authorinfo{Fabio Mascarenhas}
           {UFRJ}
           {fabiom@dcc.ufrj.br}
\authorinfo{Roberto Ierusalimschy}
           {PUC-Rio}
           {roberto@inf.puc-rio.br}

\maketitle

\begin{abstract}
Dynamically typed languages such as Lua avoid static types in order
to have a simpler and more flexible language, because their programmers
to not need to bother with abstracting types that need to be validated
by a type checker. In contrast, statically typed languages provide
earlier detection of many defects, and a better framework for structuring
large programs. The advantages of static types often lead programmers to
migrate from a dynamically typed to a statically typed language when
their simple scripts evolve into complex programs.

Optional type systems are one way to have both static and dynamic
typing in the same language while keeping its original semantics,
making evolving a program from dynamic to static typing a matter of
describing the implied types that it it using and adding annotations
to make those types explicit. Designing an optional type system for
an existing dynamically typed language is challenging, as its types
should feel natural to programmers that are already familiar with
this language.

In this work, we give a formal description of Typed Lua, an
optional type system for Lua, with a focus on its novel type system
features: incremental evolution of record and object types via
imperative update that does not sacrifice width subtyping,
and combining flow typing with multiple assignment and with functions that
return multiple values. While our type system is tailored to the
features and idioms of Lua, its features can be adapted to other
imperative scripting languages.
\end{abstract}

\category{CR-number}{subcategory}{third-level}

\terms
term1, term2

\keywords
keyword1, keyword2

\section{Introduction}
\label{sec:intro}

Dynamically typed languages forgo static type checking
in favor of using run-time {\em type tags} to classify
the values they compute, so its operations can use these
tags to perform run-time (or dynamic) checks and signal
errors in case of invalid operands~\cite{pierce2002tpl}.
The lack of static types allows programmers to write code
that might have required a complex type system to statically
type, at the cost of hiding defects that will only be caught
after deployment if the programmers do not properly
test their code.

In contrast, statically typed languages help programmers
detect defects during development, and also provide
a conceptual framework that helps programmers define modules
and interfaces that can be combined to structure the development
of large programs.

The early error detection and better tools for structuring
programs are two advantages of statically typed languages that
lead programmers to migrate their code from a dynamically
typed to a statically typed language when their scripts
evolve into complex programs~\cite{tobin-hochstadt2006ims}.
This migration from dynamic to static typing usually involves
different languages that have distinct syntax and semantics,
requiring a complexe rewrite of existing programs instead of
incremental evolution.

Ideally, programming languages should offer programmers the
option to choose between static and dynamic typing.
\emph{Optional type systems}~\cite{bracha2004pluggable} and
\emph{gradual typing}~\cite{siek2006gradual} are two similar
approaches for blending static and dynamic typing in the same
language. Their aim is to offer programmers the option
of adding type annotations where static typing is needed,
allowing the incremental migration from dynamic to static
typing. The difference between these two approaches is the
way they treat the runtime semantics of the language:
optional type systems preserve the original semantics
of a dynamically typed language, at the cost of runtime
safety, while gradual typing changes the semantics to
include more extensive runtime checking to ensure that
dynamically typed portions of the program do not violate
the type safety of statically typed portions.

Lua is a small imperative language with lexically-scoped first-class functions where the only data structuring
mechanism is the \emph{table} --
an associative array that has syntactic and runtime support
to efficiently represent arrays, records, maps, modules,
objects, etc. Unlike other scripting languages, Lua has very limited coercion among different data types.

Lua prefers to provide mechanisms instead of fixed policies due
to its primary use as an embedded language for configuration and
extension of other applications.
This means that even features such as a module system and
object orientation are a matter of convention instead of
built-in language constructs.
The result is a fragmented ecosystem of libraries, and different
ideas among Lua programmers on how they should use the language
features or on how they should structure programs.

Typed Lua is an optional type system for
Lua that is rich enough to preserve some of the idioms
that Lua programmers are already familiar with, while
helping programmers structure large Lua programs and
catch errors at compile-time. An informal
presentation of the design of Typed Lua has been
published~\cite{maidl2014tl}.

In this paper, we give a formal basis to an updated version of that design, with a focus on novel type system features
that Typed Lua uses to type some tricky Lua idioms.
One of these idioms is the use of imperative updates to
build records and objects one field at a time; Typed Lua
lets record and object types evolve incrementally along
with these updates while not sacrificing width subtyping.
Another idiom combines overloading on the return values
of a function, multiple return values, multiple assignment,
and tag checks to discriminate a several returned values
by checking just one of them; Typed Lua deconstructs unions
of tuples in a way that the dependencies between the types
of each value in the tuple can be tracked.

The lack of standard policies is a challenge for the design of
an optional type system for Lua. For this reason, the design
of Typed Lua has been informed by a mostly automated survey
of Lua idioms used by a large corpus of Lua libraries.
This paper also presents the methodology of this survey and
a summary of its results.

The rest of the paper is organized as follows:
Section \ref{sec:statistics} presents the methodology and results of
our survey of Lua idioms;
Section \ref{sec:types} presents the types of Typed Lua's
type system, and their subtyping and consistent-subtyping relations;
Section \ref{sec:rules} presents the typing rules for evolution of table types,
and the typing rules for destructuring unions of tuples into projection types,
allowing flow typing on these projections;
Section \ref{sec:related} reviews related work;
Section \ref{sec:conclusion} presents our conclusions,
and outlines our plans for future work.

\section{A Survey of Lua Idioms}
\label{sec:statistics}

In order to find out how Lua is used in practice, to inform
the design of Typed Lua, we did a survey of existing Lua
libraries. Our corpus is the repository for the LuaRocks
package manager~\cite{hisham2013luarocks}, downloaded on
February 1st, 2014. We did not survey all of the Lua
scripts in the repository: scripts that were not compatible
with Lua 5.2 (the version of Lua at that time) were ignored,
and we also ignored machine-generated scripts and test scripts,
as those could skew the result towards non-idiomatic uses
of Lua. This left 2,598 scripts out of a total of 3,928, from
262 different projects. The scripts were parsed, and their
abstract syntax trees analysed to collect the data that we
show in the rest of this section.

Tables are the core data structure of Lua programs, and
the language has features that let tables be used as
tuples, lists, maps, records, abstract data types, classes,
objects, modules, etc. We performed several analyses to
quantify the different ways that Lua programs use tables.

Our first analysis surveyed table constructors (expressions
that create and initialize a table) to find out how a table
begins its life. Of all of the 23,185 table constructors of
our corpus, 36\% of them create a record (a table with string
keys such as \texttt{\{ x = 120, y = 121 \}}), 29\% of them create
a list or a tuple (a table with consecutive integer keys such as \texttt{\{ "one", "two", "three", "four" \}}), 26\% of them create an empty table, and 8\% of them create a table with
both a record part and a list part. The remaining 1\% of them create a table with non-literal keys.

We also analysed expressions that access a table (130,448 instances), either to read a value out of it (86\% of such expressions) or to write a 
value into it (14\% of such expressions). Most of the expressions
that read a value out of a table use a literal key,
either a literal string (89\% of the reads) or a literal
number (4\% of the reads). Lua has syntactic sugar that
turns field accesses using ``dot'' notation into table
accesses through literal string keys.

In 45\% of the expressions that read a value out of a table,
its value is immediately called. These calls are split almost
evenly between function calls and method calls (25\% of reads
are function calls, 20\% of reads are method calls).
These results show that the use of tables as a namespacing
mechanism for modules, as objects, and as records, 
is prevalent.

Expressions that write into a table also mostly use
literal keys (69\% of them use a literal string, 2\%
of them use a literal number), although a large 29\%
of writes use non-literal keys, in contrast to only 7\% of
reads.

In order to gauge how frequently tables are used as
collections, we looked for the presence of code that
iterates over tables, and found out that 23\% of the
scripts in our corpus iterate over a map at least once,
and 27\% of the scripts iterate over a list at least once.

Besides measuring the number of method calls, we also
measured other kinds of expressions and statements to gauge
how frequently Lua programmers use the object-oriented paradigm.
Our corpus has 24,858 function declarations, and 23\% of these
are method declarations. Of the 262 projects in our corpus,
63\% use {\em metatables}, a metaprogramming mechanism that
lets the program extend the behavior of tables and is mostly
used for implementing prototype-based inheritance.

Typed Lua's table types, described in the next section,
reflect the way Lua programmers use Lua tables in practice,
according to the results above. A table type can express
records, tuples, maps, lists, objects, or a combination of
those.

Lua modules are tables containing the modules' exports.
The current idiomatic way to define a module is to populate
an initially empty table with the members the module wishes
to export, and return this table at the end of the script
that defines the module. Around two-thirds of the modules
in our corpus use this idiom. The other third uses a
deprecated idiom, where a call to the {\tt module} function
at the top of the script installs a new global environment,
and the exported members are global variables assigned to
in the rest of the module.

The global environment is a also a table, so the two idioms
are equivalent in terms of how Typed Lua deals with module
definitions. Modules in Typed Lua make extensive use of Typed
Lua's rules for evolution of table types
(Section~\ref{sec:tables}).

In order to gauge how common is the use of dynamic type
checking as a way to define overloaded functions, we
measured how many of the functions inspect the
tags of their input parameters, and found out that 9\%
of the functions in our corpus do this, split evenly
between using the tag in an assertion, as a form of
type checking, and using the tag to decide which code
to execute, as a form of overloading. While Typed Lua
supports both union types and flow typing, general
support for overloaded functions still has to improve.

A Lua idiom that is specially problematic for static type
systems is overloading on the number and types and
multiple return values. Around 6\% of functions use
this idiom as a way to signal errors: in case of errors,
the function returns either {\tt nil} or {\tt false} plus
either an error message or error object instead of its usual
return values or throwing an exception. The caller then
tests the first returned value to check if an error occured
or not. Standard union types and flow typing cannot deal
with this, and Typed Lua introduces {\em projection types},
a way of tracking dependencies between types in a union of
tuples after a destructuring assignment.

\section{Types}
\label{sec:types}

In this section we present the abstract syntax of Typed Lua types
and their subtyping rules.

%In the previous chapter we presented an informal overview of Typed Lua.
%We showed that programmers can use Typed Lua to combine static and dynamic
%typing in the same code, and it allows them to incrementally migrate from
%dynamic to static typing.
%This is a benefit to programmers that use dynamically typed languages
%to build large applications, as static types detect many bugs
%during the development phase, and also provide better documentation.

%In this chapter we present the abstract syntax of Typed Lua types,
%the subtyping rules, and the most interesting typing rules.
%Besides its practical contributions, Typed Lua also has some interesting
%contributions to the field of optional type systems for scripting
%languages.
%They are novel type system features that let Typed Lua cover several Lua idioms
%and features, such as refinement of tables, multiple assignment, and multiple return values.

\subsection{Types}

\begin{figure*}[!ht]
\textbf{Type Language}\\
\dstart
$$
\begin{array}{rlr}
T ::= & & \textsc{first-level types:}\\
& \;\; L & \textit{literal types}\\
& | \; B & \textit{base types}\\
& | \; \Nil & \textit{nil type}\\
& | \; \Value & \textit{top type}\\
& | \; \Any & \textit{dynamic type}\\
& | \; \Self & \textit{self type}\\
& | \; T_{1} \cup T_{2} & \textit{disjoint union types}\\
& | \; S_{1} \rightarrow S_{2} & \textit{function types}\\
& | \; \{K_{1}{:}V_{1}, ..., K_{n}{:}V_{n}\}_{unique|open|fixed|closed} & \textit{table types}\\
& | \; x & \textit{type variables}\\
& | \; \mu x.T & \textit{recursive types}\\
& | \; \phi(T_{1},T_{2}) & \textit{filter types}\\
& | \; \pi_{i}^{x} & \textit{projection types}\\
%\multicolumn{3}{c}{}\\
L ::= & & \textsc{{\small literal types:}}\\
& \;\; \False \; | \; \True \; | \; {\it int} \; | \; {\it float} \; | \; {\it string} &\\
%\multicolumn{3}{c}{}\\
B ::= & & \textsc{{\small base types:}}\\
& \;\; \Boolean \; | \; \Integer \; | \; \Number \; | \; \String &\\
%\multicolumn{3}{c}{}\\
K ::= & & \textsc{{\small key types:}}\\
& \;\; L \; | \; B \; | \; \Value &\\
%\multicolumn{3}{c}{}\\
V ::= & & \textsc{{\small value types:}}\\
& \;\; T \; | \; \Const \; T &\\ 
%\multicolumn{3}{c}{}\\
S ::= & & \textsc{second-level types:}\\
& \;\; P & \textit{tuple types}\\
& | \; S_{1} \sqcup S_{2} & \textit{unions of tuple types}\\
%\multicolumn{3}{c}{}\\
P ::= & & \textsc{{\small tuple types:}}\\
& \;\; \Void & \textit{void type}\\
& | \; T{*} & \textit{variadic types}\\
& | \; T \times P & \textit{pair types}
\end{array}
$$
\dend
\caption{The abstract syntax of Typed Lua types}
\label{fig:typelang}
\end{figure*}

Figure \ref{fig:typelang} presents the abstract syntax of
Typed Lua types.
Typed Lua splits types into two categories:
\emph{first-level types} and \emph{second-level types}.
First-level types represent first-class Lua values and
second-level types represent tuples of values that appear in 
assignments and function applications.
First-level types include literal types, base types, the type $\Nil$,
the top type $\Value$, the dynamic type $\Any$, the type $\Self$,
union types, function types, table types, recursive types,
filter types, and projection types.
Second-level types include tuple types and unions of tuple types.
Tuple types include the type $\Void$, variadic types, and pair types.
Types are ordered by a subtype relationship that we introduce
in the next section, so Lua values may belong to several distinct types.

Literal types represent the type of literal values.
They can be the boolean values $\False$ and $\True$,
an integer value, a floating point value, or a string value.
We will see that literal types are important in our treatment of
table types as records.

Typed Lua includes four base types: $\Boolean$, $\Integer$, $\Number$, and $\String$.
The base types $\Boolean$ and $\String$ represent the values that
Lua tags as \texttt{boolean} and \texttt{string} during run-time.
Lua 5.3 introduced two internal representations to the tag \texttt{number}:
\texttt{integer} for integer numbers and \texttt{float} for real numbers.
Lua does automatic promotion of \texttt{integer} values to \texttt{float}
values as needed.
We introduced the base type $\Number$ to represent \texttt{float} values,
and the base type $\Integer$ to represent \texttt{integer} values.
In the next section we will show that $\Integer$ is a subtype of $\Number$.
This allows programmers to keep using \texttt{integer} values where
\texttt{float} values are expected.

The type $\Nil$ is the type of \texttt{nil}, the value that Lua uses for
undefined variables, missing parameters, and missing table keys.

The type $\Value$ is the top type, which represents any Lua value.
This type along with variadic types help the type system to drop
extra values on assignments and function calls, thus preserving the
semantics of Lua in these cases.

Typed Lua uses the type $\Self$ to represent the \emph{receiver}
in object-oriented method definitions and method calls.
As we mentioned in our informal overview of Typed Lua \cite{maidl2014tl},
we need the type $\Self$ to prevent programs from indexing a method without
calling it with the correct receiver.

Union types $T_{1} \cup T_{2}$ represent data types that can hold a value
of two different types.

Function types have the form $S_{1} \rightarrow S_{2}$ and represent Lua functions,
where $S$ is a second-level type.

Second-level types are either tuple types or unions of tuple types.
Tuple types are tuples of first-level types that can end with
either an empty tuple or with a variadic type.
Typed Lua needs second-level types because tuples are not first-class
values in Lua, only appearing on argument passing, multiple returns,
and multiple assignments.
The type $\Void$ is the type of an empty tuple.
A variadic type $T{*}$ represents a sequence of values of type $T \cup \Nil$;
it is the type of a vararg expression.
Second-level types include unions of tuples because Lua programs
usually overload the return type of functions to denote error,
as we mentioned in Section \ref{sec:statistics}.
For clarity, we use the symbol $\sqcup$ to represent the union between
two different tuple types.
Note that $\cup$ represents the union between two first-level types,
while $\sqcup$ represents the union between two tuple types.

Back to first-level types, table types represent the various forms
that Lua tables can take.
The syntactical form of table types is $\{ K_{1}{:}V_{1}, ..., K_{n}{:}V_{n} \}_{tag}$,
where each $K_{i}$ represents the type of a table key,
and each $V_{i}$ represents the type of the value that table keys of type $K_{i}$ map to.
Key types can only be literal types, base types, or the top type.
We made this restriction to the type of the keys because the statistics
that we discussed in Section \ref{sec:statistics} showed that most
of the tables are records, lists, and hashes.
The type $\Value$ is an option when we need a loose table type.
For instance, $\{\Value:\Value\}_{closed}$ represents the type of a
table in which both indices and values can have any type.
Value types can be any first-level type, and can optionally include
the $\Const$ type to denote immutable values.

We also use the tags \emph{unique}, \emph{open}, \emph{fixed}, and \emph{closed}
to classify table types.
The tag \emph{unique} represents tables with no keys that do not
inhabit one of the table's key types, and with no alias.
In particular, the type of the table constructor has this tag.
The tag \emph{open} represents \emph{unique} table types that
have at least one alias.
The tag \emph{fixed} represents \emph{unique} table types that we
do not know how many aliases they have.
In particular, the type of a class has this tag.
The tag \emph{closed} represents table types that do not provide
any guarantees about keys with types not listed in the table type.
In particular, in the concrete syntax, type annotations, interface
declarations, and userdata declarations always describe \emph{closed} table types.
In the next sections we explain in more detail why we need
different table types.

Any table type has to be \emph{well-formed}.
Informally, a table type is well-formed if key types do not overlap.
In Section \ref{sec:tables} we formalize the definition of well-formed table types.
We delay the proper formalization of well-formed table types because we use
consistent-subtyping in this formalization.

Recursive types have the form $\mu x.T$,
where $T$ is a first-level type that $x$ represents.
As an example,
\[
\mu x.\{``info":\Integer, ``next":x \;\cup\; \Nil\}_{closed}
\]
is a type for singly-linked lists of integers.
In Typed Lua, we can use the following interface declaration
as an alias to this type:
\begin{verbatim}
    local interface Element
      info:integer
      next:Element?
    end
\end{verbatim}

Typed Lua includes filter types as a way to discriminate the type of local
variables inside conditions.
Our type system uses filter types to formalize the \texttt{type} predicates
that we introduced in our previous design of Typed Lua \cite{maidl2014tl}.
This means that \texttt{type} predicates use filter types of the form
$\phi(T_{1},T_{2})$ to discriminate local variables that are bound to
union types.
In a filter type $\phi(T_{1},T_{2})$, $T_{1}$ is the original type and
$T_{2}$ is the discriminated type.

Typed Lua includes projection types as a way to project
unions of tuple types into unions of first-level types.
In Section \ref{sec:projections} we will show in more detail how our type system
uses them as a mechanism for handling unions of tuple types,
when they appear in the right-hand side of the declaration of local variables,
as we introduced in our previous design of Typed Lua \cite{maidl2014tl}.
We also show how this feature allows our type system to constrain
the type of a local variable that depends on the type of another local variable.

Typed Lua includes the dynamic type $\Any$ for allowing programmers
to mix static and dynamic typing.
There is a subtle difference between the dynamic type $\Any$ and the top type $\Value$.
Although both types mean that they accept a value of any other type,
the type $\Value$ is not a good option for handling the
interaction between dynamically typed and statically typed code.
Gradual typing uses the dynamic type $\Any$ to identify
where it should insert run-time checks for asserting that dynamically
typed code does not violate statically typed code.
Typed Lua also uses the dynamic type $\Any$ in this sense,
though it is an optional type system.
More precisely, we use $\Any$ instead of $\Value$ to allow
programmers blending dynamic and static typing because we use the
consistent-subtyping relation to formalize our optional type system,
as it is a first step to switch Typed Lua from optional typing to
gradual typing in the future.

\subsection{Subtyping}

Our type system uses subtyping \cite{cardelli1984smi,abadi1996to} to order
types and consistent-subtyping \cite{siek2007objects,siek2013mutable}
to allow the interaction between statically and dynamically typed code.
We explain the subtyping and consistent-subtyping rules throughout this section.
However, we focus the discussion on the definition of subtyping because
we can combine the consistency and subtyping relations to achieve
consistent-subtyping \cite{siek2007objects,siek2013mutable}.
The differences between consistent-subtyping and subtyping are the way
they handle the dynamic type, and the fact that subtyping is transitive,
but consistent-subtyping is not.

We present the subtyping rules as a deduction system for the
subtyping relation $\senv \vdash T_{1} \subtype T_{2}$.
The variable $\senv$ is a set of pairs of recursion variables.
We need this set to record the hypotheses that we assume when checking
recursive types.

The subtyping rules for literal types and base types include the rules
for defining that literal types are subtypes of their respective base types,
and that $\Integer$ is a subtype of $\Number$:
\[
\begin{array}{c}
\begin{array}{c}
\mylabel{S-FALSE}\\
\senv \vdash \False \subtype \Boolean
\end{array}
\;
\begin{array}{c}
\mylabel{S-TRUE}\\
\senv \vdash \True \subtype \Boolean
\end{array}
\\ \\
\begin{array}{c}
\mylabel{S-STRING}\\
\senv \vdash {\it string} \subtype \String
\end{array}
\\ \\
\begin{array}{c}
\mylabel{S-INT1}\\
\senv \vdash {\it int} \subtype \Integer
\end{array}
\;
\begin{array}{c}
\mylabel{S-INT2}\\
\senv \vdash {\it int} \subtype \Number
\end{array}
\\ \\
\begin{array}{c}
\mylabel{S-FLOAT}\\
\senv \vdash {\it float} \subtype \Number
\end{array}
\;
\begin{array}{c}
\mylabel{S-INTEGER}\\
\senv \vdash \Integer \subtype \Number
\end{array}
\end{array}
\]

Subtyping is reflexive and transitive;
therefore, we could have omitted the rule \textsc{S-INT2}.
More precisely, we could have defined a transitive rule for first-level
types instead of defining specific rules for transitive cases.
For instance, a transitive rule would allow us to derive that
\[
\dfrac{\senv \vdash 1 \subtype \Integer \;\;\;
       \senv \vdash \Integer \subtype \Number}
      {\senv \vdash 1 \subtype \Number}
\]

However, we are using the subtyping rules as the template for defining
the consistent-subtyping rules, and consistent-subtyping is not
transitive.
More precisely, we want the subtyping and consistent-subtyping rules
to differ only in the way they handle the dynamic type.
Thus, we define the subtyping rules using an algorithmic approach
that is close to the implementation, as this approach allows us to use
subtyping to easily formalize consistent-subtyping.

Our type system includes the top type $\Value$,
so any first-level type is a subtype of $\Value$:
\[
\begin{array}{c}
\mylabel{S-VALUE}\\
\senv \vdash T \subtype \Value
\end{array}
\]

Many programming languages include a bottom type to represent
an empty value that programmers can use as a default expression,
and we could have used the type $\Nil$ for this role.
However, making $\Nil$ the bottom type would lead to several expressions
that would pass the type checker, but that would fail during run-time
in the presence of a \texttt{nil} value.
Thus, our type system does not have a bottom type, and $\Nil$ is a
subtype only of itself and of $\Value$.

Another type that is only a subtype of itself and of the type $\Value$
is the type $\Self$.

The subtyping rules for union types are standard:
\[
\begin{array}{c}
\begin{array}{c}
\mylabel{S-UNION1}\\
\dfrac{\senv \vdash T_{1} \subtype T \;\;\;
       \senv \vdash T_{2} \subtype T}
      {\senv \vdash T_{1} \cup T_{2} \subtype T}
\end{array}
\\ \\
\begin{array}{c}
\mylabel{S-UNION2}\\
\dfrac{\senv \vdash T \subtype T_{1}}
      {\senv \vdash T \subtype T_{1} \cup T_{2}}
\end{array}
\;
\begin{array}{c}
\mylabel{S-UNION3}\\
\dfrac{\senv \vdash T \subtype T_{2}}
      {\senv \vdash T \subtype T_{1} \cup T_{2}}
\end{array}
\end{array}
\]

The first rule shows that a union type $T_{1} \cup T_{2}$
is a subtype of $T$ if both $T_{1}$ and $T_{2}$ are subtypes
of $T$;
and the other rules show that a type $T$ is a subtype
of a union type $T_{1} \cup T_{2}$ if $T$ is a subtype of
either $T_{1}$ or $T_{2}$.

The subtyping rule for function types is also standard:
\[
\begin{array}{c}
\mylabel{S-FUNCTION}\\
\dfrac{\senv \vdash S_{3} \subtype S_{1} \;\;\;
       \senv \vdash S_{2} \subtype S_{4}}
      {\senv \vdash S_{1} \rightarrow S_{2} \subtype S_{3} \rightarrow S_{4}}
\end{array}
\]

The rule \textsc{S-FUNCTION} shows that subtyping between
function types is contravariant on the type of the parameter list
and covariant on the return type.
In the previous section we explained why our type system uses
second-level types to represent the type of the parameter list
and the return type.
Now, we explain their subtyping rules.

The type $\Void$ is a subtype of itself and of a variadic type:
\[
\begin{array}{c}
\mylabel{S-VOID}\\
\senv \vdash \Void \subtype T{*}
\end{array}
\]

A variadic type $T{*}$ represents a sequence of values of type
$T \cup \Nil$, and the rule \textsc{S-VOID} handles the case where
a given sequence is empty.

The subtyping rule for pair types is the standard covariant rule:
\[
\begin{array}{c}
\mylabel{S-PAIR}\\
\dfrac{\senv \vdash T_{1} \subtype T_{2} \;\;\;
       \senv \vdash P_{1} \subtype P_{2}}
      {\senv \vdash T_{1} \times P_{1} \subtype T_{2} \times P_{2}}
\end{array}
\]

The subtyping rules for variadic types are not so obvious.
We need six different subtyping rules for variadic types
to handle all the cases where they can appear.

The rule \textsc{S-VARARG1} is a special rule for handling the
case where we give a sequence of $\Nil$ to the empty tuple:
\[
\begin{array}{c}
\mylabel{S-VARARG1}\\
\senv \vdash \Nil{*} \subtype \Void
\end{array}
\]

The rule \textsc{S-VARARG2} handles the case where both tuple types end
with variadic types:
\[
\begin{array}{c}
\mylabel{S-VARARG2}\\
\dfrac{\senv \vdash T_{1} \cup \Nil \subtype T_{2} \cup \Nil}
      {\senv \vdash T_{1}{*} \subtype T_{2}{*}}
\end{array}
\]

This rule shows that $T_{1}{*}$ is a subtype of $T_{2}{*}$
if $T_{1} \cup \Nil$ is a subtype of $T_{2} \cup \Nil$.
It explicitly includes $\Nil$ in both sides because otherwise
$\Nil{*}$ would not be a subtype of several other variadic types.
For instance, $\Nil{*}$ would not be a subtype of $\Number{*}$,
as $\Nil \not\subtype \Number$.

The other rules handle the cases where only one tuple type ends with a variadic type:
\[
\begin{array}{c}
\begin{array}{c}
\mylabel{S-VARARG3}\\
\dfrac{\senv \vdash T_{1} \cup \Nil \subtype T_{2}}
      {\senv \vdash T_{1}{*} \subtype T_{2} \times \Void}
\end{array}
\;
\begin{array}{c}
\mylabel{S-VARARG4}\\
\dfrac{\senv \vdash T_{1} \subtype T_{2} \cup \Nil}
      {\senv \vdash T_{1} \times \Void \subtype T_{2}{*}}
\end{array}
\\ \\
\begin{array}{c}
\mylabel{S-VARARG5}\\
\dfrac{\senv \vdash T_{1}{*} \subtype T_{2} \times \Void \;\;\;
       \senv \vdash T_{1}{*} \subtype P_{2}}
      {\senv \vdash T_{1}{*} \subtype T_{2} \times P_{2}}
\end{array}
\\ \\
\begin{array}{c}
\mylabel{S-VARARG6}\\
\dfrac{\senv \vdash T_{1} \times \Void \subtype T_{2}{*} \;\;\;
       \senv \vdash P_{1} \subtype T_{2}{*}}
      {\senv \vdash T_{1} \times P_{1} \subtype T_{2}{*}}
\end{array}
\end{array}
\]

Note that the case where both tuple types end with the type $\Void$ does
not require any special rule.
We use the subtyping rules for variadic types along with the types
$\Value$ and $\Nil$ to make our type system reflect the semantics of
Lua on discarding extra parameters and replacing missing parameters.

The subtyping rules for unions of tuple types are similar to the
subtyping rules for unions of first-level types:
\[
\begin{array}{c}
\begin{array}{c}
\mylabel{S-UNION4}\\
\dfrac{\senv \vdash S_{1} \subtype S \;\;\;
       \senv \vdash S_{2} \subtype S}
      {\senv \vdash S_{1} \sqcup S_{2} \subtype S}
\end{array}
\\ \\
\begin{array}{c}
\mylabel{S-UNION5}\\
\dfrac{\senv \vdash S \subtype S_{1}}
      {\senv \vdash S \subtype S_{1} \sqcup S_{2}}
\end{array}
\;
\begin{array}{c}
\mylabel{S-UNION6}\\
\dfrac{\senv \vdash S \subtype S_{2}}
      {\senv \vdash S \subtype S_{1} \sqcup S_{2}}
\end{array}
\end{array}
\]

Back to the subtyping rules between first-level types,
the subtyping rule among a \emph{fixed} or \emph{closed}
table type and another \emph{closed} table type resembles the
standard subtyping rule between records:
\[
\begin{array}{c}
\mylabel{S-TABLE1}\\
\dfrac{\begin{array}{c}
       \forall i \in 1..n \; \exists j \in 1..m \\
       \senv \vdash K_{j} \subtype K_{i}' \;\;\;
       \senv \vdash K_{i}' \subtype K_{j} \;\;\;
       \senv \vdash V_{j} \subtype_{c} V_{i}'
       \end{array}}
      {\begin{array}{c}
       \senv \vdash \{K_{1}{:}V_{1}, ..., K_{m}{:}V_{m}\}_{fixed|closed} \subtype\\
                    \{K_{1}'{:}V_{1}', ..., K_{n}'{:}V_{n}'\}_{closed}
       \end{array}} \; m \ge n
\end{array}
\]

The rule \textsc{S-TABLE1} allows width subtyping and introduces the
auxiliary relation $\subtype_{c}$ to handle depth subtyping on the
type of the values stored in the table fields.
We need an auxiliary relation because the subtyping of the
type of the values stored in the table fields changes according to
the tags of the table types.
We define the relation $\subtype_{c}$ as follows:
\[
\begin{array}{c}
\begin{array}{c}
\mylabel{S-FIELD1}\\
\dfrac{\senv \vdash V_{1} \subtype V_{2} \;\;\;
       \senv \vdash V_{2} \subtype V_{1}}
      {\senv \vdash V_{1} \subtype_{c} V_{2}}
\end{array}
\;
\begin{array}{c}
\mylabel{S-FIELD2}\\
\dfrac{\senv \vdash V_{1} \subtype V_{2}}
      {\senv \vdash \Const \; V_{1} \subtype_{c} \Const \; V_{2}}
\end{array}
\\ \\
\begin{array}{c}
\mylabel{S-FIELD3}\\
\dfrac{\senv \vdash V_{1} \subtype V_{2}}
      {\senv \vdash V_{1} \subtype_{c} \Const \; V_{2}}
\end{array}
\end{array}
\]

These rules allow depth subtyping on $\Const$ fields.
The rule \textsc{S-FIELD1} defines that mutable fields are invariant,
while the rule \textsc{S-FIELD2} defines that immutable fields are covariant.
The rule \textsc{S-FIELD3} defines that it is safe to promote fields
from mutable to immutable.
We do not include a rule that allows promoting fields from immutable
to mutable because this would be unsafe due to variance.

There is a limitation on \emph{closed} table types that led us to
introduce \emph{open} and \emph{unique} table types.
If the table constructor had a \emph{closed} table type, then
programmers would not be able to use it to initialize a variable with
a table type that describes a more general type.
For instance,
\begin{verbatim}
    local t:{ "x":integer,
              "y":integer? } = { x = 1, y = 2 }
\end{verbatim}
would not type check, as the type of the table constructor would not
be a subtype of the type in the annotation.
More precisely,
\begin{align*}
& \{``x":1, ``y":2\}_{closed} \not\subtype \\
& \{``x":\Integer, ``y":\Integer \cup \Nil\}_{closed}
\end{align*}

Simply promoting the type of each table value to its supertype would
not overcome this limitation, as it still would give to the table constructor
a \emph{closed} table type without covariant mutable fields.
Thus, programmers would not be able to use the table constructor to
initialize a variable with a table type that includes an optional field.
Using the previous example,
\begin{align*}
& \{``x":\Integer, ``y":\Integer\}_{closed} \not\subtype \\
& \{``x":\Integer, ``y":\Integer \cup \Nil\}_{closed}
\end{align*}

We introduced \emph{unique} table types to avoid this limitation,
as they represent the type of tables with no keys that do not
inhabit one of the table's key types, and with no alias.
In particular, this is the case of the table constructor.
The following subtyping rule defines the subtyping relation among
\emph{unique} table types and \emph{closed} table types:
\[
\begin{array}{c}
\mylabel{S-TABLE2}\\
\dfrac{\begin{array}{c}
       \forall i \in 1..m \; \forall j \in 1..n \;
       \senv \vdash K_{i} \subtype K_{j}' \to \senv \vdash V_{i} \subtype_{u} V_{j}' \\
       \forall j \in 1..n \; \not\exists i \in 1..m \;
       \senv \vdash K_{i} \subtype K_{j}' \to \senv \vdash \Nil \subtype_{o} V_{j}'
       \end{array}}
      {\senv \vdash \{K_{1}{:}V_{1}, ..., K_{m}{:}V_{m}\}_{unique} \subtype
                    \{K_{1}'{:}V_{1}', ..., K_{n}'{:}V_{n}'\}_{closed}}
\end{array}
\]

The rule \textsc{S-TABLE2} allows width subtyping and covariant keys.
It allows covariant keys because we also want to use \emph{unique}
table types as a way to join table fields that inhabit \emph{closed} table types.
For instance, we want to use the table constructor to initialize
a variable with a table type that describes a hash.

The rule \textsc{S-TABLE2} introduced the auxiliary relations
$\subtype_{u}$ and $\subtype_{o}$.
The first allows depth subtyping on all fields,
while the second allows the omission of optional fields.
We define them as follows:
\[
\begin{array}{c}
\begin{array}{c}
\mylabel{S-FIELD4}\\
\dfrac{\senv \vdash V_{1} \subtype V_{2}}
      {\senv \vdash V_{1} \subtype_{u} V_{2}}
\end{array}
\\ \\
\begin{array}{c}
\mylabel{S-FIELD5}\\
\dfrac{\senv \vdash V_{1} \subtype V_{2}}
      {\senv \vdash \Const \; V_{1} \subtype_{u} \Const \; V_{2}}
\end{array}
\;
\begin{array}{c}
\mylabel{S-FIELD6}\\
\dfrac{\senv \vdash V_{1} \subtype V_{2}}
      {\senv \vdash V_{1} \subtype_{u} \Const \; V_{2}}
\end{array}
\\ \\
\begin{array}{c}
\mylabel{S-FIELD7}\\
\dfrac{\senv \vdash \Nil \subtype V}
      {\senv \vdash \Nil \subtype_{o} V}
\end{array}
\;
\begin{array}{c}
\mylabel{S-FIELD8}\\
\dfrac{\senv \vdash \Nil \subtype V}
      {\senv \vdash \Nil \subtype_{o} \Const \; V}
\end{array}
\end{array}
\]

Using \emph{unique} table types to represent the type of the table
constructor allows our type system to type check the previous example.
More precisely,
\begin{align*}
& \{``x":1, ``y":2\}_{unique} \subtype \\
& \{``x":\Integer, ``y":\Integer \cup \Nil\}_{closed}
\end{align*}

Even though we allow width subtyping between \emph{unique} and \emph{closed}
table types, we do not allow it among \emph{unique} and other table types
because it would violate our definition of these other table types:
\[
\begin{array}{c}
\mylabel{S-TABLE3}\\
\dfrac{\begin{array}{c}
       \forall i \in 1..m \\
       \exists j \in 1..n \;
       \senv \vdash K_{i} \subtype K_{j}' \land \senv \vdash V_{i} \subtype_{u} V_{j}' \\
       \forall j \in 1..n \; \not\exists i \in 1..m \;
       \senv \vdash K_{i} \subtype K_{j}' \to \senv \vdash \Nil \subtype_{o} V_{j}'
       \end{array}}
      {\begin{array}{c}
       \senv \vdash \{K_{1}{:}V_{1}, ..., K_{m}{:}V_{m}\}_{unique} \subtype\\
                    \{K_{1}'{:}V_{1}', ..., K_{n}'{:}V_{n}'\}_{unique|open|fixed}
       \end{array}}
\end{array}
\]

The rule that handles subtyping between \emph{open} and \emph{closed} table
types allows width subtyping:
\[
\begin{array}{c}
\mylabel{S-TABLE4}\\
\dfrac{\begin{array}{c}
       \forall i \in 1..m \; \forall j \in 1..n \;
       \senv \vdash K_{i} \subtype K_{j}' \to \senv \vdash V_{i} \subtype_{c} V_{j}' \\
       \forall j \in 1..n \; \not\exists i \in 1..m \;
       \senv \vdash K_{i} \subtype K_{j}' \to \senv \vdash \Nil \subtype_{o} V_{j}'
       \end{array}}
      {\senv \vdash \{K_{1}{:}V_{1}, ..., K_{m}{:}V_{m}\}_{open} \subtype
                    \{K_{1}'{:}V_{1}', ..., K_{n}'{:}V_{n}'\}_{closed}}
\end{array}
\]

However, the rule that handles subtyping among \emph{open} and
\emph{open} or \emph{fixed} table types does not allow width subtyping:
\[
\begin{array}{c}
\mylabel{S-TABLE5}\\
\dfrac{\begin{array}{c}
       \forall i \in 1..m \\
       \exists j \in 1..n \;
       \senv \vdash K_{i} \subtype K_{j}' \land \senv \vdash V_{i} \subtype_{c} V_{j}' \\
       \forall j \in 1..n \; \not\exists i \in 1..m \;
       \senv \vdash K_{i} \subtype K_{j}' \to \senv \vdash \Nil \subtype_{o} V_{j}'
       \end{array}}
      {\begin{array}{c}
       \senv \vdash \{K_{1}{:}V_{1}, ..., K_{m}{:}V_{m}\}_{open} \subtype\\
                    \{K_{1}'{:}V_{1}', ..., K_{n}'{:}V_{n}'\}_{open|fixed}
       \end{array}}
\end{array}
\]

The rules \textsc{S-TABLE4} and \textsc{S-TABLE5} allow joining fields
plus omitting optional fields.
Both rules use $\subtype_{c}$ to allow depth subtyping on $\Const$ fields only.

We introduced \emph{fixed} table types because we needed a safe way
to represent the type of classes that can allow single inheritance
through the refinement of table types.
The rule that handles subtyping between \emph{fixed} table types
does not allow width subtyping, joining fields, and omitting fields,
but it allows depth subtyping on $\Const$ fields:
\[
\begin{array}{c}
\mylabel{S-TABLE6}\\
\dfrac{\begin{array}{c}
       \forall i \in 1..n \; \exists j \in 1..n \\
       \senv \vdash K_{j} \subtype K_{i}' \;\;\;
       \senv \vdash K_{i}' \subtype K_{j} \;\;\;
       \senv \vdash V_{j} \subtype_{c} V_{i}'
       \end{array}}
      {\senv \vdash \{K_{1}{:}V_{1}, ..., K_{n}{:}V_{n}\}_{fixed} \subtype
                    \{K_{1}'{:}V_{1}', ..., K_{n}'{:}V_{n}'\}_{fixed}}
\end{array}
\]

In Section \ref{sec:tables} we will show in more detail how our type system
uses these tags to handle the refinement of table types.

We use the \emph{Amber rule} \cite{cardelli1986amber} to define
subtyping between recursive types:
\[
\begin{array}{c}
\begin{array}{c}
\mylabel{S-AMBER}\\
\dfrac{\senv[x_{1} \subtype x_{2}] \vdash T_{1} \subtype T_{2}}
      {\senv \vdash \mu x_{1}.T_{1} \subtype \mu x_{2}.T_{2}}
\end{array}
\;
\begin{array}{c}
\mylabel{S-ASSUMPTION}\\
\dfrac{x_{1} \subtype x_{2} \in \senv}
      {\senv \vdash x_{1} \subtype x_{2}}
\end{array}
\end{array}
\]

The rule \textsc{S-AMBER} also uses the rule \textsc{S-ASSUMPTION}
to check whether $\mu x_{1}.T_{1} \subtype \mu x_{2}.T_{2}$.
Both rules use the set of assumptions $\senv$,
where each assumption is a pair of recursion variables.
The rule \textsc{S-AMBER} extends $\senv$ with the assumption
$x_{1} \subtype x_{2}$ to check whether $T_{1} \subtype T_{2}$.
The rule \textsc{S-ASSUMPTION} allows the rule \textsc{S-AMBER}
to check whether an assumption is valid.

A recursive type may appear inside a first-level type, and our
type system includes subtyping rules to handle subtyping between
recursive types and other first-level types:
\[
\begin{array}{c}
\begin{array}{c}
\mylabel{S-UNFOLDR}\\
\dfrac{\senv \vdash T_{1} \subtype [x \mapsto \mu x.T_{2}]T_{2}}
      {\senv \vdash T_{1} \subtype \mu x.T_{2}}
\end{array}
\;
\begin{array}{c}
\mylabel{S-UNFOLDL}\\
\dfrac{\senv \vdash [x \mapsto \mu x.T_{1}]T_{1} \subtype T_{2}}
      {\senv \vdash \mu x.T_{1} \subtype T_{2}}
\end{array}
\end{array}
\]

As an example, the rule \textsc{S-UNFOLDR} allows our type system to
type check a function that inserts an element in the begining of
a singly-linked list:
\begin{verbatim}
    local function insert (e:Element?,
                           v:integer):Element
      return { info = v, next = e }
    end
\end{verbatim}
that is, the type checker uses the rule \textsc{S-UNFOLDR} to verify whether
the type of the table constructor is a subtype of \texttt{Element}:
\begin{align*}
\{ & ``info":\Integer, ``next":\mu x.\{ ``info":\Integer, \\
   & ``next":x \;\cup\; \Nil\}_{closed} \cup \Nil \}_{unique} \subtype \\
& \mu x.\{``info":\Integer, ``next":x \;\cup\; \Nil\}_{closed}
\end{align*}

Filter types are subtypes only of themselves and of $\Value$.
More precisely, a filter type $\phi(T_{1},T_{2})$ is a subtype of
the same filter type $\phi(T_{1},T_{2})$, which shares the same
types $T_{1}$ and $T_{2}$, and it is also a subtype of $\Value$.

Projection types are subtypes only of themselves and of $\Value$.
More precisely, a projection type $\pi_{i}^{x}$ is a subtype of the
same projection type $\pi_{i}^{x}$, which shares the same union of
tuples $x$ and the same index $i$, and it is also a subtype of $\Value$.

The dynamic type $\Any$ is neither the bottom nor the top type,
but a separate type that is subtype only of itself and of $\Value$.

Even though the dynamic type $\Any$ does not interact with subtyping,
it does interact with consistent-subtyping.
We present the consistent-subtyping rules as a deduction system for
the consistent-subtyping relation $\senv \vdash T_{1} \lesssim T_{2}$.
As in the subtyping relation, $\senv$ is also a set of pairs of
recursion variables.
We define the consistent-subtyping rules for the dynamic type $\Any$
as follows:
\[
\begin{array}{c}
\begin{array}{c}
\mylabel{C-ANY1}\\
\senv \vdash T \lesssim \Any
\end{array}
\;
\begin{array}{c}
\mylabel{C-ANY2}\\
\senv \vdash \Any \lesssim T
\end{array}
\end{array}
\]

If we had set the type $\Any$ as both bottom and top types of our
subtyping relation, then any type $T_{1}$ would be a subtype of
any other type $T_{2}$.
The consequence of this is that all programs would type check without errors.
This would happen due to the transitivity of subtyping, that is,
we would be able to down-cast any type $T_{1}$ to $\Any$ and then up-cast
$\Any$ to any other type $T_{2}$.
The rules \textsc{C-ANY1} and \textsc{C-ANY2} are the rules that
allow the dynamic type to interact with other first-level types,
and thus allow dynamically typed code to coexist with statically
typed code.
Because of these two rules, consistent-subtyping cannot be transitive.
These two rules are the only rules that differ between
subtyping and consistent-subtyping, if we implement the subtyping rules
as we do in this section.

In the implementation of Typed Lua we also use consistent-subtyping to
normalize and simplify union types, though we let union types free in
the formalization.
For instance, the union type \texttt{boolean|any} results in the
type \texttt{any}, because \texttt{boolean} is consistent-subtype
of \texttt{any}.
Another example is the union type \texttt{number|nil|1} that
results in the union type \texttt{number|nil}, because
\texttt{1} is consistent-subtype of \texttt{number}.

\section{Typing rules}
\label{sec:rules}

In this section we use a reduced core of Typed Lua to present
the typing rules that type check the novel features of our type system.
These rules type check table refinement and
overloading on the return type of functions.

Our core limits control flow to if and while statements;
it has explicit type annotations, explicit scope for variables,
explicit method declarations, and explicit method calls.
Here is a list of features that are not present in our reduced core:
\begin{itemize}
\item labels and goto statements (they are difficult to handle along
with our simplified form of \emph{flow typing}, and they are out of
scope for now);
\item explicit blocks (we are already using explicit scope for variables);
\item other loop structures such as repeat-until, numeric for,
and generic for (we can use while to express them);
\item table fields other than $[e_{1}] = e_{2}$
(we can use this form to express the missing forms); 
\item arithmetic operators other than $+$
(other arithmetic operators have similar typing rules);
\item relational operators other than $==$ and $<$
(inequality has similar typing rules to $==$ and
other relational operators have similar typing rules to $<$);
\item bitwise operators other than $\&$
(other bitwise operators have similar typing rules).
\end{itemize}

Our reduced core does not lose much expressiveness, as it can express
any Lua program except those that use labels and goto statements.

\begin{figure*}[!ht]
\textbf{Abstract Syntax}\\
\dstart
$$
\begin{array}{rlr}
s ::= & & \textsc{statements:}\\
& \;\; \mathbf{skip} & \textit{skip}\\
& | \; s_{1} \; ; \; s_{2} & \textit{sequence}\\
& | \; \overline{l} = el & \textit{multiple assignment}\\
& | \; \mathbf{while} \; e \; \mathbf{do} \; s \;
| \; \mathbf{if} \; e \; \mathbf{then} \; s_{1} \; \mathbf{else} \; s_{2} & \textit{control flow}\\
& | \; \mathbf{local} \; \overline{id{:}T} = el \; \mathbf{in} \; s & \textit{variable declaration}\\
& | \; \mathbf{local} \; \overline{id} = el \; \mathbf{in} \; s & \textit{variable declaration}\\
& | \; \mathbf{rec} \; id{:}T = f \; \mathbf{in} \; s & \textit{recursive function} \\
& | \; \mathbf{return} \; el & \textit{return} \\
& | \; \lfloor a \rfloor_{0} & \textit{application with no results}\\
& | \; \mathbf{fun} \; id_{1}{:}id_{2} \; (pl){:}S \; s \;;\; \mathbf{return} \; el & \textit{method declaration}\\
e ::= & & \textsc{expressions:}\\
& \;\; \mathbf{nil} & \textit{nil}\\
& | \; k & \textit{other literals}\\
& | \; id & \textit{variable access}\\
& | \; e_{1}[e_{2}] & \textit{table access}\\
& | \; {<}T{>} \; id & \textit{type coercion}\\
& | \; f & \textit{function declaration}\\
& | \; \{ \; \overline{[e_{1}] = e_{2}} \; \} \;
| \; \{ \; \overline{[e_{1}] = e_{2}},me \; \} & \textit{table constructor}\\
& | \; e_{1} + e_{2} \;
| \; e_{1} \; {..} \; e_{2} \;
| \; e_{1} == e_{2} \;
| \; e_{1} < e_{2} & \textit{binary operations}\\
& | \; e_{1} \;\&\; e_{2} \;
| \; e_{1} \; \mathbf{and} \; e_{2} \;
| \; e_{1} \; \mathbf{or} \; e_{2} & \textit{binary operations}\\
& | \; \mathbf{not} \; e \;
| \; \# \; e & \textit{unary operations} \\
& | \; \lfloor me \rfloor_{1} & \textit{expressions with one result}\\
l ::= & & \textsc{left-hand values:}\\
& \;\; id_{l} & \textit{variable assignment}\\
& | \; e_{1}[e_{2}] & \textit{table assignment}\\
& | \; id[k] \; {<}T{>} & \textit{type coercion}\\
k ::= & & \textsc{literal constants:}\\
& \;\; \mathbf{false} \; | \;
\mathbf{true} \; | \;
{\it int} \; | \;
{\it float} \; | \;
{\it string} & \\
el ::= & & \textsc{expression lists:}\\
& \;\; \overline{e} \; | \;
\overline{e}, me & \\
me ::= & & \textsc{multiple results:}\\
& \;\; a & \textit{application}\\
& | \; {...} & \textit{vararg expression}\\
a ::= & & \textsc{applications:}\\
& \;\; e(el) & \textit{function application}\\
& | \; e{:}n(el) & \textit{method application}\\
f ::= & & \textsc{function declarations:}\\
& \;\; \mathbf{fun} \; (pl){:}S \; s \;;\; \mathbf{return} \; el & \\
pl ::= & & \textsc{parameter lists:}\\
& \;\; \overline{id{:}T} \; | \;
\overline{id{:}T},{...}{:}T & \\
\end{array}
$$
\dend
\caption{The abstract syntax of Typed Lua}
\label{fig:syntax}
\end{figure*}

Figure \ref{fig:syntax} presents the abstract syntax of core Typed Lua.
It splits the syntactic categories as follows:
$s$ are statements, $e$ are expressions, $l$ are left-hand values,
$k$ are literal constants, $el$ are expression lists,
$me$ are expressions with multiple results, $a$ are function and method applications,
$f$ are function declarations, $pl$ are parameter lists,
$id$ are variable names, $T$ are first-level types, and $S$ are second-level types.
The notation $\overline{id{:}T}$ denotes the list $id_{1}{:}T_{1}, ..., id_{n}{:}T_{n}$.

Our reduced core includes two statements for declaring local variables,
one with and another without type annotations.
While we use the former to formalize how our type system handles the declaration
of annotated variables, we use the latter to formalize how our type system
handles the declaration of unannotated variables through local type inference
and also the introduction of projection types.

Our reduced core also includes a truncation operator $\lfloor \rfloor$ for
function applications, method applications, and the vararg expression.
We use $\lfloor a \rfloor_{0}$ to denote function and method applications
that produce no value, because they appear as statements.
We use $\lfloor me \rfloor_{1}$ to denote function applications,
method applications, and vararg expressions that produce only one value,
even if they return multiple values.

We also include two kinds of type coercions in our core language:
the left-hand value $id[k] \; {<}T{>}$ and the expression ${<}T{>} \;id$.
Both allow the refinement of table types.
We also split variable names into two categories to have safe aliasing
of tables in the presence of refinement.
We use $id$ when variable names appear as expressions and $id_{l}$ when
variable names appear as left-hand values.

Even though we can assign only first-level types to variables,
functions and methods can return unions of second-level types,
and our type system should be able to project these unions of
second-level types into unions of first-level types.
We use two different environments to handle this feature.
The first environment is the type environment $\env$ that maps
variables to first-level types.
We use $\env_{1}[id \mapsto T]$ to extend the environment $\env_{1}$
with the variable $id$ that maps to type $T$.
The second environment is the projection environment $\penv$ that
maps projection variables to second-level types.
We use $\penv[x \mapsto S]$ to extend the environment $\penv$
with the projection variable $x$ that maps to type $S$.
In Section \ref{sec:projections} we will show how our type system uses the
projection environment $\penv$ for handling projection types,
and also for projecting unions of second-level types into
unions of first-level types.

We present the typing rules as a deduction system for two typing relations,
one for typing statements and another for typing expressions.

We use the relation $\env_{1}, \penv \vdash s, \env_{2}$ for typing statements.
This relation means that given a type environment $\env_{1}$
and a projection environment $\penv$, we can check that a statement $s$
produces a new type environment $\env_{2}$.

We use the relation $\env_{1}, \penv \vdash e : T, \env_{2}$ for typing expressions.
This relation means that given a type environment $\env_{1}$
and a projection environment $\penv$, we can check that an expression $e$ has
type $T$ and produces a new type environment $\env_{2}$.

\subsection{Tables and refinement}
\label{sec:tables}

Our abstract syntax reduces the syntactic forms of the table constructor
into two forms: $\{\;\overline{[e_{1}] = e_{2}}\;\}$ and
$\{\;\overline{[e_{1}] = e_{2}},me\;\}$.
The first uses a list of table fields $([e_{1}] = e_{2})_{1}, ..., ([e_{1}] = e_{2})_{n}$.
The second uses a list of table fields and an expression that can
produce multiple values.

The simplest expression involving tables in the empty table constructor.
Its type checking rule is straightforward:
\[
\begin{array}{c}
\mylabel{T-CONSTRUCTOR1}\\
\env_{1}, \penv \vdash \{\}:\{\}_{unique}, \env_{1}
\end{array}
\]

As a more interesting example, let us see how our type system type checks
the table constructor $\{ [1] = ``x", [2] = ``y", [3] = ``z" \}$.

First, our type system uses the auxiliary relation
$\env_{1}, \penv \vdash [e_{1}] = e_{2} : (K,V), \env_{2}$ to type check each
table field.
This auxiliary relation means that given a type environment $\env_{1}$
and a projection environment $\penv$, checking a table field $[e_{1}] = e_{2}$
produces a pair $(K,V)$ and a new type environment $\env_{2}$.
A pair $(K,V)$ means that $e_{1}$ has type $K$ and $e_{2}$ has type $V$,
where $K$ is the type of the key and $V$ is the type of the value.

After type checking each table field, our type system uses each pair $(K,V)$
to build the table type that express the type of a given constructor, and
uses the predicate \emph{wf} to check whether this table type is well-formed.
Formally, a table type is well-formed if it obeys the following rule:
\[
\forall i \not\exists j \; i \not= j \wedge K_{i} \lesssim K_{j}
\]

Well-formed table types avoid ambiguity.
For instance, this rule detects that the table type
$\{1:\Number, \Integer:\String, \Any:\Boolean\}$ is ambiguous,
because the type of the value stored by key $1$ can be
$\Number$, $\String$, or $\Boolean$, as $1 \lesssim 1$,
$1 \lesssim \Integer$, and $1 \lesssim \Any$.
Moreover, the type of the value stored by a key of type $\Integer$,
which is not the literal type $1$, can be $\Number$ or $\Boolean$,
as $\Integer \lesssim \Integer$, and $\Integer \lesssim \Any$.

Well-formed table types also do not allow \emph{unique} and
\emph{open} table types to appear in the type of the values.
We made this restriction because our type system does not keep
track of aliases to table fields.
This means that allowing \emph{unique} and \emph{open} table
types to appear in the type of a value would allow the
creation of unsafe aliases.
The rules that type check table fields use the auxiliary function
\emph{close} to close the type of the values in a table type.

The rule \textsc{T-CONSTRUCTOR2} uses these steps to type check a
table constructor with a non-empty list of table fields:
\[
\begin{array}{c}
\mylabel{T-CONSTRUCTOR2}\\
\dfrac{\begin{array}{c}
       \env_{1}, \penv \vdash ([e_{1}] = e_{2})_{i}:(K_{i},V_{i}), \env_{i+1} \\
       T = \{K_{1}{:}V_{1}, ..., K_{n}{:}V_{n}\}_{unique} \;\;\;
       wf(T) \;\;\;
       n = |\;\overline{[e_{1}] = e_{2}}\;| \\
       \env_{f} = merge(\env_{1}, ..., \env_{n+1})
       \end{array}}
      {\env_{1}, \penv \vdash \{\;\overline{[e_{1}] = e_{2}}\;\}:T, \env_{f}}
\end{array}
\]

Back to our example, the table constructor
$\{ [1] = ``x", [2] = ``y", [3] = ``z" \}$ has type
$\{1:``x", 2:``y", 3:``z"\}_{unique}$ through rule \textsc{T-CONSTRUCTOR2}.

As another example, the table constructor $\{[``x"] = 1, [``y"] = \{[``z"] = 2\}\}$
has type $\{``x":1, ``y":\{``z":2\}_{closed}\}_{unique}$ through rule
\textsc{T-CONSTRUCTOR2}.
The inner table is \emph{closed} to prevent the creation of unsafe aliases.

After presenting some typing rules of the table constructor,
we start the discussion of the rules that define the most
unusual feature of our type system: the refinement of table types.
The first kind of refinement allows programmers to add new
fields to \emph{unique} or \emph{open} table types through
field assignment.
As an example, we can translate
\begin{verbatim}
    local person = {}
    person.firstname = "Lou"
    person.lastname = "Reed"
\end{verbatim}
to our reduced core as follows:
\begin{center}
\begin{tabular}{ll}
\multicolumn{2}{l}{$\mathbf{local} \; person = \{\} \; \mathbf{in}$}\\
& \multicolumn{1}{l}{$person[``firstname"] \; {<}\String{>} = ``Lou";$}\\
& \multicolumn{1}{l}{$person[``lastname"] \; {<}\String{>} = ``Reed"$}
\end{tabular}
\end{center}

In this example, we assign the type $\{\}_{unique}$ to the variable
$person$, then we refine its type to $\{``firstname":\String\}_{unique}$,
and then to $\{``firstname":\String, ``lastname":\String\}_{unique}$.
Rule \textsc{T-REFINE} type checks this use of refinement:
\[
\begin{array}{c}
\mylabel{T-REFINE}\\
\dfrac{\begin{array}{c}
       \env_{1}(id) = \{ K_{1}{:}V_{1}, ..., K_{n}{:}V_{n} \}_{open|unique}\\
       \env_{1}, \penv \vdash k:K, \env_{2} \;\;\;
       \not \exists i \in 1..n \; K \lesssim K_{i} \;\;\;
       V = close(T)
       \end{array}}
      {\begin{array}{l}
       \env_{1}, \penv \vdash id[k] {<}T{>}:V, \\
       \env_{2}[id \mapsto \{ K_{1}{:}V_{1}, ..., K_{n}{:}V_{n}, K{:}V\}_{open|unique}]
       \end{array}}
\end{array}
\]

We restricted the refinement of table types to include only literal
keys, because its purpose is to make it easier the construction of
table types that represent records.

We use the refinement of table types to handle the declaration of
new global variables.
In Lua, the assignment \texttt{v = v + 1} translates to
\texttt{\string_ENV["v"] = \string_ENV["v"] + 1} when \texttt{v}
is not a local variable, where \texttt{\string_ENV} is a table
that stores the global environment.
For this reason, Typed Lua treats accesses to global variables as field accesses
to an \emph{open} table in the top-level scope.
In the following examples we assume that $\string_ENV$ is in the
environment and has type $\{\}_{open}$.

As an example,
\begin{align*}
& \string_ENV[``x"] \; {<}\String{>} = ``foo" \;; \\
& \string_ENV[``y"] \; {<}\Integer{>} = 1
\end{align*}
uses field assignment to add fields $``x"$ and $``y"$ to $\string_ENV$.
Therefore, after these field assignments $\string_ENV$ has type
$\{``x":\String, ``y":\Integer\}_{open}$.

We do not allow the refinement of table types to add a field if it is
already present in the table's type.
For instance,
\begin{align*}
& \string_ENV[``x"] \; {<}\String{>} = ``foo" \;; \\
& \string_ENV[``x"] \; {<}\Integer{>} = 1
\end{align*}
does not type check, as we are trying to add $``x"$ twice.

We also do not allow the refinement of table types to introduce
fields with table types that are not \emph{closed}.
For instance,
\begin{center}
\begin{tabular}{l}
$\string_ENV[``x"] \; {<}\{\}_{unique}{>} = \{\}$
\end{tabular}
\end{center}
refines the type of $\string_ENV$ from $\{\}_{open}$ to $\{``x":\{\}_{closed}\}_{open}$.
Currently, our type system can only track \emph{unique} and
\emph{open} table types that are bound to local variables.

We can also use multiple assignment to refine table types:
\[
\string_ENV[``x"] \; {<}\String{>}, \string_ENV[``y"] \; {<}\Integer{>} = ``foo", 1
\]

This example type checks because all the environment changes are consistent, and
$``foo" \times 1 \times \Nil{*} \lesssim \String \times \Integer \times \Value{*}$.
By consistent we mean that we are only adding new fields.
Nevertheless, the next example does not type check because it tries to add
the same field to $\string_ENV$, but with different types:
\[
\string_ENV[``x"] \; {<}\String{>}, \string_ENV[``x"] \; {<}\Integer{>} = ``foo", 1
\]

Aliasing an \emph{unique} or an \emph{open} table type can produce
either a \emph{closed} or a \emph{fixed} table type, depending on
the context that we are using a variable.
We need \emph{fixed} table types to type classes in object-oriented programming.
In the implementation we fix the aliasing of \emph{unique} and \emph{open}
table types that appear in a top-level return statement, and in other cases we
close the aliasing of \emph{unique} and \emph{open} table types.
However, in the formalization we chose to define this behavior in
a not deterministic way, as it makes easier the presentation of this behavior.

As an example,
\begin{center}
\begin{tabular}{lll}
\multicolumn{3}{l}{$\mathbf{local} \; a:\{\}_{unique} = \{\} \; \mathbf{in}$}\\
& \multicolumn{2}{l}{$\mathbf{local} \; b:\{\}_{open} = a \; \mathbf{in}$}\\
& & \multicolumn{1}{l}{$a[``x"] \; {<}\String{>} = ``foo";$}\\
& & \multicolumn{1}{l}{$b[``x"] \; {<}\Integer{>} = 1$}\\
\end{tabular}
\end{center}
does not type check, as aliasing $a$ produces the type $\{\}_{closed}$
that is not a subtype of $\{\}_{open}$, the type of $b$.
Our type system has this behavior to warn programmers about
potential unsafe behaviors after this kind of alias.
In this example, it is unsafe to add the field $``x"$ to $b$,
as it changes the value that is stored in the field $``x"$ of $a$.

The rules \textsc{T-IDREAD1} and \textsc{T-IDREAD2} define this non-deterministic behavior.
Rule \textsc{T-IDREAD1} uses the auxiliary function \emph{close} to
produce a \emph{closed} alias.
It also uses the auxiliary function \emph{open} to change the type of
the original reference from \emph{unique} to \emph{open},
because aliasing an \emph{unique} table type while keeping the original
reference \emph{unique} can be unsafe.
Rule \textsc{T-IDREAD2} uses the auxiliary function \emph{fix} to
produce a \emph{fixed} alias.
It also uses \emph{fix} to change the type of the original reference
to \emph{fixed}, because a \emph{fixed} table type does not allow
width subtyping.
We define these rules as follows:
\[
\begin{array}{c}
\begin{array}{c}
\mylabel{T-IDREAD1}\\
\dfrac{\env_{1}(id) = T_{1} \;\;\; T_{2} = read(\penv, T_{1})}
      {\env_{1}, \penv \vdash id:close(T_{2}), \env_{1}[id \mapsto open(T_{1})]}
\end{array}
\\ \\
\begin{array}{c}
\mylabel{T-IDREAD2}\\
\dfrac{\env_{1}(id) = T_{1} \;\;\; T_{2} = read(\penv, T_{1})}
      {\env_{1}, \penv \vdash id:fix(T_{2}), \env_{1}[id \mapsto fix(T_{1})]}
\end{array}
\end{array}
\]

Both rules use the auxiliary function \emph{read} because they may be
accessing an identifier that is bound to a filter or projection type.
As we presented in our previous design of Typed Lua \cite{maidl2014tl},
our type system includes a small set of \texttt{type} predicates that allow
programmers to discriminate union types, and our type system uses filter and
projection types in the definition of these predicates to handle the discrimination
of unions types.
While filter types discriminate unions of first-level types, projection
types discriminate unions of second-level types and project unions of
second-level types into unions of first-level types.
We can define \emph{read} as follows:
\begin{align*}
read(\penv, \phi(T_{1},T_{2})) & = T_{2}\\
read(\penv, \pi_{i}^{x}) & = proj(\penv(x), i)\\
read(\penv, T) & = T
\end{align*}

The function \emph{read} uses the auxiliary function \emph{proj}
to project a union of first-level types, based on an union of
second-level types and an index from a projection type.
In Section \ref{sec:projections} we will discuss how our type system uses
projection types to handle overloaded return types.
We can define \emph{proj} as follows:
\begin{align*}
proj(T_{1} \times ... \times T_{n} \times T{*}, i) & =  T_{i} \;\;\; \text{if $i <= n$}\\
proj(S_{1} \sqcup S_{2}, i) & = proj(S_{1}, i) \cup proj(S_{2}, i)
\end{align*}

We also need to close \emph{unique} and \emph{open} tables that
appear in the left-hand side of assignments, as leaving them
\emph{unique} and \emph{open} would allow the creation of
unsafe references.

As an example,
\begin{center}
\begin{tabular}{lll}
\multicolumn{3}{l}{$\mathbf{local} \; a:\{\}_{unique} = \{\} \; \mathbf{in}$}\\
& \multicolumn{2}{l}{$\mathbf{local} \; b:\{\}_{open} = \{\} \; \mathbf{in}$}\\
& & \multicolumn{1}{l}{$b = a;$}\\
& & \multicolumn{1}{l}{$a[``x"] \; {<}\String{>} = ``foo";$}\\
& & \multicolumn{1}{l}{$b[``x"] \; {<}\Integer{>} = 1$}\\
\end{tabular}
\end{center}
does not type check because we cannot add the field $``x"$ to $b$,
as its type is \emph{closed}.
Aliasing $a$ changes the type of $a$ from $\emph{unique}$ to
$\emph{open}$, and that is the reason why we can add the field
$``x"$ to the type of $a$.
Aliasing $a$ also produces the type $\{\}_{closed}$, which is
the same type that $b$ has in left-hand side of the assignment.
After the assignment, the type of $b$ is \emph{closed} and thus
does not allow changing the value that is stored in the field
$``x"$ of $a$.

Rule \textsc{T-IDWRITE} defines this behavior:
\[
\begin{array}{c}
\mylabel{T-IDWRITE}\\
\dfrac{\env_{1}(id) = T_{1} \;\;\; T_{2} = write(T_{1})}
      {\env_{1}, \penv \vdash id_{l}:close(T_{2}), \env_{1}[id \mapsto close(T_{2})]}
\end{array}
\]

This rule uses the auxiliary function \emph{write} because it may be
accessing an identifier that is bound to a filter type.
In Typed Lua, assignments restore discriminated union types to their original types,
and function \emph{write} works in this purpose.
We can define \emph{write} as follows:
\begin{align*}
write(\phi(T_{1},T_{2})) & = T_{1}\\
write(T) & = T
\end{align*}

Our type system also has different rules for type checking table indexing to avoid
changing table types in these operations, as they cannot create aliases:
\[
\begin{array}{c}
\begin{array}{c}
\mylabel{T-INDEX1}\\
\dfrac{\begin{array}{c}
       \env_{1}(id) = T \;\;\;
       read(\penv, T) = \{K_{1}{:}V_{1}, ..., K_{n}{:}V_{n}\}\\
       \env_{1}, \penv \vdash e_{2}:K, \env_{2} \;\;\;
       \exists i \in 1{..}n \; K \lesssim K_{i}
       \end{array}}
      {\env_{1}, \penv \vdash id[e_{2}]:V_{i}, \env_{2}}
\end{array}
\\ \\
\begin{array}{c}
\mylabel{T-INDEX2}\\
\dfrac{\begin{array}{c}
       \env_{1}, \penv \vdash e_{1}:\{K_{1}{:}V_{1}, ..., K_{n}{:}V_{n}\}, \env_{2}\\
       \env_{2}, \penv \vdash e_{2}:K, \env_{3} \;\;\;
       \exists i \in 1{..}n \; K \lesssim K_{i}
       \end{array}}
      {\env_{1}, \penv \vdash e_{1}[e_{2}]:V_{i}, \env_{3}}
\end{array}
\end{array}
\]

A second form of refinement happens when we want to use an
\emph{unique} or \emph{open} table type in a context that expects a
\emph{fixed} or \emph{closed} table type with a different shape.
This kind of refinement allows programmers to add optional fields
or merge existing fields.
To do that, Typed Lua includes a type coercion expression ${<}T{>} \; id$.
For instance, we can use this type coercion expression to make the following
example type check:
\begin{center}
\begin{tabular}{llll}
\multicolumn{4}{l}{$\mathbf{local} \; a:\{\}_{unique} = \{ \} \; \mathbf{in}$}\\
& \multicolumn{3}{l}{$a[``x"] \; {<}\String{>} = ``foo";$}\\
& \multicolumn{3}{l}{$a[``y"] \; {<}\String{>} = ``bar";$}\\
& \multicolumn{3}{l}{$\mathbf{local} \; b:\{``x":\String, ``y":\String \cup \Nil \}_{closed} =$}\\
& & \multicolumn{2}{l}{${<}\{``x":\String, ``y":\String \cup \Nil\}_{open}{>} \; a \; \mathbf{in}$}\\
& & & \multicolumn{1}{l}{$a[``z"] \; {<}\Integer{>} = 1$}
\end{tabular}
\end{center}

We can use $a$ to initialize $b$ because the coercion converts
the type of $a$ from $\{``x":\String, ``y":\String\}_{unique}$ to
$\{``x":\String, ``y":\String \cup \Nil\}_{open}$, and results in
$\{``x":\String, ``y":\String \cup \Nil\}_{closed}$,
which is a subtype of
$\{``x":\String, ``y":\String \cup \Nil\}_{closed}$, the type of $b$.
We can continue to refine the type of $a$ after aliasing it to $b$,
as it still holds an \emph{open} table.
At the end of this example, $a$ has type
$\{``x":\String, ``y":\String \cup \Nil, ``z":\Integer\}_{open}$.

Rule \textsc{T-COERCE} defines the behavior of the type coercion expression:
\[
\begin{array}{c}
\mylabel{T-COERCE}\\
\dfrac{\env_{1}(id) \subtype T \;\;\;
       \env_{1}[id \mapsto T], \penv \vdash id:T_{1}, \env_{2}}
      {\env_{1}, \penv \vdash {<}T{>} \; id:T_{1}, \env_{2}}
\end{array}
\]

Note that rule \textsc{T-COERCE} only allows changing the type
of a variable if the new type is a supertype of the previous type,
and the resulting type is always \emph{fixed} or \emph{closed}
to prevent the creation of unsafe aliases.

We also need to make sure to close all \emph{unique} and \emph{open}
table types before we type check a nested scope.
To do that, our type system uses some auxiliary functions to change
the type of variables before type checking a nested scope and
also to change the type of assigned and referenced variables after
type checking a nested scope.
The function \emph{crall} closes all \emph{unique} and \emph{open}
table types; it also restores filter types to their original types.
The function \emph{crset} closes a given set of free assigned variables,
which is given by the function \emph{fav}, and
it also uses this set to restore filter types to their original types.
The function \emph{openset} changes from \emph{unique} to \emph{open}
a given set of referenced variables, which is given by the function \emph{rv}.

As an example,
\begin{center}
\begin{tabular}{llll}
\multicolumn{4}{l}{$\mathbf{local} \; a:\{\}_{unique}, b:\{\}_{unique} = \{\}, \{\} \; \mathbf{in}$}\\
& \multicolumn{3}{l}{$\mathbf{local} \; f:\Integer \times \Void \rightarrow \Integer \times \Void =$}\\
& & \multicolumn{2}{l}{$\mathbf{fun} \; (x:\Integer):\Integer \times \Void$}\\
& & & \multicolumn{1}{l}{$b = a \;;\; \mathbf{return} \; x + 1$}\\
& \multicolumn{3}{l}{$\mathbf{in} \; a[``x"] \; {<}\Integer{>} = 1 \;;$}\\
& & \multicolumn{2}{l}{$b[``x"] \; {<}\String{>} = ``foo" \;;\; f(a[``x"])$}
\end{tabular}
\end{center}
does not type check because we cannot add the field
$``x"$ to $b$, as its type is closed.
The assignment $b = a$ type checks because, at that point,
$a$ and $b$ have the same type: $\{\}_{closed}$.
Their type was closed by \emph{crall} before type checking
the function body.
Their type would be restored to $\{\}_{unique}$ after type checking
the function body, but that assignment also triggers other two type changes.
First, the function \emph{fav} includes $b$ in the set of variables
that should be closed by \emph{crset}.
Then, the function \emph{rv} includes $a$ in the set of variables
that should change from \emph{unique} to \emph{open} by \emph{openset}.
After declaring $f$, $a$ has type $\{\}_{open}$ and $b$ has type $\{\}_{closed}$,
so we can refine the type of $a$, but we cannot refine the type of $b$.

Rule \textsc{T-FUNCTION1} illustrates this case:
\[
\begin{array}{c}
\mylabel{T-FUNCTION1}\\
\dfrac{\begin{array}{c}
       crall(\env_{1}[\overline{id} \mapsto \overline{T}]), \penv[\ret \mapsto S] \vdash s, \env_{2}\\
       \env_{3} = crset(\env_{1}, fav(\mathbf{fun} \; (\overline{id{:}T}){:}S \; s))\\
       \env_{4} = openset(\env_{3}, rv(\mathbf{fun} \; (\overline{id{:}T}){:}S \; s))
       \end{array}}
      {\env_{1}, \penv \vdash \mathbf{fun} \; (\overline{id{:}T}){:}S \; s:\overline{T} \times \Void \rightarrow S, \env_{4}}
\end{array}
\]

This rule also extends the environment $\penv$, bounding the special
variable $\ret$ to the return type $S$.
Rule \textsc{T-RETURN} uses the type that is bound to $\ret$ in
$\penv$ to type check return statements:
\[
\begin{array}{c}
\mylabel{T-RETURN}\\
\dfrac{\env_{1} \vdash el:S_{1}, \env_{2} \;\;\;
       \penv(\ret) = S_{2} \;\;\;
       S_{1} \lesssim S_{2}}
      {\env_{1} \vdash \mathbf{return} \; el, \env_{2}}
\end{array}
\]

\subsection{Projection Types}
\label{sec:projections}

As we have seen in Section~\ref{sec:statistics}, Lua
programmers often overload the return type of functions
to denote errors, returning a {\tt nil} result and
an error message in case of an error instead of the
usual return values.

As an example, consider the integer division function
below, which returns the quotient and rest of integer
division, or a division by zero error:
\begin{verbatim}
  local function idiv(x : integer, y : integer)
    if divisor == 0 then
      return nil, "division by zero"
    else
      return x // y, x % y
    end 
  end
\end{verbatim}

Typed Lua infers the return type of this function as
$(\Integer \times \Integer \times \Nil*) \sqcup (\Nil \times \String \times \Nil{*})$. The idiom for using this such
a function is to test the first returned value:
\begin{verbatim}
  local q, r = idiv(a, b)
  if q then
    -- assume q and r are integers
  else 
    -- assume q is nil r is a string
  end
\end{verbatim}

A standard way of decomposing the union in the assignment
would let the type of {\tt q} be $\Integer \times \Nil$ and
the type of {\tt r} be $\Integer \times \String$. Flow typing
would narrow the type of {\tt q} to $\Integer$ in the {\tt then}
branch and to $\Nil$ in the {\tt else} branch, but would
leave the type of {\tt r} unchanged.

{\em Projection types}
are a general way to make the dependency between the types
of {\tt q} and {\tt r} survive the decomposition of the tuple,
so flow typing can narrow all of the components by testing
just one of them.

Intuitively, a projection type $\pi_{i}^{x}$ denotes the union of
the i-th components of all the tuples in the union of tuple types
denoted by the label $x$. Unions of tuple types referenced by projections are kept in their own {\em projection environment} $\Pi$. A fresh label is generated whenever the last term
in an expression list has a union of tuple types as its type.

Before introducing the typing rules for projections, let us translate the example above into the abstract syntax of Figure~\ref{fig:syntax}:
\begin{center}
	\begin{tabular}{ll}
		\multicolumn{2}{l}{$\mathbf{local} \, q, r = idiv(1, 2) \, \mathbf{in} \, \mathbf{if} \, q \, \mathbf{then} \, s_1 \, \mathbf{else} \, s_2$}
	\end{tabular}
\end{center}

Notice that the local variables $q$ and $r$ are declared without 
an explicit type. The rule for typing this {\bf local} statement,
{\sc T-LOCAL-UN}, introduces a new label in the projection
environment used for typing the {\bf if} statement:
\[
\begin{array}{c}
\mylabel{T-LOCAL-UN}\\
\dfrac{\begin{array}{c}
	\env_{1}, \penv \vdash el:T_1 \times \ldots T_n \times P, \env_{2}, (x,S)\\
	\env_{3} = \env_{2}[id_{1} \mapsto T_1, \ldots, id_{n} \mapsto T_n]\\
	\env_{3}, \penv[x \mapsto S] \vdash s, \env_{4} \;\;\;
	n = |\;\overline{id}\;|  
	\end{array}}
{\env_{1}, \penv \vdash \mathbf{local} \; \overline{id} = el \; \mathbf{in} \; s, \env_{4} - \{\overline{id}\}}
\end{array}
\]

This first sequent above the bar uses the auxiliary typing relation $\env_{1}, \penv \vdash el : S_{1}, \env_{2}, (x,S_{2})$.
This rule does the work of generating a fresh projection label,
and projection types associated with it, if the last term of
the expression list has a union of tuple types as its type,
as rule {\sc T-EXPLIST-PROJ} below shows:
\[
\begin{array}{c}
\mylabel{T-EXPLIST-PROJ}\\
\dfrac{\begin{array}{c}
	\env_{1}, \penv \vdash e_{i}:T_{i}, \env_{i+1} \;\;\;
	\env_{1}, \penv \vdash me:S, \env_{n+2} \;\;\; n=|\overline{e}|\\
	S = S_1 \sqcup S_2 \;\;\;m = max(|S_1|, |S_2|)-1 \;\;\; x \notin \Pi\\
	\env_{f} = merge(\env_{1}, ..., \env_{n+2})
	\end{array}}
{\env_{1}, \penv \vdash \overline{e},me:}\\ {T_{1} \times ... \times T_{n} \times \pi_{1}^{x} \times ... \times \pi_{m}^{x} \times last(S), \env_{f}, (x,S)}
\end{array}
\]

In our example, the two rules combined will bind $x$ to
$(\Integer \times \Integer \times \Nil{*}) \sqcup (\Nil \times \String \times \Nil{*})$ in the projection environment $\Pi$,
and $q$ and $r$ respectively to $\pi_{1}^{x}$ and $\pi_{2}^{x}$
in the type environment used for typing the {\bf if} statement.

At this point, using $q$ or $r$ as an r-value will project
their types into either $\Integer \cup \Nil$ or $\Integer \cup \String$. But the more interesting case is inside the
clauses of the {\bf if} statement. The particular uses the rule {\sc T-IF-FNIL-PROJ} below, which uses flow typing:
\[
\begin{array}{c}
\mylabel{T-IF-FNIL-PROJ}\\
\dfrac{\begin{array}{c}
       \env_{1}(id) = \pi_{i}^{x} \;\;\; \penv(x) = S\\
       closeall(\env_{1}), \penv[x \mapsto fpt(S,\Nil,i)] \vdash s_{1}, \env_{2} \\
       closeall(\env_{1}), \penv[x \mapsto gpt(S,\Nil,i)] \vdash s_{2}, \env_{3} \\
       \env_{4} = closeset(\env_{1}, fav(s_{1}) \cup fav(s_{2})) \\
       \env_{5} = openset(\env_{4}, rv(s_{1}) \cup rv(s_{2}))
      \end{array}}
      {\env_{1}, \penv \vdash \mathbf{if} \; id \; \mathbf{then} \; s_{1} \; \mathbf{else} \; s_{2}, \env_{5}}
\end{array}
\]

Rule \textsc{T-IF-FNIL-PROJ} uses the auxiliary functions \emph{fpt} and \emph{gpt}
to filter a projection $x$. Function {\em fpt} filters out $\Nil$ from the i-th
component of each tuple, removing the entire tuple if the i-th
component does not exist, or is $\Nil$ itself. Function {\em gpt}
filters the i-th component to $\Nil$ if the type includes
$\Nil$, removing the entire tuple if the i-th component does
not exist, or does not include $\Nil$.

In our example, $x$ would be bound to $\Integer \times \Integer \times \Nil{*}$ in the {\bf then} branch, and would be bound
to $\Nil \times \String \times \Nil{*}$ inside the {\bf else}
branch.

Thus, reading $q$ and $r$ projects $\pi_{1}^{x}$ to $\Integer$ and
$\pi_{2}^{x}$ to $\Integer$ inside the $\mathbf{if}$ branch,
but it projects $\pi_{1}^{x}$ to $\Nil$ and $\pi_{2}^{x}$ to $\String$
inside the $\mathbf{else}$ branch.
Outside the condition, $q$ and $r$ use the original projection, that is,
they project to $\Integer \cup \Nil$ and $\Integer \cup \String$, respectively.

\section{Related Work}
\label{sec:related}

Tidal Lock~\cite{tidallock} is a prototype of another optional type system for Lua. It covers just a small subset of Lua
in its current form. Its most remarkable feature is
how it structures its table types to support a form of type
evolution through imperative assignment. Typed Lua uses
the same general idea of letting the type of a table evolve
through assignment, but the structure of both the table types
and the typing rules that support this are completely different.

Sol~\cite{sol} is another experimental optional type system
for Lua. While it has some similarities to Typed Lua, it
has more limited table types: Sol tables can only be lists,
maps, and objects that follow a specific object-oriented
idiom that Sol introduces. There is no evolution of table
types.

Lua Analyzer~\cite{luaanalyzer} is another optional type
system for Lua that is specially designed to work with
Löve Studio, an IDE for game development using Lua.
It is unsound by design, and primarily for supporting
IDE autocompletion. There is no evolution of table types.

Ravi~\cite{ravi} is an experimental Lua dialect
with an LLVM-based JIT compiler. It has a basic
optional type system to improve performance
of numerical code. This type system only supports
numbers (integers and floating point) and arrays.

Typed Racket~\cite{tobin-hochstadt2008ts} is a statically typed version of the Racket language, which is a Scheme dialect.
The main purpose of Typed Racket is to allow programmers to combine untyped modules, which are written in Racket, with typed modules, which are written in Typed Racket. Typed Racket tracks
values that cross the boundary between the typed and
untyped parts to be able to correctly assign blame to type
errors that occur in the typed parts, so is a gradual type
system.

Typed Racket introduced {\em occurrence typing}, a form of
flow typing~\cite{tobin-hochstadt2010ltu}, where type
predicates are used to refining union types.
As this use of type predicates is common in other
dynamically-typed languages, related systems
have appeared~\cite{guha2011tlc,winther2011gtp,pearce2013ccf}.

Gradualtalk~\cite{allende2013gts} is a Smalltalk dialect 
with a gradual type system. Its type system combines
nominal and structural typing.
It includes function types, union types, structural object types,
nominal object types types, a self type, and parametric polymorphism. Like Typed Lua, Gradualtalk formalizes the
interaction between typed and untyped code with consistent-subtyping. Due to the performance impact of
the runtime checks that ensure the gradual typing guarantees,
Gradualtalk can be downgraded into an optional type system
through a configuration switch~\cite{allende2013cis}.

Reticulated Python~\cite{vitousek2014deg} is a
gradual type system for Python. It is structural, based on subtyping, and includes list types,
dictionary types, tuple types, function types, set types,
object types, class types, and recursive types.
Besides static type checking, Reticulated Python also introduces
three different approaches for inserting runtime assertions
that preserve the gradual typing guarantee.

Several dynamically typed languages now have optional
type systems: Clojure~\cite{bonnaire-sergeant2012typed-clojure},
JavaScript~\cite{typescript}, Python~\cite{mypy}, 
and PHP~\cite{hack}. While Lua has some similarities to
all of these languages, none of these optional type
systems have the features described in this paper.
Most of these languages do not have idioms
that inspired these features; JavaScript has the
idiom of adding fields to an initially empty
record through assignment, but TypeScript sidesteps
the issue by having allowing an empty record to
have any record or object type, as the type of
missing fields is the bottom type of its type system.

Dart~\cite{dart} and Grace~\cite{black2013sg} are
two languages that have been designed from scratch
to have an optional type system, instead of being
existing dynamically typed languages with a retrofitted 
type system. 

\section{Conclusion}
\label{sec:conclusion}

We have presented a formalization of the Typed Lua,
an optional type system for Lua, with a focus on rules
for two novel features of the system that type unusual
idioms present in Lua programs.

In the first idiom, records and objects are
defined through assignment to an initially empty
table. In Typed Lua's type system the type of a
table can change, either by assignment
or by using the table in a context that expects
a different (but compatible) type than the one it
has. The system tracks aliasing of table references
to be able to do this type evolution in a type-safe
way.

In the second idiom, a function that can return different
kinds of sequences of return values (for example, one
sequence for its usual path, another for error conditions)
is modeled by having a union of tuple types as its return
type. A destructuring assignment on a union of tuple types
decomposes the tuple in a way that the dependencies between
the types of each member of the union can be tracked.
Narrowing the type of one of the tuple's components with
a type predicate can then narrow the types of the other
components.

A key feature in optional type systems is usability.
Optional type systems retrofitted to an existing
language should fit the language's idioms, adding
static type safety to them. If the system is too
simple it may require too much of a change in the way
the programmers use the language. On the other hand,
if it is too complex it may overload the programmers
with types and error messages that are hard to
understand. The most challenging aspect of designing optional type systems is to find the right amount of complexity for a type system that feels natural to the programmers.

Usability has been a concern in the design of Typed Lua since the beginning. We realized that a design based solely on what
is possible by the semantics of Lua could lead to a 
complex type system. For this reason, we surveyed a large
corpus of Lua code to identify important idioms that
our type system should support.

We performed several
case studies to evaluate how successful we were in our goal of
providing an usable type system.
We evaluated 29 modules from 8 different case studies,
and we could give precise static types to 83\% of the 449
members that these modules export.
For half of the modules, we could give precise static types to
at least 89\% of the members from each module.

Our evaluation results show that our type system can statically
type check several Lua idioms and features, though the evaluation
also exposed several limitations of our type system.
We found that the three main limitations of our type system are
the lack of full support for overloaded functions, the lack
of parametric polymorphism, and operator overloading.
Overcoming these limitations is our major target for future work,
as it will allow us to statically type check more programs.

Unlike other optional type systems, we designed Typed Lua without
deliberate unsound parts.
However, we still do not have proofs that the novel features of
our type system are sound. We are working on a generalization
of the typing rules for evolution of table types where tagged
types can be attached to arbitrary references instead of
just local variables, and expect that a proof of soundness
will come out of this effort.

Typed Lua has a working implementation that
Lua programmers can already use as a framework to
document, test, and structure their applications,
and we already have user feedback from Lua programmers
that are using Typed Lua in their projects.
One example is ZeroBrane Studio, a Lua IDE that is
integrating Typed Lua.

Even applications where a full conversion to static
type checking is unfeasible in the current state
of the type system, or too much work, can use Typed
Lua to document the external interfaces of its
libraries, giving the benefits of static typing
checking to the users of those libraries.

\bibliographystyle{abbrvnat}
\bibliography{paper}

\end{document}
