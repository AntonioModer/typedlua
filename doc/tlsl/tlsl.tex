\documentclass[12pt]{article}

\usepackage[english]{babel}
\usepackage[utf8]{inputenc}

\title{Typing Lua Standard Libraries}

\author{André Murbach Maidl}

\begin{document}

\maketitle

\textbf{Notation}

\begin{itemize}
\item I am using ? to mark the parameters that I am still not
sure about their types.
\item I am using Any to denote the \textit{dynamic} type
(also known as the unknown type).
\item I am using the notation [A] to denote the type Table.
\item I am using the operators * and + to denote the repetition
of a certain type. The * means zero or more repetitions while +
means one or more repetitions.
\item Usually, Lua functions accept extra parameters which are
simply discarded.
I am using a star ($\star$) to mark the functions that do not
follow this behavior.
That is, the functions that do not accept extra parameters.
\end{itemize}

\newpage

\textbf{TODO}

\begin{enumerate}
\item Still need to type: \textbf{debug}.
\item Does Any include Nil or Void?
\item What is the return type of \textbf{require}?
\item Confirm the types of \textbf{table} functions.
\item Should I use Void when the function can be called without any
parameters? Does (Type $\cup$ Nil) have the equivalent meaning?
\item How should I deal with the automatic coercion between
strings and numbers?
That is, for all functions of the Lua standard libraries,
when an input parameter is of type String, actually, this
parameter is (String $\cup$ Number).
The same conversion happens when an input parameter is of type
Number.
However, Lua throws a type error when the conversion is not
possible.
\item How to denote the type of VarArg (...)?
\item How to denote the dynamic types Function, Thread, and UserData?
\item Should I include in this documentation functions like
file:f() from the \textbf{io} library?
For instance, \textbf{io.close(nil)} is not ok, but
\textbf{file:close(nil)} is ok.
\item It looks like the \textbf{debug} functions are the only ones
that perform the type checking from the last argument to the first
one.
Is there any reason for that?
\end{enumerate}

\newpage

\section{Basic}

\subsection{assert (v [, message])}

((Boolean $\cup$ Nil) $\times$ (String $\cup$ Nil)) $\cup$
(Any $\times$ Any*) $\rightarrow$
Void $\cup$ Any+

When \textbf{v} is \textbf{nil} or \textbf{false}, \textbf{assert}
shows de error \textbf{message}.
Otherwise, \textbf{assert} returns \textbf{v} and all the extra
parameters.
The typing of this function should be refined.

\subsection{collectgarbage ([opt [, arg]])}

(String $\cup$ Nil) $\times$
(Number $\cup$ Nil) $\rightarrow$
Number $\cup$ (Number $\times$ Number) $\cup$ Boolean

The return type depends on the type of \textbf{opt}:
\begin{itemize}
\item Nil $\rightarrow$ Number
\item \textbf{collect} $\rightarrow$ Number
\item \textbf{stop} $\rightarrow$ Number
\item \textbf{restart} $\rightarrow$ Number
\item \textbf{count} $\rightarrow$ (Number $\times$ Number)
\item \textbf{step} $\rightarrow$ Boolean
\item \textbf{setpause} $\rightarrow$ Number
\item \textbf{setstepmul} $\rightarrow$ Number
\item \textbf{isrunning} $\rightarrow$ Boolean
\item \textbf{generational} $\rightarrow$ Number
\item \textbf{incremental} $\rightarrow$ Number
\end{itemize}

\subsection{dofile ([filename])}

(String $\cup$ Nil $\cup$ Void) $\rightarrow$ Any*

\subsection{error (message [, level])}

Void $\cup$
((Number $\cup$ Boolean $\cup$ Nil) $\times$
(Number $\cup$ Boolean $\cup$ Nil)) $\rightarrow$
Void

\subsection{getmetatable (object)}

Any $\rightarrow$ Nil $\cup$ [A]

\subsection{ipairs (t)}

[A] $\rightarrow$ (Function $\times$ [A] $\times$ Number)

What happens when \textbf{t} has a metamethod \textbf{\_\_ipairs}?

\subsection{load (ld [, source [, mode [, env]]])}

(Function $\cup$ String) $\times$
(String $\cup$ Nil) $\times$
(String $\cup$ Nil) $\times$
Any $\rightarrow$
Function $\cup$ (Nil $\times$ String)

\subsection{loadfile ([filename [, mode [, env]]])}

Void $\cup$
(String $\cup$ Nil) $\cup$
(String $\cup$ Nil) $\cup$
Any $\rightarrow$
Function $\cup$ (Nil $\times$ String)

\subsection{next (table [, index])}

[A] $\times$ Any $\rightarrow$
Nil $\cup$ (Any $\cup$ Any)

\subsection{pairs (t)}

[A] $\rightarrow$ (Function $\times$ [A] $\times$ Nil)

\subsection{pcall (f [, arg1, ···])}

Any+ $\rightarrow$
(Boolean $\times$ Any*) $\cup$
(Boolean $\times$ String)

\subsection{print (···)}

Any $\rightarrow$ Void

\subsection{rawequal (v1, v2)}

Any $\times$ Any $\rightarrow$ Boolean

\subsection{rawget (table, index)}

[A] $\times$ Any $\rightarrow$ Any

\subsection{rawlen (v)}

([A] $\cup$ String) $\rightarrow$ Number

\subsection{rawset (table, index, value)}

[A] $\times$ Any $\times$ Any $\rightarrow$ [A]

Even though \textbf{rawset} does not generate the regular type error
message when \textbf{index} is Nil, it generates an error message in
this case.

\subsection{select (index, ···)}

Number $\times$ Any* $\rightarrow$ Void $\cup$ Any*

The result is Void when \textbf{index} does not exist in the list.
When \textbf{index} is the string `\texttt{\#}', the typing is:

String $\times$ Any* $\rightarrow$ Number

\subsection{setmetatable (table, metatable)}

[A] $\times$ (Nil $\cup$ [A])

Even though \textbf{metatable} can be Nil, it is not optional.

\subsection{tonumber (e [, base])}

String $\times$ (Number $\times$ Nil) $\rightarrow$ Number

\subsection{tostring (v)}

Any $\rightarrow$ String

\subsection{type (v)}

Any $\rightarrow$ String

\subsection{xpcall (f, msgh [, arg1, ···])}

Any $\times$ Any+ $\rightarrow$
(Boolean $\times$ Any*) $\cup$
(Boolean $\times$ String)

\newpage

\section{Coroutine}

\subsection{coroutine.create (f)}

Function $\rightarrow$ Thread

\subsection{coroutine.resume (co [, val1, ···])}

Thread $\times$ Any* $\rightarrow$ (Boolean $\times$ Any*)

\subsection{coroutine.running ()}

Void $\rightarrow$ (Thread $\times$ Boolean)

\subsection{coroutine.status (co)}

Thread $\rightarrow$ String

\subsection{coroutine.wrap (f)}

Functin $\rightarrow$ Function 

\subsection{coroutine.yield (···)}

Any* $\rightarrow$ Void

\newpage

\section{Modules}

\subsection{require (modname)}

String $\rightarrow$ ?

The function \textbf{require} returns exactly the same value that is
returned by loading \textbf{modname}.
However, if the returned value is \textbf{nil}, this value is
converted to \textbf{true}.
The parameter \textbf{modname} can be a number, provided that exists
a Lua file with this name.

\newpage

\section{String}

\subsection{string.byte (s [, i [, j]])}

String $\times$
(Number $\cup$ Nil) $\times$
(Number $\cup$ Nil) $\rightarrow$
Number$^+$

\subsection{string.char (···)}

Number$^*$ $\rightarrow$ String

\subsection{string.dump (function)}

Function $\rightarrow$ String

\subsection{string.find (s, pattern [, init [, plain]])}

String $\times$ String $\times$
(Number $\cup$ Nil) $\times$
(Boolean $\cup$ Nil) $\rightarrow$
(Number $\times$ Number) $\cup$ Nil

\subsection{string.format (formatstring, ···)}

String $\times$ Any$^*$ $\rightarrow$ String

\subsection{string.gmatch (s, pattern)}

String $\times$ String $\rightarrow$ Function

\subsection{string.gsub (s, pattern, repl [, n])}

String $\times$
String $\times$
(String $\cup$ [A] $\cup$ Function) $\times$
(Number $\cup$ Nil) $\rightarrow$
(String $\times$ Number)

\subsection{string.len (s)}

String $\rightarrow$ Number

\subsection{string.lower (s)}

String $\rightarrow$ String

\subsection{string.match (s, pattern [, init])}

String $\times$
String $\times$
(Number $\cup$ Nil) $\rightarrow$
String$^+$ $\cup$ Nil

\subsection{string.rep (s, n [, sep])}

String $\times$
Number $\times$
(String $\cup$ Nil) $\rightarrow$
String

\subsection{string.reverse (s)}

String $\rightarrow$ String

\subsection{string.sub (s, i [, j])}

String $\times$
Number $\times$
(Number $\cup$ Nil) $\rightarrow$
String

\subsection{string.upper (s)}

String $\rightarrow$ String

\newpage

\section{Table}

\subsection{table.concat (list [, sep [, i [, j]]])}

[A] $\times$
(String $\cup$ Nil) $\times$
(Number $\cup$ Nil) $\times$
(Number $\cup$ Nil) $\rightarrow$
String

The table should be a list containing only strings and numbers.
So, should I use [String $\cup$ Number] instead of [A]?

\subsection{table.insert (list, [pos,] value)$^\star$}

[A] $\times$ ((Number $\times$ A) $\cup$ A) $\rightarrow$ Void

This function does not accept/discard extra parameters.
Perhaps due to the fact that the parameter \textbf{pos} is optional.
If \textbf{value} is Void, the function inserts \textbf{pos} in the
\textbf{list}, that is, the type is [A] $\times$ A.
If \textbf{value} is Nil, the function does not insert \textbf{pos}
in the \textbf{list}, that is, the type is
[A] $\times$ Number $\times$ A, but the element is not inserted in
the list because A is actually Nil.

\subsection{table.pack (···)}

Any$^*$ $\rightarrow$ [A]

Even though \textbf{table.pack} gets any number of parameters of
any type, the Nil type generates a hole in the returned table.

\subsection{table.remove (list [, pos])}

[A] $\times$ (Number $\cup$ Nil) $\rightarrow$ (A $\cup$ Nil)

The return type depends on the type of the element that is removed
from the list.

\subsection{table.sort (list [, comp])}

[A] $\times$ (Function $\cup$ Nil) $\rightarrow$ Void

\subsection{table.unpack (list [, i [, j]])}

[A] $\times$
(Number $\cup$ Nil) $\times$
(Number $\cup$ Nil) $\rightarrow$
(A $\cup$ Nil)$^*$

The return type depends on the type of the elements that are stored in
the list.

\newpage

\section{Mathematical}

\subsection{math.abs (x)}

Number $\rightarrow$ Number

\subsection{math.acos (x)}

Number $\rightarrow$ Number

\subsection{math.asin (x)}

Number $\rightarrow$ Number

\subsection{math.atan (x)}

Number $\rightarrow$ Number

\subsection{math.atan2 (y, x)}

Number $\times$ Number $\rightarrow$ Number

\subsection{math.ceil (x)}

Number $\rightarrow$ Number

\subsection{math.cos (x)}

Number $\rightarrow$ Number

\subsection{math.cosh (x)}

Number $\rightarrow$ Number

\subsection{math.deg (x)}

Number $\rightarrow$ Number

\subsection{math.exp (x)}

Number $\rightarrow$ Number

\subsection{math.floor (x)}

Number $\rightarrow$ Number

\subsection{math.fmod (x, y)}

Number $\times$ Number $\rightarrow$ Number

\subsection{math.frexp (x)}

Number $\rightarrow$ (Number $\times$ Number)

\subsection{math.ldexp (m, e)}

Number $\times$ Number $\rightarrow$ Number

\subsection{math.log (x [, base])}

Number $\times$
(Number $\cup$ Nil) $\rightarrow$
Number

\subsection{math.max (x, ···)}

Number$^+$ $\rightarrow$ Number

\subsection{math.min (x, ···)}

Number$^+$ $\rightarrow$ Number

\subsection{math.modf (x)}

Number $\rightarrow$ (Number $\times$ Number)

\subsection{math.pow (x, y)}

Number $\times$ Number $\rightarrow$ Number

\subsection{math.rad (x)}

Number $\rightarrow$ Number

\subsection{math.random ([m [, n]])$^\star$}

Number $\cup$ (Number $\times$ Number) $\cup$ Void $\rightarrow$
Number

This function does not accept Nil as its only parameter,
although the parameter passing is optional.
This function does not accept/discard extra parameters.

\subsection{math.randomseed (x)}

Number $\rightarrow$ Void

\subsection{math.sin (x)}

Number $\rightarrow$ Number

\subsection{math.sinh (x)}

Number $\rightarrow$ Number

\subsection{math.sqrt (x)}

Number $\rightarrow$ Number

\subsection{math.tan (x)}

Number $\rightarrow$ Number

\subsection{math.tanh (x)}

Number $\rightarrow$ Number

\newpage

\section{Bitwise}

\subsection{bit32.arshift (x, disp)}

Number $\times$ Number $\rightarrow$ Number

\subsection{bit32.band (···)}

Number$^*$ $\rightarrow$ Number

\subsection{bit32.bnot (x)}

Number $\rightarrow$ Number

\subsection{bit32.bor (···)}

Number$^*$ $\rightarrow$ Number

\subsection{bit32.btest (···)}

Number$^*$ $\rightarrow$ Boolean

\subsection{bit32.bxor (···)}

Number$^*$ $\rightarrow$ Number

\subsection{bit32.extract (n, field [, width])}

Number $\times$ Number $\times$
(Number $\cup$ Nil) $\rightarrow$
Number

\subsection{bit32.replace (n, v, field [, width])}

Number $\times$ Number $\times$ Number $\times$
(Number $\cup$ Nil) $\rightarrow$
Number

\subsection{bit32.lrotate (x, disp)}

Number $\times$ Number $\rightarrow$ Number

\subsection{bit32.lshift (x, disp)}

Number $\times$ Number $\rightarrow$ Number

\subsection{bit32.rrotate (x, disp)}

Number $\times$ Number $\rightarrow$ Number

\subsection{bit32.rshift (x, disp)}

Number $\times$ Number $\rightarrow$ Number

\newpage

\section{I/O}

\subsection{io.close ([file])}

(UserData $\cup$ Void) $\rightarrow$
(Boolean $\cup$ ((Nil $\cup$ Boolean) $\times$ String $\times$ (Nil $\cup$ Number)))

Returns (Nil $\cup$ Boolean) $\times$ String $\times$ Number when
the file handled is created with \textbf{io.popen}.

\subsection{io.flush ()}

Void $\rightarrow$ Boolean

\subsection{io.input ([file])}

(String $\cup$ UserData $\cup$ Nil $\cup$ Void) $\rightarrow$
UserData

\subsection{io.lines ([filename ···])}

(String $\cup$ Nil $\cup$ Void) $\rightarrow$ Function

\subsection{io.open (filename [, mode])}

String $\times$ (String $\cup$ Nil) $\rightarrow$
UserData $\cup$ (Nil $\times$ String $\times$ Number)

\subsection{io.output ([file])}

(String $\cup$ UserData $\cup$ Nil $\cup$ Void) $\rightarrow$
UserData

\subsection{io.popen (prog [, mode])}

String $\times$ (String $\cup$ Nil) $\rightarrow$
UserData $\cup$ (Nil $\times$ String $\times$ Number)

\subsection{io.read (···)}

(String $\cup$ Number)* $\rightarrow$
Nil $\cup$ (String $\cup$ Number $\cup$ Nil)* 

This function has a weird behavior: you can run 
\textbf{io.read(1,``2'',``3'')}, but you cannot run 
\textbf{io.read(``1'',``2'',``3'')}.
That is, the first parameter must be a string when it is given.

\subsection{io.tmpfile ()}

Void $\rightarrow$ UserData

\subsection{io.type (obj)}

Any $\rightarrow$
(Nil $\cup$ String)

\subsection{io.write (···)}

(String $\cup$ Number)* $\rightarrow$
(UserData $\cup$ (Nil $\times$ String $\times$ Number))

\newpage

\section{O.S.}

\subsection{os.clock ()}

Void $\rightarrow$ Number

\subsection{os.date ([format [, time]])}

(String $\cup$ Nil) $\times$
(Number $\cup$ Nil) $\rightarrow$
(String $\cup$ [A])

Returns a table when \textbf{format} equals ``\textbf{*t}''.

\subsection{os.difftime (t2, t1)}

Number $\times$ (Number $\cup$ Nil) $\rightarrow$
Number

Returns \textbf{t2} when \textbf{t1} is not given.

\subsection{os.execute ([command])}

(String $\cup$ Nil $\cup$ Void) $\rightarrow$
(Boolean $\cup$ ((Boolean $\cup$ Nil) $\times$ String $\times$ Number))

Returns just a Boolean when \textbf{command} is Nil or Void.
This Boolean means whether there is a shell available.
Returns Boolean $\times$ String $\times$ Number when \textbf{command}
terminated successfully.
Returns Nil $\times$ String $\times$ Number when \textbf{command}
did not terminate sucessfully.

\subsection{os.exit ([code [, close]])}

Void $\cup$
((Number $\cup$ Boolean $\cup$ Nil) $\times$
(Number $\cup$ Boolean $\cup$ Nil)) $\rightarrow$
Void

It looks like \textbf{close} accepts any type.

\subsection{os.getenv (varname)}

String $\rightarrow$ (String $\cup$ Nil)

\subsection{os.remove (filename)}

String $\rightarrow$
(Boolean $\cup$ (Nil $\times$ String $\times$ Number))

\subsection{os.rename (oldname, newname)}

String $\times$ String $\rightarrow$
(Boolean $\cup$ (Nil $\times$ String $\times$ Number))

\subsection{os.setlocale (locale [, category])}

Void $\cup$
((String $\cup$ Nil) $\times$
(String $\cup$ Nil)) $\rightarrow$
(String $\cup$ Nil)

\subsection{os.time ([table])}

Void $\cup$ [A] $\cup$ Nil $\rightarrow$ Number

\subsection{os.tmpname ()}

Void $\rightarrow$ String

\newpage

\section{Debug}

\subsection{debug.debug ()}

\subsection{debug.gethook ([thread])}

\subsection{debug.getinfo ([thread,] f [, what])}

\subsection{debug.getlocal ([thread,] f, local)}

\subsection{debug.getmetatable (value)}

\subsection{debug.getregistry ()}

\subsection{debug.getupvalue (f, up)}

\subsection{debug.getuservalue (u)}

\subsection{debug.sethook ([thread,] hook, mask [, count])}

\subsection{debug.setlocal ([thread,] level, local, value)}

\subsection{debug.setmetatable (value, table)}

\subsection{debug.setupvalue (f, up, value)}

\subsection{debug.setuservalue (udata, value)}

\subsection{debug.traceback ([thread,] [message [, level]])}

\subsection{debug.upvalueid (f, n)}

\subsection{debug.upvaluejoin (f1, n1, f2, n2)}

\end{document}
