
We performed some case studies on existing Lua libraries
to evaluate the applicability of our type system.
For each library, we used Typed Lua to either annotate its modules
or to write statically typed interfaces to its modules through
Typed Lua's description files.
In Section \ref{sec:cases} we present our evaluation results
and discuss some interesting cases.

We also evaluate our contributions through the revision of related work.
In Section \ref{sec:relatedwork} we discuss what are the novel
features of our type system, what are the features that it shares
with other type systems, and whether we can incorporate some
interesting features from other type systems.

\section{Case Studies}
\label{sec:cases}

We selected seven case studies to evaluate two aspects of Typed Lua:
\begin{enumerate}
\item how precisely it can describe the type of the interface of a module;
\item whether it provides guarantees that the code matches the interface.
\end{enumerate}

The Lua Standard Library \citep{luamanual} was our first case study.
We started to think about how we would type its modules at the same time
that we started to design our type system, as it could give us some
hints on our type system.
And it did: optional parameters and overloading on the return type
are two Lua features that our type system should handle to allow us
typing some of the functions that the standard library implements.

The second case study that we chose was the MD5 library \citep{lmd5},
because we wanted a simple case study to introduce Typed Lua's description
files and \texttt{userdata} declarations.
These Typed Lua's mechanisms allow programmers to give statically typed
interfaces to Lua libraries.

LuaSocket \citep{luasocket} was the third case study that we used to
evaluate Typed Lua.
We chose LuaSocket as a case study because it is the most popular Lua library.

There are three case studies that we randomly selected from the LuaRocks
repository, they are: HTTP Digest \citep{luahttpdigest},
Typical \citep{luatypical}, and Mod 11 \citep{luamod11}.
The first one provides client side HTTP digest authentication for Lua.
The second one is an extension to the primitive function \texttt{type}.
The third one is a generator and checker of modulo 11 numbers.
We randomly selected these libraries to evaluate Typed Lua for
annotating existing Lua modules.

The Typed Lua compiler is the last case study that we evaluated.
We chose the Typed Lua compiler as a case study because it is a large application,
and bootstrapping the compiler also provides some confidence in the type system.

\begin{table}[!ht]
\begin{center}
\begin{tabular}{|l|c|c|c|c|c|c|c|}
\hline
\textbf{Case study} & \textbf{easy} & \textbf{poly} & \textbf{inter} & \textbf{hard} & \textbf{Total} & \textbf{\%} \\
\hline
Lua Standard Library & 80 & 6 & 8 & 35 & 129 & 62\% \\
\hline
MD5 & 13 & 0 & 0 & 0 & 13 & 100\% \\
\hline
LuaSocket & 110 & 1 & 2 & 10 & 123 & 89\% \\
\hline
HTTP Digest & 0 & 0 & 1 & 0 & 1 & 0\% \\
\hline
Typical & 1 & 0 & 0 & 0 & 1 & 100\% \\
\hline
Modulo 11 & 7 & 0 & 0 & 2 & 9 & 78\% \\
\hline
Typed Lua Compiler & 144 & 0 & 1 & 9 & 154 & 94\% \\
\hline
\end{tabular}
\end{center}
\caption{Evaluation results for each case study}
\label{tab:evalbycase}
\end{table}

Table \ref{tab:evalbycase} summarizes our evaluation results for each
of the case studies that we used Typed Lua for typing their members.
We split the members of each case study into four categories:
\emph{easy}, \emph{poly}, \emph{inter}, and \emph{hard}.
The \emph{easy} category groups the members that we could give
a precise static type.
The \emph{poly} category groups the members that we made minimal
use of the dynamic type as a replacement for the lack of type parameters.
The \emph{inter} category groups the members that we could give
a static type that is not precise enough because they require
intersection types.
The \emph{hard} category groups the members that we had to rely
on the dynamic type because our type system cannot give a precise
static type to them.
The last column of the table shows the percentage of members that
are under the \emph{easy} category for each case study.
This percentage gives a lower bound on how much static type safety
we could add to each one of our case studies.

In the following sections we discuss each case study in more detail.
For each case study, we split the evaluation results according
to the modules that each one of them include.
This separation by case study allows us to better discuss the
contributions and limitations of our type system.

\subsection{Lua Standard Library}

All of the modules in the standard library are implemented in C, so
we used Typed Lua to type just the interface of each module.
The \texttt{debug} module is the only one that we did not include in our
evaluation results, because it provides several functions that violate
basic assumptions about Lua code \citep{luamanual}.
For instance, we can use the function \texttt{debug.setlocal} to change the value
of a local variable that is not visible in the current scope.
Table \ref{tab:evallsl} summarizes the evaluation results for the Lua Standard Library
(version 5.3).

\begin{table}[!ht]
\begin{center}
\begin{tabular}{|c|c|c|c|c|c|c|}
\hline
\textbf{Module} & \textbf{easy} & \textbf{poly} & \textbf{inter} & \textbf{hard} & \textbf{Total} & \textbf{\%} \\
\hline
base & 9 & 1 & 2 & 14 & 26 & 35\% \\
\hline
coroutine & 1 & 0 & 0 & 6 & 7 & 14\% \\
\hline
package & 5 & 0 & 0 & 3 & 8 & 62\% \\
\hline
string & 12 & 0 & 0 & 4 & 16 & 75\% \\
\hline
utf8 & 6 & 0 & 0 & 0 & 6 & 100\% \\
\hline
table & 1 & 5 & 0 & 1 & 7 & 14\% \\
\hline
math & 22 & 0 & 4 & 1 & 27 & 81\% \\
\hline
io & 15 & 0 & 0 & 6 & 21 & 71\% \\
\hline
os & 9 & 0 & 2 & 0 & 11 & 82\% \\
\hline
\end{tabular}
\end{center}
\caption{Evaluation results for Lua Standard Library}
\label{tab:evallsl}
\end{table}

We could give precise static types to several members of
the Lua Standard Library, though some of them are still
hard to type with our current type system.
Now we will discuss why some modules are difficult to type,
and what kind of limitations they represent in our type system.

The \texttt{base} module was quite hard to type because it
includes several functions that rely on reflection.
For instance, the function \texttt{pairs} traverses all
keys and values that are stored in a given table.
As another example, the function \texttt{getmetatable}
returns the metatable of a given table.

We could not give a precise static type to \texttt{ipairs},
because our type system does not have parametric polymorphism.
This function returns an iterator function that traverses a
list of elements.

The \texttt{base} module also includes two overloaded functions:
\texttt{tonumber} and \texttt{collectgarbage}.

In the case of \texttt{tonumber}, the type of the first parameter
depends on the type of the second parameter.
For instance, we can call \texttt{tonumber(1)}, but we cannot
call \texttt{tonumber(1,2)}.
More precisely, the first argument of \texttt{tonumber} can be
a value of any type if it is the only argument, but it must
be a string if there is a second argument,
which must be an integer.

In the case of \texttt{collectgarbage}, the return type changes
according to an input literal string.
For instance, calling \texttt{collectgarbage("collect")} returns an integer,
calling \texttt{collectgarbage("count")} returns a floating point,
and calling \texttt{collectgarbage("isrunning")} returns a boolean.

We could not type most of the members of the \texttt{coroutine} module
because our type system cannot describe the computational effects of a program.
Lua has one-shot delimited continuations \citep{james2011yield}
in the form of \emph{coroutines} \citep{moura2009rc}, and
effect systems \citep{nielson1999type} are an approach that we
could use to describe control transfers with continuations.
However, for now, coroutines are out of the scope of our type
system, and we use an empty \emph{userdata} declaration
to represent the type \emph{thread}.
Still, we could type the function \texttt{coroutine.isyieldable},
as it has no input parameters, and simply returns a boolean that
indicates whether the running coroutine is yieldable.

We could type most of the members of the \texttt{package} module,
but we could not type the following tables: \texttt{package.loaded},
\texttt{package.preload}, and \texttt{package.loaders}.
The first stores loaded modules, while the others store module \emph{loaders}.
They are difficult to type because their types rely on reflection,
that is, their types depend on the modules a program loads.

The \texttt{string} module was not that hard to type, but it includes
some functions that we could not type, as they rely on format
strings.
For instance, the type of the arguments that we pass to
\texttt{string.format} depend on the format string we are using.
It is fine to call \texttt{string.format("\%d", 1)}, but
\texttt{string.format("\%d", true)} raises a run-time error.

Lua programmers often use the functions of the \texttt{string} module
in an object-oriented style, and we treat this feature as a
special case in the implementation.
For this reason, we did not include the typing of the \texttt{string}
methods in our evaluation results.

The \texttt{utf8} module was straightforward to type,
as its members are only operations over strings.

The \texttt{table} module was specially difficult to type because
most of its functions require parametric polymorphism.
These functions either receive or return a list of elements, and
parametric polymorphism would help us to describe them with a generic type.

However, the lack of parametric polymorphism did not prevent us from
giving a precise type to \texttt{table.concat}, as it operates over lists
where all elements are strings or numbers.

Even if our type system had parametric polymorphism, it would
be difficult to type \texttt{table.insert}, as its type depends on
the calling arity.
More precisely, calling \texttt{table.insert(l, v)} inserts the
element \texttt{v} at the end of the list \texttt{l},
while \texttt{table.insert(l, p, v)} inserts the element \texttt{v}
at the position \texttt{p} of the list \texttt{l},
but generates a run-time error when \texttt{p} is out of bounds.
This function also does not follow the semantics of Lua on
discarding extra arguments, and generates a run-time error whenever
we pass more than three arguments, even if the first three arguments
match its signature.

Even though the \texttt{math} module looks straightforward to type,
it includes several overloaded functions.
For instance, \texttt{math.abs(1)} returns the integer \texttt{1},
while \texttt{math.abs(1.1)} returns the floating point \texttt{1.1}.

The \texttt{math} module also includes a function that is hard to type:
the function \texttt{math.random}.
It is difficult to type because its type depends on the calling arity.
Calling \texttt{math.random()} returns a random floating point
between \texttt{0} and \texttt{1}.
Calling \texttt{math.random(10)} is equivalent to \texttt{math.random(1,10)},
and returns an integer between this interval.
This function also does not follow the semantics of Lua on discarding extra arguments,
and generates a run-time error whenever we pass more than two arguments.

The \texttt{io} module provides operations for manipulating files,
and these operations can use implicit or explicit file descriptors.
The implicit operations are functions in the \texttt{io} table,
while the explicit operations are methods of a file descriptor.
We used an \emph{userdata} declaration to introduce the type
\texttt{file} for representing the type of a file descriptor
and its methods.
For this reason, we included the typing of the \texttt{file}
methods in our evaluation results.

We could type most of members of the \texttt{io} module,
but we could not precisely type the functions \texttt{io.close},
\texttt{io.read}, and \texttt{io.lines}.
We also could not type the methods \texttt{file:close},
\texttt{file:read}, and \texttt{file:lines}.

The function \texttt{io.close} is difficult to type
because its return type depends on whether the file handle that is
being closed was created with \texttt{popen}.
Calling \texttt{io.close(f)} returns the tuple \texttt{(boolean?, string, number)}
when \texttt{f} was created with \texttt{popen},
and it returns the tuple \texttt{(boolean)?} otherwise.
The method \texttt{file:close} is difficult to type for the same reason.

The functions \texttt{io.read} and \texttt{io.lines} are difficult
to type because they are overloaded functions.
For instance, \texttt{io.read("l")} returns a string or \texttt{nil},
\texttt{io.read("n")} returns a number or \texttt{nil}, and calling
\texttt{io.read("l", "n")} returns a string or \texttt{nil} and a number or \texttt{nil}.
The function \texttt{io.lines}, and the methods \texttt{file:read} and
\texttt{file:lines} are difficult to type for the same reason.

The function \texttt{os.execute} is the only member of the \texttt{os}
module that we could not give a precise type, as it is an overloaded
function.
More precisely, \texttt{os.execute()} returns a boolean, while
\texttt{os.execute(c)} returns \texttt{(boolean?, string, number)},
where \texttt{c} is a string that represents the command to be executed.

The evaluation results show that our type system should include
intersection types, parametric polymorphism, and effect types,
as these features would help us increase the static typing of the
Lua Standard Library.
Intersection types would allow us to define overloaded function types.
Parametric polymorphism would allow us to define generic function and table types.
Effect types would allow us to type coroutines.

\subsection{MD5}

The MD5 library is an OpenSSL based message digest library for Lua.
It contains just the \texttt{md5} module that is written in C,
so we used Typed Lua's description file to type it.
Table \ref{tab:evalmd5} summarizes the evaluation results for MD5.

\begin{table}[!ht]
\begin{center}
\begin{tabular}{|c|c|c|c|c|c|c|}
\hline
\textbf{Module} & \textbf{easy} & \textbf{poly} & \textbf{inter} & \textbf{hard} & \textbf{Total} & \textbf{\%} \\
\hline
md5 & 13 & 0 & 0 & 0 & 13 & 100\% \\
\hline
\end{tabular}
\end{center}
\caption{Evaluation results for MD5}
\label{tab:evalmd5}
\end{table}

Even tough it was straightforward to type the MD5 library,
we found a little difference between its documentation and its behavior.
The documentation suggests that the type of the function \texttt{md5.update}
is \texttt{(md5\string_context, string) -> (md5\string_context)},
though there is a call to this function in the test script that passes
an extra string argument.
Reading the source code, we found that its actual type is
\texttt{(md5\string_context, string*) -> (md5\string_context)},
that is, we can pass zero or more strings to \texttt{md5.update}.

This case study shows that type annotations help programmers maintain the
documentation updated, as the type checker always validates them.

\subsection{LuaSocket}

LuaSocket is a library that adds network support to Lua,
and it is split into two parts: a core that is written in C and a set of
Lua modules.
The C core provides TCP and UDP support, while the Lua modules provide
support for SMTP, HTTP, and FTP client protocols, MIME encoding,
URL manipulation, and LTN12 filters \citep{nehab2008ltn012}.
We used Typed Lua's description files to type both parts, as we also
wanted to use LuaSocket to test description files to statically type
the interface of modules that are written in Lua.
Table \ref{tab:evalsocket} summarizes the evaluation results for LuaSocket.

\begin{table}[!ht]
\begin{center}
\begin{tabular}{|c|c|c|c|c|c|c|}
\hline
\textbf{Module} & \textbf{easy} & \textbf{poly} & \textbf{inter} & \textbf{hard} & \textbf{Total} & \textbf{\%} \\
\hline
socket & 50 & 0 & 0 & 10 & 60 & 83\% \\
\hline
ftp & 5 & 0 & 1 & 0 & 6 & 83\% \\
\hline
http & 4 & 0 & 1 & 0 & 5 & 80\% \\
\hline
smtp & 7 & 0 & 0 & 0 & 7 & 100\% \\
\hline
mime & 17 & 0 & 0 & 0 & 17 & 100\% \\
\hline
ltn12 & 19 & 1 & 0 & 0 & 20 & 95\% \\
\hline
url & 8 & 0 & 0 & 0 & 8 & 100\% \\
\hline
\end{tabular}
\end{center}
\caption{Evaluation results for LuaSocket}
\label{tab:evalsocket}
\end{table}

We could type most of the members in the \texttt{socket} module,
which is the C core.
However, this module includes some functions such as
\texttt{socket.try}, \texttt{socket.protect}, and \texttt{socket.skip}
that are hard to type because they rely on reflection.
For instance, we can use \texttt{socket.skip} to choose the number of
values that we want to return.
As an example, calling \texttt{socket.skip(1, nil, "hello")}
returns only the string \texttt{"hello"}, because \texttt{1} indicates
that we do not want to return the first value.
Passing a negative number to \texttt{socket.skip} can be
dangerous, as it returns anything that might be in the stack.
As an example, calling \texttt{socket.skip(-1, nil, "hello")}
returns the tuple \texttt{(-1, nil, "hello")}, because \texttt{-1} makes
\texttt{socket.skip} does not skip any values.
As another example, \texttt{f = socket.skip(-2)} assigns \texttt{socket.skip}
to \texttt{f}, as \texttt{-2} gets \texttt{socket.skip} from the stack.
Our type system cannot handle the type of negative numbers, as
this requires more complex types such as the refinement types from
hybrid type checking \citep{flanagan2006htc}.

We could type most of the members of the modules \texttt{ftp} and
\texttt{http}, but we could not precisely type
the functions \texttt{ftp.get} and \texttt{http.request} because
they are overloaded functions.

The function \texttt{ftp.get} downloads data from a given URL,
which can be either a string or a table.
More precisely, \texttt{ftp.get(url)} returns the tuple
\texttt{(string) | (nil, string)} if \texttt{url} is a string,
and it returns the tuple \texttt{(number) | (nil, string)} if
\texttt{url} is a table.

The function \texttt{http.request} downloads data from a given URL,
which can be either a string or a table.
More precisely, \texttt{http.request(url, body)} returns the tuple
\texttt{(string, number, \{string:string\}, number) | (nil, string)}
if \texttt{url} is a string and \texttt{body} is another string or \texttt{nil},
but it returns the tuple
\texttt{(number, number, \{string:string\}, number) | (nil, string)}
if \texttt{url} is a table and \texttt{body} is \texttt{nil}.

The modules \texttt{mime} and \texttt{ltn12} have a strong connection.
The \texttt{mime} module offers low-level and high-level filters
that apply and remove some text encoding.
The low-level filters are written in C, while the high-level filters
use the function \texttt{ltn12.filter.cycle} along with the low level
filters to create standard filters.

Even though we could type all the members of the \texttt{mime} module,
the function \texttt{ltn12.filter.cycle} is the only member of the
\texttt{ltn12} module that we could not give a precise type.
This function is difficult to type because it is polymorphic.

The modules \texttt{smtp} and \texttt{url} were straightforward to type.
The \texttt{smtp} module provides functions that send e-mails.
The \texttt{url} module provides functions that manipulate URLs.

\subsection{HTTP Digest}

The HTTP Digest library implements client side HTTP digest authentication for Lua.
Table \ref{tab:evalhttpdigest} summarizes the evaluation results for HTTP Digest.

\begin{table}[!ht]
\begin{center}
\begin{tabular}{|c|c|c|c|c|c|c|}
\hline
\textbf{Module} & \textbf{easy} & \textbf{poly} & \textbf{inter} & \textbf{hard} & \textbf{Total} & \textbf{\%} \\
\hline
http-digest & 0 & 0 & 1 & 0 & 1 & 0\% \\
\hline
\end{tabular}
\end{center}
\caption{Evaluation results for HTTP Digest}
\label{tab:evalhttpdigest}
\end{table}

It is difficult to type the interface of the \texttt{http-digest} module
because it is an extension to the \texttt{http} module from LuaSocket.
The \texttt{http-digest} module only exports the function
\texttt{http-digest.request}, which extends the function
\texttt{http.request} with MD5 authentication.
Like \texttt{http.request}, \texttt{http-digest.request}
is also an overloaded function.

Even though we could not precisely type the interface that \texttt{http-digest}
exports, we could use only static types to annotate it, and they pointed a bug
in the code.
The problem was related to the way the library was loading the MD5 library
that should be used. 
This part of the code checks the existence of three different MD5 libraries
in the system, and uses the first one that is available, or generates an
error when none is available.
The code that loads the first option was fine, but the code that loads the
second and third options were trying to access an undefined global variable.

\subsection{Typical}

Typical is a library that extends the behavior of the function \texttt{type}. 
Table \ref{tab:evaltypical} summarizes the evaluation results for Typical.

\begin{table}[!ht]
\begin{center}
\begin{tabular}{|c|c|c|c|c|c|c|}
\hline
\textbf{Module} & \textbf{easy} & \textbf{poly} & \textbf{inter} & \textbf{hard} & \textbf{Total} & \textbf{\%} \\
\hline
typical & 1 & 0 & 0 & 0 & 1 & 100\% \\
\hline
\end{tabular}
\end{center}
\caption{Evaluation results for Typical}
\label{tab:evaltypical}
\end{table}

The interface of the \texttt{typical} module is straightforward to type,
as it contains only the function \texttt{typical.type},
which has the same type of the function \texttt{type}: \texttt{(value) -> (string)}.

However, we hit some limitations of our type system while annotating this module.
First, it uses the function \texttt{getmetatable} to get a table and
checks whether this table has the field \texttt{\string_\string_type} or not.
We could not give a precise type to \texttt{getmetatable}, so we used the dynamic
type \texttt{any} as its return type, and this generates a warning.
Second, the module returns a metatable that extends \texttt{\string_\string_call}
with \texttt{typical.type}, that is, we can use the module itself as a function,
though it is a table.
Our type system still does not support metatables, so we did not extend our version
of the \texttt{typical} module to support \texttt{\string_\string_call}.
Third, the module uses \texttt{ipairs} to iterate over an array of functions,
but our type system also has limited support to \texttt{ipairs}, and generates
a warning when we try to use the indexed value inside the \texttt{for} body.
We removed this warning using the numeric \texttt{for} to perform the same loop.

\subsection{Modulo 11}

Modulo 11 is a library that generates and verifies modulo 11 numbers.
Table \ref{tab:evalmod11} summarizes the evaluation results for Typical.

\begin{table}[!ht]
\begin{center}
\begin{tabular}{|c|c|c|c|c|c|c|}
\hline
\textbf{Module} & \textbf{easy} & \textbf{poly} & \textbf{inter} & \textbf{hard} & \textbf{Total} & \textbf{\%} \\
\hline
mod11 & 7 & 0 & 0 & 2 & 9 & 78\% \\
\hline
\end{tabular}
\end{center}
\caption{Evaluation results for Modulo 11}
\label{tab:evalmod11}
\end{table}

The \texttt{mod11} module was written using an object-oriented idiom that
our type system does not support, and that is the reason why we could not
type all the members of its interface.
More precisely, the original code uses \texttt{setmetatable} to hide
two attributes, which our type system cannot hide.

In addition, it returns a metatable that extends \texttt{\string_\string_call}
with the class constructor.
This allows us to use the module itself to create new instances of a
Modulo 11 number.
However, our type system does not support this feature, and we need to
make explicit calls to the constructor whenever we want to create a
new instance.

Even though we had these two issues to annotate the \texttt{mod11} module,
we could use only static types to annotate it, and we found some interesting
points.
The code relies on implicit conversions between strings and numbers,
and some parts of the code keep on changing the type of local variables.
These are two practices that may hide bugs.

\subsection{Typed Lua Compiler}

The Typed Lua compiler is the last case study that we evaluated.
Table \ref{tab:evaltlc} summarizes its evaluation results.

\begin{table}[!ht]
\begin{center}
\begin{tabular}{|c|c|c|c|c|c|c|}
\hline
\textbf{Module} & \textbf{easy} & \textbf{poly} & \textbf{inter} & \textbf{hard} & \textbf{Total} & \textbf{\%} \\
\hline
tlast & 46 & 0 & 1 & 0 & 47 & 98\% \\
\hline
tltype & 65 & 0 & 0 & 0 & 65 & 100\% \\
\hline
tlst & 26 & 0 & 0 & 0 & 26 & 100\% \\
\hline
tllexer & 2 & 0 & 0 & 9 & 11 & 18\% \\
\hline
tlparser & 1 & 0 & 0 & 0 & 1 & 100\% \\
\hline
tldparser & 1 & 0 & 0 & 0 & 1 & 100\% \\
\hline
tlchecker & 2 & 0 & 0 & 0 & 2 & 100\% \\
\hline
tlcode & 1 & 0 & 0 & 0 & 1 & 100\% \\
\hline
\end{tabular}
\end{center}
\caption{Evaluation results for Typed Lua Compiler}
\label{tab:evaltlc}
\end{table}

The \texttt{tlast} module implements the Abstract Syntax Tree for
the compiler.
We could not precisely type just one function, because it has
an overloaded type that requires intersection types.

The \texttt{tltype} module implements the types introduced by Typed Lua.
It also implements the subtyping and consistent-subtyping relations.
The interface that this module exports was straightforward to type.

The \texttt{tlst} module implements the symbols table for the compiler.
The interface that this module exports was also straightforward to type.

The \texttt{tllexer} module defines common lexical rules for
the Typed Lua parser and the description file parser.
This module is hard to type because it uses LPeg \citep{lpeg,ierusalimschy2009lpeg}
patterns, and LPeg uses overloaded arithmetic operators to
build LPeg patterns. 
We could not precisely type LPeg patterns because our
type system still does not support overloading arithmetic
operators.
In this module, we could only type the two error reporting functions
that it exports.

The \texttt{tlparser} and \texttt{tldparser} modules implement the Typed Lua
parser and the description file parser, respectively.
Even though they use LPeg to implement the grammar rules,
they only export a parsing function.
Both use LPeg to parse a string and return the corresponding AST.

The interface that the \texttt{tlchecker} and \texttt{tlcode}
modules export are also straightforward to type.
The former traverses the AST to perform type checking,
while the latter traverses the AST to perform code generation.

Even though we could precisely type the interface that most of
the modules export, we had issues to write mutually recursive functions.
This kind of functions often appear on compilers construction
to traverse the data structures that they use.
However, Typed Lua still does not support mutually recursive functions.
A way to overcome this limitation was to predeclare these functions
with an empty body, and then redeclare them with they actual body.
The first declaration specifies the function type, while the
second specifies what the function actually does without changing
any type definition.

Traversing the AST would also be problematic if we had not included
a way to discriminate unions of table types, as we mentioned in Section \ref{sec:alias}.

Bootstraping the compiler also helped revealing some bugs.
We found some accesses to undeclared global variables and also
to undeclared table fields.
The compiler also helped pointing the places where we should narrow
a nilable value before using it.
In fact, this point appeared in all the case studies that we used
Typed Lua to annotate Lua code.
This means that Lua programmers often use possibly \texttt{nil} values
before checking whether it is \texttt{nil}.

\section{Related Work}
\label{sec:relatedwork}

In this section we review related work.
We split this section into two parts: in the first part we review
other Lua projects, while in the second part we review other
projects that are not related to Lua.

\subsection{Other Lua projects}

Metalua \citep{metalua} is a Lua compiler that supports compile-time
metaprogramming (CTMP).
CTMP is a kind of macro system that allows the programmers to interact
with the compiler \citep{fleutot2007contrasting}. 
Metalua extends Lua 5.1 syntax to include its macro system,
and allows programmers to define their own syntax.
Metalua can provide syntactical support for several object-oriented
styles, and can also provide syntax for turning simple type
annotations into run-time assertions.

MoonScript \citep{moonscript} is a programming language that supports
class-based object-oriented programming.
MoonScript compiles to idiomatic Lua code, but 
it does not perform compile-time type checking.

LuaInspect \citep{luainspect} is a tool that uses MetaLua to perform
some code analysis.
For instance, it flags unknown global variables and table fields,
it checks the number of function arguments against signatures, and
it infers function return values.
However, it does not try to analyze object-oriented code and
it does not perform compile-time type checking.

Tidal Lock \citep{tidallock} is a prototype of another optional type
system for Lua, which is written in Metalua.
Tidal Lock covers a little subset of Lua.
Statements include declaration of local variables, multiple assignments,
function application, and the return statement.
This means that Tidal Lock does not include any control-flow statement.
Expressions include primitive literals, table indexing, function application,
function declaration, and the table constructor, but they do not include
binary operations.

A remarkable feature of Tidal Lock is the refinement of table types.
This feature inspired us to also include it in Typed Lua,
but in a simpler way and with different formalization.

The table type from Tidal Lock can only represent records, that is,
it cannot describe hash tables and arrays yet, though we can refine them.
Tidal Lock also includes field types to describe the type of the fields
of a table type.
The field types describe if a table field is mutable or immutable
in a table type.
Field types are the feature that allow the refinement of table types in
Tidal Lock.

Tidal Lock is also a structural type system that relies on subtyping and
local type inference.
However, it does not support union types, recursive types, and variadic types.
It also does not type any object-oriented idiom.

Luacheck \citep{luacheck} is a tool that performs static analysis on Lua code.
It can flag access to undeclared globals and unused local variables,
but it does not perform static type checking.

Ravi \citep{ravi} is an experimental Lua dialect.
Ravi introduces optional static typing for Lua to improve run-time performance.
To do that, Ravi extends the Lua Virtual Machine to include new
operations that take into account static type information.

\subsection{Other projects}

Typed Racket \citep{tobin-hochstadt2008ts} is a statically typed version
of the Racket language, which is a Scheme dialect.
The main purpose of Typed Racket is to allow programmers to combine
untyped modules, which are written in Racket, with typed modules, which are
written in Typed Racket.
It also uses local type inference to deduce the type of unannotated expressions.

The main feature of Typed Racket's type system is \emph{occurrence typing}
\citep{tobin-hochstadt2010ltu}.
It is a novel way to use type predicates in control flow statements
to refine union types.
Occurrence typing is not sound in the presence of mutation.
As these kinds of checks are common in other languages, related systems
have appeared \citep{guha2011tlc,winther2011gtp,pearce2013ccf}. 

The type system of Typed Racket also includes function types, recursive
types, and structure types.
Its function types also handle multiple return values, and there is
also a way to describe function types that have optional arguments.
Its structure types are similar to our interfaces, as they describe record types.
The type system is also structural and based on subtyping.
It also includes the dynamic type \texttt{Any}, which is the top type in the system.
Typed Racket also supports polymorphic functions and data structures.

Typed Clojure \citep{bonnaire-sergeant2012typed-clojure} is an
optional type system for Clojure.
Although Clojure is a Lisp dialect that runs on the Java Virtual Machine,
Common Language Runtime, and JavaScript, Typed Clojure runs only on
the Java Virtual Machine.
Perhaps, this restriction pushed Typed Clojure to support Java classes
and some Java types such as \texttt{Long}, \texttt{Double}, and \texttt{String}.
Typed Clojure also provides optional type annotations and uses
local type inference to deduce the type of unannotated expressions.
It also assigns the type \texttt{Any} to unannotated function parameters,
which is the top type in the type system.

The type system of Typed Clojure includes polymorphic function types,
union types, intersection types, lists, vectors, maps, sets, and recursive types.
Function types can also have rest parameters, which are similar
to our variadic types, but can only appear on the input parameter
of function types.
In fact, its function types cannot return multiple results.
It also uses occurrence typing to allow control flow statements to
refine union types.
The type system is also structural and based on subtyping.

Dart \citep{dart} is a new class-based object-oriented programming
language.
It includes optional type annotations and compiles to JavaScript.
The type system of Dart is nominal and includes base types,
function types, lists, and maps.
It also supports generics, and the programmer can define
generic functions, lists, and maps.
Unlike Typed Lua, Dart is unsound by design.

Even though Dart has optional typing and static types by
default do not affect run-time semantics, it has an
execution mode that affects run-time.
The \emph{checked mode} inserts run-time assertions that
verifies whether static types match run-time tags.
The \emph{production mode} is the default execution mode
that does not include any assertions.

TypeScript \citep{typescript} is a JavaScript extension
that includes optional type annotations and class-based
object-oriented programming.
It also uses local type inference to deduce the type
of unannotated expressions.
The type system of TypeScript is structural, based
on subtyping, and supports generics.
It includes the dynamic type, primitive types, union types,
function types, array types, tuple types, recursive types, and
object types.
Unlike Typed Lua, TypeScript uses arrays to represent variadic
functions and multiple return values.

Even though TypeScript is unsound by design,
\citet{bierman2014typescript} shows how to make TypeScript sound.
They use a reduced core of TypeScript to formalize a
sound type system for TypeScript, but also to formalize
its current unsound type system.

TeJaS \citep{lerner2013tejas} is a framework for the construction of
different type systems for JavaScript.
The authors created a base type system for JavaScript with
extensible typing rules that allow the experimentation of
different static analysis.
They used TeJaS to create a type system that simulates the
type system of TypeScript.

\citet{politz2012semantics} proposes semantics and types for objects
with first-class member names, a well-known feature from scripting languages.
Their type system uses string patterns to describe the members of
an object, and define a complex subtyping relation to validate
these patterns.
They also provide an implementation of their system to JavaScript.

Gradualtalk \citep{allende2013gts} is a Smalltalk dialect that
supports gradual typing.
The type system combines nominal and structural typing.
It includes function types, union types, structural types,
nominal types, a self type, and parametric polymorphism.
The type system also relies on subtyping and consistent-subtyping.

Gradualtalk inserts run-time checks that ensure dynamically
typed code does not violate statically typed code.
\citet{allende2013cis} perform a carefull evaluation about
cast insertion on Gradualtalk.
They report that usually cast insertions impact on execution
performance, so Gradualtalk also has an option that allows
programmers to turn them off, downgrading Gradualtalk
to an optional type system.

Reticulated Python \citep{vitousek2014deg} is a Python compiler
that supports gradual typing.
The type system is structural and based on subtyping.
It includes base types, the dynamic type, list types,
dictionary types, tuple types, function types, set types,
object types, class types, and recursive types.
It includes class and object types to differentiate the
type of class declarations and instances, respectively.
It also uses local type inference.
Besides static type checking, Reticulated Python also introduces
three different approaches for inserting run-time assertions.

Hack \citep{hack} is a new programming language that runs on the
Hip Hop Virtual Machine (HHVM).
The HHVM is a virtual machine that executes Hack and PHP programs.
We can view Hack as an extension to PHP that combines static and
dynamic typing.
The type system of Hack includes generics, nullable types, collections,
and function types.

The Ruby Type Checker \citep{ren2013rtc} is a library that
performs type checking during run-time.
The library provides type annotations that the programmer
can use on classes and methods.
Its type system includes nominal types, union types,
intersection types, method types, parametric polymorphism,
and type casts.

Grace \citep{black2013sg} is an object-oriented language
with optional typing.
Grace is not a dynamically typed language that has been
extended with an optional type system, but a language
that has been designed from scratch to have both
static and dynamic typing.
\citet{homer2013modules} explores some useful patterns
that derive from Grace's use of objects as modules
and its brand of optional structural typing, which
can also be expressed with Typed Lua's modules as tables.
