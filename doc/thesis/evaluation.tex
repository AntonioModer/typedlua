We performed some case studies on existing Lua libraries
to evaluate the design of our type system.
For each library, we used Typed Lua to either rewrite its modules
or to write statically typed interfaces to its modules through
Typed Lua's description files.
In this chapter we present our evaluation results and discuss
some interesting cases.

Tables \ref{tab:evalbylib} and \ref{tab:evalbymod} sumarizes our
evaluation results:
the first presents the results agruped by library while
the second presents the results agruped by module.
We split the members that we typed into three categories:
\emph{easy}, \emph{poly}, and \emph{hard}.
The \emph{easy} category represents the members that we could give
a precise static type for them.
The \emph{poly} category represents the members that we had to
rely on the dynamic type, and we believe we could give a precise
static type for them if our type system had parametric polymorphism.
The \emph{hard} category represents the members that we could not
give a precise static type for them, relying on the dynamic type
in most of the cases, but with no hopes that parametric polymorphism
would help us to give a precise static type for them.
The last column of each table contains the actual evaluation result
of each case study, that is basically the percentage of members
that are under the \emph{easy} category.

\begin{table}[!ht]
\begin{center}
\begin{tabular}{|l|c|c|c|c|c|c|}
\hline
\textbf{Case study} & \textbf{easy} & \textbf{poly} & \textbf{hard} & \textbf{Total} & \textbf{\%} \\
\hline
Lua Standard Libraries & 92 & 11 & 27 & 130 & 71\% \\
\hline
Lua Socket & 85 & 22 & 19 & 126 & 67\% \\
\hline
MD5 & 13 & 0 & 0 & 13 & 100\% \\
\hline
\end{tabular}
\end{center}
\caption{Evaluation of Typed Lua by library}
\label{tab:evalbylib}
\end{table}

\begin{table}[!ht]
\begin{center}
\begin{tabular}{|l|c|c|c|c|c|c|}
\hline
\textbf{Case study} & \textbf{Module} & \textbf{easy} & \textbf{poly} & \textbf{hard} & \textbf{Total} & \textbf{\%} \\
\hline
\multirow{9}{*}{Lua Standard Libraries}
& base & 9 & 5 & 12 & 26 & 35\% \\
\cline{2-7}
& coroutine & 0 & 0 & 6 & 6 & 0\% \\
\cline{2-7}
& package & 5 & 2 & 1 & 8 & 62\% \\
\cline{2-7}
& string & 13 & 0 & 1 & 14 & 93\% \\
\cline{2-7}
& table & 1 & 4 & 1 & 6 & 17\% \\
\cline{2-7}
& math & 28 & 0 & 1 & 29 & 97\% \\
\cline{2-7}
& bit32 & 12 & 0 & 0 & 12 & 100\% \\
\cline{2-7}
& io & 14 & 0 & 4 & 18 & 78\% \\
\cline{2-7}
& os & 10 & 0 & 1 & 11 & 91\% \\
\hline
\multirow{7}{*}{Lua Socket}
& socket & 50 & 3 & 7 & 60 & 83\% \\
\cline{2-7}
& ftp & 6 & 1 & 1 & 8 & 75\% \\
\cline{2-7}
& http & 3 & 0 & 2 & 5 & 60\% \\
\cline{2-7}
& ltn12 & 10 & 10 & 0 & 20 & 50\% \\
\cline{2-7}
& mime & 2 & 8 & 7 & 17 & 12\% \\
\cline{2-7}
& smtp & 6 & 0 & 2 & 8 & 75\% \\
\cline{2-7}
& url & 8 & 0 & 0 & 8 & 100\% \\
\hline
\multirow{1}{*}{MD5}
& md5 & 13 & 0 & 0 & 13 & 100\% \\
\hline
\end{tabular}
\end{center}
\caption{Evaluation of Typed Lua by module}
\label{tab:evalbymod}
\end{table}
