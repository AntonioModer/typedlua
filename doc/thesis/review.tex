We begin this chapter presenting a little bit of the history about
combining static and dynamic typing in the same language.
After that, we introduce optional type systems and gradual typing.
We end this section discussing why optional type systems and two
other approaches are often called gradual typing.

\section{A little bit of history}

Common LISP \citep{steele1982ocl} introduced optional type annotations
in the early eighties, but not for static type checking.
Instead, programmers may choose to declare types of variables as
optimization hints to the compiler, that is, type declarations are
just one way to help the compiler to optimize code.
These annotations are unsafe because they can crash the program
when they are wrong.

\citet{abadi1989dts} extended the simply typed lambda calculus with the
\texttt{Dynamic} type and the \texttt{dynamic} and \texttt{typecase}
constructs, with the aim to safely integrate dynamic code in
statically typed languages.
The \texttt{Dynamic} type is a pair \texttt{(v,T)} where \texttt{v} is a
value and \texttt{T} is the tag that represents the type of \texttt{v}.
The constructs \texttt{dynamic} and \texttt{typecase} are explicit
injection and projection operations, respectively.
That is, \texttt{dynamic} builds values of type \texttt{Dynamic} and
\texttt{typecase} safely inspects the type of a \texttt{Dynamic} value.
Thus, migrating code between dynamic and static type checking requires
changing type annotations and adding or removing \texttt{dynamic} and
\texttt{typecase} constructs throughout the code.

The \emph{quasi-static} type system proposed by \citet{thatte1990qst}
performs implicit coercions and run-time checks to replace the
\texttt{dynamic} and \texttt{typecase} constructs that were proposed by
\citet{abadi1989dts}.
To do that, quasi-static typing relies on subtyping with a top type
$\Omega$ that represents the dynamic type, and splits type checking
into two phases.
The first phase inserts implicit coercions from the dynamic type to
the expected type, while the second phase performs what Thatte calls
\emph{plausibility checking}, that is, it rewrites the program to
guarantee that sequences of upcasts and downcasts always have a
common subtype.

\emph{Soft typing} \citep{cartwright1991soft} is another approach
to combine static and dynamic typing in the same language.
The main goal of soft typing is to add static type checking to
dynamically typed languages without compromising their flexibility.
To do that, soft typing relies on type inference for
translating dynamically typed code to statically typed code.
The type checker inserts run-time checks around inconsistent code and
warns the programmer about the insertion of these run-time checks,
as they indicate the existence of potential type errors.
However, the programmer is free to choose between inspecting the
run-time checks or simply running the code.
Soft typing gives this choice to the programmer because it is one of
its key ideas, that is, type inference and static type checking should
not prevent the programmer from running inconsistent code.
One advantage of soft typing is the fact that the compiler for
softly typed languages can use the translated code to generate
more efficient code, as the translated code statically type checks.
One disadvantage of soft typing is that it can be cumbersome when
the inferred types are meaningless large types that just confuse the
programmer.

\emph{Dynamic typing} \citep{henglein1994dts} is an approach
that optimizes code from dynamically typed languages by eliminating
unnecessary checks of tags.
The paper describes how to translate dynamically typed code to
statically typed code that uses a \texttt{Dynamic} type.
The translation is done through a coercion calculus that uses type
inference to insert the operations that are necessary to type check
the \texttt{Dynamic} type during run-time.
Although soft typing and dynamic typing may seem similar, they are not.
They are not similar because soft typing targets statically type
checking of dynamically typed languages for detecting programming
errors while dynamic typing targets the optimization of dynamically
typed code through the elimination of unnecessary run-time checks.
In other words, soft typing sees code optimization as a side effect
that comes with static type checking.

\citet{findler2002chf} proposed contracts for higher-order functions
and blame annotations for run-time checks.
Contracts perform dynamic type checking instead of static type checking,
but deferring all verifications to run-time can lead to defects
that are difficult to fix, because run-time errors can show a
stack trace where it is not clear to programmers if the cause of a
certain run-time error is in application code or library code.
Even if programmers identify that the source of a certain run-time
error is in library code, they still may have problems to identify
if this run-time error is due to a violation of library's contract,
or due to a bug, when the library is poorly documented.
In this approach, programmers can insert assertions in the form of
contracts that check the input and output of higher-order functions;
and the compiler add blame annotations in the generated code as a way
to track assertion failures back to the source of the error.

BabyJ \citep{anderson2003babyj} is an object-oriented language
without inheritance that allows programmers to incrementally annotate
the code with more specific types.
Programmers can choose between using the dynamically typed version
of BabyJ, when they do not need types at all, and the statically
typed version of BabyJ, when they need to annotate the code.
In the statically typed BabyJ, programmers can use the
\emph{permissive type} $*$ to annotate the parts of the code that
still do not have a specific type or the parts of the code that should
have dynamic behavior.
The type system of BabyJ is nominal, so types are either class names
or the permissive type $*$.
However, the type system does not use type equality or subtyping,
but the relation $\approx$ between two types.
The relation $\approx$ holds when both types have the same name or
any of them is the permissive type $*$.
Even though the permissive type $*$ is similar to the dynamic type
from previous approaches, BabyJ does not provide any way to add
implicit or explicit run-time checks.

\citet{ou2004dtd} specified a language that combines static types
with dependent types.
To ensure safety, the compiler automatically inserts coercions
between dependent code and static code.
The coercions are run-time checks that ensure static code does not
crash dependent code during run-time.

\section{Optional Type Systems}

Optional type systems \citep{bracha2004pluggable} are an approach for
plugging static typing in dynamically typed languages through
optional type annotations that are used to perform compile-time type
checking, though not influencing the original run-time semantics of
the language.
That is, the run-time semantics should still catch type errors
independently of the static type checking.
For instance, we can view the typed lambda calculus as an optional
type system for the untyped lambda calculus, because both have the
same semantic rules and the type system serves only for discarding
programs that may have undesired behaviors \citep{bracha2004pluggable}.

Strongtalk \citep{bracha1993strongtalk,bracha1996strongtalk} is
a version of Smalltalk that comes with an optional type system.
It has a polymorphic type system that programmers can use to annotate
Smalltalk code or leave type annotations out.
Strongtalk assigns a dynamic type to unannotated expressions and allows
programmers to cast unannotated expressions to any static type.
This means that the interaction of the dynamic type with the rest of
the type system is unsound, so Strongtalk uses the original run-time
semantics of Smalltalk when executing programs, even if programs are
statically typed.

\emph{Pluggable type systems} \citep{bracha2004pluggable} generalize
the idea of optional type systems that Strongtalk put in practice,
to have different optional type systems that can be layered on top
of a dynamically typed language without affecting its original run-time
semantics.
Although these systems can be unsound on their interaction with the
dynamically typed part of the language or even by design, their
unsoundness does not affect run-time safety, as the language run-time
semantics still cathes any run-time errors caused by an unsound
type system.

Dart \citep{dart} and TypeScript \citep{typescript} are new
languages that are designed with an optional type system, and both
use JavaScript as their code generation target because their main
purpose is web development.
In fact, Dart is a new class-based object-oriented language with
optional type annotations and a semantics that resembles the
semantics of Smalltalk, while TypeScript is a strict superset of
JavaScript that provides optional type annotations and class-based
object-oriented programming.
Dart has a nominal type system, while TypeScript has a structural
one, but both are unsound by design.
For instance, Dart has covariant arrays, while TypeScript has
covariant function return types;
without counting the interaction between statically and dynamically
typed code that is also unsound.

Unfortunately, there is no common formalization for optional type
systems, so each language ends up implementing its optional type
system on its own way.
Strongtalk, Dart, and TypeScript provide an informal description of
their optional type systems rather than a formal one.
In the next section we shall see that we can use some aspects
of gradual typing \citep{siek2006gradual,siek2007objects} to
formalize optional type systems.

\section{Gradual Typing}

The main goal of gradual typing \citep{siek2006gradual} is to allow
programmers to choose between static and dynamic typing in the same
language.
To do that, \citet{siek2006gradual} extended the simply typed
lambda calculus with the dynamic type $?$.
In gradual typing, type annotations are optional, so an untyped
variable is syntactic sugar for a variable whose declared type is
the dynamic type $?$.
Under these circumstances, we view gradual typing as a way to add
a dynamic type to statically typed languages.

The central idea of gradual typing is the \emph{consistency}
relation, written $\tau_{1} \sim \tau_{2}$, that allows implicit
conversions to and from the dynamic type, and disallows conversions
between inconsistent types \citep{siek2006gradual}.
For instance, $\mathbf{int} \sim \;?$, $? \sim \mathbf{int}$,
$\mathbf{string} \sim \;?$, and $? \sim \mathbf{string}$,
but $\mathbf{int} \not\sim \mathbf{string}$, and
$\mathbf{string} \not\sim \mathbf{int}$.
The consistency relation is neither commutative nor transitive.

Gradual typing is similar to the approaches proposed by
\citet{abadi1989dts} and \citet{thatte1990qst} by including a
dynamic type to a statically typed language.
However, these three approaches differ in the way they handle the
dynamic type.
While \citet{siek2006gradual} rely on the consistency relation,
\citet{abadi1989dts} rely on type equality with explicit projections
and injections, and \citet{thatte1990qst} relies on subtyping.
The subtyping relation actually is a pitfall on Thatte's quasi-static
typing, because the dynamic type is put as the top and the bottom of
the subtyping relation.
Therefore, downcasts combined with the transitivity of subtyping
accepts programs that should be rejected.

Later, \citet{siek2007objects} discovered that the consistency relation
is orthogonal to subtyping, so we can combine the two relations to
achieve both consistency and subtyping.
The \emph{consistent-subtyping} relation,
written $\tau_{1} \lesssim \tau_{2}$,
is essential to design object-oriented languages that have a
gradual type system.
Like the consistency relation, and unlike the subtyping relation,
the consistent-subtyping relation is not transitive.

Another important feature of gradual typing is the theoretic
foundation that it provides for inserting run-time checks that
prove dynamically typed code does not violate the invariants of
statically typed code, thus preserving type safety.
To do that, \citet{siek2006gradual} defined the run-time semantics
of gradual typing as a translation to an intermediate language with
explicit casts at the frontiers between statically and dynamically
typed code.
The semantics of these casts is based on the higher-order contracts
proposed by \citet{findler2002chf}.

However, \citet{herman2007sgt} showed that there is an efficiency
concern regarding the run-time checks, because there are two
ways that casts can lead to unbounded space consumption.
The first affects tail recursion while the second appears when
first-class functions or objects cross the border between
static code and dynamic code, that is, some programs can apply
repeated casts to the same function or object.
\citet{herman2007sgt} use the coercion calculus of
\citet{henglein1994dts} to solve the problem of space efficiency,
that is, they use coercions to express casts.
Their approach normalizes an arbitrary sequence of coercions to a
coercion of bounded size.

Another concern about casts is how to improve debugging support,
because a cast application can be delayed and the error related
to that cast application can appear far way from the real error.
\citet{wadler2009wpc} developed the \emph{blame calculus} as a way
to attack this issue, and \citet{ahmed2011bfa} extended the
blame calculus with polymorphism.
The blame calculus is an intermediate language to integrate
static and dynamic typing that uses the blame tracking approach
proposed by \citet{findler2002chf}.

On the one hand, the blame calculus solves the issue regarding
error reporting;
on the other hand, it has the space efficiency problem reported
by \citet{herman2007sgt}.
Thus, \citet{siek2009casts} extended the coercion calculus of
\citet{herman2007sgt} with blame tracking to achieve an
implementation of the blame calculus that is space efficient.
After that, \citet{siek2010blame} proposed a new solution that also
attacks both problems.
This new solution is based on a concept called \emph{threesome},
which is a way to split a cast between two parties into two casts
among three parties.
A cast has a source and a target type (a \emph{twosome}),
so we can split any cast into a downcast from the source to an
intermediate type that is followed by an upcast from the intermediate
type to the target type (a \emph{threesome}).

ActionScript \citep{moock2007as3} is the first language that
incorporated a gradual type system to its implementation and
Perl 6 \citep{tang2007pri} is also being designed with a
gradual type system, though there is little documentation about
the gradual type systems of these languages.
Grace \citep{black2012grace,black2013sg} is a new object-oriented,
gradually typed, educational language.
In Grace, modules are gradually typed objects, that is, modules
may have types and methods as attributes, and can have a state
\citep{homer2013modules}.
Gradual Python \citep{reticulated} is an attempt to develop a
gradual type system for Python that deals with mutable objects
\citep{siek2013mutable}.
Gradualtalk \citep{allende2013gts} is a gradually-typed Smalltalk
that introduces a new cast insertion strategy for gradually-typed
objects \citep{allende2013cis}.

\section{Approaches that are often called Gradual Typing}

Gradual typing is similar to optional type systems in that type
annotations are optional, and unannotated code is dynamically
typed, but unlike optional type systems, gradual typing changes
the run-time semantics to preserve type safety, and it is a way to
add a dynamic type to statically typed languages.
More precisely, programming languages that include a gradual type
system implement the semantics of statically typed languages, so
the gradual type system inserts casts in the translated code to
guarantee that types are consistent before execution, while
programming languages that include an optional type system still
implement the semantics of dynamically typed languages, so all
the type checking also belongs to the semantics of each operation.

Still, we can view gradual typing as a way to formalize an optional
type system when the gradual type system does not insert run-time
checks.
BabyJ \citep{anderson2003babyj} and Alore \citep{lehtosalo2011alore}
are two examples of object-oriented languages that use gradual
typing to formalize their optional type systems.
The optional type systems of both BabyJ and Alore are nominal.
BabyJ uses the relation $\approx$ that is similar to the consistency
relation while Alore combines  subtyping along with the consistency
relation to define a \emph{consistent-or-subtype} relation.
The consistent-or-subtype relation is different from the
consistent-subtyping relation of \citet{siek2007objects}, but it is
also written $\tau_{1} \lesssim \tau_{2}$.
The consistent-or-subtype relation holds when $\tau_{1} \sim \tau_{2}$
or $\tau_{1} <: \tau_{2}$, where $<:$ is transitive and $\sim$ is not.
Alore also extends its optional type system to include optional
monitoring of run-time type errors in the gradual typing style.

Hence, optional type annotations for software evolution probably are
the reasons why optional type systems are commonly called
gradual type systems.
Typed Clojure \citep{bonnaire-sergeant2012typed-clojure} is an
optional type system for Clojure that now is adopting the
gradual typing slogan.

\citet{flanagan2006htc} introduced \emph{hybrid type checking},
an approach that combines static types and \emph{refinement} types.
For instance, programmers can specify the refinement type
$\{x:Int \;|\; x \ge 0\}$ when they need a type for natural numbers.
The programmer can also choose between explicit or implicit casts.
When casts are not explicit, the type checker uses a theorem prover
to insert casts.
In our example of natural numbers, a cast would be inserted to check
whether an integer is greater than or equal to zero.

Sage \citep{gronski2006sage} is a programming language that
extended hybrid type checking with a dynamic type to
support dynamic and static typing in the same language.
Sage also offers optional type annotations in the gradual typing
style, that is, unannotated code is syntactic sugar for
code whose declared type is the dynamic type.

Thus, the inclusion of a dynamic type to hybrid type checking
along with optional type annotations, and the insertion of run-time
checks probably are the reasons why hybrid type checking is
also viewed as a form of gradual typing.

\citet{tobin-hochstadt2006ims} proposed another approach for gradually
migrating from dynamically typed to statically typed code,
and they coined the term \emph{from scripts to programs} for
referring to this kind of interlanguage migration.
In their approach, the migration from dynamically typed to
statically typed code happens module-by-module, so they designed
and implemented Typed Racket \citep{tobin-hochstadt2008ts} for
this purpose.
Typed Racket is a statically typed version of Racket
(a Scheme dialect) that allows the programmer to write typed modules,
so Typed Racket modules can coexist with Racket modules,
which are untyped.

The approach used by \citet{tobin-hochstadt2008ts} to design and
implement Typed Racket probably is also called gradual typing
because it allows the programmer to gradually migrate from untyped
scripts to typed programs.
However, note that Typed Racket is a statically typed language
and type annotations are mandatory.
So, what makes it gradual is a type system with a dynamic type
that handles the interaction between Racket and Typed Racket modules.

