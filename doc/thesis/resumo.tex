Linguagens dinamicamente tipadas, tais como Lua, não usam tipos estáticos em
favor de simplicidade e flexibilidade, porque a ausência de tipos estáticos
significa que programadores não precisam se preocupar em abstrair tipos que
devem ser validados por um verificador de tipos.
Por outro lado, linguagens estaticamente tipadas ajudam na detecção prévia de
diversos \emph{bugs} e também ajudam na estruturação de programas grandes.
Tais pontos geralmente são vistos como duas vantagens que levam programadores
a migrar de uma linguagem dinamicamente tipada para uma linguagem estaticamente tipada,
quando os pequenos \emph{scripts} deles evoluem para programas complexos.

Sistemas de tipos opcionais nos permitem combinar tipagem dinâmica e estática na
mesma linguagem, sem afetar a semântica original da linguagem, tornando mais
fácil a evolução de código tipado dinamicamente para código tipado estaticamente.
Desenvolver um sistema de tipos opcional para uma linguagem dinamicamente tipada é
uma tarefa desafiadora, pois ele deve ser o mais natural possível para os programadores
que já estão familiarizados com essa linguagem.

Neste trabalho nós apresentamos e formalizamos Typed Lua, um sistema de tipos opcional
para Lua, o qual introduz novas características para tipar estaticamente alguns idiomas
e características de Lua.
Embora Lua compartilhe várias características com outras linguagens dinamicamente
tipadas, em particular JavaScript, Lua também possui várias características não usuais,
as quais não estão presentes nos sistemas de tipos dessas linguagens.
Essas características incluem funções com aridade flexível, atribuições múltiplas,
funções que são sobrecarregadas no número de valores de retorno e
a evolução incremental de registros e objetos.
Nós discutimos como Typed Lua tipa estaticamente essas características e
também discutimos nossas decisões de projeto.
Finalmente, apresentamos uma avaliação de resultados,
a qual obtivemos ao usar Typed Lua para tipar código Lua existente.
