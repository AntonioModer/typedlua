Dynamically typed languages such as Lua avoid static types in favor of
simplicity and flexibility, because the absence of static types means
that programmers do not need to bother about abstracting types that
should be validated by a type checker.
Instead, dynamically typed languages use run-time \emph{type tags}
to classify the values they compute, so their implementation can use
these tags to perform run-time (or dynamic) type checking
\citep{pierce2002tpl}.

This simplicity and flexibility allow programmers to write code that
might require a quite complex type system to statically type check,
though it may also hide bugs that will be caught only after deployment
if programmers do not properly test their code.
In contrast, static type checking helps programmers detect many
bugs during the development phase.
Static types also provide a conceptual framework that helps
programmers define modules and interfaces that can be combined to
structure the development of programs.

Thus, early error detection and better program structure are two
advantages of static type checking that can lead programmers to
migrate their code from a dynamically typed to a statically
typed language, when their simple scripts become complex programs
\citep{tobin-hochstadt2006ims}.
Dynamically typed languages certainly help programmers during the
beginning of a project, because their simplicity and flexibility
allow quick development and make it easier to change code according to
changing requirements.
However, programmers tend to migrate from dynamically typed to
statically typed code as soon as the project has consolidated its
requirements, because the robustness of static types helps
programmers link requirements to abstractions.
This migration usually involves different languages that have
different syntax and semantics, which usually require a complete
rewrite of existing programs instead of incremental evolution from
dynamic to static types.

Ideally, programming languages should offer programmers the
option to choose between static and dynamic typing:
\emph{optional type systems} \citep{bracha2004pluggable} and
\emph{gradual typing} \citep{siek2006gradual} are two similar
approaches for blending static and dynamic typing in the same
language.
The aim of both approaches is to offer programmers the option
to use type annotations where static typing is needed and to not use
them where dynamic typing is sufficient.
The difference between these two approaches is the way they treat
run-time semantics.
While optional type systems do not affect the run-time semantics,
gradual typing uses run-time checks to ensure that dynamically typed
code does not violate the invariants of statically typed code.

Programmers and researchers sometimes use the term gradual typing
to mean the incremental evolution of dynamically typed code to
statically typed code.
For this reason, gradual typing may also refer to optional type
systems and other approaches that blend static and dynamic typing to
help programmers incrementally migrate from dynamic to static typing
without having to switch to a different language, though all these
approaches differ in the way they handle static and dynamic typing
together.
We use the term gradual typing for referring to the work of
\citet{siek2006gradual}.

In this work we present the design and implementation of Typed Lua:
an optional type system for Lua that is complex enough to
preserve some of the Lua idioms that programmers are already used to,
but that also includes new constructs that help programmers
structure Lua programs.

Lua is a small scripting language, so designing an optional type
system for Lua should shed some light on optional type systems for
scripting languages in general.
Lua provides mechanisms for organizing a program in modules and
writing object-oriented programs, but it does not set policy on how
these features should behave, due to its use as an embedded and
extension language.
Thus, implementing an optional type system for Lua should offer Lua
programmers one way to obtain most of the advantages of static typing
without compromising the simplicity and flexibility of dynamic typing.

So far, Typed Lua is a strict superset of Lua that
provides optional type annotations, compile-time type checking, and
class-based object-oriented programming through the definition of
classes, interfaces, and modules.
In practice, we implement Typed Lua as a programming language that
extends Lua syntax to add optional type annotations and standard
constructions for structuring Lua code.
The compiler uses static types to perform compile-time type
checking, but also allows Lua code to coexist with Typed Lua code,
and generates Lua code that runs in unmodified Lua implementations.
We have an ongoing implementation of the Typed Lua compiler that is
available online\footnote{https://github.com/andremm/typedlua}.

Typed Lua intended use is as an application language, and we view
that policies for organizing a program in modules and writing
object-oriented programs should be part of the language and
enforced by its optional type system.
An application language is a programming language that helps
programmers develop applications from scratch until these
applications evolve to complex systems rather than just scripts.

We also believe that Typed Lua will help programmers give a more
formal documentation to already existing Lua code, as static types
are also a useful source of documentation in languages that provide
type annotations, because type annotations are always validated by
the type checker and therefore never get outdated.
Thus, programmers can use Typed Lua to define axioms about the
interfaces and types of dynamically typed modules.
We enforce this point by using Typed Lua to statically type
the interface of the Lua standard libraries and other commonly used
Lua libraries, so our compiler can check their usage by Typed Lua
code.

Typed Lua performs a very limited form of local type inference
\citep{pierce2000lti}, as static typing does not necessarily mean
that programmers need to insert type annotations in the code.
Several statically typed languages such as Haskell provide some
amount of type inference that automatically deduces the types of
expressions.
Still, Typed Lua needs a small amount of type annotations due
to the nature of its optional type system.

Typed Lua does not deal with code optimization, although another
important advantage of static types is that they help the compiler
perform optimizations and generate more efficient code.
However, we believe that the formalization of our optional type
system is precise enough to aid optimization in some Lua implementations.

We use some of the ideas of gradual typing to formalize Typed Lua.
Even though Typed Lua is an optional type system and thus does not
include run-time checks between dynamic and static regions of the
code, we believe that using the foundations of gradual typing to
formalize our optional type system will allow us to include run-time
checks in the future.

In Chapter \ref{chap:review} we review the literature about static and
dynamic typing in the same language, which includes the discussion
about the particularities between optional type systems and gradual
typing.
In Chapter \ref{chap:typedlua} we present the design and implementation
of Typed Lua along with the reasons why it includes features of
object-oriented programming.
In Chapter \ref{chap:cases} we present some case studies.
In Chapter \ref{chap:conc} we outline our contributions.

