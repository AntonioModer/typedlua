\documentclass[pdftex,12pt,a4paper]{report}

\usepackage[utf8]{inputenc}
\usepackage[square,semicolon]{natbib}
\usepackage{amsmath}
\usepackage{amssymb}
\usepackage{url}

\newcommand{\Any}{\mathbf{any}}
\newcommand{\Nil}{\mathbf{nil}}
\newcommand{\False}{\mathbf{false}}
\newcommand{\True}{\mathbf{true}}
\newcommand{\Boolean}{\mathbf{boolean}}
\newcommand{\Number}{\mathbf{number}}
\newcommand{\String}{\mathbf{string}}
\newcommand{\Void}{\mathbf{void}}
\newcommand{\X}{$\bullet$}

\def\dstart{\hbox to \hsize{\vrule depth 4pt\hrulefill\vrule depth 4pt}}
\def\dend{\hbox to \hsize{\vrule height 4pt\hrulefill\vrule height 4pt}}

\begin{document}

\begin{titlepage}
\begin{center}

\textsc{\LARGE PUC--Rio}\\[1.5cm]
\textsc{\Large Informatics Department}\\[1.0cm]
\textsc{\Large PhD Thesis}\\[0.5cm]

\newcommand{\HRule}{\rule{\linewidth}{0.5mm}}
\HRule \\[0.4cm]
{\huge \bfseries Typed Lua: An Optional Type System for Lua}\\[0.4cm]
\HRule \\[1.5cm]

\begin{minipage}{0.4\textwidth}
\begin{flushleft} \large
\emph{Author:}\\
André Murbach Maidl
\end{flushleft}
\end{minipage}
\begin{minipage}{0.4\textwidth}
\begin{flushright} \large
\emph{Advisors:} \\
Roberto Ierusalimschy \\
Fabio Mascarenhas
\end{flushright}
\end{minipage}

\vfill
{\large \today}

\end{center}
\end{titlepage}

\newpage
\pagenumbering{roman}

\tableofcontents

\listoffigures
\addcontentsline{toc}{chapter}{List of Figures}

%\listoftables
%\addcontentsline{toc}{chapter}{List of Tables}

%\chapter*{Resumo}
%\addcontentsline{toc}{chapter}{Resumo}

%\chapter*{Abstract}
%\addcontentsline{toc}{chapter}{Abstract}

\newpage
\pagenumbering{arabic}

\chapter{Introduction}
\label{chap:intro}
Dynamically typed languages such as Lua avoid static types in favor of
simplicity and flexibility, because the absence of static types means
that programmers do not need to bother about abstracting types that
should be validated by a type checker.
Instead, dynamically typed languages use run-time \emph{type tags}
to classify the values they compute, so their implementation can use
these tags to perform run-time (or dynamic) type checking
\citep{pierce2002tpl}.

This simplicity and flexibility allow programmers to write code that
might require a quite complex type system to statically type check,
though it may also hide bugs that will be caught only after deployment
if programmers do not properly test their code.
In contrast, static type checking helps programmers detect many
bugs during the development phase.
Static types also provide a conceptual framework that helps
programmers define modules and interfaces that can be combined to
structure the development of programs.

Thus, early error detection and better program structure are two
advantages of static type checking that can lead programmers to
migrate their code from a dynamically typed to a statically
typed language, when their simple scripts evolve into complex programs
\citep{tobin-hochstadt2006ims}.
Dynamically typed languages certainly help programmers during the
beginning of a project, because their simplicity and flexibility
allow quick development and make it easier to change code according to
changing requirements.
However, programmers tend to migrate from dynamically typed to
statically typed code as soon as the project has consolidated its
requirements, because the robustness of static types helps
programmers link requirements to abstractions.
This migration usually involves different languages that have
different syntax and semantics, which usually require a complete
rewrite of existing programs instead of incremental evolution from
dynamic to static types.

Ideally, programming languages should offer programmers the
option to choose between static and dynamic typing:
\emph{optional type systems} \citep{bracha2004pluggable} and
\emph{gradual typing} \citep{siek2006gradual} are two similar
approaches for blending static and dynamic typing in the same
language.
The aim of both approaches is to offer programmers the option
to use type annotations where static typing is needed and to not use
them where dynamic typing is sufficient.
The difference between these two approaches is the way they treat
run-time semantics.
While optional type systems do not affect the run-time semantics,
gradual typing uses run-time checks to ensure that dynamically typed
code does not violate the invariants of statically typed code.

Programmers and researchers sometimes use the term gradual typing
to mean the incremental evolution of dynamically typed code to
statically typed code.
For this reason, gradual typing may also refer to optional type
systems and other approaches that blend static and dynamic typing to
help programmers incrementally migrate from dynamic to static typing
without having to switch to a different language, though all these
approaches differ in the way they handle static and dynamic typing
together.
We use the term gradual typing for referring to the work of
\citet{siek2006gradual}.

In this work we present the design and evaluation of Typed Lua:
an optional type system for Lua that is rich enough to
preserve some of the Lua idioms that programmers are already used to,
but that also includes new constructs that help programmers
structure Lua programs.

Lua is a small scripting language, so designing an optional type
system for Lua may shed some light on optional type systems for
scripting languages in general.
Lua provides mechanisms for organizing a program in modules and
writing object-oriented programs, but it does not set policy on how
these features should behave, due to its use as an embedded and
extension language.
Thus, implementing an optional type system for Lua offers Lua
programmers one way to obtain most of the advantages of static typing
without compromising the simplicity and flexibility of dynamic typing.

So far, Typed Lua is a strict superset of Lua that
provides optional type annotations, compile-time type checking, and
class-based object-oriented programming through the definition of
classes, interfaces, and modules.
In practice, we implement Typed Lua as a programming language that
extends Lua syntax to add optional type annotations and standard
constructions for structuring Lua code.
The compiler uses static types to perform compile-time type
checking, but also allows Lua code to coexist with Typed Lua code,
and generates Lua code that runs in unmodified Lua implementations.
We have an implementation of the Typed Lua compiler that is
available online\footnote{https://github.com/andremm/typedlua}.

Typed Lua intended use is as an application language, and we view
that policies for organizing a program in modules and writing
object-oriented programs should be part of the language and
enforced by its optional type system.
An application language is a programming language that helps
programmers develop applications from scratch until these
applications evolve to complex systems rather than just scripts.

We also believe that Typed Lua help programmers give a more
formal documentation to already existing Lua code, as static types
are also a useful source of documentation in languages that provide
type annotations, because type annotations are always validated by
the type checker and therefore never get outdated.
Thus, programmers can use Typed Lua to define axioms about the
interfaces and types of dynamically typed modules.
We enforce this point by using Typed Lua to statically type
the interface of the Lua standard libraries and other commonly used
Lua libraries, so our compiler can check their usage by Typed Lua
code.

Typed Lua performs a very limited form of local type inference
\citep{pierce2000lti}, as static typing does not necessarily mean
that programmers need to insert type annotations in the code.
Several statically typed languages such as Haskell provide some
amount of type inference that automatically deduces the types of
expressions.
Still, Typed Lua needs a small amount of type annotations due
to the nature of its optional type system.

Typed Lua does not deal with code optimization, although another
important advantage of static types is that they help the compiler
perform optimizations and generate more efficient code.
However, we believe that the formalization of our optional type
system is precise enough to aid optimization in some Lua implementations.

We use some of the ideas of gradual typing to formalize Typed Lua.
Even though Typed Lua is an optional type system and thus does not
include run-time checks between dynamic and static regions of the
code, we believe that using the foundations of gradual typing to
formalize our optional type system will allow us to include run-time
checks in the future.

In Chapter \ref{chap:review} we review the literature about static and
dynamic typing in the same language, which includes the discussion
about the particularities between optional type systems and gradual
typing.
In Chapter \ref{chap:typedlua} we use code examples to present the
design of Typed Lua.
In Chapther \ref{chap:system} we use typing rules to present the
formalization of Typed Lua.
In Chapter \ref{chap:evaluation} we discuss some case studies that
we used to evaluate our design.
In Chapter \ref{chap:conc} we outline our contributions.



\chapter{Blending static and dynamic typing}
\label{chap:review}
We begin this chapter presenting a little bit of the history behind
combining static and dynamic typing in the same language.
Then, we introduce optional type systems and gradual typing.
After that, we discuss why optional type systems and two
other approaches are often called gradual typing.
We end this chapter presenting some statistics about the usage of
some Lua features and idioms that helped us identify how we should
combine static and dynamic typing in Lua.

\section{A little bit of history}
\label{sec:history}

Common LISP \cite{steele1982ocl} introduced optional type annotations
in the early eighties, but not for static type checking.
Instead, programmers could choose to declare types of variables as
optimization hints to the compiler, that is, type declarations are
just one way to help the compiler to optimize code.
These annotations are unsafe because they can crash the program
when they are wrong.

Abadi and associates \cite{abadi1989dts} extended the simply typed
lambda calculus with the \texttt{Dynamic} type and the \texttt{dynamic}
and \texttt{typecase} constructs, with the aim to safely integrate dynamic
code in statically typed languages.
The \texttt{Dynamic} type is a pair \texttt{(v,T)} where \texttt{v} is a
value and \texttt{T} is the tag that represents the type of \texttt{v}.
The constructs \texttt{dynamic} and \texttt{typecase} are explicit
injection and projection operations, respectively.
That is, \texttt{dynamic} builds values of type \texttt{Dynamic} and
\texttt{typecase} safely inspects the type of a \texttt{Dynamic} value.
Thus, migrating code between dynamic and static type checking requires
changing type annotations and adding or removing \texttt{dynamic} and
\texttt{typecase} constructs throughout the code.

The \emph{quasi-static} type system proposed by Thatte \cite{thatte1990qst}
performs implicit coercions and run-time checks to replace the
\texttt{dynamic} and \texttt{typecase} constructs that were proposed by
Abadi and associates \cite{abadi1989dts}.
To do that, quasi-static typing relies on subtyping with a top type
$\Omega$ that represents the dynamic type, and splits type checking
into two phases.
The first phase inserts implicit coercions from the dynamic type to
the expected type, while the second phase performs what Thatte calls
\emph{plausibility checking}, that is, it rewrites the program to
guarantee that sequences of upcasts and downcasts always have a
common subtype.

\emph{Soft typing} \cite{cartwright1991soft} is another approach
to combine static and dynamic typing in the same language.
The main goal of soft typing is to add static type checking to
dynamically typed languages without compromising their flexibility.
To do that, soft typing relies on type inference for
translating dynamically typed code to statically typed code.
The type checker inserts run-time checks around inconsistent code and
warns the programmer about the insertion of these run-time checks,
as they indicate the existence of potential type errors.
However, the programmer is free to choose between inspecting the
run-time checks or simply running the code.
This means that type inference and static type checking do
not prevent the programmer from running inconsistent code.
One advantage of soft typing is the fact that the compiler for
softly typed languages can use the translated code to generate
more efficient code, as the translated code statically type checks.
One disadvantage of soft typing is that it can be cumbersome when
the inferred types are meaningless large types that just confuse the
programmer.

\emph{Dynamic typing} \cite{henglein1994dts} is an approach
that optimizes code from dynamically typed languages by eliminating
unnecessary checks of tags.
Henglein describes how to translate dynamically typed code into
statically typed code that uses a \texttt{Dynamic} type.
The translation is done through a coercion calculus that uses type
inference to insert the operations that are necessary to type check
the \texttt{Dynamic} type during run-time.
Although soft typing and dynamic typing may seem similar, they are not.
Soft typing targets statically type checking of dynamically typed
languages for detecting programming errors, while
dynamic typing targets the optimization of dynamically
typed code through the elimination of unnecessary run-time checks.
In other words, soft typing sees code optimization as a side effect
that comes with static type checking.

Findler and Felleisen \cite{findler2002chf} proposed contracts for
higher-order functions and blame annotations for run-time checks.
Contracts perform dynamic type checking instead of static type checking,
but deferring all verifications to run-time can lead to defects
that are difficult to fix, because run-time errors can show a
stack trace where it is not clear to programmers if the cause of a
certain run-time error is in application code or library code.
Even if programmers identify that the source of a certain run-time
error is in library code, they still may have problems to identify
if this run-time error is due to a violation of library's contract
or due to a bug, when the library is poorly documented.
In this approach, programmers can insert assertions in the form of
contracts that check the input and output of higher-order functions;
and the compiler adds blame annotations in the generated code
to track assertion failures back to the source of the error.

BabyJ \cite{anderson2003babyj} is an object-oriented language
without inheritance that allows programmers to incrementally annotate
the code with more specific types.
Programmers can choose between using the dynamically typed version
of BabyJ when they do not need types at all, and the statically
typed version of BabyJ when they need to annotate the code.
In statically typed BabyJ, programmers can use the
\emph{permissive type} $*$ to annotate the parts of the code that
still do not have a specific type or the parts of the code that should
have dynamic behavior.
The type system of BabyJ is nominal, so types are either class names
or the permissive type $*$.
However, the type system does not use type equality or subtyping,
but the relation $\approx$ between two types.
The relation $\approx$ holds when both types have the same name or
any of them is the permissive type $*$.
Even though the permissive type $*$ is similar to the dynamic type
from previous approaches, BabyJ does not provide any way to add
implicit or explicit run-time checks.

Ou and associates \cite{ou2004dtd} specified a language that combines
static types with dependent types.
To ensure safety, the compiler automatically inserts coercions
between dependent code and static code.
The coercions are run-time checks that ensure static code does not
crash dependent code during run-time.

\section{Optional Type Systems}
\label{sec:optional}

Optional type systems \cite{bracha2004pluggable} are an approach for
plugging static typing in dynamically typed languages.
They use optional type annotations to perform compile-time type checking,
though they do not influence the original run-time semantics
of the language.
This means that the run-time semantics should still catch type errors
independently of the static type checking.
For instance, we can view the typed lambda calculus as an optional
type system for the untyped lambda calculus, because both have the
same semantic rules and the type system serves only for discarding
programs that may have undesired behaviors \cite{bracha2004pluggable}.

Strongtalk \cite{bracha1993strongtalk,bracha1996strongtalk} is
a version of Smalltalk that comes with an optional type system.
It has a polymorphic type system that programmers can use to annotate
Smalltalk code or leave type annotations out.
Strongtalk assigns a dynamic type to unannotated expressions and allows
programmers to cast unannotated expressions to any static type.
This means that the interaction of the dynamic type with the rest of
the type system is unsound, so Strongtalk uses the original run-time
semantics of Smalltalk when executing programs, even if programs are
statically typed.

\emph{Pluggable type systems} \cite{bracha2004pluggable} generalize
the idea of optional type systems that Strongtalk put in practice.
The idea is to have different optional type systems that can be layered
on top of a dynamically typed language without affecting its original
run-time semantics.
Although these systems can be unsound in their interaction with the
dynamically typed part of the language or even by design, their
unsoundness does not affect run-time safety, as the language run-time
semantics still catches any run-time errors caused by an unsound
type system.

Dart \cite{dart} and TypeScript \cite{typescript} are new
languages that are designed with an optional type system.
Both use JavaScript as their code generation target because
their main purpose is Web development.
In fact, Dart is a new class-based object-oriented language with
optional type annotations and semantics that resembles the
semantics of Smalltalk, while TypeScript is a strict superset of
JavaScript that provides optional type annotations and class-based
object-oriented programming.
Dart has a nominal type system, while TypeScript has a structural
one, but both are unsound by design.
For instance, Dart has covariant arrays, while TypeScript has
covariant parameter types in function signatures,
besides the interaction between statically and dynamically
typed code that is also unsound.

There is no common formalization for optional type systems, and
each language ends up implementing its optional type system in
its own way.
Strongtalk, Dart, and TypeScript provide an informal description of
their optional type systems rather than a formal one.
In the next section we will show that we can use some features
of gradual typing \cite{siek2006gradual,siek2007objects} to
formalize optional type systems.

\section{Gradual Typing}
\label{sec:gradual}

The main goal of gradual typing \cite{siek2006gradual} is to allow
programmers to choose between static and dynamic typing in the same
language.
To do that, Siek and Taha \cite{siek2006gradual} extended the simply
typed lambda calculus with the dynamic type $?$, as we can see in
Figure \ref{fig:gtlc}.
In gradual typing, type annotations are optional, and an untyped
variable is syntactic sugar for a variable whose declared type is
the dynamic type $?$, that is, $\lambda x.e$ is equivalent to $\lambda x{:}?.e$.
Under these circumstances, we view gradual typing as a way to add
a dynamic type to statically typed languages.

\begin{figure}[!ht]
\dstart
$$
\begin{array}{llr}
T ::= & & \textsc{types:}\\
& \;\; \Number & \textit{base type number}\\
& | \; \String & \textit{base type string}\\
& | \; ? & \textit{dynamic type}\\
& | \; T \rightarrow T & \textit{function types}\\
e ::= & & \textsc{expressions:}\\
& \;\; l & \textit{literals}\\
& | \; x & \textit{variables}\\
& | \; \lambda x{:}T{.}e & \textit{abstractions}\\
& | \; e_{1} e_{2} & \textit{application}
\end{array}
$$
\dend
\caption{Syntax of the gradually-typed lambda calculus}
\label{fig:gtlc}
\end{figure}

The central idea of gradual typing is the \emph{consistency}
relation, written $T_{1} \sim T_{2}$.
The consistency relation allows implicit conversions to and from the
dynamic type, and disallows conversions between inconsistent types
\cite{siek2006gradual}.
For instance, $\Number \sim \;?$, $? \sim \Number$,
$\String \sim \;?$, and $? \sim \String$,
but $\Number \not\sim \String$, and
$\String \not\sim \Number$.
The consistency relation is both reflexive and symmetric, but
it is neither commutative nor transitive.

\begin{figure}[!ht]
\dstart
$$
\begin{array}{c}
\begin{array}{c}
T \sim T \mylabel{C-REFL}
\end{array}
\;
\begin{array}{c}
T \sim \;? \mylabel{C-DYNR}
\end{array}
\;
\begin{array}{c}
? \sim T \mylabel{C-DYNL}
\end{array}
\\ \\
\begin{array}{c}
\dfrac{T_{3} \sim T_{1} \;\;\; T_{2} \sim T_{4}}
      {T_{1} \rightarrow T_{2} \sim T_{3} \rightarrow T_{4}} \mylabel{C-FUNC}
\end{array}
\end{array}
$$
\dend
\caption{The consistency relation}
\label{fig:consistency}
\end{figure}

Figure \ref{fig:consistency} defines the consistency relation.
The rule \textsc{C-REFL} is the reflexive rule.
Rules \textsc{C-DYNR} and \textsc{C-DYNL} are the rules that allow
conversions to and from the dynamic type $?$.
The rule \textsc{C-FUNC} resembles subtyping between function types,
because it is contravariant on the argument type and covariant on the
return type.

Figure \ref{fig:gts} uses the consistency relation in the typing rules
of the gradual type system of the simply typed lambda calculus extended
with the dynamic type $?$.
The environment $\env$ is a function from variables to types, and
the directive $type$ is a function from literal values to types.
The rule \textsc{T-VAR} uses the environment function $\env$ to get the
type of a variable $x$.
The rule \textsc{T-LIT} uses the directive $type$ to get the type of
a literal $l$.
The rule \textsc{T-ABS} evaluates the expression $e$ with an environment
$\env$ that binds the variable $x$ to the type $T_{1}$, and the resulting
type is the the function type $T_{1} \rightarrow T_{2}$.
The rule \textsc{T-APP1} handles function calls where the type of a
function is dynamically typed; in this case, the argument type may have
any type and the resulting type has the dynamic type.
The rule \textsc{T-APP2} handles function calls where the type of a
function is statically typed; in this case, the argument type should
be consistent with the argument type of the function's signature.

\begin{figure}[!ht]
\dstart
$$
\begin{matrix}
\dfrac{\env(x) = T}
      {\env \vdash x:T} \mylabel{T-VAR}
\;\;\;
\dfrac{type(l) = T}
      {\env \vdash l:T} \mylabel{T-LIT}
\\ \\
\dfrac{\env[x \mapsto T_{1}] \vdash e:T_{2}}
      {\env \vdash \lambda x:T_{1}.e:T_{1} \rightarrow T_{2}} \mylabel{T-ABS}
\;\;\;
\dfrac{\env \vdash e_{1}:\;? \;\;\;
       \env \vdash e_{2}:T}
      {\env \vdash e_{1} e_{2}:\;?} \mylabel{T-APP1}
\\ \\
\dfrac{\env \vdash e_{1}:T_{1} \rightarrow T_{2} \;\;\;
       \env \vdash e_{2}:T_{3} \;\;\;
       T_{3} \sim T_{1}}
      {\env \vdash e_{1} e_{2}:T_{2}} \mylabel{T-APP2}
\end{matrix}
$$
\dend
\caption{Gradual type system gradually-typed lambda calculus}
\label{fig:gts}
\end{figure}

Gradual typing is similar to the approaches proposed by
Abadi and associates \cite{abadi1989dts} and Thatte \cite{thatte1990qst}
by including a dynamic type in a statically typed language.
However, these three approaches differ in the way they handle the
dynamic type.
While Siek and Taha \cite{siek2006gradual} rely on the consistency relation,
Abadi and associates \cite{abadi1989dts} rely on type equality with explicit
projections and injections, and Thatte \cite{thatte1990qst} relies on subtyping.

The subtyping relation $\subtype$ is actually a pitfall on Thatte's
quasi-static typing, because it sets the dynamic type
as the top and the bottom of the subtying relation:
$T \subtype \;?$ and $? \subtype T$.
Subtyping is transitive, so we know that
\[
\frac{\Number \subtype \;? \;\;\;
      ? \subtype \String}
     {\Number \subtype \String}
\]
Therefore, downcasts combined with the transitivity of subtyping
accepts programs that should be rejected.

Later, Siek and Taha \cite{siek2007objects} reported that the consistency relation
is orthogonal to the subtyping relation, so we can combine them to achieve
the \emph{consistent-subtyping} relation, written $T_{1} \lesssim T_{2}$.
This relation is essential for designing gradual type systems for
object-oriented languages.
Like the consistency relation, and unlike the subtyping relation,
the consistent-subtyping relation is not transitive.
Therefore, $\Number \lesssim \;?$, $? \lesssim \Number$,
$\String \lesssim \;?$, and $? \lesssim \String$,
but $\Number \not\lesssim \String$, and
$\String \not\lesssim \Number$.

Now, we will show how we can combine consistency and subtyping
to compose a consistent-subtyping relation for the simply typed
lambda calculus extended with the dynamic type $?$.

\begin{figure}[!ht]
\dstart
$$
\begin{array}{c}
\begin{array}{c}
\Number \subtype \Number \mylabel{S-NUM}
\end{array}
\;
\begin{array}{c}
\String \subtype \String \mylabel{S-STR}
\end{array}
\\ \\
\begin{array}{c}
? \subtype \;? \mylabel{S-ANY}
\end{array}
\;
\begin{array}{c}
\dfrac{T_{3} \subtype T_{1} \;\;\; T_{2} \subtype T_{4}}
      {T_{1} \rightarrow T_{2} \subtype T_{3} \rightarrow T_{4}} \mylabel{S-FUN}
\end{array}
\end{array}
$$
\dend
\caption{The subtyping relation}
\label{fig:subtyping}
\end{figure}

Figure \ref{fig:subtyping} presents the subtyping relation for the simply
typed lambda calculus extended with the dynamic type $?$.
Even though we could have used the reflexive rule $T \subtype T$ to express
the rules \textsc{S-NUM}, \textsc{S-STR}, and \textsc{S-ANY},
we did not combine them into a single rule to make explicit the
neutrality of the dynamic type $?$ to the subtyping rules.
The dynamic type $?$ must be neutral to subtyping to avoid the pitfall
from Thatte's quasi-static typing.
The rule \textsc{S-FUN} defines the subtyping relation for function types,
which are contravariant on the argument type and covariant on the return type.

\begin{figure}[!ht]
\dstart
$$
\begin{array}{c}
\begin{array}{c}
\Number \lesssim \Number \mylabel{C-NUM}
\end{array}
\;
\begin{array}{c}
\String \lesssim \String \mylabel{C-STR}
\end{array}
\\ \\
\begin{array}{c}
T \lesssim \;? \mylabel{C-ANY1}
\end{array}
\;
\begin{array}{c}
? \lesssim T \mylabel{C-ANY2}
\end{array}
\\ \\
\begin{array}{c}
\dfrac{T_{3} \lesssim T_{1} \;\;\; T_{2} \lesssim T_{4}}
      {T_{1} \rightarrow T_{2} \lesssim T_{3} \rightarrow T_{4}} \mylabel{C-FUN}
\end{array}
\end{array}
$$
\dend
\caption{The consistent-subtyping relation}
\label{fig:consistent-subtyping}
\end{figure}

Figure \ref{fig:consistent-subtyping} combines the consistency and subtyping
relations to compose the consistent-subtyping relation for the simply typed
lambda calculus extended with the dynamic type $?$.
When we combine consistency and subtyping, we are making subtyping handle
what casts are safe among static types, and we are making consistency
handle the casts that involve the dynamic type $?$.
The consistent-subtyping relation is not transitive, and thus
the dynamic type $?$ is not neutral to this relation.

So far, gradual typing looks like a mere formalization to optional
type systems, as a gradual type system uses the consistency or
consistent-subtyping relation to statically check the interaction
between statically and dynamically typed code, without influencing
the run-time semantics.

However, another important feature of gradual typing is the theoretic
foundation that it provides for inserting run-time checks that
prove dynamically typed code does not violate the invariants of
statically typed code, thus preserving type safety.
To do that, Siek and Taha \cite{siek2006gradual,siek2007objects}
defined the run-time semantics of gradual typing as a translation to an
intermediate language with explicit casts at the frontiers between
statically and dynamically typed code.
The semantics of these casts is based on the higher-order contracts
proposed by Findler and Felleisen \cite{findler2002chf}.

Herman and associates \cite{herman2007sgt} showed that there is an
efficiency concern regarding the run-time checks, because there are
two ways that casts can lead to unbounded space consumption.
The first affects tail recursion while the second appears when
first-class functions or objects cross the border between
static code and dynamic code, that is, some programs can apply
repeated casts to the same function or object.
Herman and associates \cite{herman2007sgt} use the coercion calculus
outlined by Henglein \cite{henglein1994dts} to express casts
as coercions and solve the problem of space efficiency.
Their approach normalizes an arbitrary sequence of coercions to a
coercion of bounded size.

Another concern about casts is how to improve debugging support,
because a cast application can be delayed and the error related
to that cast application can appear considerable distance
from the real error.
Wadler and Findler \cite{wadler2009wpc} developed \emph{blame calculus}
as a way to handle this issue, and Ahmed and associates \cite{ahmed2011bfa}
extended blame calculus with polymorphism.
Blame calculus is an intermediate language to integrate
static and dynamic typing along with the blame tracking approach
proposed by Findler and Felleisen \cite{findler2002chf}.

On the one hand, blame calculus solves the issue regarding
error reporting;
on the other hand, it has the space efficiency problem reported
by Herman and associates \cite{herman2007sgt}.
Thus, Siek and associates \cite{siek2009casts} extended the coercion
calculus outlined by Herman and associates \cite{herman2007sgt} with
blame tracking to achieve an implementation of the blame calculus that
is space efficient.
After that, Siek and Wadler \cite{siek2010blame} proposed a new solution
that also handles both problems.
This new solution is based on a concept called \emph{threesome},
which is a way to split a cast between two parties into two casts
among three parties.
A cast has a source and a target type (a \emph{twosome}),
so we can split any cast into a downcast from the source to an
intermediate type that is followed by an upcast from the intermediate
type to the target type (a \emph{threesome}).

There are some projects that incorporate gradual typing into some
programming languages.
Reticulated Python \cite{reticulated,vitousek2014deg} is a research
project that evaluates the costs of gradual typing in Python.
Gradualtalk \cite{allende2013gts} is a gradually-typed Smalltalk
that introduces a new cast insertion strategy for gradually-typed
objects \cite{allende2013cis}.
Grace \cite{black2012grace,black2013sg} is a new object-oriented,
gradually-typed, educational language.
In Grace, modules are gradually-typed objects, that is, modules
may have types and methods as attributes, and can have a state
\cite{homer2013modules}.
ActionScript \cite{moock2007as3} is one the first languages that
incorporated gradual typing to its implementation and
Perl 6 \cite{tang2007pri} is also being designed with gradual typing,
though there is few documentation about the gradual type systems
of these languages.

\section{Approaches that are often called Gradual Typing}
\label{sec:approaches}

Gradual typing is similar to optional type systems in that type
annotations are optional, and unannotated code is dynamically
typed, but unlike optional type systems, gradual typing changes
the run-time semantics to preserve type safety, and it is a way to
add a dynamic type to statically typed languages.
More precisely, programming languages that include a gradual type
system implement the semantics of statically typed languages, so
the gradual type system inserts casts in the translated code to
guarantee that types are consistent before execution, while
programming languages that include an optional type system still
implement the semantics of dynamically typed languages, so all
the type checking also belongs to the semantics of each operation.

Still, we can view gradual typing as a way to formalize an optional
type system when the gradual type system does not insert run-time
checks.
BabyJ \cite{anderson2003babyj} and Alore \cite{lehtosalo2011alore}
are two examples of object-oriented languages that have an
optional type system with a formalization that relates to gradual typing,
though the optional type systems of both BabyJ and Alore are nominal.
BabyJ uses the relation $\approx$ that is similar to the consistency
relation while Alore combines subtyping along with the consistency
relation to define a \emph{consistent-or-subtype} relation.
The consistent-or-subtype relation is different from the
consistent-subtyping relation of Siek and Taha \cite{siek2007objects},
but it is also written $T_{1} \lesssim T_{2}$.
The consistent-or-subtype relation holds when $T_{1} \sim T_{2}$
or $T_{1} <: T_{2}$, where $<:$ is transitive and $\sim$ is not.
Alore also extends its optional type system to include optional
monitoring of run-time type errors in the gradual typing style.

Hence, optional type annotations for software evolution are likely
the reason why optional type systems are commonly called
gradual type systems.
Typed Clojure \cite{bonnaire-sergeant2012typed-clojure} is an
optional type system for Clojure that is now adopting the
gradual typing slogan.

Flanagan \cite{flanagan2006htc} introduced \emph{hybrid type checking},
an approach that combines static types and \emph{refinement} types.
For instance, programmers can specify the refinement type
$\{x:Int \;|\; x \ge 0\}$ when they need a type for natural numbers.
The programmer can also choose between explicit or implicit casts.
When casts are not explicit, the type checker uses a theorem prover
to insert casts.
In our example of natural numbers, a cast would be inserted to check
whether an integer is greater than or equal to zero.

Sage \cite{gronski2006sage} is a programming language that
extends hybrid type checking with a dynamic type to
support dynamic and static typing in the same language.
Sage also offers optional type annotations in the gradual typing
style, that is, unannotated code is syntactic sugar for
code whose declared type is the dynamic type.

Thus, the inclusion of a dynamic type in hybrid type checking
along with optional type annotations, and the insertion of run-time
checks are likely the reason why hybrid type checking is
also viewed as a form of gradual typing.

Tobin-Hochstadt and Felleisen \cite{tobin-hochstadt2006ims} proposed
another approach for gradually migrating from dynamically typed to
statically typed code, and they coined the term
\emph{from scripts to programs} for referring to this kind of
interlanguage migration.
In their approach, the migration from dynamically typed to
statically typed code happens module-by-module, so they designed
and implemented Typed Racket \cite{tobin-hochstadt2008ts} for
this purpose.
Typed Racket is a statically typed version of Racket
(a Scheme dialect) that allows the programmer to write typed modules,
so Typed Racket modules can coexist with Racket modules,
which are untyped.

The approach used by Tobin-Hochstadt and Felleisen \cite{tobin-hochstadt2008ts}
to design and implement Typed Racket is probably also called gradual typing
because it allows the programmer to gradually migrate from untyped
scripts to typed programs.
However, Typed Racket is a statically typed language,
and what makes it gradual is a type system with a dynamic type
that handles the interaction between Racket and Typed Racket modules.

\section{Statistics about the usage of Lua}
\label{sec:statistics}

In this section we present statistics about the usage of Lua
features and idioms.
We collected statistics about how programmers use tables, functions,
dynamic type checking, object-oriented programming, and modules.
We shall see that these statistics informed important design decisions
on our optional type system.

We used the LuaRocks repository to build our statistics database;
LuaRocks \cite{hisham2013luarocks} is a package manager for Lua
modules.
We downloaded the 3928 \texttt{.lua} files that were available in
the LuaRocks repository at February 1st 2014.
However, we ignored files that were not compatible with Lua 5.2,
the latest version of Lua at that time.
We also ignored \emph{machine-generated} files and test files,
because these files may not represent idiomatic Lua code,
and might skew our statistics towards non-typical uses of Lua.
This left 2598 \texttt{.lua} files from 262 different projects for
our statistics database;
we parsed these files and processed their abstract syntax tree
to gather the statistics that we show in this section.

To verify how programmers use tables, we measured how they
initialize, index, and iterate tables.
We present these statistics in the next three paragraphs to discuss
their influence on our type system.

The table constructor appears 23185 times.
In 36\% of the occurrences it is a constructor that initializes a
record (e.g., \texttt{\{ x = 120, y = 121 \}});
in 29\% of the occurrences it is a constructor that initializes a
list (e.g., \texttt{\{ "one", "two", "three", "four" \}});
in 8\% of the occurrences it is a constructor that initializes a
record with a list part;
and in less than 1\% of the occurrences (4 times) it is a constructor
that uses only the booleans \texttt{true} and \texttt{false} as indexes.
At all, in 73\% of the occurrences it is a constructor that uses
only literal keys;
in 26\% of the occurrences it is the empty constructor;
in 1\% of the occurrences it is a constructor with non-literal keys
only, that is, a constructor that uses variables and function calls
to create the indexes of a table;
and in less than 1\% of the occurrences (19 times) it is a constructor
that mixes literal keys and non-literal keys.

The indexing of tables appears 130448 times:
86\% of them are for reading a table field while
14\% of them are for writing into a table field.
We can classify the indexing operations that are reads as follows:
89\% of the reads use a literal string key,
4\% of the reads use a literal number key,
and less than 1\% of the reads (10 times) use a literal boolean key.
At all, 93\% of the reads use literals to index a table while
7\% of the reads use non-literal expressions to index a table.
It also worth mentioning that 45\% of the reads are actually
function calls.
More precisely, 25\% of the reads use literals to call a function,
20\% of the reads use literals to call a method, that is,
a function call that uses the colon syntactic sugar, 
and less than 1\% of the reads (195 times) use non-literal expressions
to call a function.
We can also classify the indexing operations that are writes as follows: 
69\% of the writes use a literal string key,
2\% of the writes use a literal number key,
and less than 1\% of the writes (1 time) uses a literal boolean key.
At all, 71\% of the writes use literals to index a table while
29\% of the writes use non-literal expressions to index a table.

We also measured how many files have code that iterate over tables to
observe how frequently iteration is used.
We observed that 23\% of the files have code that iterate over keys
of any value, that is, the call to \texttt{pairs} appears at least
once in these files (the median is twice per file);
21\% of the files have code that iterate over integer keys, that is,
the call to \texttt{ipairs} appears at least once in these files
(the median is also twice per file);
and 10\% of the files have code that use the numeric \texttt{for}
along with the length operator (the median is once per file).

The numbers about table initialization, indexing, and iteration
show us that tables are mostly used to represent records, lists,
and associative arrays.
Therefore, Typed Lua should include a table type for handling
these uses of Lua tables.
Even though the statistics show that programmers initialize tables
more often than they use the empty constructor to
dynamically initialize tables, the statistics of the empty
constructor are still expressive and indicate that Typed Lua should
also include a way to handle this style of defining table types.

We found a total of 24858 function declarations in our database
(the median is six per file).
Next, we discuss how frequently programmers use dynamic type
checking and multiple return values inside these functions.

We observed that 9\% of the functions perform dynamic type checking
on their input parameters, that is, these functions use \texttt{type}
to inspect the tags of Lua values (the median is once per function).
We randomly selected 20 functions to sample how programmers are
using \texttt{type}, and we got the following data:
50\% of these functions use \texttt{type} for asserting the tags of
their input parameters, that is, they raise an error when the tag of a
certain parameter does not match the expected tag, and
50\% of these functions use \texttt{type} for overloading, that is,
they execute different code according to the inspected tag.

These numbers show us that Typed Lua should include union types,
because the use of the \texttt{type} idiom shows that disjoint unions
would help programmers define data structures that can hold a value of
several different, but fixed types.
Typed Lua should also use \texttt{type} as a mechanism for decomposing
unions, though it may be restricted to base types only.

We observed that 10\% of the functions explicitly return multiple
values.
We also observed that 5\% of the functions return \texttt{nil} plus
something else, for signaling an unexpected behavior;
and 1\% of the functions return \texttt{false} plus something else,
also for signaling an unexpected behavior.

Typed Lua should include function types to represent Lua functions,
and tuple types to represent the signatures of Lua functions,
multiple return values, and multiple assignments.
Tuple types require some special attention, because Typed Lua
should be able to adjust tuple types during compile-time, in a
similar way to what Lua does with function calls and multiple
assignments during run-time.
In addition, the number of functions that return \texttt{nil} and
\texttt{false} plus something else show us that overloading on the
return type is also useful to the type system.

We also measured how frequently programmers use the object-oriented
paradigm in Lua.
We observed that 23\% of the function declarations are actually
method declarations.
More precisely, 14\% of them use the colon syntactic sugar while
9\% of them use \texttt{self} as their first parameter.
We also observed that 63\% of the projects extend tables with
metatables, that is, they call \texttt{setmetatable} at least once,
and 27\% of the projects access the metatable of a given table,
that is, they call \texttt{getmetatable} at least once.
In fact, 45\% of the projects extend tables with metatables and
declare methods:
13\% using the colon syntactic sugar, 14\% using \texttt{self}, and
18\% using both.

Based on these observations, Typed Lua should include support
to object-oriented programming.
Even though Lua does not have standard policies for object-oriented
programming, it provides mechanisms that allow programmers to
abstract their code in terms of objects, and our statistics confirm
that an expressive number of programmers are relying on these mechanisms
to use the object-oriented paradigm in Lua.
Typed Lua should include some standard way of defining interfaces and classes
that the compiler can use to type check object-oriented code,
but without changing the semantics of Lua.

We also measured how programmers are defining modules.
We observed that 38\% of the files use the current way of defining
modules, that is, these files return a table that contains the
exported members of the module at the end of the file;
22\% of the files still use the deprecated way of defining modules,
that is, these files call the function \texttt{module};
and 1\% of the files use both ways.
At all, 61\% of the files are modules while 39\% of the files are
plain scripts.
The number of plain scripts is high considering the origin of
our database.
However, we did not ignore sample scripts, which usually serve to
help the users of a given module on how to use this module, and
that is the reason why we have a high number of plain scripts.

Based on these observations, Typed Lua should include a way
for defining table types that represent the type of modules.
Typed Lua should also support the deprecated style of module
definition, using global names as exported members of the module.

Typed Lua should also include some way to define the types of
userdata.
This feature should also allow programmers to define userdata
that can be used in an object-oriented style, as this is another
common idiom from modules that are written in C.

The last statistics that we collected were about variadic functions
and vararg expressions.
We observed that 8\% of the functions are variadic, that is,
their last parameter is the vararg expression.
We also observed that 5\% of the initialization of lists
(or 2\% of the occurrences of the table constructor) use solely the
vararg expression.
Typed Lua should include a \emph{vararg type} to handle variadic
functions and vararg expressions.




\chapter{The design \& implementation of Typed Lua}
\label{chap:typedlua}
\documentclass[mathserif]{beamer}

\usepackage[english]{babel}
\usepackage[utf8]{inputenc}
\usepackage{amsmath}
\usepackage{amssymb}
\usepackage{beamerthemesplit}
%\usecolortheme{dove}

\newcommand{\mylabel}[1]{\; (\textsc{#1})}
\newcommand{\subtype}{<:}
\newcommand{\pipe}{|\;}
\newcommand{\kw}[1]{\mathbf{#1} \;}
\newcommand{\env}{\Gamma}

\begin{document}

\title{Typed Lua}
\subtitle{An Optional Type System for Lua}
\author{André Murbach Maidl}
\institute{Colóquios do LabLua}
\date{September 13th 2013}

\frame{\titlepage}

\begin{frame}
\frametitle{What is Typed Lua?}
\begin{itemize}
\item A typed superset of Lua that compiles to plain Lua.
\item \textcolor{blue}{Type annotations}.
\item \textcolor{blue}{Compile-time type checking}.
\item \textcolor{gray}{Classes}.
\item \textcolor{gray}{Interfaces}.
\item \textcolor{gray}{Modules}.
\end{itemize}
\end{frame}

\begin{frame}
\frametitle{Why Optional and not Gradual?}
\begin{itemize}
\item Because, first of all, gradual is optional!
\end{itemize}
\end{frame}

\begin{frame}
\frametitle{Gradual Typing}
Gradual type systems use consistency ($\sim$) instead of equality ($=$).
\begin{Large}
\[
\tau ::= \gamma \;|\; ? \;|\; \tau \rightarrow \tau
\]
\[
\tau \sim \tau
\]
\[
\tau \sim \;?
\]
\[
?\; \sim \tau
\]

\[
\frac{\sigma_{1} \sim \tau_{1} \;\;\; \sigma_{2} \sim \tau_{2}}
     {\sigma_{1} \rightarrow \sigma_{2} \sim \tau_{1} \rightarrow \tau_{2}}
\]
\end{Large}
\end{frame}

\begin{frame}
\frametitle{Gradual Typing}
Gradual type systems use consistency (\textcolor{blue}{$\sim$}) instead of equality ($=$).
\begin{Large}
\[
e ::= c \;|\; x \;|\; \lambda x:\tau.e \;|\; e\;e \;\;\;
(\lambda x.e \equiv \lambda x:?.e)
\]
\[
\frac{\Gamma x = \lfloor\tau\rfloor}
     {\Gamma \vdash_{G} x:\tau} \;\;\;\;\;
\frac{\Delta c = \tau}
     {\Gamma \vdash_{G} c:\tau}
\]

\[
\frac{\Gamma(x \mapsto \sigma) \vdash_{G} e:\tau}
     {\Gamma \vdash_{G} \lambda x:\sigma.e:\sigma \rightarrow \tau} \;\;\;\;\;
\frac{\Gamma \vdash_{G} e_{1}:\;? \;\;\; \Gamma \vdash_{G} e_{2}:\tau_{2}}
     {\Gamma \vdash_{G} e_{1}\;e_{2}:\;?}
\]

\[
\frac{\Gamma \vdash_{G} e_{1}:\tau \rightarrow \tau' \;\;\;
      \Gamma \vdash_{G} e_{2}:\tau_{2} \;\;\; \textcolor{blue}{\tau_{2} \sim \tau}}
     {\Gamma \vdash_{G} e_{1}\;e_{2}:\tau'}
\]
\end{Large}
\end{frame}

\begin{frame}
\frametitle{Optional versus Gradual}
\begin{center}
\begin{tabular}{|r|c|c|}
\hline
& Optional & Gradual\\
\hline
Optional type annotations & Yes & Yes \\ 
\hline
Compile-time type checking & Yes & Yes \\
\hline
Influence the run-time semantics & No & Yes \\
\hline
\end{tabular}
\end{center}
\end{frame}

\begin{frame}
\frametitle{The levels of Gradual Typing according to Jeremy Siek}
\begin{center}
\begin{tabular}{|r|c|c|c|}
\hline
& Level 1 & Level 2 & Level 3\\
\hline
Optional type annotations & Yes & Yes & Yes \\ 
\hline
Compile-time type checking & Yes & Yes & Yes \\
\hline
Run-time checking$^{1}$ & No & Yes & Yes \\
\hline
Blame tracking$^{2}$ & No & No & Yes\\
\hline
\end{tabular}
\end{center}
$1$ -- A compiler for a gradually typed language (level $>$ 1) infers
where dynamic checks are needed and inserts casts into the intermediate
language to performe these checks.\\
$2$ -- Blame tracking solves the problem of tracing a run-time cast
failure back to the source of the error.
\end{frame}

\begin{frame}
\frametitle{Examples of Gradually Typed Languages}
\begin{enumerate}
\item Strongtalk, TypeScript, and Typed Lua.
\item ActionScript.
\item Typed Racket.
\end{enumerate}
\end{frame}

\begin{frame}
\frametitle{Overview of rest of the presentation}
\begin{itemize}
\item Changes in the syntax of Lua.
\item Typed Lua Type System.
\end{itemize}
\end{frame}

\begin{frame}
\frametitle{Changes in the syntax of Lua}
\begin{align*}
stat ::= & \; ... \; |\\
& \textcolor{blue}{varlist} \; \texttt{`='} \; explist \; |\\
& \; ... \; |\\
& \textbf{function} \; funcname \; \textcolor{blue}{funcbody} \; |\\
& \textbf{local} \; \textbf{function} \; Name \; \textcolor{blue}{funcbody} \; |\\
& \textbf{local} \; \textcolor{blue}{namelist} \; [\texttt{`='} \; explist]\\
exp ::= & \; ... \; |\\
& functiondef \; |\\
& \; ...\\
functiondef ::= & \; \textbf{function} \; \textcolor{blue}{funcbody}
\end{align*}
\end{frame}

\begin{frame}
\frametitle{Declaration of variables}
\begin{align*}
stat ::= & \; ... \; |\\
& \textcolor{blue}{varlist} \; \texttt{`='} \; explist \; |\\
& \; ... \; |\\
& \textbf{local} \; \textcolor{blue}{namelist} \; [\texttt{`='} \; explist]\\
varlist ::= & \; \textcolor{blue}{var} \; \{\texttt{`,'} \; \textcolor{blue}{var}\}\\
var ::= & \; Name \; \textcolor{blue}{[\texttt{`:'} \; type]} \; |\\
& prefixexp \; \texttt{`['} \; exp \; \texttt{`]'} \; |\\
& prefixexp \; \texttt{`.'} \; Name\\
namelist ::= & \; Name \; \textcolor{blue}{[\texttt{`:'} \; type]} \;
\{\texttt{`,'} \; Name \; \textcolor{blue}{[\texttt{`:'} \; type]}\}
\end{align*}
\end{frame}

\begin{frame}
\frametitle{Declaration of functions}
\begin{align*}
stat ::= & \; ... \; |\\
& \textbf{function} \; funcname \; \textcolor{blue}{funcbody} \; |\\
& \textbf{local} \; \textbf{function} \; Name \; \textcolor{blue}{funcbody} \; |\\
& \; ... \\
exp ::= & \; ... \; |\\
& functiondef \; |\\
& \; ...\\
functiondef ::= & \; \textbf{function} \; \textcolor{blue}{funcbody}\\
funcbody ::= & \; \texttt{`('} \; [\textcolor{blue}{parlist}] \; \texttt{`)'} \;
\textcolor{blue}{[\texttt{`:'} \; type]} \; block \; \textbf{end}\\
parlist ::= & \; \textcolor{blue}{namelist} \; [\texttt{`,'} \; \texttt{`...'} \;
\textcolor{blue}{[\texttt{`:'} \; type]}] \; | \;
\texttt{`...'} \; \textcolor{blue}{[\texttt{`:'} \; type]}\\
namelist ::= & \; Name \; \textcolor{blue}{[\texttt{`:'} \; type]} \;
\{\texttt{`,'} \; Name \; \textcolor{blue}{[\texttt{`:'} \; type]}\}
\end{align*}
\end{frame}

\begin{frame}
\frametitle{Type annotations}
\begin{align*}
type ::= & \; \textbf{object} \;|\; \textbf{any} \;|\; \textbf{nil} \;|\\
& basetype \;|\; uniontype \;|\; functiontype\\
basetype ::= & \; \textbf{boolean} \;|\; \textbf{number} \;|\; \textbf{string}\\
uniontype ::= & \; type \;\texttt{`|'}\; type\\
functiontype ::= & \; \texttt{`('} \; [2ndclasstype] \; \texttt{`)'} \;
\texttt{`->'} \; 2ndclasstype\\
2ndclasstype ::= & \; type \; \{\texttt{`,'} \; type\} \; [\texttt{`*'}]
\end{align*}
\end{frame}

\begin{frame}
\frametitle{Example}
\textit{Pensar em um exemplo legal para colocar aqui.}
\end{frame}

\begin{frame}
\frametitle{Typed Lua Type System}
\begin{itemize}
\item Our aim is to design an object oriented language.
\item Most OO languages use the following subsumption rule:
\[
\frac{\env \vdash e:T_{1} \;\;\; T_{1} \subtype T_{2}}
     {\env \vdash e:T_{2}}
\]
\item We also aim to have gradual typing, so we also
need the consistency relation.
\end{itemize}
\end{frame}

\begin{frame}
\frametitle{Typed Lua Type System}
We do not use the subsumption rule, but subtyping
instead of equality.
\[
\frac{\env \vdash e:S_{1} \rightarrow S_{2} \;\;\;
      \env \vdash el:S_{3} \;\;\;
      \textcolor{red}{S_{3} \subtype S_{1}}}
     {\env \vdash e(el):S_{2}}
\]
and also the consistency rule
\[
\frac{\env \vdash e:S_{1} \rightarrow S_{2} \;\;\;
      \env \vdash el:S_{3} \;\;\;
      \textcolor{red}{S_{3} \subtype S_{1}'} \;\;\;
      \textcolor{blue}{S_{1}' \sim S_{1}}}
     {\env \vdash e(el):S_{2}}
\]
so we can compose the two relations:
\[
\frac{\env \vdash e:S_{1} \rightarrow S_{2} \;\;\;
      \env \vdash el:S_{3} \;\;\;
      \textcolor{red!50!blue}{S_{3} \lesssim S_{1}}}
     {\env \vdash e(el):S_{2}}
\]
which we call consistent-subtyping.
\end{frame}

\begin{frame}
\frametitle{Type language}
\begin{align*}
T ::= \; & C \; \pipe B \; \pipe Object \; \pipe Any \; \pipe
S \rightarrow S \; \pipe T \cup T\\
C ::= \; & \mathbf{nil} \; \pipe \mathbf{false} \; \pipe \mathbf{true} \;
\pipe {<}double{>} \; \pipe {<}integer{>} \; \pipe {<}string{>}\\
B ::= \; & Boolean \; \pipe Number \; \pipe String\\
S ::= \; & T \; \pipe {T*} \; \pipe T \times S\\ 
\end{align*}
\end{frame}

\begin{frame}
\frametitle{Typing rules}
\begin{itemize}
\item Type consistency.
\item Subtyping.
\item Typed Lua typing rules.
\end{itemize}
\end{frame}

\begin{frame}
\textit{Colocar as regras do sistema de tipos.}
\end{frame}

\end{document}


\chapter{Case studies}
\label{chap:cases}

\section{Lua standard libraries}

\subsection{Basic functions}

\begin{enumerate}
\item assert (v [, message]) :
$\Top \times
(\String \cup \Nil) \times
\TopStar \rightarrow
\NilStar \cup
(\Top \times
(\String \cup \Nil) \times
\TopStar)$ 
\\
When \textbf{v} is $\Nil$ or $\False$, \textbf{assert}
shows the error \textbf{message}.
Otherwise, \textbf{assert} returns \textbf{v} and all the extra
parameters.
\item collectgarbage ([opt [, arg]]) :
$(\String \cup \Nil) \times
(\Number \cup \Nil) \times
\TopStar \rightarrow
(\Number \times \Nil \times \NilStar) \cup
(\Number \times \Number \times \NilStar) \cup
(\Boolean \times \Nil \times \NilStar)$
\\
The return type depends on the type of \textbf{opt}:
\begin{itemize}
\item $\Nil \times \TopStar \rightarrow \Number \times \NilStar$
\item $``collect" \times \TopStar \rightarrow \Number \times \NilStar$
\item $``stop" \times \TopStar \rightarrow \Number \times \NilStar$
\item $``restart" \times \TopStar \rightarrow \Number \times \NilStar$
\item $``count" \times \TopStar \rightarrow \Number \times \Number \times \NilStar$
\item $``step" \times \TopStar \rightarrow \Boolean \times \NilStar$
\item $``setpause" \times \TopStar \rightarrow \Number \times \NilStar$
\item $``setstepmul" \times \TopStar \rightarrow \Number \times \NilStar$
\item $``isrunning" \times \TopStar \rightarrow \Boolean \times \NilStar$
\item $``generational" \times \TopStar \rightarrow \Number \times \NilStar$
\item $``incremental" \times \TopStar \rightarrow \Number \times \NilStar$
\end{itemize}
\item dofile ([filename]) :
$(\String \cup \Nil) \times
\TopStar \rightarrow
\AnyStar
$
\item error (message [, level]) :
$\String \times
(0 \cup 1 \cup 2 \cup \Nil) \times
\TopStar \rightarrow
\NilStar$
\item \_G : $\{ \String:\Top \}$
\item getmetatable (object) :
$\mu X.\{ \_\_metatable:\Top \} \times
\TopStar \rightarrow
(\Nil \cup \{ \_\_metatable:\Top \}) \times
\NilStar$
\item ipairs (t) :
$\{ \Number:\Any \} \times
\TopStar \rightarrow
\Number \times
\Any \times
\NilStar$
\begin{itemize}
\item What happens when \textbf{t} has a metamethod \textbf{\_\_ipairs}?
\item Will $\{ ``tag":\String,\; \Number:\Any \}$ type check?
\item In fact, \textbf{ipairs} and \textbf{pairs} return three values.
Need to check this.
\end{itemize}
\item load (ld [, source [, mode [, env]]]) :
$(\String \cup (\TopStar \rightarrow \TopStar)) \times
(\String \cup \Nil) \times
(``b" \cup ``t" \cup ``bt" \cup \Nil) \times
(\Top \cup \Nil) \times
\TopStar \rightarrow
((\TopStar \rightarrow \TopStar) \times \NilStar) \cup
(\Nil \times \String \times \NilStar)$
\item loadfile ([filename [, mode [, env]]]) :
$(\String \cup \Nil) \times
(``b" \cup ``t" \cup ``bt" \cup \Nil) \times
(\Top \cup \Nil) \times
\TopStar \rightarrow
((\TopStar \rightarrow \TopStar) \times \NilStar) \cup
(\Nil \times \String \times \NilStar)$
\item next (table [, index]) :
$\{ \Any:\Any \} \times
(\Any \cup \Nil) \times
\TopStar \rightarrow
(\Nil \times \NilStar) \cup
(\Any \times \Any \times \NilStar)$
\item pairs (t) :
$\{ \Any:\Any \} \times
\TopStar \rightarrow
\Any \times 
\Any \times
\NilStar$
\item pcall (f [, arg1, ···]) :
$(\TopStar \rightarrow \TopStar) \times
\TopStar \rightarrow
(\Boolean \times \AnyStar) \cup
(\Boolean \times \String \times \NilStar)$
\item print (···) : $\TopStar \rightarrow \NilStar$
\item rawequal (v1, v2) :
$\Top \times
\Top \times
\TopStar \rightarrow
\Boolean \times
\NilStar$
\item rawget (table, index) :
$\{ \Top:\Top \} \times
\Top \times
\TopStar \rightarrow
\Any \times
\NilStar$
\item rawlen (v) :
$(\{ \Top:\Top \} \cup \String) \times
\TopStar \rightarrow
\Number \times
\NilStar$
\item rawset (table, index, value) :
$\{ \Top:\Top \} \times
\Top \times
\Top \times
\TopStar \rightarrow
\{ \Any:\Any \} \times
\NilStar$
\\
In fact, \textbf{index} cannot be $\Nil$ or \texttt{NaN}.
\item select (index, ···) :
$\Number \times
\AnyStar \rightarrow
\AnyStar$
\\
When \textbf{index} is the string ``\texttt{\#}", the typing is:
$\String \times \AnyStar \rightarrow \Number \times \NilStar$
\item setmetatable (table, metatable) :
$\{ \Top:\Top \} \times
(\Nil \cup \{ \_\_index:\mu X.\{ \Top:\Top \} \}) \times
\TopStar \rightarrow
\{ \_\_index:\mu X.\{ \Top:\Top \} \} \times
\NilStar$
\\
Even though \textbf{metatable} can be $\Nil$, it is not optional.
\item tonumber (e [, base]) :
$\String \times
(\Number \cup \Nil) \times
\TopStar \rightarrow
\Number \times
\NilStar$
\item tostring (v) :
$\Top \times
\TopStar \rightarrow
\String \times
\NilStar$
\item type (v) :
$\Top \times
\TopStar \rightarrow
(``nil" \cup
``number" \cup
``string" \cup
``boolean" \cup
``table" \cup
``function" \cup
``thread" \cup
``userdata") \times
\NilStar$
\item \_VERSION : $\String$
\item xpcall (f, msgh [, arg1, ···]) :
$(\TopStar \rightarrow \TopStar) \times
\Top \times
\TopStar \rightarrow
(\Boolean \times \AnyStar) \cup
(\Boolean \times \String \times \NilStar)$
\end{enumerate}

\subsection{Coroutine manipulation}

\subsection{Modules}

\subsection{String manipulation}

\begin{enumerate}
\item string.byte (s [, i [, j]]) :
$\String \times
(\Number \cup \Nil) \times
(\Number \cup \Nil) \times
\TopStar \rightarrow
\Number \times
\Number{*}$
\item string.char (···) :
$\Number{*} \rightarrow
\String \times
\NilStar$
\item string.dump (function) :
$(\TopStar \rightarrow \TopStar) \times
\TopStar \rightarrow
\String \times
\NilStar$
\item string.find (s, pattern [, init [, plain]]) :
$\String \times
\String \times
(\Number \cup \Nil) \times
(\Boolean \cup \Nil) \times
\TopStar \rightarrow
(\Number \times \Number \times \NilStar) \cup
(\Nil \times \NilStar)$
\item string.format (formatstring, ···) :
$\String \times
\TopStar \rightarrow
\String \times
\NilStar$
\item string.gmatch (s, pattern) :
$\String \times
\String \times
\TopStar \rightarrow
\String{*}$
\\
This function actually returns an iterator,
so I still need to confirm its return type.
\item string.gsub (s, pattern, repl [, n]) :
$\String \times
\String \times
(\String \cup \{ \Top:\Top \} \cup (\TopStar \rightarrow \TopStar)) \times
(\Number \cup \Nil) \times
\TopStar \rightarrow
\String \times
\Number \times
\NilStar$
\item string.len (s) :
$\String \times
\TopStar \rightarrow
\Number \times
\NilStar$
\item string.lower (s) :
$\String \times
\TopStar \rightarrow
\String \times
\NilStar$
\item string.match (s, pattern [, init]) :
$\String \times
\String \times
(\Number \cup \Nil) \times
\TopStar \rightarrow
(\String \times \String{*}) \cup
(\Nil \times \NilStar)$
\item string.rep (s, n [, sep]) :
$\String \times
\Number \times
(\String \cup \Nil) \times
\TopStar \rightarrow
\String \times
\NilStar$
\item string.reverse (s) :
$\String \times
\TopStar \rightarrow
\String \times
\NilStar$
\item string.sub (s, i [, j]) :
$\String \times
\Number \times
(\Number \cup \Nil) \times
\TopStar \rightarrow
\String \times
\NilStar$
\item string.upper (s) :
$\String \times
\TopStar \rightarrow
\String \times
\NilStar$
\end{enumerate}

\subsection{Table manipulation}

\begin{enumerate}
\item table.concat (list [, sep [, i [, j]]]) :
$\{ \Number:\String \cup \Number \} \times
(\String \cup \Nil) \times
(\Number \cup \Nil) \times
(\Number \cup \Nil) \times
\TopStar \rightarrow
\String \times
\NilStar$
\item table.insert (list, [pos,] value) :
$\{ \Number:\Top \} \times
(\Number \times \Top) \cup
(\Top \times \Nil) \times
\TopStar \rightarrow
\NilStar$
\\
This function does not accept/discard extra parameters.
Perhaps due to the fact that the parameter \textbf{pos} is optional.
If \textbf{value} is $\TopStar$, the function inserts \textbf{pos}
in the \textbf{list}, that is, the input type is
$\{ \Number:\Top \} \times \Top \times \TopStar$.
If \textbf{value} is $\Nil$, the function does not insert \textbf{pos}
in the \textbf{list}, that is, the input type is
$\{ \Number:\Top \} \times \Number \times \Top \times \TopStar$,
but the element is not inserted in the list because $\Top$ is actually $\Nil$.
\item table.pack (···) :
$\TopStar \rightarrow
\{ \Number:\Top \} \times
\NilStar$
\\
Even though \textbf{table.pack} gets any number of parameters of
any type, the $\Nil$ type generates a hole in the returned table.
\item table.remove (list [, pos]) :
$\{ \Number:\Top \} \times
(\Number \cup \Nil) \times
\TopStar \rightarrow
\Top \times
\NilStar
$
\\
The return type depends on the type of the element that is removed
from the list.
\item table.sort (list [, comp]) :
$\{ \Number:\Top \} \times
((\TopStar \rightarrow \TopStar) \cup \Nil) \times
\TopStar \rightarrow
\NilStar$
\item table.unpack (list [, i [, j]]) :
$\{ \Number:\Top \} \times
(\Number \cup \Nil) \times
(\Number \cup \Nil) \times
\TopStar \rightarrow
\TopStar$
\\
The return type depends on the type of the elements that are stored in
the list.
\end{enumerate}

\subsection{Mathematical functions}

\begin{enumerate}
\item math.abs (x) :
$\Number \times \TopStar \rightarrow \Number \times \NilStar$
\item math.acos (x) :
$\Number \times \TopStar \rightarrow \Number \times \NilStar$
\item math.asin (x) :
$\Number \times \TopStar \rightarrow \Number \times \NilStar$
\item math.atan (x) :
$\Number \times \TopStar \rightarrow \Number \times \NilStar$
\item math.atan2 (y, x) :
$\Number \times \Number \times \TopStar \rightarrow \Number \times \NilStar$
\item math.ceil (x) :
$\Number \times \TopStar \rightarrow \Number \times \NilStar$
\item math.cos (x) :
$\Number \times \TopStar \rightarrow \Number \times \NilStar$
\item math.cosh (x) :
$\Number \times \TopStar \rightarrow \Number \times \NilStar$  
\item math.deg (x) :
$\Number \times \TopStar \rightarrow \Number \times \NilStar$
\item math.exp (x) :
$\Number \times \TopStar \rightarrow \Number \times \NilStar$
\item math.floor (x) :
$\Number \times \TopStar \rightarrow \Number \times \NilStar$
\item math.fmod (x, y) :
$\Number \times \Number \times \TopStar \rightarrow \Number \times \NilStar$
\item math.frexp (x) :
$\Number \times \TopStar \rightarrow \Number \times \Number \times \NilStar$
\item math.huge : $\Number$
\item math.ldexp (m, e) :
$\Number \times \Number \times \TopStar \rightarrow \Number \times \NilStar$
\item math.log (x [, base]) :
$\Number \times
(\Number \cup \Nil) \times
\TopStar \rightarrow
\Number \times
\NilStar$
\item math.max (x, ···) :
$\Number \times \Number{*} \rightarrow \Number \times \NilStar$
\item math.min (x, ···) :
$\Number \times \Number{*} \rightarrow \Number \times \NilStar$
\item math.modf (x) :
$\Number \times \TopStar \rightarrow \Number \times \Number \times \NilStar$
\item math.pi : $\Number$
\item math.pow (x, y) :
$\Number \times \Number \times \TopStar \rightarrow \Number \times \NilStar$
\item math.rad (x) :
$\Number \times \TopStar \rightarrow \Number \times \NilStar$
\item math.random ([m [, n]]) :
$(\Number \cup \Nil) \times
(\Number \cup \Nil) \rightarrow
\Number \times
\NilStar$
\\
This function does not accept $\Nil$ as its only parameter,
although the parameter passing is optional.
This function does not accept/discard extra parameters.
\item math.randomseed (x) :
$\Number \times \TopStar \rightarrow \NilStar$
\item math.sin (x) :
$\Number \times \TopStar \rightarrow \Number \times \NilStar$
\item math.sinh (x) :
$\Number \times \TopStar \rightarrow \Number \times \NilStar$
\item math.sqrt (x) :
$\Number \times \TopStar \rightarrow \Number \times \NilStar$
\item math.tan (x) :
$\Number \times \TopStar \rightarrow \Number \times \NilStar$
\item math.tanh (x) :
$\Number \times \TopStar \rightarrow \Number \times \NilStar$
\end{enumerate}

\subsection{Bitwise operations}

\subsection{Input and Output facilities}

\subsection{Operating System facilities}

\subsection{The debug library}


\chapter{Conclusions}
\label{chap:conc}
In this work we presented Typed Lua, an optional type system for Lua.
We implemented Typed Lua as a Lua extension that allows programmers to
combine static and dynamic typing in Lua code, making easier the evolution
of simple scripts into large programs.

Our main contribution is the formalization of a complete optional type
system that introduces several novel type system features to statically
type check Lua programs.
Even though Lua shares several features with other dynamically
typed languages such as JavaScript, Lua also has several unusual features.
These unusual features include tables (or associative arrays) as the sole
mechanism for structured data, besides functions with multiple return values
and flexible arity that interact with multiple assignment.
We highlight the following novel features of our type system:
\begin{itemize}
\item type refinement allows the incremental evolution of record and
object types, playing an important role in statically type checking
the idiomatic way in which Lua programmers use tables to define modules
and objects;
\item projection types handle functions that are overloaded on the
number and types of return values, allowing programmers to narrow the
types of a set of variables by narrowing the type of a single component
of this set;
\item union types and variadic types help our type system handle
functions with flexible arity, that is, union types are helpful in
describing optional parameters while variadic types are helpful in
describing the type of the vararg expression and the type of functions
that can receive or return any number of values.
\end{itemize}

A key feature in optional type systems is usability.
This means that optional type systems should not change the idioms
that programmers are already familiar with.
Instead, optional type systems should fit existing idioms to
statically type check them.
Designing a too simple type system can overload programmers by forcing
them to change the way they program in the language to fit the type system,
while designing a too complex type system can overload programmers with
types and error messages that are hard to understand, even if type inference
removes the necessity of annotating the program with these complex types.
The most challenging aspect of designing optional type systems is to find
the right amount of complexity for a type system that feels natural to the programmers.

Usability has been a concern in the design of Typed Lua since the beginning.
We realized that we should not rely on the semantics of Lua only,
as this could lead to a cumbersome type system that would not support
several Lua idioms.
For this reason, we performed a mostly automated survey of Lua idioms
and features to inform our design choices.

After designing and implementing Typed Lua, we performed several
case studies to evaluate how successful we were in our goal of
providing an usable type system.
We evaluated 29 modules from 8 different case studies,
and we could give precise static types to 83\% of the 449
members that these modules export.
In the median, we could give precise static types to 89\%
of the members from each module.
Our evaluation results showed that our type system can statically
type check several Lua idioms and features, though the evaluation
results also exposed several limitations of our type system.
We found that the three main limitations of our type system are
the lack of intersection types, parametric polymorphism, and operator overloading.
Overcoming these limitations is our major target for future work,
as it will allow us to statically type check more programs.

Unlike other optional type systems, we designed Typed Lua without
deliberate unsound parts.
However, we still do not have proofs that the novel features of
our type system are sound.
We see a soundness proof as another major future work, as it is
necessary to use static types for code optimization.

Finally, we believe that Typed Lua is a major contribution to the Lua community,
because it offers a framework that programmers can use to document,
test, and better structure their applications.
For libraries where a full conversion to static type checking should
prove unfeasible or too much work, the community can use Typed Lua
just to document the external interfaces of the libraries,
giving the benefits of static type checking to the users of these
libraries.
In fact, we already have user feedback from Lua programmers that are
using Typed Lua in their projects.
For instance, ZeroBrane Studio is an IDE for Lua development that is
starting to use Typed Lua to perform static analysis in Lua code.



\bibliographystyle{plainnat}
\bibliography{thesis_andre}

\end{document}
