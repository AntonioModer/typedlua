We begin this chapter presenting statistics about the usage of Lua
features and idioms.
We collected statistics about how programmers use tables, functions,
dynamic type checking, object-oriented programming, and modules.
We shall see that these statistics informed important design decisions
on our optional type system.
We end this chapter introducing Typed Lua through examples, and
describing a few Lua projects that are similar to Typed Lua.

\section{Statistics about the usage of Lua}
\label{sec:statistics}

We used the LuaRocks repository to build our statistics database;
LuaRocks \citep{hisham2013luarocks} is a package manager for Lua
modules.
We downloaded the 3928 \texttt{.lua} files that were available in
the LuaRocks repository at February 1st 2014.
However, we ignored files that are not compatible with Lua 5.2,
the latest version of Lua.
We also ignored \emph{machine-generated} files and test files,
because these files may not represent idiomatic Lua code,
and might skew our statistics towards non-typical uses of Lua.
This left 2598 \texttt{.lua} files from 262 different projects for
our statistics database;
we parsed these files and processed their abstract syntax tree
to gather the statistics that we show in this section.

To verify how programmers use tables, we measured how they
initialize, index, and iterate tables.
We present these statistics in the next three paragraphs to discuss
their influence on our type system.

The table constructor appears 23185 times.
In 36\% of the occurrences it is a constructor that initializes a
record (e.g., \texttt{\{ x = 120, y = 121 \}});
in 29\% of the occurrences it is a constructor that initializes a
list (e.g., \texttt{\{ "one", "two", "three", "four" \}});
in 8\% of the occurrences it is a constructor that initializes a
record with a list part;
and in less than 1\% of the occurrences (4 times) it is a constructor
that uses only the booleans \texttt{true} and \texttt{false} as indexes.
At all, in 73\% of the occurrences it is a constructor that uses
only literal keys;
in 26\% of the occurrences it is the empty constructor;
in 1\% of the occurrences it is a constructor with non-literal keys
only, that is, a constructor that uses variables and function calls
to create the indexes of a table;
and in less than 1\% of the occurrences (19 times) it is a constructor
that mixes literal keys and non-literal keys.

The indexing of tables appears 130448 times:
86\% of them are for reading a table field while
14\% of them are for writing into a table field.
We can classify the indexing operations that are reads as follows:
89\% of the reads use a literal string key,
4\% of the reads use a literal number key,
and less than 1\% of the reads (10 times) use a literal boolean key.
At all, 93\% of the reads use literals to index a table while
7\% of the reads use non-literal expressions to index a table.
It also worth mentioning that 45\% of the reads are actually
function calls.
More precisely, 25\% of the reads use literals to call a function,
20\% of the reads use literals to call a method, that is,
a function call that uses the colon syntactic sugar, 
and less than 1\% of the reads (195 times) use non-literal expressions
to call a function.
We can also classify the indexing operations that are writes as follows: 
69\% of the writes use a literal string key,
2\% of the writes use a literal number key,
and less than 1\% of the writes (1 time) uses a literal boolean key.
At all, 71\% of the writes use literals to index a table while
29\% of the writes use non-literal expressions to index a table.

We also measured how many files have code that iterate over tables to
observe how frequently iteration is used.
We observed that 23\% of the files have code that iterate over keys
of any value, that is, the call to \texttt{pairs} appears at least
once in these files (the median is twice per file);
21\% of the files have code that iterate over integer keys, that is,
the call to \texttt{ipairs} appears at least once in these files
(the median is also twice per file);
and 10\% of the files have code that use the numeric \texttt{for}
along with the length operator (the median is once per file).

The numbers about table initialization, indexing, and iteration
show us that tables are mostly used to represent records, lists,
and associative arrays.
Therefore, Typed Lua should include a table type for handling
these uses of Lua tables.
Even though the statistics show that programmers initialize tables
more often than they use the empty constructor to
dynamically initialize tables, the statistics of the empty
constructor are still expressive and indicate that Typed Lua should
also include a way to handle this style of defining table types.

We found a total of 24858 function declarations in our database
(the median is six per file).
Next, we discuss how frequently programmers use dynamic type
checking and multiple return values inside these functions.

We observed that 9\% of the functions perform dynamic type checking
on their input parameters, that is, these functions use \texttt{type}
to inspect the tags of Lua values (the median is once per function).
We randomly selected 20 functions to sample how programmers are
using \texttt{type}, and we got the following data:
50\% of these functions use \texttt{type} for asserting the tags of
their input parameters, that is, they raise an error when the tag of a
certain parameter does not match the expected tag, and
50\% of these functions use \texttt{type} for overloading, that is,
they execute different code according to the inspected tag.

These numbers show us that Typed Lua should include union types,
because the use of the \texttt{type} idiom shows that disjoint unions
will help programmers define data structures that can hold a value of
several different, but fixed types.
Typed Lua should also use \texttt{type} as a mechanism for decomposing
unions, though it may be restricted to base types only.

We observed that 10\% of the functions explicitly return multiple
values.
We also observed that 5\% of the functions return \texttt{nil} plus
something else, for signaling an unexpected behavior;
and 1\% of the functions return \texttt{false} plus something else,
also for signaling an unexpected behavior.

Typed Lua should include function types to represent Lua functions,
and tuple types to represent the signatures of Lua functions,
multiple return values, and multiple assignments.
Tuple types will require some special attention, because Typed Lua
should be able to adjust tuple types during compile-time, in a
similar way to what Lua does with function calls and multiple
assignments during run-time.
In addition, the number of functions that return \texttt{nil} and
\texttt{false} plus something else show us that overloading on the
return type will be useful to the type system.

We also measured how frequently programmers use the object-oriented
paradigm in Lua.
We observed that 23\% of the function declarations are actually
method declarations.
More precisely, 14\% of them use the colon syntactic sugar while
9\% of them use \texttt{self} as their first parameter.
We also observed that 63\% of the projects extend tables with
metatables, that is, they call \texttt{setmetatable} at least once,
and 27\% of the projects access the metatable of a given table,
that is, they call \texttt{getmetatable} at least once.
In fact, 45\% of the projects extend tables with metatables and
declare methods:
13\% using the colon syntactic sugar, 14\% using \texttt{self}, and
18\% using both.

Based on these observations, Typed Lua should include interfaces and
classes as syntactic sugar for table types that encode objects.
Proposing an optional type system for Lua that includes classes may
sound unusual at first glance, because Lua does not have standard
policies for object-oriented programming.
However, Lua provides mechanisms that allow programmers to abstract
their code in terms of objects, and our statistics confirm that
an expressive number of programmers are relying on these mechanisms to
use the object-oriented paradigm in Lua.
Typed Lua should include interfaces and classes as merely syntactic
mechanisms that help the compiler type check object-oriented code,
and they should not influence the semantics of Lua.

We also measured how programmers are defining modules.
We observed that 38\% of the files use the current way of defining
modules, that is, these files return a table that contains the
exported members of the module at the end of the file;
22\% of the files still use the deprecated way of defining modules,
that is, these files call the function \texttt{module};
and 1\% of the files use both ways.
At all, 61\% of the files are modules while 39\% of the files are
plain scripts.
The number of plain scripts is high considering the origin of
our database.
However, we did not ignore sample scripts, which usually serve to
help the users of a given module on how to use this module, and
that is the reason why we have a high number of plain scripts.

Based on these observations, Typed Lua should also use interfaces as
syntactic sugar for table types that define modules.
Typed Lua should also support the deprecated style of module
definition, using global names as exported members of the module.

Besides using interfaces for encoding objects and defining modules,
Typed Lua should also use interfaces for defining the types of
userdata, as programmers often use userdata in an object-oriented
style.

The last statistics that we collected were about variadic functions
and vararg expressions.
We observed that 8\% of the functions are variadic, that is,
their last parameter is the vararg expression.
We also observed that 5\% of the initialization of lists
(or 2\% of the occurrences of the table constructor) use solely the
vararg expression.
Typed Lua should include a \emph{vararg type} to handle variadic
functions and vararg expressions.

\section{Typed Lua}
\label{sec:typedlua}

Typed Lua allows statically typed Lua code to coexist and interact
with dynamically typed Lua code.
The Typed Lua compiler checks for type errors, but alwayas removes
the annotations, generating Lua code that runs in unmodified Lua
implementations.
The programmer can enjoy some of the benefits of static types even
without converting existing Lua modules to Typed Lua:
a dynamically typed module can export a statically typed interface,
and statically typed users of the module will have their use of the
module checked by the compiler.

Unlike gradual typing, Typed Lua does not insert run-time checks
between dynamically typed and statically typed parts of the program.
Unlike some optional type systems, the statically typed subset of
Typed Lua aims to be sound, so a later version of the Typed Lua
compiler can switch to gradual typing instead of just optional typing.

In this section we cover the design of Typed Lua:
In Section \ref{sec:types} we present the types that are available
in Typed Lua along with the subtyping and consistent-subtyping
relations.
In Section \ref{sec:examples} we present the design of Typed Lua
through code examples.
In Section \ref{sec:rules} we present the design of Typed Lua
through typing rules.

\subsection{Type Language}
\label{sec:types}

Lua values can have one of eight tags:
\emph{nil}, \emph{boolean}, \emph{number}, \emph{string},
\emph{function}, \emph{table}, \emph{userdata}, and \emph{thread}.
In this section we present the types that Typed Lua introduces
to represent these tags.

\begin{figure}[!ht]
\textbf{Type Language}\\
\dstart
$$
\begin{array}{rlr}
\multicolumn{3}{c}{\textbf{First-level types}}\\
T ::= & \;\; L & \textit{literal types}\\
& | \; B & \textit{base types}\\
& | \; \Nil & \textit{nil type}\\
& | \; \Value & \textit{top type}\\
& | \; \Any & \textit{dynamic type}\\
& | \; \Self & \textit{self type}\\
& | \; T \cup T & \textit{disjoint union types}\\
& | \; P \rightarrow R & \textit{function types}\\
& | \; \{ \}_{o|c} & \textit{empty table type}\\
& | \; \{F, ..., F\}_{o|c} & \textit{table types}\\
& | \; X & \textit{type variables}\\
& | \; \mu X.T & \textit{recursive types}\\
& | \; X_{i} & \textit{projection types}\\
L ::= & \False \; | \; \True \; | \; {<}{\it number}{>} \; | \; {<}{\it string}{>} & \\
B ::= & \Boolean \; | \; \Number \; | \; \String & \\
F ::= & K:T \; | \; \Const \; K:T & \textit{field types}\\ 
K ::= & L \; | \; B \; | \; \Value \; | \; \Any & \textit{key types}\\
\multicolumn{3}{c}{}\\
\multicolumn{3}{c}{\textbf{Second-level types}}\\
S ::= & \;\; T &\\
& | \; T* & \textit{vararg types}\\
& | \; T \times S & \textit{tuple types}\\
P ::= & \Void \; | \; S & \textit{parameters list type}\\
R ::= & \Void \; | \; S \; | \; S \cup R & \textit{return types}\\
\end{array}
$$
\dend
\caption{Typed Lua types}
\label{fig:typelang}
\end{figure}

Figure \ref{fig:typelang} presents the abstract syntax of the
Typed Lua types.
We split Typed Lua types into two categories:
\emph{first-level types} and \emph{second-level types}.
We use first-level types to represent Lua values and
second-level types to type expression lists, multiple assignments,
and function applications.

Figure \ref{fig:subtyping} presents the subtyping relation among
Typed Lua types.
Any first-level type is a subtype of $\Value$.
The rest of the subtyping relationship is standard:
union types are supertypes of their respective literal types,
function types are related by contravariance on the input
and covariance on the output, table types have width subtyping,
with depth subtyping on $\Const$ fields, tuple and vararg types
are covariant.

\begin{figure}[!ht]
\textbf{Subtyping}\\
\dstart
\begin{footnotesize}
$$
\begin{array}{c}
\begin{array}{c}
\mylabel{S-LITERAL}\\
\senv \vdash L \subtype L
\end{array}
\;
\begin{array}{c}
\mylabel{S-FALSE}\\
\senv \vdash \False \subtype \Boolean
\end{array}
\;
\begin{array}{c}
\mylabel{S-TRUE}\\
\senv \vdash \True \subtype \Boolean
\end{array}
\\ \\
\begin{array}{c}
\mylabel{S-NUMBER}\\
\senv \vdash {<}{\it number}{>} \subtype \Number
\end{array}
\;
\begin{array}{c}
\mylabel{S-STRING}\\
\senv \vdash {<}{\it string}{>} \subtype \String
\end{array}
\;
\begin{array}{c}
\mylabel{S-BASE}\\
\senv \vdash B \subtype B
\end{array}
\\ \\
\begin{array}{c}
\mylabel{S-NIL}\\
\senv \vdash \Nil \subtype \Nil
\end{array}
\;
\begin{array}{c}
\mylabel{S-VALUE}\\
\senv \vdash T \subtype \Value
\end{array}
\;
\begin{array}{c}
\mylabel{S-ANY}\\
\senv \vdash \Any \subtype \Any
\end{array}
\;
\begin{array}{c}
\mylabel{S-SELF}\\
\senv \vdash \Self \subtype \Self
\end{array}
\\ \\
\begin{array}{c}
\mylabel{S-UNION1}\\
\dfrac{\senv \vdash T_{1} \subtype T \;\;\;
       \senv \vdash T_{2} \subtype T}
      {\senv \vdash T_{1} \cup T_{2} \subtype T}
\end{array}
\;
\begin{array}{c}
\mylabel{S-UNION2}\\
\dfrac{\senv \vdash T \subtype T_{1}}
      {\senv \vdash T \subtype T_{1} \cup T_{2}}
\end{array}
\;
\begin{array}{c}
\mylabel{S-UNION3}\\
\dfrac{\senv \vdash T \subtype T_{2}}
      {\senv \vdash T \subtype T_{1} \cup T_{2}}
\end{array}
\\ \\
\begin{array}{c}
\mylabel{S-FUNCTION}\\
\dfrac{\senv \vdash P_{2} \subtype P_{1} \;\;\;
       \senv \vdash R_{1} \subtype R_{2}}
      {\senv \vdash P_{1} \rightarrow R_{1} \subtype P_{2} \rightarrow R_{2}}
\end{array}
\;
\begin{array}{c}
\mylabel{S-TABLE1}\\
\dfrac{\forall i \in 1..n \; \exists j \in 1..m \;\;\;
       \senv \vdash F_{j} \subtype F_{i}}
      {\senv \vdash \{F_{1}, ..., F_{m}\}_{c} \subtype \{F_{1}, ..., F_{n}\}}
\end{array}
\\ \\
\begin{array}{c}
\mylabel{S-TABLE2}\\
\dfrac{\forall i \in 1..m \; \senv \vdash F_{i}' \subtype F_{i}}
      {\senv \vdash \{F_{1}, ..., F_{m}\}_{o} \subtype \{F_{1}', ..., F_{m}', ..., F_{n}'\}}
\end{array}
\;
\begin{array}{c}
\mylabel{S-TABLE3}\\
\dfrac{\forall i \in 1..m \; \senv \vdash F \subtype F_{i}}
      {\senv \vdash \{F_{1}, ..., F_{m}\}_{o} \subtype \{F\}} 
\end{array}
\\ \\
\begin{array}{c}
\mylabel{S-FIELD1}\\
\dfrac{\senv \vdash K_{2} \subtype K_{1} \;\;\;
       \senv \vdash T_{1} \subtype T_{2} \;\;\;
       \senv \vdash T_{2} \subtype T_{1}}
      {\senv \vdash K_{1}:T_{1} \subtype K_{2}:T_{2}}
\end{array}
\;
\begin{array}{c}
\mylabel{S-FIELD2}\\
\dfrac{\senv \vdash K_{2} \subtype K_{1} \;\;\;
       \senv \vdash T_{1} \subtype T_{2}}
      {\senv \vdash \Const \; K_{1}:T_{1} \subtype \Const \; K_{2}:T_{2}}
\end{array}
\\ \\
\begin{array}{c}
\mylabel{S-FIELD3}\\
\dfrac{\senv \vdash K_{2} \subtype K_{1} \;\;\;
       \senv \vdash T_{1} \subtype T_{2}}
      {\senv \vdash K_{1}:T_{1} \subtype \Const \; K_{2}:T_{2}}
\end{array}
\;
\begin{array}{c}
\mylabel{S-VARIABLE}\\
\dfrac{X_{1} \subtype X_{2} \in \senv}
      {\senv \vdash X_{1} \subtype X_{2}}
\end{array}
\;
\begin{array}{c}
\mylabel{S-RECURSIVE1}\\
\dfrac{\senv, X_{1} \subtype X_{2} \vdash T_{1} \subtype T_{2}}
      {\senv \vdash \mu X_{1}.T_{1} \subtype \mu X_{2}.T_{2}}
\end{array}
\\ \\
\begin{array}{c}
\mylabel{S-RECURSIVE2}\\
\dfrac{X \subtype X \not\in \senv \;\;\;
      \senv, X \subtype X \vdash T_{1} \subtype T_{2}}
      {\senv \vdash \mu X.T_{1} \subtype T_{2}}
\end{array}
\;
\begin{array}{c}
\mylabel{S-RECURSIVE3}\\
\dfrac{X \subtype X \in \senv \;\;\;
       \senv \vdash X \subtype T_{2}}
      {\senv \vdash \mu X.T_{1} \subtype T_{2}}
\end{array}
\\ \\
\begin{array}{c}
\mylabel{S-RECURSIVE4}\\
\dfrac{X \subtype X \not\in \senv \;\;\;
       \senv, X \subtype X \vdash T_{1} \subtype T_{2}}
      {\senv \vdash T_{1} \subtype \mu X.T_{2}}
\end{array}
\;
\begin{array}{c}
\mylabel{S-RECURSIVE5}\\
\dfrac{X \subtype X \in \senv \;\;\;
       \senv \vdash T_{1} \subtype X}
      {\senv \vdash T_{1} \subtype \mu X.T_{2}}
\end{array}
\\ \\
\begin{array}{c}
\mylabel{S-VOID}\\
\senv \vdash \Void \subtype \Void
\end{array}
\;
\begin{array}{c}
\mylabel{S-VARARG}\\
\dfrac{\senv \vdash T_{1} \cup \Nil \subtype T_{2} \cup \Nil}
      {\senv \vdash T_{1}* \subtype T_{2}*}
\end{array}
\;
\begin{array}{c}
\mylabel{S-TUPLE1}\\
\dfrac{\senv \vdash T_{1} \subtype T_{2} \;\;\;
       \senv \vdash S_{1} \subtype S_{2}}
      {\senv \vdash T_{1} \times S_{1} \subtype T_{2} \times S_{2}}
\end{array}
\\ \\
\begin{array}{c}
\mylabel{S-TUPLE2}\\
\dfrac{\senv \vdash T_{1} \cup \Nil \subtype T_{2} \;\;\;
       \senv \vdash T_{1}* \subtype S_{2}}
      {\senv \vdash T_{1}* \subtype T_{2} \times S_{2}}
\end{array}
\;
\begin{array}{c}
\mylabel{S-TUPLE3}\\
\dfrac{\senv \vdash T_{1} \subtype T_{2} \cup \Nil \;\;\;
       \senv \vdash S_{1} \subtype T_{2}*}
      {\senv \vdash T_{1} \times S_{1} \subtype T_{2}*}
\end{array}
\end{array}
$$
\end{footnotesize}
\dend
\caption{The subtyping relation}
\label{fig:subtyping}
\end{figure}

Figure \ref{fig:consistent_subtyping} presents the consistent-subtyping
relation that allows the dynamic type $\Any$ interoperate between
dynamically typed and statically typed code.
The dynamic type $\Any$ is a subtype of $\Value$, but neither a
supertype nor a subtype of any other type.
Our consistent-subtyping relationship follows the standards defined
by the gradual typing of objects \citep{siek2007objects,siek2013mutable}.
In practice, we can pass a value of the dynamic type anytime we want
a value of some other type, and can pass any value where where a
value of the dynamic type is expected, but these operations are tracked
by the type system, and the programmer can choose to be warned about them.

\begin{figure}[!ht]
\textbf{Consistent-Subtyping}\\
\dstart
\begin{footnotesize}
$$
\begin{array}{c}
\begin{array}{c}
\mylabel{C-LITERAL}\\
\senv \vdash L \lesssim L
\end{array}
\;
\begin{array}{c}
\mylabel{C-FALSE}\\
\senv \vdash \False \lesssim \Boolean
\end{array}
\;
\begin{array}{c}
\mylabel{C-TRUE}\\
\senv \vdash \True \lesssim \Boolean
\end{array}
\\ \\
\begin{array}{c}
\mylabel{C-NUMBER}\\
\senv \vdash {<}{\it number}{>} \lesssim \Number
\end{array}
\;
\begin{array}{c}
\mylabel{C-STRING}\\
\senv \vdash {<}{\it string}{>} \lesssim \String
\end{array}
\;
\begin{array}{c}
\mylabel{C-BASE}\\
\senv \vdash B \lesssim B
\end{array}
\\ \\
\begin{array}{c}
\mylabel{C-NIL}\\
\senv \vdash \Nil \lesssim \Nil
\end{array}
\;
\begin{array}{c}
\mylabel{C-VALUE}\\
\senv \vdash T \lesssim \Value
\end{array}
\;
\begin{array}{c}
\mylabel{C-ANY1}\\
\senv \vdash T \lesssim \Any
\end{array}
\;
\begin{array}{c}
\mylabel{C-ANY2}\\
\senv \vdash \Any \lesssim T
\end{array}
\;
\begin{array}{c}
\mylabel{C-SELF}\\
\senv \vdash \Self \lesssim \Self
\end{array}
\\ \\
\begin{array}{c}
\mylabel{C-UNION1}\\
\dfrac{\senv \vdash T_{1} \lesssim T \;\;\;
       \senv \vdash T_{2} \lesssim T}
      {\senv \vdash T_{1} \cup T_{2} \lesssim T}
\end{array}
\;
\begin{array}{c}
\mylabel{C-UNION2}\\
\dfrac{\senv \vdash T \lesssim T_{1}}
      {\senv \vdash T \lesssim T_{1} \cup T_{2}}
\end{array}
\;
\begin{array}{c}
\mylabel{C-UNION3}\\
\dfrac{\senv \vdash T \lesssim T_{2}}
      {\senv \vdash T \lesssim T_{1} \cup T_{2}}
\end{array}
\\ \\
\begin{array}{c}
\mylabel{C-FUNCTION}\\
\dfrac{\senv \vdash P_{2} \lesssim P_{1} \;\;\;
       \senv \vdash R_{1} \lesssim R_{2}}
      {\senv \vdash P_{1} \rightarrow R_{1} \lesssim P_{2} \rightarrow R_{2}}
\end{array}
\;
\begin{array}{c}
\mylabel{C-TABLE1}\\
\dfrac{\forall i \in 1..n \; \exists j \in 1..m \;\;\;
       \senv \vdash F_{j} \lesssim F_{i}}
      {\senv \vdash \{F_{1}, ..., F_{m}\}_{c} \lesssim \{F_{1}, ..., F_{n}\}}
\end{array}
\\ \\
\begin{array}{c}
\mylabel{C-TABLE2}\\
\dfrac{\forall i \in 1..m \; \senv \vdash F_{i}' \lesssim F_{i}}
      {\senv \vdash \{F_{1}, ..., F_{m}\}_{o} \lesssim \{F_{1}', ..., F_{m}', ..., F_{n}'\}}
\end{array}
\;
\begin{array}{c}
\mylabel{C-TABLE3}\\
\dfrac{\forall i \in 1..m \; \senv \vdash F \lesssim F_{i}}
      {\senv \vdash \{F_{1}, ..., F_{m}\}_{o} \lesssim \{F\}} 
\end{array}
\\ \\
\begin{array}{c}
\mylabel{C-FIELD1}\\
\dfrac{\senv \vdash K_{2} \lesssim K_{1} \;\;\;
       \senv \vdash T_{1} \lesssim T_{2} \;\;\;
       \senv \vdash T_{2} \lesssim T_{1}}
      {\senv \vdash K_{1}:T_{1} \lesssim K_{2}:T_{2}}
\end{array}
\;
\begin{array}{c}
\mylabel{C-FIELD2}\\
\dfrac{\senv \vdash K_{2} \lesssim K_{1} \;\;\;
       \senv \vdash T_{1} \lesssim T_{2}}
      {\senv \vdash \Const \; K_{1}:T_{1} \lesssim \Const \; K_{2}:T_{2}}
\end{array}
\\ \\
\begin{array}{c}
\mylabel{C-FIELD3}\\
\dfrac{\senv \vdash K_{2} \lesssim K_{1} \;\;\;
       \senv \vdash T_{1} \lesssim T_{2}}
      {\senv \vdash K_{1}:T_{1} \lesssim \Const \; K_{2}:T_{2}}
\end{array}
\;
\begin{array}{c}
\mylabel{C-VARIABLE}\\
\dfrac{X_{1} \lesssim X_{2} \in \senv}
      {\senv \vdash X_{1} \lesssim X_{2}}
\end{array}
\;
\begin{array}{c}
\mylabel{C-RECURSIVE1}\\
\dfrac{\senv, X_{1} \lesssim X_{2} \vdash T_{1} \lesssim T_{2}}
      {\senv \vdash \mu X_{1}.T_{1} \lesssim \mu X_{2}.T_{2}}
\end{array}
\\ \\
\begin{array}{c}
\mylabel{C-RECURSIVE2}\\
\dfrac{X \lesssim X \not\in \senv \;\;\;
      \senv, X \lesssim X \vdash T_{1} \lesssim T_{2}}
      {\senv \vdash \mu X.T_{1} \lesssim T_{2}}
\end{array}
\;
\begin{array}{c}
\mylabel{C-RECURSIVE3}\\
\dfrac{X \lesssim X \in \senv \;\;\;
       \senv \vdash X \lesssim T_{2}}
      {\senv \vdash \mu X.T_{1} \lesssim T_{2}}
\end{array}
\\ \\
\begin{array}{c}
\mylabel{C-RECURSIVE4}\\
\dfrac{X \lesssim X \not\in \senv \;\;\;
       \senv, X \lesssim X \vdash T_{1} \lesssim T_{2}}
      {\senv \vdash T_{1} \lesssim \mu X.T_{2}}
\end{array}
\;
\begin{array}{c}
\mylabel{C-RECURSIVE5}\\
\dfrac{X \lesssim X \in \senv \;\;\;
       \senv \vdash T_{1} \lesssim X}
      {\senv \vdash T_{1} \lesssim \mu X.T_{2}}
\end{array}
\\ \\
\begin{array}{c}
\mylabel{C-VOID}\\
\senv \vdash \Void \lesssim \Void
\end{array}
\;
\begin{array}{c}
\mylabel{C-VARARG}\\
\dfrac{\senv \vdash T_{1} \cup \Nil \lesssim T_{2} \cup \Nil}
      {\senv \vdash T_{1}* \lesssim T_{2}*}
\end{array}
\;
\begin{array}{c}
\mylabel{C-TUPLE1}\\
\dfrac{\senv \vdash T_{1} \lesssim T_{2} \;\;\;
       \senv \vdash S_{1} \lesssim S_{2}}
      {\senv \vdash T_{1} \times S_{1} \lesssim T_{2} \times S_{2}}
\end{array}
\\ \\
\begin{array}{c}
\mylabel{C-TUPLE2}\\
\dfrac{\senv \vdash T_{1} \cup \Nil \lesssim T_{2} \;\;\;
       \senv \vdash T_{1}* \lesssim S_{2}}
      {\senv \vdash T_{1}* \lesssim T_{2} \times S_{2}}
\end{array}
\;
\begin{array}{c}
\mylabel{C-TUPLE3}\\
\dfrac{\senv \vdash T_{1} \lesssim T_{2} \cup \Nil \;\;\;
       \senv \vdash S_{1} \lesssim T_{2}*}
      {\senv \vdash T_{1} \times S_{1} \lesssim T_{2}*}
\end{array}
\end{array}
$$
\end{footnotesize}
\dend
\caption{The consistent-subtyping relation}
\label{fig:consistent_subtyping}
\end{figure}

\subsection{Typed Lua through examples}
\label{sec:examples}

The main goal of Typed Lua is to allow programmers to combine static
and dynamic typing in Lua, so Typed Lua is an extension to Lua 5.2
syntax that introduces optional type annotations, and class-based
object-oriented programming through the definition of classes,
interfaces, and modules.
An important feature of Typed Lua is its backwards compatibility with
Lua, that is, unannotated Typed Lua code is valid Lua code.
For this reason, the syntactic extensions introduced by Typed Lua
include new keywords that are not reserved words.
Appendix \ref{app:syntax} presents the complete syntax of Typed Lua,
which we cover in this section through code examples. 

\subsection{Atomic types and Functions}

Typed Lua allows optional type annotations in variable and function
declarations.
Whenever it is possible to use local type inference, the compiler
assigns a more specific type to an unannotated declaration.
When it is not possible to use local type inference, the compiler
assigns the type $\Any$ to an unannotated declaration.
For instance, in function declarations, the compiler does not infer
a more specific type to unannotated input parameters, because,
in this case, global type inference often is undecidable in a
language with subtyping and overloading \citep{wells1999typability}.

In the following example we use type annotations in a function
declaration, and we do not use type annotations in a declaration of a
local variable:
\begin{verbatim}
    local function fact (n:number):number
      if n == 0 then
        return 1
      else
        return n * fact(n - 1)
      end
    end
    local x = 5
    print(fact(x))
\end{verbatim}

Even though we did not add type annotations in the declaration of
the local variable \texttt{x}, the compiler uses local type inference to
assign the type $\Number$ to \texttt{x}.
The compiler does not use the literal type $5$ instead of the base
type $\Number$ because this would prevent
programmers from assigning other numeric values to the variable
\texttt{x}.
The inference that we implement in Typed Lua is quite simple, as it
uses only the type of the local expression.

Typed Lua allows programmers to combine annotated code with
unannotated code, as we show in the following example:
\begin{verbatim}
    local function abs (n:number):number
      if n < 0 then
        return -n
      else
        return n
      end
    end

    local function dist (x, y)
      return abs(x - y)
    end
\end{verbatim}

The function \texttt{dist} receives two parameters of type $\Any$
and returns a value of type $\Number$.
The compiler assigns the dynamic type $\Any$ to the input
parameters of \texttt{dist} because they do not have type annotations,
and the compiler does not use global type inference, as we mentioned
previously.
The compiler infers the return type of \texttt{dist} because it is
not a recursive function, that is, Typed Lua does not infer the return
type of recursive functions.

We are using the consistent-subtyping relation of gradual
typing to check the interaction among the dynamic type and the other
types.
More precisely, we can call \texttt{abs} inside \texttt{dist} because
\texttt{x - y} is of type $\Any$, and the type $\Any$ is
consistent with the type $\Number$.
The type $\Any$ is not the union of all types, but it is a separate
type.

Unlike gradual typing, we do not insert run-time checks that
inspect the interaction between dynamically typed code and statically
typed code during run-time.
In other words, Typed Lua does not guarantee that dynamically typed
code does not violate statically typed code.
Typed Lua does not insert run-time checks because they can decrease
run-time performance \citep{allende2013cis}.
We believe that a careful evaluation of run-time checks should be done
before inserting them on the type system.
In the previous example, we cannot guarantee that \texttt{dist} is never
going to call \texttt{abs} with a parameter that is not a number,
because in the semantics of Lua the minus operator can result in a
value that is not a number.
Still, we can implement run-time checks in the future because we are
using the consistent-subtyping relation of gradual typing to
formalize our optional type system, and the type system is being
designed to be sound.

\subsection{Unions}

Typed Lua includes union types as a way to discriminate data
structures that can hold values of several different, but fixed types.
Next, we show an example:
\begin{verbatim}
    local x:number?            -- OK
    local y:number|string = 1  -- OK
    local z:number = x         -- Error
\end{verbatim}

The annotation \texttt{number?} is syntactic sugar for
\texttt{number|nil} that means the type $\Number\;\cup\;\Nil$.
Typed Lua includes the syntactic sugar \texttt{t?} because the
type $t\;\cup\;\Nil$ is very common in Lua code.

The last line of the previous example gives a type error because we
are attempting to assign a value of type $\Number\;\cup\;\Nil$
to a variable that accepts only values of type $\Number$.
Although we can solve this type error changing the annotation of
variable \texttt{z} from \texttt{number} to \texttt{number?}, there is
another way that we can use to solve this type error without changing
the annotation:
\begin{verbatim}
    local z:number = x or 0  -- OK
\end{verbatim}

Typed Lua has a typing rule for checking that
$t\;\cup\;\Nil\;\mathbf{or}\;t$ always has the type $t$.

\subsection{Tables and Interfaces}

Typed Lua includes table types to represent Lua's tables.
Although Typed Lua has just one table type constructor, it has three
different syntaxes for defining the type of a table.
The first syntax defines table types for hash tables;
it is written \texttt{\{t1:t2\}},
and maps to the table type $\{t_{1}:t_{2}\;\cup\;\Nil\}$.
This table type represents a hash table that maps values of type
$t_{1}$ to values of type $t_{2}\;\cup\;\Nil$, as Lua returns
$\Nil$ when we use a non-existing key to index a table.
The second syntax defines table types for lists;
it is written \texttt{\{t\}},
and maps to the table type $\{\Number:t\;\cup\;\Nil\}$.
This table type is just a syntactic sugar to the definition of a
hash table that maps values of type $\Number$ to values of type
$t\;\cup\;\Nil$.
The third syntax defines table types for records;
it is written \texttt{\{s1:t1, ..., sn:tn\}},
and maps to the table type $\{s_{1}:t_{1}, ..., s_{n}:t_{n}\}$.
This table type represents a record that maps the literals
$s_{i}$\texttt{, ..., }$s_{n}$ to values of types
$t_{i}$\texttt{, ..., }$t_{n}$, where $i$ may vary from one to $n$.

Next, we show one example of table type to type a hash table:
\begin{verbatim}
    local t:{string:number} = { foo = 1 }  -- OK
    local x:number = t.foo                 -- Error
    local y:number? = t.bar                -- OK
\end{verbatim}

The second line of this example gives a type error for the same
reason that the last line of the previous example gives a type error
too, that is, we are trying to assign a value that can be of two
different types to a variable that accepts values of only one type.
Although the field \texttt{bar} does not exist in \texttt{t}, the last
line of this example does not give a type error because the
annotated type matches the type of the values that can be stored in
\texttt{t}.

Next, we show one example of table type to type a list:
\begin{verbatim}
    local days:{string} = { "Sunday", "Monday", "Tuesday",
                            "Wednesday", "Thursday", "Friday",
                            "Saturday" }  -- OK
\end{verbatim}

When we know that a list has fixed elements, we can leave the
variable declaration unannotated and let local type inference assign
a more specific table type to the list.
If we remove the annotation in this example, the compiler uses the
syntax of records to infer the following table type to \texttt{days}:
\begin{align*}
\{{1:\String},\;{2:\String},\;{3:\String},\;{4:\String},\;\\
{5:\String},\;{6:\String},\;{7:\String}\}
\end{align*}
This inferred table type is a subtype of the type
$\{\Number:\String\;\cup\;\Nil\}$, which is the type that we used to
annotate the variable \texttt{days}.

When we use the inferred table type while initializing a table, the
compiler gives an error message when we try to access an index that
is out of bounds.

It also worth mentioning that we can use the style of the inferred
type to define heterogeneous tuples.

Next, we show one example of table type to type a record:
\begin{verbatim}
    local person:{"firstname":string, "lastname":string} =
      { firstname = "Lou", lastname = "Reed" } 
\end{verbatim}

Type inference would infer the very same table type that we used to
annotate the variable \texttt{person}.

Typed Lua includes interfaces as syntactic sugar to table types
that specify records.
For instance, we can use the following interface as an alias to the
type of the record we showed in the previous example:
\begin{verbatim}
    interface Person
      firstname:string
      lastname:string
    end
\end{verbatim}

Now we can use the type \texttt{Person} as an alias to a table type
that certainly has the fields \texttt{firstname} and \texttt{lastname}.
Next, we show an example of code that uses the type \texttt{Person}:
\begin{verbatim}
    local function greeter (person:Person):string
      return "Hello " .. person.firstname ..
             " " .. person.lastname
    end

    local user1 = { firstname = "Lewis",
                    middlename = "Allan",
                    lastname = "Reed" }
    local user2 = { firstname = "Lou" }
    local user3 = { lastname = "Reed",
                    firstname = "Lou" }
    local user4 = { "Lou", "Reed" }

    print(greeter(user1)) -- OK
    print(greeter(user2)) -- Error
    print(greeter(user3)) -- OK
    print(greeter(user4)) -- Error
\end{verbatim}

This example shows that our optional type system is structural rather
than nominal, that is, it checks the structure of types instead of
their names, and it uses subtyping and consistent-subtyping for
checking types.

Using interfaces we can also define recursive types such as linked
lists:

\begin{verbatim}
    interface Element
      info:number
      next:Element?
    end
\end{verbatim}

We need to explicitly annotate \texttt{next} being of type
\texttt{Element}$\;\cup\;\Nil$ because in our type system $\Nil$
is not the bottom type, that is, $\Nil$ is subtype of itself only.
Typed Lua has this behavior because in Lua $\Nil$ represents
the absence of an useful value.

Besides interfaces, Typed Lua will also introduce classes.
We want to offer classes as a standard way for writing code that
Lua programmers are already using, but without tying them to a
specific model.
Our classes will not include static attributes and will also not offer
privacy rules.

Typed Lua will also introduce a specific syntax for defining modules.
For now, each file is a module.
We use all global names and their types to build the interface that
represents the type of the module, in case the module is not returned
at the end of the file.

Typed Lua includes function types because functions are first-class
values in Lua.
The syntax for defining a function type is \texttt{(s) -> (s)},
where \texttt{s} is a second-class type.
As we saw in Figure \ref{fig:typelang}, a second-class type in
Typed Lua can be the type $\Void$, a first-class type, a variadic
type, or a tuple type.
We use tuple types to represent multiple parameters, multiple return
values, and multiple assignments, and we use the variadic type to
represent variadic functions and the vararg expression.
Even though we already included the tuple type and the vararg type,
we are still working on these types.

The next example shows how we can annotate a \texttt{map} function
that receives another function as its parameter and uses this
function to alter the elements of a list of numbers, which is
also a parameter of \texttt{map}:
\begin{verbatim}
    local function map (f:(number)->(number), l:{number}):void
      for i=1, #l do
        l[i] = f(l[i] or 0)
      end
    end
\end{verbatim}

To conclude this section, we just started the design and
implementation of Typed Lua, and there is still several design
decisions to be taken on this subject.
For instance, our type system should handle the subtleties that
come with metatables.

\subsection{Modules}

\subsection{Objects and Classes}

\subsection{The typing rules of Typed Lua}
\label{sec:rules}

\begin{figure}[!ht]
\textbf{Abstract Syntax}\\
\dstart
$$
\begin{array}{rl}
s ::= & \;\; \mathbf{skip} \; | \;
s_{1} \; ; \; s_{2} \; | \;
\vec{l} = \vec{e}  \; | \;
\mathbf{while} \; e \; \mathbf{do} \; s \; | \;
\mathbf{if} \; e \; \mathbf{then} \; s_{1} \; \mathbf{else} \; s_{2}\\
& | \; \mathbf{local} \; \vec{n}{:}\vec{T} = \vec{e} \; \mathbf{in} \; s \; | \;
\mathbf{rec} \; n{:}T = f \; \mathbf{in} \; s \; | \;
\mathbf{return} \; \vec{e} \; | \;
a\\
e ::= & \;\; \mathbf{nil} \; | \;
\mathbf{false} \; | \;
\mathbf{true} \; | \;
{<}{\it number}{>} \; | \;
{<}{\it string}{>} \; | \;
{...} \; | \;
f \; | \;
\{ \; \vec{c}\; \} \\
& | \; e_{1} + e_{2} \; | \;
e_{1} \; {..} \; e_{2} \; | \;
e_{1} == e_{2} \; | \;
e_{1} < e_{2} \; | \;
e_{1} \; \mathbf{and} \; e_{2} \; | \;
e_{1} \; \mathbf{or} \; e_{2} \\
& | \; \mathbf{not} \; e \; | \;
- e \; | \;
\# \; e \; | \;
a \; | \;
l \\
l ::= & \; n \; | \;
e_{1}[e_{2}]\\
a ::= & \;\; e_{1}(\vec{e_{2}}) \; | \;
e_{1}{:}n(\vec{e_{2}})\\
f ::= & \; \mathbf{fun} \; (){:}R \; s \; | \;
\mathbf{fun} \; ({...}{:}T){:}R \; s \; | \;
\mathbf{fun} \; (\vec{n}{:}\vec{T}){:}R \; s \; | \;
\mathbf{fun} \; (\vec{n}{:}\vec{T},{...}{:}T){:}R \; s\\
c ::= & \; [e_{1}] = e_{2}\\
n ::= & \; {<}{\it name}{>}\\
\end{array}
$$
\dend
\caption{Typed Lua abstract syntax}
\label{fig:syntax}
\end{figure}

\section{Related Lua projects}
\label{sec:related}

Metalua \citep{metalua} is a Lua compiler that supports compile-time
metaprogramming (CTMP).
CTMP is a kind of macro system that allows the programmers to interact
with the compiler \citep{fleutot2007contrasting}. 
Metalua extends Lua 5.1 syntax to include its macro system,
and allows programmers to define their own syntax.
Metalua can provide syntactical support for several object-oriented
styles, and can also provide syntax for turning simple type
annotations into run-time assertions.

MoonScript \citep{moonscript} is a programming language that supports
class-based object-oriented programming, and compiles to idiomatic
Lua code.
However, it does not perform compile-time type checking.

LuaInspect \citep{luainspect} is a tool that uses MetaLua to perform
some code analysis, such as flagging unknown global variables and
table fields, checking the number of function arguments against
signatures, and inferring function return values, but it does not
try to analyze object-oriented code and does not perform compile-time
type checking.

Tidal Lock \citep{tidallock} is a prototype of another optional type
system for Lua, which is written in Metalua.
A remarkable feature of Tidal Lock is the way that it handles tables,
because it allows the programmer to incrementally build the type of
a table.
More precisely, Tidal Lock allows the programmer to create an empty
table to build its type according to the keys that the programmer
inserts in it, so the programmer does not need to build the table in
a single table constructor.
Typed Lua has a simpler form of this feature, with a different
formalization.
Like Typed Lua, Tidal Lock also has local type inference.
Unlike Typed Lua, Tidal Lock does not type the common object-oriented
idioms used by Lua programmers.

