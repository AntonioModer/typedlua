\begin{description}
\item[bottom type] It is a type that is subtype of all types.

\item[closed table type] It is the type of table annotations in Typed Lua.
A closed table type does not provide any guarantees that about keys
with types not listed in the table type, and it also does not
allow adding fields to the table type. 

\item[coercion] It is a relation that allows converting values from one type to values
of another type without error.

\item[consistency] Gradual typing uses the consistency relation instead of type equality
to check the interaction among the dynamic type and other types.
This relation allows us combining dynamic and static typing in the
same language, but still catching static type errors.
The consistency relation is reflexive and symmetric, but it is not transitive.

\item[consistent-subtyping] It is a relation that combines consistency and subtyping, allowing
the definition of gradual type systems for object-oriented languages.
The consistent-subtyping relation is reflexive and symmetric, but it is not transitive.

\item[contravariant] Subtyping is contravariant when it reverses the subtyping order,
that is, a subtyping rule is contravariant when it orders
types from more generic types (supertypes) to more specific types (subtypes).

\item[covariant] Subtyping is covariant when it preserves the subtyping order,
that is, a subtyping rule is covariant when it orders
types from more specific types (subtypes) to more general types (supertypes).

\item[depth subtyping] It allows the supertype to change the type of individual fields in the subtype.

\item[dynamic type] It is a type that allows combining dynamic and static typing in the same code.
It is neither the bottom nor the top type in the subtyping relation, but
a subtype only of itself.
Gradual typing uses the dynamic type along with the consistency relation to
identify where run-time casts should be inserted to prevent that dynamically
typed code violates statically typed code.

\item[filter type] It is a type that allows Typed Lua to discriminate and restore the type of
local variables inside control flow statements.

\item[fixed table type] It is a table type that does not allow width subtyping to make single
inheritance safe in Typed Lua, and it also does not allow adding fields to
the table type.

\item[flow typing] It is a combination of static typing and flow analysis to allow variables
to have different types at different parts of the program.

\item[free assigned variable] It is a free variable that appears in an assignment.

\item[gradual type system] It is a type system that uses the consistency relation instead of type equality
to check the interaction among the dynamic type and other types.

\item[gradual typing] It is an approach that uses a gradual type system to allow static and dynamic
typing in the same code, but inserting run-time checks between statistically
typed and dynamically typed code.
This run-time checks ensure that dynamically typed code does not violate
statically typed code.

\item[invariant] Subtyping is invariant when it does not allow ordering types, that is,
it is a way to define type equality through subtyping.

\item[metatable] It is a Lua table that allows changing the behavior of other tables
it is attached to.

\item[nominal type system] It is a type system that uses the name of the types to check the
compatibility among them.

\item[open table type] It is the type of the tables with keys that do not inhabit one of
the table's key types, have at least one alias, and allows adding
fields to the table type.

\item[optional type system] It is a type system that allows combining static and dynamic typing in the same
language, but without affecting the run-time semantics.

\item[projection environment] It is an environment that Typed Lua uses to assign projection variables to
second-level types.

\item[projection type] It is a type that allows Typed Lua to discriminate the type of local variables
that have a dependency relation, as they include a projection variable and an
index that allow Typed Lua to project unions of second-level types into
unions of first-level types.

\item[prototype object] It is an object that works like a class, that is, it is an object from
which other objects inherit its attributes.

\item[self-like delegation] It is a technique to implement inheritance in dynamically typed languages
through prototype objects. 

\item[sound type system] It is a type system that does not type check all programs that contain a type error.

\item[structural type system] It is a type system that uses the structure of types to check the compatibility among them.

\item[table refinement] It is an operation that allows changing a table type to include new fields or
to specialize existing fields.

\item[top type] It is a type that is supertype of all types.

\item[type environment] It is an environment that Typed Lua uses to assign variable names to first-level types.

\item[type tag] In dynamically typed languages a type tag describes the type of a value during
run-time.

\item[unique table type] It is the type of the tables with keys that do not inhabit one of
the table's key types, does not have any alias, and allow adding
fields to the table type.
It is also the table type that describes the type of the table constructor.

\item[unsound type system] It is a type system that type checks certain programs that contain type errors.

\item[userdata] It is a Lua data type that allows Lua variables to hold values from applications
or libraries that are written in C.

\item[vararg expression] It is a Lua expression that can result in an arbitrary number of values.

\item[variadic function] It is a Lua function that can receive an arbitrary number of arguments.

\item[variance] It is the way types are ordered.

\item[width subtyping] It allows the subtype to have fields that do not exist in the supertype.

\end{description}
