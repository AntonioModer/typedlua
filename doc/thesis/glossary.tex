\begin{description}
\item[bottom type] A type that is subtype of all types.

\item[closed table type] A table type that does not provide any guarantees
about keys with types not listed in the table type.
See complete definition in page \pageref{def:tabletype}.

\item[coercion] A relation that allows converting values from one type to
values of another type without error.

\item[consistency] A relation used by gradual typing to check the interaction
between the dynamic type and other types.
See complete definition in page \pageref{def:consistency}.

\item[consistent-subtyping] A relation that combines consistency and subtyping.
See complete definition in page \pageref{def:consistent-subtyping}.

\item[contravariant] A part of the type constructor is contravariant when
it reverses the subtyping order.

\item[covariant] A part of the type constructor is covariant when
it preserves the subtyping order.

\item[depth subtyping] An operation that allows variance in the type of
record fields.

\item[dynamic type] A type used by gradual typing to denote unknown values.
See complete definition in page \pageref{def:dynamictype}.

\item[filter type] A type used by Typed Lua to discriminate the type of
local variables inside control flow statements.
See complete definition in page \pageref{def:filtertype}.

\item[fixed table type] A table type which guarantees that there are no
keys with a type that is not one of its key types, and that can have
any number of \emph{fixed} or \emph{closed} references point to it.
See complete definition in page \pageref{def:tabletype}.

\item[flow typing] An approach that combines static typing and flow analysis to
allow variables to have different types at different parts of the program.

\item[free assigned variable] A free variable that appears in an assignment.

\item[gradual type system] A type system that uses the consistency relation
instead of type equality to perform static type checking.
See complete definition in page \pageref{sec:gradual}.

\item[gradual typing] An approach that uses a gradual type system to allow
static and dynamic typing in the same code, but inserting run-time checks
between statically typed and dynamically typed code.
See complete definition in page \pageref{sec:gradual}.

\item[invariant] A part of the type constructor is invariant when it forbids variance.
It is also a way to define type equality through subtyping.

\item[metatable] A Lua table that allows changing the behavior of other tables
it is attached to.

\item[nominal type system] A type system that uses the type names to check the
compatibility among them.

\item[open table type] A table type which guarantees that there are no
keys with a type that is not one of its key types, and that only have
\emph{closed} references point to it.
See complete definition in page \pageref{def:tabletype}.

\item[optional type system] A type system that allows combining static and
dynamic typing in the same language, but without affecting the run-time semantics.
See complete definition in page \pageref{sec:optional}.

\item[projection environment] An environment used by Typed Lua to handle unions of
second-level types that are bound to projection types.

\item[projection type] A type used by Typed Lua to discriminate the type of local
variables that have a dependency relation.
See complete definition in page \pageref{def:projectiontype}.

\item[prototype object] An object that works like a class, that is, it is an object from
which other objects inherit its attributes.

\item[self-like delegation] A technique to implement inheritance in dynamically typed
languages through prototype objects.

\item[sound type system] A type system that does not type check all programs that contain a type error.

\item[structural type system] A type system that uses type structures to check the compatibility among them.

\item[table refinement] An operation from Typed Lua that allows programmers to change a table type
to include new fields or to specialize existing fields.
See complete definition in page \pageref{sec:refinement}.

\item[top type] A type that is supertype of all types.

\item[type environment] An environment used by Typed Lua to assign variable names to first-level types.

\item[type tag] A tag that describes the type of a value during run-time in dynamically
typed languages.

\item[unique table type] A table type which guarantees that there are no
keys with a type that is not one of its key types, and that do not have
any references point to it.
See complete definition in page \pageref{def:tabletype}.

\item[unsound type system] A type system that type checks certain programs that contain type errors.

\item[userdata] A Lua data type that allows Lua to hold values from applications
or libraries that are written in C.

\item[vararg expression] A Lua expression that can result in an arbitrary number of values.

\item[variadic function] A function that can receive an arbitrary number of arguments.

\item[variance] A way to define the subtyping order between the components
of a type constructor.

\item[width subtyping] An operation that allows adding fields to a record.

\end{description}
