Our main contribution is the design of a complete optional type system
for Lua, a procedural scripting language.
TypeScript is a complete optional type system for JavaScript,
but it does not include union types, and it uses arrays to represent
variadic functions and multiple return values.
Gradualtalk is a complete gradual type system for Smalltalk,
but Smalltalk is an object-oriented language that does
not handle multiple return values, multiple assignments, and
variadic functions.
Gradualtalk also mixes nominal and structural typing because of
the object-oriented nature of Smalltalk.
Typed Racket is a complete static type system for Racket, a functional
language.
Typed Racket makes it easier to migrate from untyped to typed code,
but the migration happens module-by-module and type annotations are
mandatory on typed modules.
There are research on the design of optional and gradual type systems
for JavaScript, Perl, and Python, but they are either experimental or
limited in their coverage.
Furthermore, Typed Lua includes table types that use types as keys,
instead of identifiers, and the fact that Typed Lua also includes
literal types makes its table types subsume the usual idea of
record types.

Although each language is different, we believe that several parts of
the design of Typed Lua should be relevant to other scripting languages.
Most notably, JavaScript shares with Lua the use of tests of type tags
as a way of encoding optional and overloaded parameters, and the
proliferation of class-based object systems built on top of the
language's delegation mechanisms.

Typed Lua should also be a major contribution to the Lua community,
because it offers a framework that programmers can use to document,
test, and better structure their applications.
For libraries where a full conversion to static type checking should
prove unfeasible or too much work, the community can use Typed Lua
just to document the external interfaces of the libraries,
giving the benefits of static type checking to the users of these
libraries.

Finally, Typed Lua can be a base for further research and development
of static type systems for the Lua language, such as the use of static
analysis of Lua API calls for checking that a native library is
exporting the interface to Lua code that it declares, the use of
effect types for typing Lua coroutines, and the use of static types
for code optimization.

