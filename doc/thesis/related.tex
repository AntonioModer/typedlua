In this chapter we review related work, and we split it into two sections:
in the first section we review other Lua projects,
while in the second section we review other projects that are not related to Lua.

\section{Other Lua projects}

Metalua \cite{metalua} is a Lua compiler that supports compile-time
metaprogramming (CTMP).
CTMP is a kind of macro system that allows the programmers to interact
with the compiler \cite{fleutot2007contrasting}.
Metalua extends Lua 5.1 syntax to include its macro system,
and allows programmers to define their own syntax.
Metalua can provide syntactical support for several object-oriented
styles, and can also provide syntax for turning simple type
annotations into run-time assertions.

MoonScript \cite{moonscript} is a programming language that supports
class-based object-oriented programming.
MoonScript compiles to idiomatic Lua code, but
it does not perform compile-time type checking.

LuaInspect \cite{luainspect} is a tool that uses MetaLua to perform
some code analysis.
For instance, it flags unknown global variables and table fields,
it checks the number of function arguments against signatures, and
it infers function return values.
However, it does not try to analyze object-oriented code and
it does not perform compile-time type checking.

Tidal Lock \cite{tidallock} is a prototype of another optional type
system for Lua, which is written in Metalua.
Tidal Lock covers a little subset of Lua.
Statements include declaration of local variables, multiple assignment,
function application, and the return statement.
This means that Tidal Lock does not include any control-flow statement.
Expressions include primitive literals, table indexing, function application,
function declaration, and the table constructor, but they do not include
binary operations.

A remarkable feature of Tidal Lock is the refinement of table types.
This feature inspired us to also include it in Typed Lua,
but in a simpler way and with different formalization.

The table type from Tidal Lock can only represent records, that is,
it cannot describe hash tables and arrays yet, though we can refine them.
Tidal Lock also includes field types to describe the type of the fields
of a table type.
The field types describe if a table field is mutable or immutable
in a table type.
Field types are the feature that allow the refinement of table types in
Tidal Lock.

Tidal Lock is also a structural type system that relies on subtyping and
local type inference.
However, it does not support union types, recursive types, and variadic types.
It also does not type any object-oriented idiom.

Sol \cite{sol} is an experimental optional type system for Lua.
Its type system is similar to ours, as it includes literal types,
union types, and function types that handle variadic functions.
However, it does not handle the refinement of tables and it
includes different types for tables.
Sol types tables as lists, maps, and objects.
Its object types handle a specific object-oriented idiom that
Sol introduces.

Lua Analyzer \cite{luaanalyzer} is an optional type system for Lua
that is specially designed to work in the Löve Studio,
an IDE for game developing using the Löve framework.
It works in Lua 5.1 only, and uses type annotations inside comments.
It is unsound by design because its dynamic type is both
top and bottom in the subtyping relation.

Lua Analyzer shares some features with Typed Lua, and also
has some interesting features that we do not have in Typed Lua.
It has similar rules for handling the \texttt{or} idiom and
discriminating union types inside conditions.
However, these rules are limited to the \texttt{nil} tag only.
It also includes different types for typing tables.
It includes regular record types that maps names to types,
array types, and map types.
Even though it does not support the refinement of tables,
it allows the definition of nominal table types that simulate classes.
This system allows it to type check custom class systems,
which are common in Lua.
Function types also support multiple return values and
variadic functions, but they do not support overloading the
return type.
Recently, it included experimental support for type aliases and generics.

Luacheck \cite{luacheck} is a tool that performs static analysis on Lua code.
It can flag access to undeclared globals and unused local variables,
but it does not perform static type checking.

Ravi \cite{ravi} is an experimental Lua dialect.
Ravi introduces optional static typing for Lua to improve run-time performance.
To do that, Ravi extends the Lua Virtual Machine to include new
operations that take into account static type information.
Currently, Ravi extends the Lua Virtual Machine to support few types:
\texttt{integer}, \texttt{number}, arrays of integers, and arrays of numbers.

\section{Other projects}

Typed Racket \cite{tobin-hochstadt2008ts} is a statically typed version
of the Racket language, which is a Scheme dialect.
The main purpose of Typed Racket is to allow programmers to combine
untyped modules, which are written in Racket, with typed modules, which are
written in Typed Racket.
It also uses local type inference to deduce the type of unannotated expressions.

The main feature of Typed Racket's type system is \emph{occurrence typing}
\cite{tobin-hochstadt2010ltu}.
It is a novel way to use type predicates in control flow statements
to refine union types.
Occurrence typing is not sound in the presence of mutation.
As these kinds of checks are common in other languages, related systems
have appeared \cite{guha2011tlc,winther2011gtp,pearce2013ccf}.

The type system of Typed Racket also includes function types, recursive
types, and structure types.
Its function types also handle multiple return values, and there is
also a way to describe function types that have optional arguments.
Its structure types are similar to our interfaces, as they describe record types.
The type system is also structural and based on subtyping.
It also includes the dynamic type \texttt{Any}, which is the top type in the system.
Typed Racket also supports polymorphic functions and data structures.

Typed Clojure \cite{bonnaire-sergeant2012typed-clojure} is an
optional type system for Clojure.
Although Clojure is a Lisp dialect that runs on the Java Virtual Machine,
Common Language Runtime, and JavaScript, Typed Clojure runs only on
the Java Virtual Machine.
Perhaps, this restriction pushed Typed Clojure to support Java classes
and some Java types such as \texttt{Long}, \texttt{Double}, and \texttt{String}.
Typed Clojure also provides optional type annotations and uses
local type inference to deduce the type of unannotated expressions.
It also assigns the type \texttt{Any} to unannotated function parameters,
which is the top type in the type system.

The type system of Typed Clojure includes polymorphic function types,
union types, intersection types, lists, vectors, maps, sets, and recursive types.
Function types can also have rest parameters, which are similar
to our variadic types, but can only appear on the input parameter
of function types.
In fact, its function types cannot return multiple results.
It also uses occurrence typing to allow control flow statements to
refine union types.
The type system is also structural and based on subtyping.

Dart \cite{dart} is a new class-based object-oriented programming
language.
It includes optional type annotations and compiles to JavaScript.
The type system of Dart is nominal and includes base types,
function types, lists, and maps.
It also supports generics, and the programmer can define
generic functions, lists, and maps.
Unlike Typed Lua, Dart is unsound by design.

Even though Dart has optional typing and static types by
default do not affect run-time semantics, it has an
execution mode that affects run-time.
The \emph{checked mode} inserts run-time assertions that
verifies whether static types match run-time tags.
The \emph{production mode} is the default execution mode
that does not include any assertions.

TypeScript \cite{typescript} is a JavaScript extension
that includes optional type annotations and class-based
object-oriented programming.
It also uses local type inference to deduce the type
of unannotated expressions.
The type system of TypeScript is structural, based
on subtyping, and supports generics.
It includes the dynamic type, primitive types, union types,
function types, array types, tuple types, recursive types, and
object types.
Unlike Typed Lua, TypeScript uses arrays to represent variadic
functions and multiple return values.

Even though TypeScript is unsound by design,
Bierman et al. \cite{bierman2014typescript} shows how to
make TypeScript sound.
They use a reduced core of TypeScript to formalize a
sound type system for TypeScript, but also to formalize
its current unsound type system.

TeJaS \cite{lerner2013tejas} is a framework for the construction of
different type systems for JavaScript.
The authors created a base type system for JavaScript with
extensible typing rules that allow the experimentation of
different static analysis.
They used TeJaS to create a type system that simulates the
type system of TypeScript.

Politz et al. \cite{politz2012semantics} proposes semantics
and types for objects with first-class member names, a well-known
feature from scripting languages.
Their type system uses string patterns to describe the members of
an object, and define a complex subtyping relation to validate
these patterns.
They also provide an implementation of their system to JavaScript.

Gradualtalk \cite{allende2013gts} is a Smalltalk dialect that
supports gradual typing.
The type system combines nominal and structural typing.
It includes function types, union types, structural types,
nominal types, a self type, and parametric polymorphism.
The type system also relies on subtyping and consistent-subtyping.

Gradualtalk inserts run-time checks that ensure dynamically
typed code does not violate statically typed code.
Allende et al. \cite{allende2013cis} perform a careful
evaluation about cast insertion in Gradualtalk.
They report that usually cast insertions impact on execution
performance, so Gradualtalk also has an option that allows
programmers to turn them off, downgrading Gradualtalk
to an optional type system.

Reticulated Python \cite{vitousek2014deg} is a Python compiler
that supports gradual typing.
The type system is structural and based on subtyping.
It includes base types, the dynamic type, list types,
dictionary types, tuple types, function types, set types,
object types, class types, and recursive types.
It includes class and object types to differentiate the
type of class declarations and instances, respectively.
It also uses local type inference.
Besides static type checking, Reticulated Python also introduces
three different approaches for inserting run-time assertions.

Mypy \cite{mypy} is an optional type system for Python.
The type system of mypy is similar to the type system of
Reticulated Python, but mypy does not insert run-time checks
and it has parametric polymorphism.
In contrast, Reticulated Python can type variadic functions,
but mypy cannot.
Recently, Guido van Rossum, Python's author, proposed a
standard syntax for type annotations in Python \cite{PEP483}
that is extremely inspired by mypy \cite{PEP484}.
The main goal of this proposal is to make easier building
static analysis tools for Python.
Typing \cite{typing} is a tool that is being developed to
implement this proposal.

Hack \cite{hack} is a new programming language that runs on the
Hip Hop Virtual Machine (HHVM).
The HHVM is a virtual machine that executes Hack and PHP programs.
We can view Hack as an extension to PHP that combines static and
dynamic typing.
The type system of Hack includes generics, nullable types, collections,
and function types.

The Ruby Type Checker \cite{ren2013rtc} is a library that
performs type checking during run-time.
The library provides type annotations that the programmer
can use on classes and methods.
Its type system includes nominal types, union types,
intersection types, method types, parametric polymorphism,
and type casts.

Grace \cite{black2013sg} is an object-oriented language
with optional typing.
Grace is not a dynamically typed language that has been
extended with an optional type system, but a language
that has been designed from scratch to have both
static and dynamic typing.
Homer et al. \cite{homer2013modules} explores some
useful patterns that derive from Grace's use of objects as modules
and its brand of optional structural typing, which
can also be expressed with Typed Lua's modules as tables.
