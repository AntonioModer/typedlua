\documentclass[phd,oneside,british]{ThesisPUC_uk}

\usepackage[utf8]{inputenc}
\usepackage{amsmath}
\usepackage{amssymb}
\usepackage{url}
\usepackage{color}
\usepackage{multirow}

\newcommand{\Value}{\mathbf{value}}
\newcommand{\Any}{\mathbf{any}}
\newcommand{\Nil}{\mathbf{nil}}
\newcommand{\Self}{\mathbf{self}}
\newcommand{\False}{\mathbf{false}}
\newcommand{\True}{\mathbf{true}}
\newcommand{\Boolean}{\mathbf{boolean}}
\newcommand{\Integer}{\mathbf{integer}}
\newcommand{\Number}{\mathbf{number}}
\newcommand{\String}{\mathbf{string}}
\newcommand{\Void}{\mathbf{void}}
\newcommand{\Const}{\mathbf{const}}

\newcommand{\mylabel}[1]{\; (\textsc{#1})}
\newcommand{\env}{\Gamma}
\newcommand{\penv}{\Pi}
\newcommand{\senv}{\Sigma}
\newcommand{\subtype}{<:}
\newcommand{\ret}{\rho}
\newcommand{\self}{\sigma}

\def\dstart{\hbox to \hsize{\vrule depth 4pt\hrulefill\vrule depth 4pt}}
\def\dend{\hbox to \hsize{\vrule height 4pt\hrulefill\vrule height 4pt}}

\author{André Murbach Maidl}
\authorR{Maidl, André Murbach}
\adviser{Roberto Ierusalimschy}
\adviserR{Ierusalimschy, Roberto}
\coadviser{Fabio Mascarenhas de Queiroz}
\coadviserR{Queiroz, Fabio Mascarenhas de}
\coadviserInst{UFRJ}

\title{Typed Lua: An Optional Type System for Lua}
\titlebr{Typed Lua: um sistema de tipos opcional para Lua}

\day{10} \month{April} \year{2015}

\city{Rio de Janeiro}
\CDD{004}
\department{Informática}
\program{Informática}
\school{Centro Técnico Científico}
\university{Pontifícia Universidade Católica do Rio de Janeiro}
\uni{PUC--Rio}

\jury {
\jurymember{Ana Lúcia de Moura}
           {Departamento de Informatica --- PUC-Rio}
\jurymember{Edward Hermann Haeusler}
           {Departamento de Informática --- PUC-Rio}
\jurymember{Anamaria Martins Moreira}
           {UFRJ}
\jurymember{Roberto da Silva Bigonha}
           {UFMG}
\schoolhead{José Eugênio Leal}
}

\resume{
}

\acknowledgment{
}

\abstract{
Dynamically typed languages such as Lua avoid static types in favor of
simplicity and flexibility, because the absence of static types means
that programmers do not need to bother with abstracting types that
should be validated by a type checker.
In contrast, statically typed languages provide the early detection of
many bugs, and a better framework for structuring large programs.
These are two advantages of static typing that may lead programmers
to migrate from a dynamically typed to a statically typed language,
when their simple scripts evolve into complex programs.

Optional type systems allow combining dynamic and static typing in
the same language, without affecting its original semantics,
making easier this code evolution from dynamic to static typing.
Designing an optional type system for a dynamically typed language
is challenging, as it should feel natural to programmers that are
already familiar with this language.

In this work we present and formalize the design of Typed Lua,
an optional type system for Lua that introduces novel features
to statically type check some Lua idioms and features.
Even though Lua shares several characteristics with other dynamically
typed languages such as JavaScript, Lua also has several unusual features
that are not present in the type system of these languages.
These features include functions with flexible arity, multiple assignment,
functions that are overloaded on the number of return values, and the
incremental evolution of record and object types.
We discuss how Typed Lua handles these features and our design decisions.
Finally, we present the evaluation results that we achieved while using
Typed Lua to type existing Lua code.

}

\keywords{
\key{Scripting languages}
\key{Lua}
\key{Type systems}
\key{Optional type systems}
\key{Gradual typing}
}

\abstractbr{
Linguagens dinamicamente tipadas, tais como Lua, não usam tipos estáticos em
favor de simplicidade e flexibilidade, porque a ausência de tipos estáticos
significa que programadores não precisam se preocupar em abstrair tipos que
devem ser validados por um verificador de tipos.
Por outro lado, linguagens estaticamente tipadas ajudam na detecção prévia de
diversos \emph{bugs} e também ajudam na estruturação de programas grandes.
Tais pontos geralmente são vistos como duas vantagens que levam programadores
a migrar de uma linguagem dinamicamente tipada para uma linguagem estaticamente tipada,
quando os pequenos \emph{scripts} deles evoluem para programas complexos.
Sistemas de tipos opcionais nos permitem combinar tipagem dinâmica e estática na
mesma linguagem, sem afetar a semântica original da linguagem, tornando mais
fácil a evolução de código tipado dinamicamente para código tipado estaticamente.
Desenvolver um sistema de tipos opcional para uma linguagem dinamicamente tipada é
uma tarefa desafiadora, pois ele deve ser o mais natural possível para os programadores
que já estão familiarizados com essa linguagem.
Neste trabalho nós apresentamos e formalizamos Typed Lua, um sistema de tipos opcional
para Lua, o qual introduz novas características para tipar estaticamente alguns idiomas
e características de Lua.
Embora Lua compartilhe várias características com outras linguagens dinamicamente
tipadas, em particular JavaScript, Lua também possui várias características não usuais,
as quais não estão presentes nos sistemas de tipos dessas linguagens.
Essas características incluem funções com aridade flexível, atribuições múltiplas,
funções que são sobrecarregadas no número de valores de retorno e
a evolução incremental de registros e objetos.
Nós discutimos como Typed Lua tipa estaticamente essas características e
também discutimos nossas decisões de projeto.
Finalmente, apresentamos uma avaliação de resultados,
a qual obtivemos ao usar Typed Lua para tipar código Lua existente.

}

\keywordsbr{
\key{Linguagens de script}
\key{Lua}
\key{Sistemas de tipos}
\key{Sistemas de tipos opcionais}
\key{Tipagem gradual}
}

\tablesmode{figtab}

\begin{document}

\chapter{Introduction}
\label{chap:intro}
Dynamically typed languages such as Lua avoid static types in favor of
simplicity and flexibility, because the absence of static types means
that programmers do not need to bother about abstracting types that
should be validated by a type checker.
Instead, dynamically typed languages use run-time \emph{type tags}
to classify the values they compute, so their implementation can use
these tags to perform run-time (or dynamic) type checking
\citep{pierce2002tpl}.

This simplicity and flexibility allow programmers to write code that
might require a quite complex type system to statically type check,
though it may also hide bugs that will be caught only after deployment
if programmers do not properly test their code.
In contrast, static type checking helps programmers detect many
bugs during the development phase.
Static types also provide a conceptual framework that helps
programmers define modules and interfaces that can be combined to
structure the development of programs.

Thus, early error detection and better program structure are two
advantages of static type checking that can lead programmers to
migrate their code from a dynamically typed to a statically
typed language, when their simple scripts evolve into complex programs
\citep{tobin-hochstadt2006ims}.
Dynamically typed languages certainly help programmers during the
beginning of a project, because their simplicity and flexibility
allow quick development and make it easier to change code according to
changing requirements.
However, programmers tend to migrate from dynamically typed to
statically typed code as soon as the project has consolidated its
requirements, because the robustness of static types helps
programmers link requirements to abstractions.
This migration usually involves different languages that have
different syntax and semantics, which usually require a complete
rewrite of existing programs instead of incremental evolution from
dynamic to static types.

Ideally, programming languages should offer programmers the
option to choose between static and dynamic typing:
\emph{optional type systems} \citep{bracha2004pluggable} and
\emph{gradual typing} \citep{siek2006gradual} are two similar
approaches for blending static and dynamic typing in the same
language.
The aim of both approaches is to offer programmers the option
to use type annotations where static typing is needed and to not use
them where dynamic typing is sufficient.
The difference between these two approaches is the way they treat
run-time semantics.
While optional type systems do not affect the run-time semantics,
gradual typing uses run-time checks to ensure that dynamically typed
code does not violate the invariants of statically typed code.

Programmers and researchers sometimes use the term gradual typing
to mean the incremental evolution of dynamically typed code to
statically typed code.
For this reason, gradual typing may also refer to optional type
systems and other approaches that blend static and dynamic typing to
help programmers incrementally migrate from dynamic to static typing
without having to switch to a different language, though all these
approaches differ in the way they handle static and dynamic typing
together.
We use the term gradual typing for referring to the work of
\citet{siek2006gradual}.

In this work we present the design and evaluation of Typed Lua:
an optional type system for Lua that is rich enough to
preserve some of the Lua idioms that programmers are already used to,
but that also includes new constructs that help programmers
structure Lua programs.

Lua is a small scripting language, so designing an optional type
system for Lua may shed some light on optional type systems for
scripting languages in general.
Lua provides mechanisms for organizing a program in modules and
writing object-oriented programs, but it does not set policy on how
these features should behave, due to its use as an embedded and
extension language.
Thus, implementing an optional type system for Lua offers Lua
programmers one way to obtain most of the advantages of static typing
without compromising the simplicity and flexibility of dynamic typing.

So far, Typed Lua is a strict superset of Lua that
provides optional type annotations, compile-time type checking, and
class-based object-oriented programming through the definition of
classes, interfaces, and modules.
In practice, we implement Typed Lua as a programming language that
extends Lua syntax to add optional type annotations and standard
constructions for structuring Lua code.
The compiler uses static types to perform compile-time type
checking, but also allows Lua code to coexist with Typed Lua code,
and generates Lua code that runs in unmodified Lua implementations.
We have an implementation of the Typed Lua compiler that is
available online\footnote{https://github.com/andremm/typedlua}.

Typed Lua intended use is as an application language, and we view
that policies for organizing a program in modules and writing
object-oriented programs should be part of the language and
enforced by its optional type system.
An application language is a programming language that helps
programmers develop applications from scratch until these
applications evolve to complex systems rather than just scripts.

We also believe that Typed Lua help programmers give a more
formal documentation to already existing Lua code, as static types
are also a useful source of documentation in languages that provide
type annotations, because type annotations are always validated by
the type checker and therefore never get outdated.
Thus, programmers can use Typed Lua to define axioms about the
interfaces and types of dynamically typed modules.
We enforce this point by using Typed Lua to statically type
the interface of the Lua standard libraries and other commonly used
Lua libraries, so our compiler can check their usage by Typed Lua
code.

Typed Lua performs a very limited form of local type inference
\citep{pierce2000lti}, as static typing does not necessarily mean
that programmers need to insert type annotations in the code.
Several statically typed languages such as Haskell provide some
amount of type inference that automatically deduces the types of
expressions.
Still, Typed Lua needs a small amount of type annotations due
to the nature of its optional type system.

Typed Lua does not deal with code optimization, although another
important advantage of static types is that they help the compiler
perform optimizations and generate more efficient code.
However, we believe that the formalization of our optional type
system is precise enough to aid optimization in some Lua implementations.

We use some of the ideas of gradual typing to formalize Typed Lua.
Even though Typed Lua is an optional type system and thus does not
include run-time checks between dynamic and static regions of the
code, we believe that using the foundations of gradual typing to
formalize our optional type system will allow us to include run-time
checks in the future.

In Chapter \ref{chap:review} we review the literature about static and
dynamic typing in the same language, which includes the discussion
about the particularities between optional type systems and gradual
typing.
In Chapter \ref{chap:typedlua} we use code examples to present the
design of Typed Lua.
In Chapther \ref{chap:system} we use typing rules to present the
formalization of Typed Lua.
In Chapter \ref{chap:evaluation} we discuss some case studies that
we used to evaluate our design.
In Chapter \ref{chap:conc} we outline our contributions.



\chapter{Blending static and dynamic typing}
\label{chap:review}
We begin this chapter presenting a little bit of the history behind
combining static and dynamic typing in the same language.
Then, we introduce optional type systems and gradual typing.
After that, we discuss why optional type systems and two
other approaches are often called gradual typing.
We end this chapter presenting some statistics about the usage of
some Lua features and idioms that helped us identify how we should
combine static and dynamic typing in Lua.

\section{A little bit of history}
\label{sec:history}

Common LISP \cite{steele1982ocl} introduced optional type annotations
in the early eighties, but not for static type checking.
Instead, programmers could choose to declare types of variables as
optimization hints to the compiler, that is, type declarations are
just one way to help the compiler to optimize code.
These annotations are unsafe because they can crash the program
when they are wrong.

Abadi and associates \cite{abadi1989dts} extended the simply typed
lambda calculus with the \texttt{Dynamic} type and the \texttt{dynamic}
and \texttt{typecase} constructs, with the aim to safely integrate dynamic
code in statically typed languages.
The \texttt{Dynamic} type is a pair \texttt{(v,T)} where \texttt{v} is a
value and \texttt{T} is the tag that represents the type of \texttt{v}.
The constructs \texttt{dynamic} and \texttt{typecase} are explicit
injection and projection operations, respectively.
That is, \texttt{dynamic} builds values of type \texttt{Dynamic} and
\texttt{typecase} safely inspects the type of a \texttt{Dynamic} value.
Thus, migrating code between dynamic and static type checking requires
changing type annotations and adding or removing \texttt{dynamic} and
\texttt{typecase} constructs throughout the code.

The \emph{quasi-static} type system proposed by Thatte \cite{thatte1990qst}
performs implicit coercions and run-time checks to replace the
\texttt{dynamic} and \texttt{typecase} constructs that were proposed by
Abadi and associates \cite{abadi1989dts}.
To do that, quasi-static typing relies on subtyping with a top type
$\Omega$ that represents the dynamic type, and splits type checking
into two phases.
The first phase inserts implicit coercions from the dynamic type to
the expected type, while the second phase performs what Thatte calls
\emph{plausibility checking}, that is, it rewrites the program to
guarantee that sequences of upcasts and downcasts always have a
common subtype.

\emph{Soft typing} \cite{cartwright1991soft} is another approach
to combine static and dynamic typing in the same language.
The main goal of soft typing is to add static type checking to
dynamically typed languages without compromising their flexibility.
To do that, soft typing relies on type inference for
translating dynamically typed code to statically typed code.
The type checker inserts run-time checks around inconsistent code and
warns the programmer about the insertion of these run-time checks,
as they indicate the existence of potential type errors.
However, the programmer is free to choose between inspecting the
run-time checks or simply running the code.
This means that type inference and static type checking do
not prevent the programmer from running inconsistent code.
One advantage of soft typing is the fact that the compiler for
softly typed languages can use the translated code to generate
more efficient code, as the translated code statically type checks.
One disadvantage of soft typing is that it can be cumbersome when
the inferred types are meaningless large types that just confuse the
programmer.

\emph{Dynamic typing} \cite{henglein1994dts} is an approach
that optimizes code from dynamically typed languages by eliminating
unnecessary checks of tags.
Henglein describes how to translate dynamically typed code into
statically typed code that uses a \texttt{Dynamic} type.
The translation is done through a coercion calculus that uses type
inference to insert the operations that are necessary to type check
the \texttt{Dynamic} type during run-time.
Although soft typing and dynamic typing may seem similar, they are not.
Soft typing targets statically type checking of dynamically typed
languages for detecting programming errors, while
dynamic typing targets the optimization of dynamically
typed code through the elimination of unnecessary run-time checks.
In other words, soft typing sees code optimization as a side effect
that comes with static type checking.

Findler and Felleisen \cite{findler2002chf} proposed contracts for
higher-order functions and blame annotations for run-time checks.
Contracts perform dynamic type checking instead of static type checking,
but deferring all verifications to run-time can lead to defects
that are difficult to fix, because run-time errors can show a
stack trace where it is not clear to programmers if the cause of a
certain run-time error is in application code or library code.
Even if programmers identify that the source of a certain run-time
error is in library code, they still may have problems to identify
if this run-time error is due to a violation of library's contract
or due to a bug, when the library is poorly documented.
In this approach, programmers can insert assertions in the form of
contracts that check the input and output of higher-order functions;
and the compiler adds blame annotations in the generated code
to track assertion failures back to the source of the error.

BabyJ \cite{anderson2003babyj} is an object-oriented language
without inheritance that allows programmers to incrementally annotate
the code with more specific types.
Programmers can choose between using the dynamically typed version
of BabyJ when they do not need types at all, and the statically
typed version of BabyJ when they need to annotate the code.
In statically typed BabyJ, programmers can use the
\emph{permissive type} $*$ to annotate the parts of the code that
still do not have a specific type or the parts of the code that should
have dynamic behavior.
The type system of BabyJ is nominal, so types are either class names
or the permissive type $*$.
However, the type system does not use type equality or subtyping,
but the relation $\approx$ between two types.
The relation $\approx$ holds when both types have the same name or
any of them is the permissive type $*$.
Even though the permissive type $*$ is similar to the dynamic type
from previous approaches, BabyJ does not provide any way to add
implicit or explicit run-time checks.

Ou and associates \cite{ou2004dtd} specified a language that combines
static types with dependent types.
To ensure safety, the compiler automatically inserts coercions
between dependent code and static code.
The coercions are run-time checks that ensure static code does not
crash dependent code during run-time.

\section{Optional Type Systems}
\label{sec:optional}

Optional type systems \cite{bracha2004pluggable} are an approach for
plugging static typing in dynamically typed languages.
They use optional type annotations to perform compile-time type checking,
though they do not influence the original run-time semantics
of the language.
This means that the run-time semantics should still catch type errors
independently of the static type checking.
For instance, we can view the typed lambda calculus as an optional
type system for the untyped lambda calculus, because both have the
same semantic rules and the type system serves only for discarding
programs that may have undesired behaviors \cite{bracha2004pluggable}.

Strongtalk \cite{bracha1993strongtalk,bracha1996strongtalk} is
a version of Smalltalk that comes with an optional type system.
It has a polymorphic type system that programmers can use to annotate
Smalltalk code or leave type annotations out.
Strongtalk assigns a dynamic type to unannotated expressions and allows
programmers to cast unannotated expressions to any static type.
This means that the interaction of the dynamic type with the rest of
the type system is unsound, so Strongtalk uses the original run-time
semantics of Smalltalk when executing programs, even if programs are
statically typed.

\emph{Pluggable type systems} \cite{bracha2004pluggable} generalize
the idea of optional type systems that Strongtalk put in practice.
The idea is to have different optional type systems that can be layered
on top of a dynamically typed language without affecting its original
run-time semantics.
Although these systems can be unsound in their interaction with the
dynamically typed part of the language or even by design, their
unsoundness does not affect run-time safety, as the language run-time
semantics still catches any run-time errors caused by an unsound
type system.

Dart \cite{dart} and TypeScript \cite{typescript} are new
languages that are designed with an optional type system.
Both use JavaScript as their code generation target because
their main purpose is Web development.
In fact, Dart is a new class-based object-oriented language with
optional type annotations and semantics that resembles the
semantics of Smalltalk, while TypeScript is a strict superset of
JavaScript that provides optional type annotations and class-based
object-oriented programming.
Dart has a nominal type system, while TypeScript has a structural
one, but both are unsound by design.
For instance, Dart has covariant arrays, while TypeScript has
covariant parameter types in function signatures,
besides the interaction between statically and dynamically
typed code that is also unsound.

There is no common formalization for optional type systems, and
each language ends up implementing its optional type system in
its own way.
Strongtalk, Dart, and TypeScript provide an informal description of
their optional type systems rather than a formal one.
In the next section we will show that we can use some features
of gradual typing \cite{siek2006gradual,siek2007objects} to
formalize optional type systems.

\section{Gradual Typing}
\label{sec:gradual}

The main goal of gradual typing \cite{siek2006gradual} is to allow
programmers to choose between static and dynamic typing in the same
language.
To do that, Siek and Taha \cite{siek2006gradual} extended the simply
typed lambda calculus with the dynamic type $?$, as we can see in
Figure \ref{fig:gtlc}.
In gradual typing, type annotations are optional, and an untyped
variable is syntactic sugar for a variable whose declared type is
the dynamic type $?$, that is, $\lambda x.e$ is equivalent to $\lambda x{:}?.e$.
Under these circumstances, we view gradual typing as a way to add
a dynamic type to statically typed languages.

\begin{figure}[!ht]
\dstart
$$
\begin{array}{llr}
T ::= & & \textsc{types:}\\
& \;\; \Number & \textit{base type number}\\
& | \; \String & \textit{base type string}\\
& | \; ? & \textit{dynamic type}\\
& | \; T \rightarrow T & \textit{function types}\\
e ::= & & \textsc{expressions:}\\
& \;\; l & \textit{literals}\\
& | \; x & \textit{variables}\\
& | \; \lambda x{:}T{.}e & \textit{abstractions}\\
& | \; e_{1} e_{2} & \textit{application}
\end{array}
$$
\dend
\caption{Syntax of the gradually-typed lambda calculus}
\label{fig:gtlc}
\end{figure}

The central idea of gradual typing is the \emph{consistency}
relation, written $T_{1} \sim T_{2}$.
The consistency relation allows implicit conversions to and from the
dynamic type, and disallows conversions between inconsistent types
\cite{siek2006gradual}.
For instance, $\Number \sim \;?$, $? \sim \Number$,
$\String \sim \;?$, and $? \sim \String$,
but $\Number \not\sim \String$, and
$\String \not\sim \Number$.
The consistency relation is both reflexive and symmetric, but
it is neither commutative nor transitive.

\begin{figure}[!ht]
\dstart
$$
\begin{array}{c}
\begin{array}{c}
T \sim T \mylabel{C-REFL}
\end{array}
\;
\begin{array}{c}
T \sim \;? \mylabel{C-DYNR}
\end{array}
\;
\begin{array}{c}
? \sim T \mylabel{C-DYNL}
\end{array}
\\ \\
\begin{array}{c}
\dfrac{T_{3} \sim T_{1} \;\;\; T_{2} \sim T_{4}}
      {T_{1} \rightarrow T_{2} \sim T_{3} \rightarrow T_{4}} \mylabel{C-FUNC}
\end{array}
\end{array}
$$
\dend
\caption{The consistency relation}
\label{fig:consistency}
\end{figure}

Figure \ref{fig:consistency} defines the consistency relation.
The rule \textsc{C-REFL} is the reflexive rule.
Rules \textsc{C-DYNR} and \textsc{C-DYNL} are the rules that allow
conversions to and from the dynamic type $?$.
The rule \textsc{C-FUNC} resembles subtyping between function types,
because it is contravariant on the argument type and covariant on the
return type.

Figure \ref{fig:gts} uses the consistency relation in the typing rules
of the gradual type system of the simply typed lambda calculus extended
with the dynamic type $?$.
The environment $\env$ is a function from variables to types, and
the directive $type$ is a function from literal values to types.
The rule \textsc{T-VAR} uses the environment function $\env$ to get the
type of a variable $x$.
The rule \textsc{T-LIT} uses the directive $type$ to get the type of
a literal $l$.
The rule \textsc{T-ABS} evaluates the expression $e$ with an environment
$\env$ that binds the variable $x$ to the type $T_{1}$, and the resulting
type is the the function type $T_{1} \rightarrow T_{2}$.
The rule \textsc{T-APP1} handles function calls where the type of a
function is dynamically typed; in this case, the argument type may have
any type and the resulting type has the dynamic type.
The rule \textsc{T-APP2} handles function calls where the type of a
function is statically typed; in this case, the argument type should
be consistent with the argument type of the function's signature.

\begin{figure}[!ht]
\dstart
$$
\begin{matrix}
\dfrac{\env(x) = T}
      {\env \vdash x:T} \mylabel{T-VAR}
\;\;\;
\dfrac{type(l) = T}
      {\env \vdash l:T} \mylabel{T-LIT}
\\ \\
\dfrac{\env[x \mapsto T_{1}] \vdash e:T_{2}}
      {\env \vdash \lambda x:T_{1}.e:T_{1} \rightarrow T_{2}} \mylabel{T-ABS}
\;\;\;
\dfrac{\env \vdash e_{1}:\;? \;\;\;
       \env \vdash e_{2}:T}
      {\env \vdash e_{1} e_{2}:\;?} \mylabel{T-APP1}
\\ \\
\dfrac{\env \vdash e_{1}:T_{1} \rightarrow T_{2} \;\;\;
       \env \vdash e_{2}:T_{3} \;\;\;
       T_{3} \sim T_{1}}
      {\env \vdash e_{1} e_{2}:T_{2}} \mylabel{T-APP2}
\end{matrix}
$$
\dend
\caption{Gradual type system gradually-typed lambda calculus}
\label{fig:gts}
\end{figure}

Gradual typing is similar to the approaches proposed by
Abadi and associates \cite{abadi1989dts} and Thatte \cite{thatte1990qst}
by including a dynamic type in a statically typed language.
However, these three approaches differ in the way they handle the
dynamic type.
While Siek and Taha \cite{siek2006gradual} rely on the consistency relation,
Abadi and associates \cite{abadi1989dts} rely on type equality with explicit
projections and injections, and Thatte \cite{thatte1990qst} relies on subtyping.

The subtyping relation $\subtype$ is actually a pitfall on Thatte's
quasi-static typing, because it sets the dynamic type
as the top and the bottom of the subtying relation:
$T \subtype \;?$ and $? \subtype T$.
Subtyping is transitive, so we know that
\[
\frac{\Number \subtype \;? \;\;\;
      ? \subtype \String}
     {\Number \subtype \String}
\]
Therefore, downcasts combined with the transitivity of subtyping
accepts programs that should be rejected.

Later, Siek and Taha \cite{siek2007objects} reported that the consistency relation
is orthogonal to the subtyping relation, so we can combine them to achieve
the \emph{consistent-subtyping} relation, written $T_{1} \lesssim T_{2}$.
This relation is essential for designing gradual type systems for
object-oriented languages.
Like the consistency relation, and unlike the subtyping relation,
the consistent-subtyping relation is not transitive.
Therefore, $\Number \lesssim \;?$, $? \lesssim \Number$,
$\String \lesssim \;?$, and $? \lesssim \String$,
but $\Number \not\lesssim \String$, and
$\String \not\lesssim \Number$.

Now, we will show how we can combine consistency and subtyping
to compose a consistent-subtyping relation for the simply typed
lambda calculus extended with the dynamic type $?$.

\begin{figure}[!ht]
\dstart
$$
\begin{array}{c}
\begin{array}{c}
\Number \subtype \Number \mylabel{S-NUM}
\end{array}
\;
\begin{array}{c}
\String \subtype \String \mylabel{S-STR}
\end{array}
\\ \\
\begin{array}{c}
? \subtype \;? \mylabel{S-ANY}
\end{array}
\;
\begin{array}{c}
\dfrac{T_{3} \subtype T_{1} \;\;\; T_{2} \subtype T_{4}}
      {T_{1} \rightarrow T_{2} \subtype T_{3} \rightarrow T_{4}} \mylabel{S-FUN}
\end{array}
\end{array}
$$
\dend
\caption{The subtyping relation}
\label{fig:subtyping}
\end{figure}

Figure \ref{fig:subtyping} presents the subtyping relation for the simply
typed lambda calculus extended with the dynamic type $?$.
Even though we could have used the reflexive rule $T \subtype T$ to express
the rules \textsc{S-NUM}, \textsc{S-STR}, and \textsc{S-ANY},
we did not combine them into a single rule to make explicit the
neutrality of the dynamic type $?$ to the subtyping rules.
The dynamic type $?$ must be neutral to subtyping to avoid the pitfall
from Thatte's quasi-static typing.
The rule \textsc{S-FUN} defines the subtyping relation for function types,
which are contravariant on the argument type and covariant on the return type.

\begin{figure}[!ht]
\dstart
$$
\begin{array}{c}
\begin{array}{c}
\Number \lesssim \Number \mylabel{C-NUM}
\end{array}
\;
\begin{array}{c}
\String \lesssim \String \mylabel{C-STR}
\end{array}
\\ \\
\begin{array}{c}
T \lesssim \;? \mylabel{C-ANY1}
\end{array}
\;
\begin{array}{c}
? \lesssim T \mylabel{C-ANY2}
\end{array}
\\ \\
\begin{array}{c}
\dfrac{T_{3} \lesssim T_{1} \;\;\; T_{2} \lesssim T_{4}}
      {T_{1} \rightarrow T_{2} \lesssim T_{3} \rightarrow T_{4}} \mylabel{C-FUN}
\end{array}
\end{array}
$$
\dend
\caption{The consistent-subtyping relation}
\label{fig:consistent-subtyping}
\end{figure}

Figure \ref{fig:consistent-subtyping} combines the consistency and subtyping
relations to compose the consistent-subtyping relation for the simply typed
lambda calculus extended with the dynamic type $?$.
When we combine consistency and subtyping, we are making subtyping handle
what casts are safe among static types, and we are making consistency
handle the casts that involve the dynamic type $?$.
The consistent-subtyping relation is not transitive, and thus
the dynamic type $?$ is not neutral to this relation.

So far, gradual typing looks like a mere formalization to optional
type systems, as a gradual type system uses the consistency or
consistent-subtyping relation to statically check the interaction
between statically and dynamically typed code, without influencing
the run-time semantics.

However, another important feature of gradual typing is the theoretic
foundation that it provides for inserting run-time checks that
prove dynamically typed code does not violate the invariants of
statically typed code, thus preserving type safety.
To do that, Siek and Taha \cite{siek2006gradual,siek2007objects}
defined the run-time semantics of gradual typing as a translation to an
intermediate language with explicit casts at the frontiers between
statically and dynamically typed code.
The semantics of these casts is based on the higher-order contracts
proposed by Findler and Felleisen \cite{findler2002chf}.

Herman and associates \cite{herman2007sgt} showed that there is an
efficiency concern regarding the run-time checks, because there are
two ways that casts can lead to unbounded space consumption.
The first affects tail recursion while the second appears when
first-class functions or objects cross the border between
static code and dynamic code, that is, some programs can apply
repeated casts to the same function or object.
Herman and associates \cite{herman2007sgt} use the coercion calculus
outlined by Henglein \cite{henglein1994dts} to express casts
as coercions and solve the problem of space efficiency.
Their approach normalizes an arbitrary sequence of coercions to a
coercion of bounded size.

Another concern about casts is how to improve debugging support,
because a cast application can be delayed and the error related
to that cast application can appear considerable distance
from the real error.
Wadler and Findler \cite{wadler2009wpc} developed \emph{blame calculus}
as a way to handle this issue, and Ahmed and associates \cite{ahmed2011bfa}
extended blame calculus with polymorphism.
Blame calculus is an intermediate language to integrate
static and dynamic typing along with the blame tracking approach
proposed by Findler and Felleisen \cite{findler2002chf}.

On the one hand, blame calculus solves the issue regarding
error reporting;
on the other hand, it has the space efficiency problem reported
by Herman and associates \cite{herman2007sgt}.
Thus, Siek and associates \cite{siek2009casts} extended the coercion
calculus outlined by Herman and associates \cite{herman2007sgt} with
blame tracking to achieve an implementation of the blame calculus that
is space efficient.
After that, Siek and Wadler \cite{siek2010blame} proposed a new solution
that also handles both problems.
This new solution is based on a concept called \emph{threesome},
which is a way to split a cast between two parties into two casts
among three parties.
A cast has a source and a target type (a \emph{twosome}),
so we can split any cast into a downcast from the source to an
intermediate type that is followed by an upcast from the intermediate
type to the target type (a \emph{threesome}).

There are some projects that incorporate gradual typing into some
programming languages.
Reticulated Python \cite{reticulated,vitousek2014deg} is a research
project that evaluates the costs of gradual typing in Python.
Gradualtalk \cite{allende2013gts} is a gradually-typed Smalltalk
that introduces a new cast insertion strategy for gradually-typed
objects \cite{allende2013cis}.
Grace \cite{black2012grace,black2013sg} is a new object-oriented,
gradually-typed, educational language.
In Grace, modules are gradually-typed objects, that is, modules
may have types and methods as attributes, and can have a state
\cite{homer2013modules}.
ActionScript \cite{moock2007as3} is one the first languages that
incorporated gradual typing to its implementation and
Perl 6 \cite{tang2007pri} is also being designed with gradual typing,
though there is few documentation about the gradual type systems
of these languages.

\section{Approaches that are often called Gradual Typing}
\label{sec:approaches}

Gradual typing is similar to optional type systems in that type
annotations are optional, and unannotated code is dynamically
typed, but unlike optional type systems, gradual typing changes
the run-time semantics to preserve type safety, and it is a way to
add a dynamic type to statically typed languages.
More precisely, programming languages that include a gradual type
system implement the semantics of statically typed languages, so
the gradual type system inserts casts in the translated code to
guarantee that types are consistent before execution, while
programming languages that include an optional type system still
implement the semantics of dynamically typed languages, so all
the type checking also belongs to the semantics of each operation.

Still, we can view gradual typing as a way to formalize an optional
type system when the gradual type system does not insert run-time
checks.
BabyJ \cite{anderson2003babyj} and Alore \cite{lehtosalo2011alore}
are two examples of object-oriented languages that have an
optional type system with a formalization that relates to gradual typing,
though the optional type systems of both BabyJ and Alore are nominal.
BabyJ uses the relation $\approx$ that is similar to the consistency
relation while Alore combines subtyping along with the consistency
relation to define a \emph{consistent-or-subtype} relation.
The consistent-or-subtype relation is different from the
consistent-subtyping relation of Siek and Taha \cite{siek2007objects},
but it is also written $T_{1} \lesssim T_{2}$.
The consistent-or-subtype relation holds when $T_{1} \sim T_{2}$
or $T_{1} <: T_{2}$, where $<:$ is transitive and $\sim$ is not.
Alore also extends its optional type system to include optional
monitoring of run-time type errors in the gradual typing style.

Hence, optional type annotations for software evolution are likely
the reason why optional type systems are commonly called
gradual type systems.
Typed Clojure \cite{bonnaire-sergeant2012typed-clojure} is an
optional type system for Clojure that is now adopting the
gradual typing slogan.

Flanagan \cite{flanagan2006htc} introduced \emph{hybrid type checking},
an approach that combines static types and \emph{refinement} types.
For instance, programmers can specify the refinement type
$\{x:Int \;|\; x \ge 0\}$ when they need a type for natural numbers.
The programmer can also choose between explicit or implicit casts.
When casts are not explicit, the type checker uses a theorem prover
to insert casts.
In our example of natural numbers, a cast would be inserted to check
whether an integer is greater than or equal to zero.

Sage \cite{gronski2006sage} is a programming language that
extends hybrid type checking with a dynamic type to
support dynamic and static typing in the same language.
Sage also offers optional type annotations in the gradual typing
style, that is, unannotated code is syntactic sugar for
code whose declared type is the dynamic type.

Thus, the inclusion of a dynamic type in hybrid type checking
along with optional type annotations, and the insertion of run-time
checks are likely the reason why hybrid type checking is
also viewed as a form of gradual typing.

Tobin-Hochstadt and Felleisen \cite{tobin-hochstadt2006ims} proposed
another approach for gradually migrating from dynamically typed to
statically typed code, and they coined the term
\emph{from scripts to programs} for referring to this kind of
interlanguage migration.
In their approach, the migration from dynamically typed to
statically typed code happens module-by-module, so they designed
and implemented Typed Racket \cite{tobin-hochstadt2008ts} for
this purpose.
Typed Racket is a statically typed version of Racket
(a Scheme dialect) that allows the programmer to write typed modules,
so Typed Racket modules can coexist with Racket modules,
which are untyped.

The approach used by Tobin-Hochstadt and Felleisen \cite{tobin-hochstadt2008ts}
to design and implement Typed Racket is probably also called gradual typing
because it allows the programmer to gradually migrate from untyped
scripts to typed programs.
However, Typed Racket is a statically typed language,
and what makes it gradual is a type system with a dynamic type
that handles the interaction between Racket and Typed Racket modules.

\section{Statistics about the usage of Lua}
\label{sec:statistics}

In this section we present statistics about the usage of Lua
features and idioms.
We collected statistics about how programmers use tables, functions,
dynamic type checking, object-oriented programming, and modules.
We shall see that these statistics informed important design decisions
on our optional type system.

We used the LuaRocks repository to build our statistics database;
LuaRocks \cite{hisham2013luarocks} is a package manager for Lua
modules.
We downloaded the 3928 \texttt{.lua} files that were available in
the LuaRocks repository at February 1st 2014.
However, we ignored files that were not compatible with Lua 5.2,
the latest version of Lua at that time.
We also ignored \emph{machine-generated} files and test files,
because these files may not represent idiomatic Lua code,
and might skew our statistics towards non-typical uses of Lua.
This left 2598 \texttt{.lua} files from 262 different projects for
our statistics database;
we parsed these files and processed their abstract syntax tree
to gather the statistics that we show in this section.

To verify how programmers use tables, we measured how they
initialize, index, and iterate tables.
We present these statistics in the next three paragraphs to discuss
their influence on our type system.

The table constructor appears 23185 times.
In 36\% of the occurrences it is a constructor that initializes a
record (e.g., \texttt{\{ x = 120, y = 121 \}});
in 29\% of the occurrences it is a constructor that initializes a
list (e.g., \texttt{\{ "one", "two", "three", "four" \}});
in 8\% of the occurrences it is a constructor that initializes a
record with a list part;
and in less than 1\% of the occurrences (4 times) it is a constructor
that uses only the booleans \texttt{true} and \texttt{false} as indexes.
At all, in 73\% of the occurrences it is a constructor that uses
only literal keys;
in 26\% of the occurrences it is the empty constructor;
in 1\% of the occurrences it is a constructor with non-literal keys
only, that is, a constructor that uses variables and function calls
to create the indexes of a table;
and in less than 1\% of the occurrences (19 times) it is a constructor
that mixes literal keys and non-literal keys.

The indexing of tables appears 130448 times:
86\% of them are for reading a table field while
14\% of them are for writing into a table field.
We can classify the indexing operations that are reads as follows:
89\% of the reads use a literal string key,
4\% of the reads use a literal number key,
and less than 1\% of the reads (10 times) use a literal boolean key.
At all, 93\% of the reads use literals to index a table while
7\% of the reads use non-literal expressions to index a table.
It also worth mentioning that 45\% of the reads are actually
function calls.
More precisely, 25\% of the reads use literals to call a function,
20\% of the reads use literals to call a method, that is,
a function call that uses the colon syntactic sugar, 
and less than 1\% of the reads (195 times) use non-literal expressions
to call a function.
We can also classify the indexing operations that are writes as follows: 
69\% of the writes use a literal string key,
2\% of the writes use a literal number key,
and less than 1\% of the writes (1 time) uses a literal boolean key.
At all, 71\% of the writes use literals to index a table while
29\% of the writes use non-literal expressions to index a table.

We also measured how many files have code that iterate over tables to
observe how frequently iteration is used.
We observed that 23\% of the files have code that iterate over keys
of any value, that is, the call to \texttt{pairs} appears at least
once in these files (the median is twice per file);
21\% of the files have code that iterate over integer keys, that is,
the call to \texttt{ipairs} appears at least once in these files
(the median is also twice per file);
and 10\% of the files have code that use the numeric \texttt{for}
along with the length operator (the median is once per file).

The numbers about table initialization, indexing, and iteration
show us that tables are mostly used to represent records, lists,
and associative arrays.
Therefore, Typed Lua should include a table type for handling
these uses of Lua tables.
Even though the statistics show that programmers initialize tables
more often than they use the empty constructor to
dynamically initialize tables, the statistics of the empty
constructor are still expressive and indicate that Typed Lua should
also include a way to handle this style of defining table types.

We found a total of 24858 function declarations in our database
(the median is six per file).
Next, we discuss how frequently programmers use dynamic type
checking and multiple return values inside these functions.

We observed that 9\% of the functions perform dynamic type checking
on their input parameters, that is, these functions use \texttt{type}
to inspect the tags of Lua values (the median is once per function).
We randomly selected 20 functions to sample how programmers are
using \texttt{type}, and we got the following data:
50\% of these functions use \texttt{type} for asserting the tags of
their input parameters, that is, they raise an error when the tag of a
certain parameter does not match the expected tag, and
50\% of these functions use \texttt{type} for overloading, that is,
they execute different code according to the inspected tag.

These numbers show us that Typed Lua should include union types,
because the use of the \texttt{type} idiom shows that disjoint unions
would help programmers define data structures that can hold a value of
several different, but fixed types.
Typed Lua should also use \texttt{type} as a mechanism for decomposing
unions, though it may be restricted to base types only.

We observed that 10\% of the functions explicitly return multiple
values.
We also observed that 5\% of the functions return \texttt{nil} plus
something else, for signaling an unexpected behavior;
and 1\% of the functions return \texttt{false} plus something else,
also for signaling an unexpected behavior.

Typed Lua should include function types to represent Lua functions,
and tuple types to represent the signatures of Lua functions,
multiple return values, and multiple assignments.
Tuple types require some special attention, because Typed Lua
should be able to adjust tuple types during compile-time, in a
similar way to what Lua does with function calls and multiple
assignments during run-time.
In addition, the number of functions that return \texttt{nil} and
\texttt{false} plus something else show us that overloading on the
return type is also useful to the type system.

We also measured how frequently programmers use the object-oriented
paradigm in Lua.
We observed that 23\% of the function declarations are actually
method declarations.
More precisely, 14\% of them use the colon syntactic sugar while
9\% of them use \texttt{self} as their first parameter.
We also observed that 63\% of the projects extend tables with
metatables, that is, they call \texttt{setmetatable} at least once,
and 27\% of the projects access the metatable of a given table,
that is, they call \texttt{getmetatable} at least once.
In fact, 45\% of the projects extend tables with metatables and
declare methods:
13\% using the colon syntactic sugar, 14\% using \texttt{self}, and
18\% using both.

Based on these observations, Typed Lua should include support
to object-oriented programming.
Even though Lua does not have standard policies for object-oriented
programming, it provides mechanisms that allow programmers to
abstract their code in terms of objects, and our statistics confirm
that an expressive number of programmers are relying on these mechanisms
to use the object-oriented paradigm in Lua.
Typed Lua should include some standard way of defining interfaces and classes
that the compiler can use to type check object-oriented code,
but without changing the semantics of Lua.

We also measured how programmers are defining modules.
We observed that 38\% of the files use the current way of defining
modules, that is, these files return a table that contains the
exported members of the module at the end of the file;
22\% of the files still use the deprecated way of defining modules,
that is, these files call the function \texttt{module};
and 1\% of the files use both ways.
At all, 61\% of the files are modules while 39\% of the files are
plain scripts.
The number of plain scripts is high considering the origin of
our database.
However, we did not ignore sample scripts, which usually serve to
help the users of a given module on how to use this module, and
that is the reason why we have a high number of plain scripts.

Based on these observations, Typed Lua should include a way
for defining table types that represent the type of modules.
Typed Lua should also support the deprecated style of module
definition, using global names as exported members of the module.

Typed Lua should also include some way to define the types of
userdata.
This feature should also allow programmers to define userdata
that can be used in an object-oriented style, as this is another
common idiom from modules that are written in C.

The last statistics that we collected were about variadic functions
and vararg expressions.
We observed that 8\% of the functions are variadic, that is,
their last parameter is the vararg expression.
We also observed that 5\% of the initialization of lists
(or 2\% of the occurrences of the table constructor) use solely the
vararg expression.
Typed Lua should include a \emph{vararg type} to handle variadic
functions and vararg expressions.




\chapter{Typed Lua}
\label{chap:typedlua}
\documentclass[mathserif]{beamer}

\usepackage[english]{babel}
\usepackage[utf8]{inputenc}
\usepackage{amsmath}
\usepackage{amssymb}
\usepackage{beamerthemesplit}
%\usecolortheme{dove}

\newcommand{\mylabel}[1]{\; (\textsc{#1})}
\newcommand{\subtype}{<:}
\newcommand{\pipe}{|\;}
\newcommand{\kw}[1]{\mathbf{#1} \;}
\newcommand{\env}{\Gamma}

\begin{document}

\title{Typed Lua}
\subtitle{An Optional Type System for Lua}
\author{André Murbach Maidl}
\institute{Colóquios do LabLua}
\date{September 13th 2013}

\frame{\titlepage}

\begin{frame}
\frametitle{What is Typed Lua?}
\begin{itemize}
\item A typed superset of Lua that compiles to plain Lua.
\item \textcolor{blue}{Type annotations}.
\item \textcolor{blue}{Compile-time type checking}.
\item \textcolor{gray}{Classes}.
\item \textcolor{gray}{Interfaces}.
\item \textcolor{gray}{Modules}.
\end{itemize}
\end{frame}

\begin{frame}
\frametitle{Why Optional and not Gradual?}
\begin{itemize}
\item Because, first of all, gradual is optional!
\end{itemize}
\end{frame}

\begin{frame}
\frametitle{Gradual Typing}
Gradual type systems use consistency ($\sim$) instead of equality ($=$).
\begin{Large}
\[
\tau ::= \gamma \;|\; ? \;|\; \tau \rightarrow \tau
\]
\[
\tau \sim \tau
\]
\[
\tau \sim \;?
\]
\[
?\; \sim \tau
\]

\[
\frac{\sigma_{1} \sim \tau_{1} \;\;\; \sigma_{2} \sim \tau_{2}}
     {\sigma_{1} \rightarrow \sigma_{2} \sim \tau_{1} \rightarrow \tau_{2}}
\]
\end{Large}
\end{frame}

\begin{frame}
\frametitle{Gradual Typing}
Gradual type systems use consistency (\textcolor{blue}{$\sim$}) instead of equality ($=$).
\begin{Large}
\[
e ::= c \;|\; x \;|\; \lambda x:\tau.e \;|\; e\;e \;\;\;
(\lambda x.e \equiv \lambda x:?.e)
\]
\[
\frac{\Gamma x = \lfloor\tau\rfloor}
     {\Gamma \vdash_{G} x:\tau} \;\;\;\;\;
\frac{\Delta c = \tau}
     {\Gamma \vdash_{G} c:\tau}
\]

\[
\frac{\Gamma(x \mapsto \sigma) \vdash_{G} e:\tau}
     {\Gamma \vdash_{G} \lambda x:\sigma.e:\sigma \rightarrow \tau} \;\;\;\;\;
\frac{\Gamma \vdash_{G} e_{1}:\;? \;\;\; \Gamma \vdash_{G} e_{2}:\tau_{2}}
     {\Gamma \vdash_{G} e_{1}\;e_{2}:\;?}
\]

\[
\frac{\Gamma \vdash_{G} e_{1}:\tau \rightarrow \tau' \;\;\;
      \Gamma \vdash_{G} e_{2}:\tau_{2} \;\;\; \textcolor{blue}{\tau_{2} \sim \tau}}
     {\Gamma \vdash_{G} e_{1}\;e_{2}:\tau'}
\]
\end{Large}
\end{frame}

\begin{frame}
\frametitle{Optional versus Gradual}
\begin{center}
\begin{tabular}{|r|c|c|}
\hline
& Optional & Gradual\\
\hline
Optional type annotations & Yes & Yes \\ 
\hline
Compile-time type checking & Yes & Yes \\
\hline
Influence the run-time semantics & No & Yes \\
\hline
\end{tabular}
\end{center}
\end{frame}

\begin{frame}
\frametitle{The levels of Gradual Typing according to Jeremy Siek}
\begin{center}
\begin{tabular}{|r|c|c|c|}
\hline
& Level 1 & Level 2 & Level 3\\
\hline
Optional type annotations & Yes & Yes & Yes \\ 
\hline
Compile-time type checking & Yes & Yes & Yes \\
\hline
Run-time checking$^{1}$ & No & Yes & Yes \\
\hline
Blame tracking$^{2}$ & No & No & Yes\\
\hline
\end{tabular}
\end{center}
$1$ -- A compiler for a gradually typed language (level $>$ 1) infers
where dynamic checks are needed and inserts casts into the intermediate
language to performe these checks.\\
$2$ -- Blame tracking solves the problem of tracing a run-time cast
failure back to the source of the error.
\end{frame}

\begin{frame}
\frametitle{Examples of Gradually Typed Languages}
\begin{enumerate}
\item Strongtalk, TypeScript, and Typed Lua.
\item ActionScript.
\item Typed Racket.
\end{enumerate}
\end{frame}

\begin{frame}
\frametitle{Overview of rest of the presentation}
\begin{itemize}
\item Changes in the syntax of Lua.
\item Typed Lua Type System.
\end{itemize}
\end{frame}

\begin{frame}
\frametitle{Changes in the syntax of Lua}
\begin{align*}
stat ::= & \; ... \; |\\
& \textcolor{blue}{varlist} \; \texttt{`='} \; explist \; |\\
& \; ... \; |\\
& \textbf{function} \; funcname \; \textcolor{blue}{funcbody} \; |\\
& \textbf{local} \; \textbf{function} \; Name \; \textcolor{blue}{funcbody} \; |\\
& \textbf{local} \; \textcolor{blue}{namelist} \; [\texttt{`='} \; explist]\\
exp ::= & \; ... \; |\\
& functiondef \; |\\
& \; ...\\
functiondef ::= & \; \textbf{function} \; \textcolor{blue}{funcbody}
\end{align*}
\end{frame}

\begin{frame}
\frametitle{Declaration of variables}
\begin{align*}
stat ::= & \; ... \; |\\
& \textcolor{blue}{varlist} \; \texttt{`='} \; explist \; |\\
& \; ... \; |\\
& \textbf{local} \; \textcolor{blue}{namelist} \; [\texttt{`='} \; explist]\\
varlist ::= & \; \textcolor{blue}{var} \; \{\texttt{`,'} \; \textcolor{blue}{var}\}\\
var ::= & \; Name \; \textcolor{blue}{[\texttt{`:'} \; type]} \; |\\
& prefixexp \; \texttt{`['} \; exp \; \texttt{`]'} \; |\\
& prefixexp \; \texttt{`.'} \; Name\\
namelist ::= & \; Name \; \textcolor{blue}{[\texttt{`:'} \; type]} \;
\{\texttt{`,'} \; Name \; \textcolor{blue}{[\texttt{`:'} \; type]}\}
\end{align*}
\end{frame}

\begin{frame}
\frametitle{Declaration of functions}
\begin{align*}
stat ::= & \; ... \; |\\
& \textbf{function} \; funcname \; \textcolor{blue}{funcbody} \; |\\
& \textbf{local} \; \textbf{function} \; Name \; \textcolor{blue}{funcbody} \; |\\
& \; ... \\
exp ::= & \; ... \; |\\
& functiondef \; |\\
& \; ...\\
functiondef ::= & \; \textbf{function} \; \textcolor{blue}{funcbody}\\
funcbody ::= & \; \texttt{`('} \; [\textcolor{blue}{parlist}] \; \texttt{`)'} \;
\textcolor{blue}{[\texttt{`:'} \; type]} \; block \; \textbf{end}\\
parlist ::= & \; \textcolor{blue}{namelist} \; [\texttt{`,'} \; \texttt{`...'} \;
\textcolor{blue}{[\texttt{`:'} \; type]}] \; | \;
\texttt{`...'} \; \textcolor{blue}{[\texttt{`:'} \; type]}\\
namelist ::= & \; Name \; \textcolor{blue}{[\texttt{`:'} \; type]} \;
\{\texttt{`,'} \; Name \; \textcolor{blue}{[\texttt{`:'} \; type]}\}
\end{align*}
\end{frame}

\begin{frame}
\frametitle{Type annotations}
\begin{align*}
type ::= & \; \textbf{object} \;|\; \textbf{any} \;|\; \textbf{nil} \;|\\
& basetype \;|\; uniontype \;|\; functiontype\\
basetype ::= & \; \textbf{boolean} \;|\; \textbf{number} \;|\; \textbf{string}\\
uniontype ::= & \; type \;\texttt{`|'}\; type\\
functiontype ::= & \; \texttt{`('} \; [2ndclasstype] \; \texttt{`)'} \;
\texttt{`->'} \; 2ndclasstype\\
2ndclasstype ::= & \; type \; \{\texttt{`,'} \; type\} \; [\texttt{`*'}]
\end{align*}
\end{frame}

\begin{frame}
\frametitle{Example}
\textit{Pensar em um exemplo legal para colocar aqui.}
\end{frame}

\begin{frame}
\frametitle{Typed Lua Type System}
\begin{itemize}
\item Our aim is to design an object oriented language.
\item Most OO languages use the following subsumption rule:
\[
\frac{\env \vdash e:T_{1} \;\;\; T_{1} \subtype T_{2}}
     {\env \vdash e:T_{2}}
\]
\item We also aim to have gradual typing, so we also
need the consistency relation.
\end{itemize}
\end{frame}

\begin{frame}
\frametitle{Typed Lua Type System}
We do not use the subsumption rule, but subtyping
instead of equality.
\[
\frac{\env \vdash e:S_{1} \rightarrow S_{2} \;\;\;
      \env \vdash el:S_{3} \;\;\;
      \textcolor{red}{S_{3} \subtype S_{1}}}
     {\env \vdash e(el):S_{2}}
\]
and also the consistency rule
\[
\frac{\env \vdash e:S_{1} \rightarrow S_{2} \;\;\;
      \env \vdash el:S_{3} \;\;\;
      \textcolor{red}{S_{3} \subtype S_{1}'} \;\;\;
      \textcolor{blue}{S_{1}' \sim S_{1}}}
     {\env \vdash e(el):S_{2}}
\]
so we can compose the two relations:
\[
\frac{\env \vdash e:S_{1} \rightarrow S_{2} \;\;\;
      \env \vdash el:S_{3} \;\;\;
      \textcolor{red!50!blue}{S_{3} \lesssim S_{1}}}
     {\env \vdash e(el):S_{2}}
\]
which we call consistent-subtyping.
\end{frame}

\begin{frame}
\frametitle{Type language}
\begin{align*}
T ::= \; & C \; \pipe B \; \pipe Object \; \pipe Any \; \pipe
S \rightarrow S \; \pipe T \cup T\\
C ::= \; & \mathbf{nil} \; \pipe \mathbf{false} \; \pipe \mathbf{true} \;
\pipe {<}double{>} \; \pipe {<}integer{>} \; \pipe {<}string{>}\\
B ::= \; & Boolean \; \pipe Number \; \pipe String\\
S ::= \; & T \; \pipe {T*} \; \pipe T \times S\\ 
\end{align*}
\end{frame}

\begin{frame}
\frametitle{Typing rules}
\begin{itemize}
\item Type consistency.
\item Subtyping.
\item Typed Lua typing rules.
\end{itemize}
\end{frame}

\begin{frame}
\textit{Colocar as regras do sistema de tipos.}
\end{frame}

\end{document}


\chapter{The type system}
\label{chap:system}

In the previous chapter we presented an informal overview of Typed Lua.
We showed that programmers can use Typed Lua to combine static and dynamic
typing in the same code, and it allows them to incrementally migrate from
dynamic to static typing.
This is a benefit to programmers that use dynamically typed languages
to build large applications, as static types detect many bugs
during the development phase, and also provide better documentation.

In this chapter we present the formalization of Typed Lua's type system.
Besides its practical contribution, Typed Lua also has some interesting
contributions to the field of optional type systems for scripting
languages.
They are novel type system features that let Typed Lua cover several Lua idioms
and features, such as the refinement of tables, multiple return values,
and optional parameters.

\section{Types}
\label{sec:types}

\begin{figure}[!ht]
\textbf{Type Language}\\
\dstart
$$
\begin{array}{rlr}
\multicolumn{3}{c}{\textbf{First-level types}}\\
t ::= & \;\; l & \textit{literal types}\\
& | \; b & \textit{base types}\\
& | \; \Nil & \textit{nil type}\\
& | \; \Value & \textit{top type}\\
& | \; \Any & \textit{dynamic type}\\
& | \; \Self & \textit{self type}\\
& | \; t \cup t & \textit{disjoint union types}\\
& | \; s \rightarrow r & \textit{function types}\\
& | \; \{k_{1}{:}v_{1}, ..., k_{n}{:}v_{n}\}_{closed|open|unique} & \textit{table types}\\
& | \; x & \textit{type variables}\\
& | \; \mu x.t & \textit{recursive types}\\
& | \; \pi_{i}^{x} & \textit{projection types}\\
l ::= & \;\; \False \; | \; \True \; | \; {\it int} \; | \; {\it float} \; | \; {\it string} & \\
b ::= & \;\; \Boolean \; | \; \Integer \; | \; \Number \; | \; \String & \\
k ::= & \;\; l \; | \; b \; | \; \Any & \textit{key types}\\
v ::= & \;\; t \; | \; \Const \; t & \textit{value types}\\ 
\multicolumn{3}{c}{}\\
\multicolumn{3}{c}{\textbf{Second-level types}}\\
s ::= & \;\; \Void & \textit{void type}\\
& | \; t* & \textit{variadic types}\\
& | \; t \times s & \textit{tuple types}\\
r ::= & \;\; s \; | \; s \sqcup r & \textit{return types}\\
\end{array}
$$
\dend
\caption{The abstract syntax of Typed Lua types}
\label{fig:typelang}
\end{figure}

Figure \ref{fig:typelang} presents the abstract syntax of the
Typed Lua types.
Typed Lua splits types into two categories:
\emph{first-level types} and \emph{second-level types}.
First-level types represent first-class Lua values and
second-level types represent tuples of values that appear in 
assignments and function applications.
First-level types include literal types, base types, the type $\Nil$,
the top type $\Value$, the dynamic type $\Any$, the type $\Self$,
union types, function types, table types, recursive types, and
projection types.
Second-level types include the type $\Void$, variadic types,
tuple types, and unions of tuple types.

Literal types $l$ represent the type of literal values.
They can be the boolean values $\False$ and $\True$,
an integer value, a floating point value, or a string value.
We shall see that literal types are important in our treatment of
table types as records.

Typed Lua includes four base types $b$: $\Boolean$, $\Integer$, $\Number$, and $\String$.
The base types $\Boolean$ and $\String$ represent the values that during
run-time Lua tags as \texttt{boolean} and \texttt{string}, respectively.
Lua 5.3 introduced two internal representations to the tag \texttt{number}:
\texttt{integer} for integer numbers and \texttt{float} for real numbers.
Both \texttt{integer} and \texttt{float} values are a subtype of \texttt{number}.
We introduced the base type $\Number$ to represent either the tag
\texttt{number} or \texttt{float} values, and the type $\Integer$ to represent
\texttt{integer} values.
In the next section we will show that $\Integer$ is subtype of $\Number$.
This allows programmers to use \texttt{integer} values in the
place of \texttt{float} values.

The type $\Nil$ represents the type of the value that during run-time
Lua tags as \texttt{nil}.

The type $\Value$ is the top type.
In Section \ref{sec:rules} we will show that this type,
along with variadic types, helps the type system to drop extra values
on assignments and function calls, thus preserving the
semantics of Lua in these cases.

Typed Lua includes the dynamic type $\Any$ for allowing programmers
to mix static and dynamic typing.

Typed Lua uses the type $\Self$ to represent the \emph{receiver}
in object-oriented method definitions and method calls.
As we mentioned in Chapter \ref{chap:typedlua}, we need the type
$\Self$ to prevent programs from indexing a method without
calling it with the correct receiver.

Union types $t_{1} \cup t_{2}$ represent types that can hold a value
of two different types.

Function types have the form $s \rightarrow r$ to represent Lua functions,
where $s$ is the type of the parameter list and $r$ is the return type.
Both $s$ and $r$ are second-level types.
In the next paragraphs we explain why we need different representations
for the type of the parameter list and the return type.

Second-level types $s$ are tuples of first-level types that can end
with either an empty tuple or with a variadic type.
Typed Lua needs second-level types because tuples are not first-class
values in Lua, though they may appear on argument passing,
multiple returns, and multiple assignments.
The type $\Void$ is the type of an empty tuple.
A variadic type $t*$ is a generator for a sequence of values
of type $t \cup \Nil$, and it represents the type of a vararg expression.

Second-level types $r$ include the same second-level types introduced
by $s$ plus unions of tuple types.
Typed Lua includes unions of tuples because Lua programs
usually overload the return type of functions to denote error,
as we mentioned in Section \ref{sec:statistics}.
We could not find a nice concrete syntax to represent unions of
tuples that appear in the type of the parameter list, and
that is the reason why we restricted them to appear in the return type only.
The same reason led us to split second-level types into two
different symbols.
We use the symbol $\sqcup$ to represent the union between two
different tuple types.
We have different unions because $\cup$ represents the union
between two first-level types, while $\sqcup$ represents the
union between two tuple types.

Back to first-level types, table types represent the various forms
that Lua tables can take.
The syntactical form of table types is $\{ k_{1}{:}v_{1}, ..., k_{n}{:}v_{n} \}_{tag}$,
where each $k_{i}$ represents the type of a table key,
and each $v_{i}$ represents the type of the value that table keys of type $k_{i}$ map to.
Key types can only be literal types, base types, or the dynamic type.
We made this restriction to the type of the keys because the statistics
that we discussed in Section \ref{sec:statistics} showed that most
of the tables are records, lists, and hashes.
The dynamic type is an option when we need a loose table type.
Value types can be any first-level type, and can optionally include
the $\Const$ type to denote immutable values.

We also use the tags \emph{closed}, \emph{open}, and \emph{unique}
to classify table types.
The tag \emph{unique} represents tables that we know their contents,
and that there is no alias to them.
In particular, the type of the table constructor has this tag.
The tag \emph{open} represents \emph{unique} table types that
have at least one alias.
The tag \emph{closed} represents table types that do not provide
any guarantees about the types not listed in the table type.
In particular, type annotations as well as interfaces and userdata
declarations always describe \emph{closed} table types.
In the next sections we explain in more detail why we need
different table types.

Any table type has to be \emph{well-formed}.
Informally, a table type is well-formed if key types do not overlap.
In Section \ref{sec:rules} we formalize the definition of well-formed table types.
We delay the proper formalization of well-formed table types because we use
consistent-subtyping in this formalization.

Recursive types have the form $\mu x.t$,
where $t$ is a first-level type that $x$ represents.
For instance, $\mu x.\{``info":\Integer, ``next":x \;\cup\; \Nil\}_{closed}$
is a type for singly-linked lists of integers.
In Section \ref{sec:alias} we mentioned that we can use the following
interface declaration as an alias to this type:
\begin{verbatim}
    local interface Element
      info:integer
      next:Element?
    end
\end{verbatim}

Type Lua includes projection types as a way to project
unions of tuple types into unions of first-level types.
Projection types have the form $\pi_{i}^{x}$, where $x$ is a variable
that stores an union of tuple types in a special type environment,
and $i$ is an index that indicates how to project a first-level
union type from $x$.
For instance, if $x$ is the union of tuple types
$(\Integer \times \Integer) \sqcup (\Nil \times \String)$,
then $\pi_{1}^{x}$ represents the type $\Integer \cup \Nil$
while $\pi_{2}^{x}$ represents the type $\Integer \cup \String$.
In Section \ref{sec:rules} we show in more detail how our type system
uses them as a mechanism for handling unions of tuple types,
when they appear on the right-hand side of the declaration of local variables.
We also show how this feature allows our type system to constrain
the type of a local variable that depends on the type of another local variable.

\section{Subtyping}
\label{sec:subtyping}

Our type system uses subtyping \citep{cardelli1984smi,abadi1996to} to order
types and consistent-subtyping \citep{siek2007objects,siek2013mutable}
to allow the interaction between statically and dynamically typed code.
We explain the subtyping and consistent-subtyping rules along this section.
However, we focus the discussion on the definition of subtyping because,
as we mentioned in Chapter \ref{chap:review}, we can combine the
consistency and subtyping relations to achieve consistent-subtyping.
The differences between subtyping and consistent-subtyping are the way
they handle the dynamic type, and the fact that subtyping is transitive,
but consistent-subtyping is not.

We present the subtyping rules as a deduction system for the
subtyping relation $\senv \vdash t_{1} \subtype t_{2}$.
The variable $\senv$ is a set of pairs of recursion variables.
We shall notice that we need this set to record the hypotheses
that we assume to check subtyping between recursive types.

The subtyping rules for literal types and base types include the rules
for defining that literal types are subtypes of their respective base types,
and that $\Integer$ is subtype of $\Number$.
We define these subtyping rules as follows:
\[
\begin{array}{c}
\begin{array}{c}
\mylabel{S-FALSE}\\
\senv \vdash \False \subtype \Boolean
\end{array}
\;
\begin{array}{c}
\mylabel{S-TRUE}\\
\senv \vdash \True \subtype \Boolean
\end{array}
\;
\begin{array}{c}
\mylabel{S-STRING}\\
\senv \vdash {\it string} \subtype \String
\end{array}
\\ \\
\begin{array}{c}
\mylabel{S-INT1}\\
\senv \vdash {\it int} \subtype \Integer
\end{array}
\;
\begin{array}{c}
\mylabel{S-INT2}\\
\senv \vdash {\it int} \subtype \Number
\end{array}
\;
\begin{array}{c}
\mylabel{S-FLOAT}\\
\senv \vdash {\it float} \subtype \Number
\end{array}
\\ \\
\begin{array}{c}
\mylabel{S-NUMBER}\\
\senv \vdash \Integer \subtype \Number
\end{array}
\end{array}
\]

Subtyping is reflexive and transitive;
due to the transitivity of subtyping, we could have omitted the rule \textsc{S-INT2}.
More precisely, we could have defined a transitive rule for first-level
types instead of defining specific rules to transitive cases.
For instance, a transitive rule would allow us to derive that
\[
\dfrac{\senv \vdash 1 \subtype \Integer \;\;\;
       \senv \vdash \Integer \subtype \Number}
      {\senv \vdash 1 \subtype \Number}
\]

However, we are using the subtyping rules as the template for defining
the consistent-subtyping rules, and consistent-subtyping is not
transitive.
More precisely, we want the subtyping and consistent-subtyping rules
to differ only in the way they handle the dynamic type.
Thus, we define the subtyping rules using an algorithmic
way that is close to the implementation, as this way allows us to use
subtyping to formalize consistent-subtyping.

That is the reason why the dynamic type $\Any$ is neither the bottom nor the top type,
but a separate type that is subtype only of itself through the reflexive rule.

Even though the dynamic type $\Any$ does not interact with subtyping,
it does interact with consistent-subtyping:
\[
\begin{array}{c}
\begin{array}{c}
\mylabel{C-ANY1}\\
\senv \vdash t \lesssim \Any
\end{array}
\;
\begin{array}{c}
\mylabel{C-ANY2}\\
\senv \vdash \Any \lesssim t
\end{array}
\end{array}
\]

If we had set the type $\Any$ as both bottom and top types of our
subtyping relation, then any type $t_{1}$ would be subtype of
any other type $t_{2}$.
The consequence of this is that all programs would type check without errors.
This would happen due to the transitivity of subtyping, that is,
we would be able to down-cast any type $t_{1}$ to $\Any$ and then up-cast
$\Any$ to any other type $t_{2}$.
The rules \textsc{C-ANY1} and \textsc{C-ANY2} are the rules that
allow the dynamic type to interact with other first-level types,
and thus allow dynamically typed code to coexist with statically
typed code.
Because of these two rules, consistent-subtyping cannot be transitive.
Still, these two rules are the only rules that differ between
subtyping and consistent-subtyping.

Our type system includes the top type $\Value$,
so any first-level type is a subtype of $\Value$:
\[
\begin{array}{c}
\mylabel{S-VALUE}\\
\senv \vdash t \subtype \Value
\end{array}
\]

Some people might expect that $\Nil$ is the bottom type
of our type system, but it is not,
and it is subtype only of itself through the reflexive rule.
Many programming languages include a bottom type to represent
an empty value that programmers can use as a default expression.
Typed Lua does not have this behavior because in Lua the tag \texttt{nil}
represents the absence of an useful value.
Thus, making $\Nil$ the bottom type would lead to several expressions
that would pass the type checker, but that would fail during run-time
in the presence of a \texttt{nil} value.

Another type that only follows the reflexive rule is the type $\Self$,
that is, it is subtype only of itself.

The subtyping rules for union types are standard:
\[
\begin{array}{c}
\begin{array}{c}
\mylabel{S-UNION1}\\
\dfrac{\senv \vdash t_{1} \subtype t \;\;\;
       \senv \vdash t_{2} \subtype t}
      {\senv \vdash t_{1} \cup t_{2} \subtype t}
\end{array}
\;
\begin{array}{c}
\mylabel{S-UNION2}\\
\dfrac{\senv \vdash t \subtype t_{1}}
      {\senv \vdash t \subtype t_{1} \cup t_{2}}
\end{array}
\;
\begin{array}{c}
\mylabel{S-UNION3}\\
\dfrac{\senv \vdash t \subtype t_{2}}
      {\senv \vdash t \subtype t_{1} \cup t_{2}}
\end{array}
\end{array}
\]

The first rule shows that a union type $t_{1} \cup t_{2}$
is subtype of $t$ if both $t_{1}$ and $t_{2}$ are subtypes
of $t$;
and the other rules show that a type $t$ is subtype
of a union type $t_{1} \cup t_{2}$ if $t$ is subtype of
$t_{1}$ or if $t$ is subtype of $t_{2}$.

In the implementation of Typed Lua we use consistent-subtyping to
normalize and simplify union types, but we let them free in the
formalization.
For instance, the union type \texttt{boolean|any} results in the
type \texttt{any}, because \texttt{boolean} is consistent-subtype
of \texttt{any}.
Another example is the union type \texttt{number|nil|1} that
results in the union type \texttt{number|nil}, because
\texttt{1} is consistent-subtype of \texttt{number}.

The subtyping rule for function types is also standard:
\[
\begin{array}{c}
\mylabel{S-FUNCTION}\\
\dfrac{\senv \vdash s_{2} \subtype s_{1} \;\;\;
       \senv \vdash r_{1} \subtype r_{2}}
      {\senv \vdash s_{1} \rightarrow s_{1} \subtype s_{2} \rightarrow r_{2}}
\end{array}
\]

The rule \textsc{S-FUNCTION} shows that subtyping between
function types is contravariant on the type of the parameter list
and covariant on the return type.
In the last section we explained why our type system uses
second-level types to represent the type of the parameter list
and the return type.
Now, we explain their subtyping rules.

The simplest second-level type is the type $\Void$ that
is subtype only of itself through the reflexive rule.

The subtyping rule for tuple types is standard, but the subtyping
rules for variadic types are not so obvious:
\[
\begin{array}{c}
\begin{array}{c}
\mylabel{S-TUPLE}\\
\dfrac{\senv \vdash t_{1} \subtype t_{2} \;\;\;
       \senv \vdash s_{1} \subtype s_{2}}
      {\senv \vdash t_{1} \times s_{1} \subtype t_{2} \times s_{2}}
\end{array}
\;
\begin{array}{c}
\mylabel{S-VARARG1}\\
\dfrac{\senv \vdash t_{1} \cup \Nil \subtype t_{2} \cup \Nil}
      {\senv \vdash t_{1}{*} \subtype t_{2}{*}}
\end{array}
\\ \\
\begin{array}{c}
\mylabel{S-VARARG2}\\
\dfrac{\senv \vdash t_{1} \cup \Nil \subtype t_{2}}
      {\senv \vdash t_{1}{*} \subtype t_{2} \times \Void}
\end{array}
\;
\begin{array}{c}
\mylabel{S-VARARG3}\\
\dfrac{\senv \vdash t_{1} \subtype t_{2} \cup \Nil}
      {\senv \vdash t_{1} \times \Void \subtype t_{2}{*}}
\end{array}
\\ \\
\begin{array}{c}
\mylabel{S-VARARG4}\\
\dfrac{\senv \vdash t_{1}{*} \subtype t_{2} \times \Void \;\;\;
       \senv \vdash t_{1}{*} \subtype s_{2}}
      {\senv \vdash t_{1}{*} \subtype t_{2} \times s_{2}}
\end{array}
\;
\begin{array}{c}
\mylabel{S-VARARG5}\\
\dfrac{\senv \vdash t_{1} \times \Void \subtype t_{2}{*} \;\;\;
       \senv \vdash s_{1} \subtype t_{2}{*}}
      {\senv \vdash t_{1} \times s_{1} \subtype t_{2}{*}}
\end{array}
\end{array}
\]

The subtyping rules for tuple and variadic types show that they are covariant.
While we need just one subtyping rule for tuple types,
we need five different subtyping rules for variadic types
to handle all the cases where they can appear.
The rule \textsc{S-VARARG1} handles the subtyping between two variadic types,
while the other rules handle the subtyping between a tuple type and a variadic type.
The rule \textsc{S-VARARG1} shows that $t_{1}{*}$ is subtype of $t_{2}{*}$
if $t_{1} \cup \Nil$ is subtype of $t_{2} \cup \Nil$.
This rule explicit includes $\Nil$ in both sides because otherwise
$\Nil{*}$ would not be a subtype of several other variadic types.
For instance, $\Nil{*}$ would not be subtype of $\Number{*}$,
as $\Nil \not\subtype \Number$.
The rule \textsc{S-VARARG2} shows that a variadic type $t_{1}{*}$ is
subtype of a tuple type $t_{2} \times \Void$ if $t_{1} \cup \Nil$ is subtype of $t_{2}$.
The rule \textsc{S-VARARG3} shows that a tuple type $t_{1} \times \Void$
is subtype of a variadic type $t_{2}{*}$ if $t_{1}$ is subtype of $t_{2} \cup \Nil$.
The rule \textsc{S-VARARG4} shows that a variadic type is subtype of
a tuple type whenever the type of the value that it generates is
subtype of all the elements of the tuple.
The rule \textsc{S-VARARG5} shows that a tuple type is subtype of a
variadic type whenever all the elements of the tuple are subtype of
the type of the value that it generates.
In the next section we will show that we use these subtyping relations,
along with the types $\Value$ and $\Nil$, to make our type system reflect
the semantics of Lua on discarding extra parameters, and
replacing missing parameters.

The subtyping rules for unions of tuple types are similar to the
subtyping rules for unions of first-level types:
\[
\begin{array}{c}
\begin{array}{c}
\mylabel{S-UNION4}\\
\dfrac{\senv \vdash s_{1} \subtype s \;\;\;
       \senv \vdash s_{2} \subtype s}
      {\senv \vdash s_{1} \sqcup s_{2} \subtype s}
\end{array}
\;
\begin{array}{c}
\mylabel{S-UNION5}\\
\dfrac{\senv \vdash s \subtype s_{1}}
      {\senv \vdash s \subtype s_{1} \sqcup s_{2}}
\end{array}
\;
\begin{array}{c}
\mylabel{S-UNION6}\\
\dfrac{\senv \vdash s \subtype s_{2}}
      {\senv \vdash s \subtype s_{1} \sqcup s_{2}}
\end{array}
\end{array}
\]

Back to the subtyping rules between first-level types,
the subtyping rule between \emph{closed} table types resembles the
standard subtyping rule between records:
\[
\begin{array}{c}
\mylabel{S-TABLE1}\\
\dfrac{\forall i \in 1..n \; \exists j \in 1..m \;\;\;
       \senv \vdash k_{j} \subtype k_{i}' \;\;\;
       \senv \vdash k_{i}' \subtype k_{j} \;\;\;
       \senv \vdash v_{j} \subtype_{c} v_{i}'}
      {\senv \vdash \{k_{1}{:}v_{1}, ..., k_{m}{:}v_{m}\}_{closed} \subtype \{k_{1}'{:}v_{1}', ..., k_{n}'{:}v_{n}'\}_{closed}} \; m \ge n
\end{array}
\]

The rule \textsc{S-TABLE1} introduced the relation $\subtype_{c}$.
It is just an auxiliary relation that defines the subtyping of the
type of the values stored in the table fields.
We need an auxiliary relation because the subtyping of the
type of the values stored in the table fields changes according to
the tags of the table types.
We define the relation $\subtype_{c}$ as follows:
\[
\begin{array}{c}
\begin{array}{c}
\mylabel{S-FIELD1}\\
\dfrac{\senv \vdash v_{1} \subtype v_{2} \;\;\;
       \senv \vdash v_{2} \subtype v_{1}}
      {\senv \vdash v_{1} \subtype_{c} v_{2}}
\end{array}
\;
\begin{array}{c}
\mylabel{S-FIELD2}\\
\dfrac{\senv \vdash v_{1} \subtype v_{2}}
      {\senv \vdash \Const \; v_{1} \subtype_{c} \Const \; v_{2}}
\end{array}
\\ \\
\begin{array}{c}
\mylabel{S-FIELD3}\\
\dfrac{\senv \vdash v_{1} \subtype v_{2}}
      {\senv \vdash v_{1} \subtype_{c} \Const \; v_{2}}
\end{array}
\end{array}
\]

These rules allow closed table types to have width subtyping
and depth subtyping on $\Const$ fields.
The rule \textsc{S-FIELD1} defines that mutable fields are invariant,
while the rule \textsc{S-FIELD2} defines that immutable fields are covariant.
The rule \textsc{S-FIELD3} defines that it is safe to promote fields
from mutable to immutable.
We do not include a rule that allows promoting fields from immutable
to mutable because it would not be safe, as this rule would allow
programmers to change the value that is stored in an immutable field.

There is a limitation on \emph{closed} table types that led us to
introduce \emph{open} and \emph{unique} table types.
If the table constructor had a \emph{closed} table type, then
programmers would not be able to use it to initialize a variable with
a table type that describes a more general type.
For instance,
\begin{verbatim}
    local t:{"x":integer, "y":integer?} = { x = 1, y = 2 }
\end{verbatim}
would not type check, as the type of the table constructor would not
be subtype of the type in the annotation.
More precisely,
\[
\{``x":1, ``y":2\}_{closed} \not\subtype \{``x":\Integer, ``y":\Integer \cup \Nil\}_{closed}
\]

Simply promoting the type of each table value to its supertype would
not overcome this limitation, as it still would give to the table constructor
a closed table type without covariant mutable fields.
Thus, programmers would not be able to use the table constructor to
initialize a variable with a table type that includes an optional field.
Using the previous example,
\begin{align*}
& \{``x":\Integer, ``y":\Integer\}_{closed} \not\subtype \\
& \{``x":\Integer, ``y":\Integer \cup \Nil\}_{closed}
\end{align*}

We introduced \emph{unique} table types to avoid this limitation,
as they represent the type of the tables that we known their contents.
In particular, this is the case of the table constructor.
The following subtyping rule defines the subtyping relation among
\emph{unique} table types and other table types:
\[
\begin{array}{c}
\mylabel{S-TABLE2}\\
\dfrac{\forall i \in 1..m \; \senv \vdash k_{i} \subtype k_{i}' \;\;\;
       \senv \vdash v_{i} \subtype_{u} v_{i}' \;\;\;
       \forall j \in m+1..n \; \senv \vdash \Nil \subtype_{o} v_{j}'}
      {\senv \vdash \{k_{1}{:}v_{1}, ..., k_{m}{:}v_{m}\}_{unique} \subtype
                    \{k_{1}'{:}v_{1}', ..., k_{m}'{:}v_{m}', ..., k_{n}'{:}v_{n}'\}_{closed|open|unique}}
\end{array}
\]

Unlike the rule \textsc{S-TABLE1}, the rule \textsc{S-TABLE2} allows
covariant keys.
We allow this behavior to \emph{unique} table types because we also
want to use them as a way to join table fields.
For instance, we may want to use the table constructor to initialize
a variable with a table type that describes a hash.

The rule \textsc{S-TABLE2} introduced the auxiliary relations
$\subtype_{u}$ and $\subtype_{o}$.
The first allows width and depth subtyping on all fields,
while the second allows the omission of optional fields.
We define them as follows:
\[
\begin{array}{c}
\begin{array}{c}
\mylabel{S-FIELD4}\\
\dfrac{\senv \vdash v_{1} \subtype v_{2}}
      {\senv \vdash v_{1} \subtype_{u} v_{2}}
\end{array}
\;
\begin{array}{c}
\mylabel{S-FIELD5}\\
\dfrac{\senv \vdash v_{1} \subtype v_{2}}
      {\senv \vdash \Const \; v_{1} \subtype_{u} \Const \; v_{2}}
\end{array}
\;
\begin{array}{c}
\mylabel{S-FIELD6}\\
\dfrac{\senv \vdash v_{1} \subtype v_{2}}
      {\senv \vdash v_{1} \subtype_{u} \Const \; v_{2}}
\end{array}
\\ \\
\begin{array}{c}
\mylabel{S-FIELD7}\\
\dfrac{\senv \vdash \Nil \subtype v}
      {\senv \vdash \Nil \subtype_{o} v}
\end{array}
\;
\begin{array}{c}
\mylabel{S-FIELD8}\\
\dfrac{\senv \vdash \Nil \subtype v}
      {\senv \vdash \Nil \subtype_{o} \Const \; v}
\end{array}
\end{array}
\]

Using \emph{unique} table types to represent the type of the table
constructor allows our type system to type check the previous example.
More precisely,
\[
\{``x":1, ``y":2\}_{unique} \subtype \{``x":\Integer, ``y":\Integer \cup \Nil\}_{closed}
\]

Even though \emph{unique} table types overcome our limitation,
its subtyping rule allows unsafe aliasing.
The covariance on mutable fields allows programs that create unsafe
aliases to pass the type checker.
For instance,
\begin{verbatim}
    local t1 = { foo = 5 }
    local t2:{"foo":integer?} = t1
    t2.foo = nil
\end{verbatim}
should not type check, as the last line is erasing the field \texttt{foo}
from \texttt{t1}.
This program would type checks as it is, because our implementation uses
local type inference to assign static types to unannotated locals.
In this example, the type of \texttt{t1} would be subtype of the type of \texttt{t2}.
More precisely,
\[
\{``foo":\Integer\}_{unique} \subtype \{``foo":\Integer \cup \Nil\}_{closed}
\]

In the next section we will show in more detail that this example
does not type check, because our type system promotes an alias
expression from \emph{unique} to \emph{closed}, and promotes the
aliased variable from \emph{unique} to \emph{open}.
This means that the previous example does not type check because
the type of the expression \texttt{t1} is not subtype of the type
of the variable \texttt{t2}.
More precisely,
\[
\{``foo":\Integer\}_{closed} \not\subtype \{``foo":\Integer \cup \Nil\}_{closed}
\]

Thus, we introduced \emph{open} table types to handle the aliasing of table types.
We define the subtyping relation among \emph{open} table types
and \emph{closed} or \emph{open} table types as follows:
\[
\begin{array}{c}
\mylabel{S-TABLE3}\\
\dfrac{\forall i \in 1..m \; \senv \vdash k_{i} \subtype k_{i}' \;\;\;
       \senv \vdash v_{i} \subtype_{c} v_{i}' \;\;\;
       \forall j \in m+1..n \; \senv \vdash \Nil \subtype_{o} v_{j}'}
      {\senv \vdash \{k_{1}{:}v_{1}, ..., k_{m}{:}v_{m}\}_{open} \subtype
                    \{k_{1}'{:}v_{1}', ..., k_{m}'{:}v_{m}', ..., k_{n}'{:}v_{n}'\}_{closed|open}}
\end{array}
\]

The rule \textsc{S-TABLE3} uses the same rules from \textsc{S-TABLE1}
for handling the type of the values of each key, and it uses the same
rules from \textsc{S-TABLE2} to allow joining fields plus omitting optional fields.

In the next section we will show in more detail how our type system
keeps track of \emph{open} and \emph{unique} table types to make them safe.
Moreover, we will also show that we can use them to allow the refinement
of table types.

We use the \emph{Amber rule} \citep{cardelli1986amber} to define
subtyping between recursive types:
\[
\begin{array}{c}
\begin{array}{c}
\mylabel{S-AMBER}\\
\dfrac{\senv[x_{1} \subtype x_{2}] \vdash t_{1} \subtype t_{2}}
      {\senv \vdash \mu x_{1}.t_{1} \subtype \mu x_{2}.t_{2}}
\end{array}
\;
\begin{array}{c}
\mylabel{S-ASSUMPTION}\\
\dfrac{x_{1} \subtype x_{2} \in \senv}
      {\senv \vdash x_{1} \subtype x_{2}}
\end{array}
\end{array}
\]

The rule \textsc{S-AMBER} also uses the rule \textsc{S-ASSUMPTION}
to check whether $\mu x_{1}.t_{1} \subtype \mu x_{2}.t_{2}$.
Both rules use the set of assumptions $\senv$,
where each assumption is a pair of recursion variables.
The rule \textsc{S-AMBER} extends $\senv$ with the assumption
$x_{1} \subtype x_{2}$ to check whether $t_{1} \subtype t_{2}$.
The rule \textsc{S-ASSUMPTION} allows the rule \textsc{S-AMBER}
to check whether an assumption is valid.

A recursive type may appear inside a first-level type, and our
type system includes subtyping rules to handle subtyping between
recursive types and other first-level types:
\[
\begin{array}{c}
\begin{array}{c}
\mylabel{S-UNFOLDR}\\
\dfrac{\senv \vdash t_{1} \subtype [x \mapsto \mu x.t_{2}]t_{2}}
      {\senv \vdash t_{1} \subtype \mu x.t_{2}}
\end{array}
\;
\begin{array}{c}
\mylabel{S-UNFOLDL}\\
\dfrac{\senv \vdash [x \mapsto \mu x.t_{1}]t_{1} \subtype t_{2}}
      {\senv \vdash \mu x.t_{1} \subtype t_{2}}
\end{array}
\end{array}
\]

The rules \textsc{S-UNFOLDR}, \textsc{S-AMBER}, and \textsc{S-ASSUMPTION}
allow our type system to type check the function \texttt{insert}
from Section \ref{sec:alias}:
\begin{verbatim}
    local function insert (e:Element?, v:integer):Element
      return { info = v, next = e }
    end
\end{verbatim}

First, a recursive type appears inside the type of the table constructor.
Thus, the type checker uses the rule \textsc{S-UNFOLDR} to verify whether
the type of the table constructor is a subtype of \texttt{Element}:
\begin{align*}
\{ & ``info":\Integer, \\
   & ``next":\mu x.\{``info":\Integer, ``next":x \;\cup\; \Nil\}_{closed} \cup \Nil \}_{unique} \subtype \\
& \mu x.\{``info":\Integer, ``next":x \;\cup\; \Nil\}_{closed}
\end{align*}

After unfolding, the recursive types appear only as part of the type
of the value of the key type $``next"$.
Thus, the type checker uses the rules \textsc{S-AMBER} and \textsc{S-ASSUMPTION}
to verify whether these recursive types are subtypes between each other:
\begin{align*}
& \mu x.\{``info":\Integer, ``next":x \;\cup\; \Nil\}_{closed} \subtype \\
& \mu x.\{``info":\Integer, ``next":x \;\cup\; \Nil\}_{closed}
\end{align*}

Projection types follow the reflexive subtyping rule.
A projection type $\pi_{i}^{x}$ is a subtype of another projection type $\pi_{i}^{x}$
that shares the same union of tuples $x$ and the same index $i$.

\section{Typing rules}
\label{sec:rules}

In this section we use a reduced core of Typed Lua to present the
most interesting rules of our type system.
This core limits control flow to if and while statements,
has explicit type annotations, and explicit scope for variables.
It also has explicit method declarations and explicit method calls
instead of treating them as syntactic sugar.
We use this reduced core because it simplifies the presentation
of our type system.
Appendix \ref{app:rules} presents the full set of typing rules.

\begin{figure}[!ht]
\textbf{Abstract Syntax}\\
\dstart
$$
\begin{array}{rl}
s ::= & \;\; \mathbf{skip} \; | \;
s_{1} \; ; \; s_{2} \; | \;
\vec{l} = el \; | \;
m \; | \;
\mathbf{while} \; e \; \mathbf{do} \; s \; | \;
\mathbf{if} \; e \; \mathbf{then} \; s_{1} \; \mathbf{else} \; s_{2}\\
& | \; \mathbf{local} \; \vec{n{:}t} = el \; \mathbf{in} \; s \; | \;
\mathbf{rec} \; n{:}t = f \; \mathbf{in} \; s \; | \;
\mathbf{return} \; el \; | \;
e(el)_{s} \; | \;
e{:}n(el)_{s}\\
e ::= & \;\; \mathbf{nil} \; | \;
c \; | \;
{...}_{e} \; | \;
n_{e} \; | \;
e_{1}[e_{2}] \; | \;
{<}t{>} \; n \; | \;
f \; | \;
\{ \} \; | \;
\{ \; \vec{p} \; \} \; | \;
\{ \; {...}_{m} \; \} \; | \;
\{ \; \vec{p},{...}_{m} \; \}\\
& | \; e_{1} + e_{2} \; | \;
e_{1} \; {..} \; e_{2} \; | \;
e_{1} == e_{2} \; | \;
e_{1} < e_{2} \; | \;
e_{1} \; \mathbf{and} \; e_{2} \; | \;
e_{1} \; \mathbf{or} \; e_{2} \\
& | \; \mathbf{not} \; e \; | \;
- e \; | \;
\# \; e \; | \;
e(el)_{e} \; | \;
e{:}n(el)_{e}\\
l ::= & \;\; n_{l} \; | \;
e_{1}[e_{2}] \; | \;
n[c] \; {<}t{>}\\
c ::= & \;\; \mathbf{false} \; | \;
\mathbf{true} \; | \;
{\it int} \; | \;
{\it float} \; | \;
{\it string}\\
p ::= & \;\; [e_{1}] = e_{2}\\
el ::= & \;\; \mathbf{nothing} \; | \;
\vec{e} \; | \;
me \; | \;
\vec{e}, me\\
me ::= & \;\; e(el)_{m} \; | \;
e{:}n(el)_{m} \; | \;
{...}_{m}\\
m ::= & \;\; \mathbf{fun} \; n_{1}{:}n_{2} \; (){:}r \; fb \; | \;
\mathbf{fun} \; n_{1}{:}n_{2} \; ({...}{:}t){:}r \; fb\\
& | \; \mathbf{fun} \; n_{1}{:}n_{2} \; (\vec{n{:}t}){:}r \; fb \; | \;
\mathbf{fun} \; n_{1}{:}n_{2} \; (\vec{n{:}t},{...}{:}t){:}r \; fb\\
f ::= & \;\; \mathbf{fun} \; (){:}r \; fb \; | \;
\mathbf{fun} \; ({...}{:}t){:}r \; fb \; | \;
\mathbf{fun} \; (\vec{n{:}t}){:}r \; fb \; | \;
\mathbf{fun} \; (\vec{n{:}t},{...}{:}t){:}r \; fb\\
fb ::= & \;\; s \;;\; \mathbf{return} \; el\\
n ::= & \;\; {\it name}\\
\end{array}
$$
\dend
\caption{Typed Lua abstract syntax}
\label{fig:syntax}
\end{figure}

Figure \ref{fig:syntax} presents the abstract syntax of core Typed Lua.
It splits the syntactic categories as follows:
$s$ are statements, $e$ are expressions, $l$ are left-hand values,
$el$ are expression lists, $c$ are literal constants, $p$ are expression pairs,
$me$ are expressions with multiple results, $f$ are function declarations,
$m$ are method declarations, $n$ are variable names,
$t$ are first-level types, and $r$ are second-level types.
The notation $\vec{n{:}t}$ denotes the non-empty list
$n_{1}{:}t_{1}, ..., n_{n}{:}t_{n}$.

Our reduced core also splits function application, method application, and
the vararg expression (${...}$) into different syntactic categories.
It uses $e(el)_{s}$ to denote function applications that return no value
because they appear as statements,
$e(el)_{e}$ to denote function applications that return only one value,
and $e(el)_{m}$ to denote function applications that return multiple values.
It uses the same categories for method applications, but only ${...}_{e}$
and ${...}_{m}$, as the vararg expression cannot appear as a statement.

We also include two kinds of type cast in our core language:
the expression ${<}t{>} \;n$ and the left-hand value $n[c] \; {<}t{>}$.
We will show that the first helps allowing safe aliasing between
table types, while the second allows the refinement of table types.
Due to the refinement of table types we also split variable names
into two categories: $n_{e}$ when they appear as expressions and
$n_{l}$ when they appear as left-hand values.

We will present the typing rules as a deduction system for the typing
relations $\env_{1} \vdash s : \env_{2}$ and $\env_{1} \vdash e : t, \env_{2}$. 
We will use the first one for typing statements, and the second one
for typing expressions.
The first relation means that given the type environment $\env_{1}$,
typing the statement $s$ produces the new type environment $\env_{2}$. 
The second relation means that given the type environment $\env_{1}$,
the expression $e$ has type $t$, and produces the new type environment $\env_{2}$.
We need two typing relations because only expressions have types,
though both expressions and statements can modify the environment.
The environment maps variables to types, and we will use $\env_{1}[n \mapsto t]$
for meaning that we extended the environment $\env_{1}$ with the
variable $n$ that maps to type $t$, and this mapping does not affect
the other mappings that existed before.

We start with the rule that defines the most unusual feature of
our type system: the refinment of table types.
The rule \textsc{T-REFINE} allow us to add new keys to \emph{open}
and \emph{unique} table types:
\[
\begin{array}{c}
\mylabel{T-REFINE}\\
\dfrac{\begin{array}{c}
       \env_{1}(n) = \{ k_{1}{:}v_{1}, ..., k_{n}{:}v_{n} \}_{open|unique}\\
       \env_{1} \vdash c:t_{1}, \env_{2} \;\;\;
       \not \exists i \in 1..n \; t_{1} \lesssim k_{i} \;\;\;
       \Nil \not\lesssim t_{1} \;\;\;
       t_{2} = close(t)
       \end{array}}
      {\env_{1} \vdash n[c] {<}t{>}:t, \env_{2}[n \mapsto \{ k_{1}{:}v_{1}, ..., k_{n}{:}v_{n}, t_{1}{:}t_{2}\}_{open|unique}]}
\end{array}
\]

We restricted the refinement of table types to include only literal
keys, because its purpose is to make it easier the construction of
table types that represent records.

We use the refinement of table types to handle global variables.
In Lua, the assignment \texttt{v = v + 1} translates to
\texttt{\string_ENV["v"] = \string_ENV["v"] + 1} when \texttt{v}
is not a local variable, where \texttt{\string_ENV} is a table
that stores the global environment.
For this reason, Typed Lua treats accesses to global variables as field accesses
to an open table in the top-level scope.
For instance,
\[
\string_ENV[``x"] \; {<}\String{>} = ``foo" \;;\; \string_ENV[``y"] \; {<}\Integer{>} = 1
\]
uses field assignment to add fields $``x"$ and $``y"$ to $\string_ENV$.
In all the examples we assume that $\string_ENV$ is in the environment and
has type $\{\}_{open}$.
Therefore, after these field assignments $\string_ENV$ has type
$\{``x":\String, ``y":\Integer\}_{open}$.

We do not allow the refinement of table types to change the type of
a field that is already present in the table type.
For instance,
\[
\string_ENV[``x"] \; {<}\String{>} = ``foo" \;;\; \string_ENV[``x"] \; {<}\Integer{>} = 1
\]
do not type check, as we are trying to add a field that already exists
in $\string_ENV$.
We could have used union types to allow the refinement of existing fields,
but this could lead to fields that have a too general type, which could
decrease the ammount of static type checking.
This means that our type system does not allow the table type to change
towards a type that is not a subtype of the previous type.
In this example,
\[
\{``x":\String \cup \Integer\}_{open} \not\subtype \{``x":\String\}_{open}
\]

We also do not allow the refinement of table types to introduce
\emph{open} or \emph{unique} table fields.
For instance,
\begin{center}
\begin{tabular}{l}
$\string_ENV[``x"] \; {<}\{\}_{unique}{>} = \{\}$
\end{tabular}
\end{center}
refines the type of $\string_ENV$ from $\{\}_{open}$ to $\{``x":\{\}_{closed}\}_{open}$,
though we are trying to refine to $\{``x":\{\}_{unique}\}_{open}$.
We made this restriction because we still need to prove that the refinement
of table types is safe.

Lua has multiple assignments, so Typed Lua also has to take this into
consideration while refining table types.
The rule \textsc{T-ASSINGMENT} shows that tuple types make it easy
to use the consistent-subtyping relation to type check multiple assignments:
\[
\begin{array}{c}
\mylabel{T-ASSIGNMENT}\\
\begin{array}{c}
\dfrac{\env_{1} \vdash el:r_{1}, \env_{2} \;\;\;
       \env_{2} \vdash \vec{l}:r_{2}, \env_{3} \;\;\;
       r_{1} \lesssim r_{2}}
      {\env_{1} \vdash \vec{l} = el:\env_{3}}
\end{array}
\end{array}
\]

This rule is so simple because we left the difficult part to the
type checking of expression lists.
The list of left-hand values is also a kind of expression list.
The difficult part is to check whether there is an expression in
the expression list that messes the environment, and the rule
\textsc{T-LHSLIST} captures this intuition:
\[
\begin{array}{c}
\mylabel{T-LHSLIST}\\
\dfrac{\env \vdash l_{k}:t_{k}, \env_{k} \;\;\;
       \env_{m} = merge(\env_{1}, ..., \env_{n}) \;\;\;
       n = |\;\vec{l}\;|}
      {\env \vdash \vec{l}:t_{1} \times ... \times t_{n} \times \Value{*}, \env_{m}}
\end{array}
\]

First, this rule uses the same environment $\env$ to type check each
left-hand side value $l_{k}$, as they can perform different changes
to the environment.
Then, it takes each type $t_{k}$ and builds the tuple type for
the assignment rule.
It places the type $\Value{*}$ in the end to discard extra values.
The rule succeeds if the predicate \emph{merge} can
produce a new environment, which includes all the modifications
of each environment $\env_{k}$.
Intuitively, the predicate \emph{merge} fails when an environment
tries to change the type of a variable towards a type that is not
a subtype of its previous type.

We can also use multiple assignments to refine table types:
\[
\string_ENV[``x"] \; {<}\String{>}, \string_ENV[``y"] \; {<}\Integer{>} = ``foo", 1
\]

This example type checks because all the environment changes are consistent, and
$``foo" \times 1 \times \Nil{*} \lesssim \String \times \Integer \times \Value{*}$.
The type of the expression list includes the type $\Nil{*}$ in the end
because our type system has typing rules to add this type,
when the type of the expression list does not end with a variadic type.
Our type system has this behavior to use $\Nil$ in the place of missing
values, like Lua does.
This example uses the rule \textsc{T-EXPLIST2}, which is really close
to the rule \textsc{T-LHSLIST}:
\[
\begin{array}{c}
\mylabel{T-EXPLIST2}\\
\dfrac{\env \vdash e_{k}:t_{k}, \env_{k} \;\;\;
       \env_{m} = merge(\env_{1}, ..., \env_{n}) \;\;\;
       n = |\;\vec{e}\;|}
      {\env \vdash \vec{e}:t_{1} \times ... \times t_{n} \times \Nil{*}, \env_{m}}
\end{array}
\]

The next example does not type check because it tries to add
the same field to $\string_ENV$, but with different types:
\[
\string_ENV[``x"] \; {<}\String{>}, \string_ENV[``x"] \; {<}\Integer{>} = ``foo", 1
\]

Typed Lua also keeps track whether it is safe or not safe to
change the type of a table, but before discussing this subject,
we will discuss the different table constructors that our
abstract syntax includes, as they will appear in the next
examples.

The simplest form of the table constructor is the empty table $\{\}$.
Its inference rule is straightforward:
\[
\begin{array}{c}
\mylabel{T-CONSTRUCTOR1}\\
\env \vdash \{\}:\{\}_{unique}, \env
\end{array}
\]

Our abstract syntax reduces the more complex uses of the table
constructor into three forms: $\{\;\vec{p}\;\}$, $\{\;{...}_{m}\;\}$,
and $\{\;\vec{p},{...}_{m}\;\}$.
The first one uses a list of expression pairs $[e_{1}] = e_{2}$,
the second one uses just the vararg expression, and the third one uses both.
We did not include in the abstract syntax a table constructor
that uses a list of expressions because we can write it using the first form.
For instance, we can write the table constructor
$\{ [1] = ``x", [2] = ``y", [3] = ``z" \}$ to express $\{ ``x", ``y", ``z" \}$.
We chose to not include this form because its inference rule is similar
to the inference rule of the first form, and just controls the type of
the omitted indices.
For the same reason we did not include table constructors with
multiple expressions, which can return a tuple type that does not end
with a variadic type.
We did include the vararg expression in the last position to show how
our type system can infer a table type that has a record part and
also an array part.

The inference rule \textsc{T-CONSTRUCTOR2} infers a table type for
the first form:
\[
\begin{array}{c}
\mylabel{T-CONSTRUCTOR2}\\
\dfrac{\env_{i} \vdash p_{i}:k_{i}{:}v_{i}, \env_{i+1} \;\;\;
       n = |\;\vec{p}\;| \;\;\;
       t = wf(\{k_{1}{:}v_{1}, ..., k_{n}{:}v_{n}\}_{unique})}
      {\env_{1} \vdash \{\;\vec{p}\;\}:t, \env_{n+1}}
\end{array}
\]

This inference rule infers the type of each pair, uses these types
to build a table type, and uses the predicate \emph{wf} to check
whether the table type is well formed.
Formally, a table type is well-formed if it obeys the following rule:
\[
\forall i \not\exists j \; i \not= j \wedge k_{i} \lesssim k_{j}
\]

Well-formed table types avoid ambiguous table types.
For instance, this rule detects that the table type
$\{1:\Number, \Integer:\String, \Any:\Boolean\}$ is ambiguous,
because the type of the value stored by key $1$ can be
$\Number$, $\String$, or $\Boolean$, as $1 \lesssim 1$,
$1 \lesssim \Integer$, and $1 \lesssim \Any$.
Moreover, the type of the value stored by a key of type $\Integer$,
which is not the literal type $1$, can be $\Number$ or $\Boolean$,
as $\Integer \lesssim \Integer$, and $\Integer \lesssim \Any$.

The inference rules for expression pairs would be straightforward
if our table types allowed any type in the keys.
The rules \textsc{T-PAIR1}, \textsc{T-PAIR2}, and \textsc{T-PAIR3}
check if the type of the key is a subtype of $\Boolean$, $\Number$,
or $\String$, respectively:
\[
\begin{array}{c}
\begin{array}{c}
\mylabel{T-PAIR1}\\
\dfrac{\env_{1} \vdash e_{1}:t_{1}, \env_{2} \;\;\;
       \env_{2} \vdash e_{2}:t_{2}, \env_{3} \;\;\;
       t_{1} \subtype \Boolean}
      {\env_{1} \vdash [e_{1}] = e_{2}: t_{1}{:}close(t_{2}), \env_{3}}
\end{array}
\\ \\
\begin{array}{c}
\mylabel{T-PAIR2}\\
\dfrac{\env_{1} \vdash e_{1}:t_{1}, \env_{2} \;\;\;
       \env_{2} \vdash e_{2}:t_{2}, \env_{3} \;\;\;
       t_{1} \subtype \Number}
      {\env_{1} \vdash [e_{1}] = e_{2}: t_{1}{:}close(t_{2}), \env_{3}}
\end{array}
\\ \\
\begin{array}{c}
\mylabel{T-PAIR3}\\
\dfrac{\env_{1} \vdash e_{1}:t_{1}, \env_{2} \;\;\;
       \env_{2} \vdash e_{2}:t_{2}, \env_{3} \;\;\;
       t_{1} \subtype \String}
      {\env_{1} \vdash [e_{1}] = e_{2}: t_{1}{:}close(t_{2}), \env_{3}}
\end{array}
\end{array}
\]

The rule \textsc{T-PAIR4} is a fallback for the previous rules,
but it works only if the type of the key does not include $\Nil$:
\[
\begin{array}{c}
\mylabel{T-PAIR4}\\
\dfrac{\env_{1} \vdash e_{1}:t_{1}, \env_{2} \;\;\;
       \env_{2} \vdash e_{2}:t_{2}, \env_{3} \;\;\;
       \Nil \not\subtype t_{1}}
      {\env_{1} \vdash [e_{1}] = e_{2}: \Any{:}close(t_{2}), \env_{3}}
\end{array}
\]

The rule \textsc{T-CONSTRUCTOR2} along with the rules for expression
pairs allow Typed Lua to type check the following example:
\begin{center}
\begin{tabular}{ll}
\multicolumn{2}{l}{$\mathbf{local} \; a:\{\String:\Integer \cup \Nil\}_{closed} = \{ [``x"] = 1, [``y"] = 2 \} \; \mathbf{in}$}\\
& \multicolumn{1}{l}{$\mathbf{local} \; b:\Integer \cup \Nil = a[``x"] \; \mathbf{in} \; b = a[``z"]$}
\end{tabular}
\end{center}

This example type checks because Typed Lua infers the table type
$\{``x":1, ``y":2\}_{unique}$, which is subtype of
$\{\String:\Integer \cup \Nil\}_{closed}$.
Even though the key $``z"$ does not exist in the table $a$,
the two table indexations type check with type $\Integer \cup \Nil$,
which is subtype of the type of $b$.
The rule \textsc{T-INDEX1} describes this behavior:
\[
\begin{array}{c}
\mylabel{T-INDEX1}\\
\dfrac{\env_{1} \vdash e_{1}:\{k_{1}{:}v_{1}, ..., k_{n}{:}v_{n}\}, \env_{2} \;\;\;
       \env_{2} \vdash e_{2}:t, \env_{3} \;\;\;
       \exists i \in 1{..}n \; t \lesssim k_{i}}
      {\env_{1} \vdash e_{1}[e_{2}]:v_{i}, \env_{3}}
\end{array}
\]

Typed Lua handles arrays as hashes that maps integers to some type $t$.
In Lua, programmers often use the vararg expression to initialize arrays.
The rules \textsc{T-CONSTRUCTOR3} and \textsc{T-CONSTRUCTOR4} define
the behavior of the vararg expression for this case:
\[
\begin{array}{c}
\begin{array}{c}
\mylabel{T-CONSTRUCTOR3}\\
\dfrac{\env_{1}({...}) = t}
      {\env_{1} \vdash \{{...}_{m}\}:\{\Integer{:}t \cup \Nil\}_{unique}, \env_{1}}
\end{array}
\\ \\
\begin{array}{c}
\mylabel{T-CONSTRUCTOR4}\\
\dfrac{\begin{array}{c}
       \env_{i} \vdash p_{i}:k_{i}{:}v_{i}, \env_{i+1} \;\;\;
       n = |\;\vec{p}\;| \;\;\;
       \env_{i+1}({...}) = t_{v}\\
       wf(\{k_{1}{:}v_{1}, ..., k_{n}{:}v_{n}, \Integer{:}t_{v} \cup \Nil\}_{unique}) = t_{t}
       \end{array}}
      {\env_{1} \vdash \{\;\vec{p},\;{...}_{m}\;\}:t_{t}, \env_{n+1}}
\end{array}
\end{array}
\]

The rule \textsc{T-CONSTRUCTOR3} type checks the case where we use
only the vararg expression inside a table constructor to initialize
an array.
If we assume that the $...$ is in the environment and has type $\String$,
the following example type checks through this rule:
\begin{center}
\begin{tabular}{ll}
\multicolumn{2}{l}{$\mathbf{local} \; a:\{\Integer:\String \cup \Nil\}_{closed} = \{ {...}_{m} \} \; \mathbf{in}$}\\
& \multicolumn{1}{l}{$\mathbf{local} \; b:\String \cup \Nil = t[1] \; \mathbf{in} \; b = t[5]$}
\end{tabular}
\end{center}

The rule \textsc{T-CONSTRUCTOR4} type checks the case where we use
the table constructor as a record that includes an array part.
If we assume that the $...$ is in the environment and has type $\String$,
the following example type checks through this rule:
\begin{center}
\begin{tabular}{ll}
\multicolumn{2}{l}{$\mathbf{local} \; a:\{``x":\Integer, ``y":\Boolean, \Integer:\String \cup \Nil\}_{closed} =$}\\
& \multicolumn{1}{l}{$\{ [``x"] = ``foo", [``y"] = \mathbf{true}, {...}_{m} \}$}\\
\multicolumn{2}{l}{$\mathbf{in} \; a[``x"] = ``bar"$}
\end{tabular}
\end{center}
but the next example does not type check because the record part
includes fields that the key types are subtype of the array part:
\begin{center}
\begin{tabular}{ll}
\multicolumn{2}{l}{$\mathbf{local} \; a:\{1:\String, 10:\Boolean, \Integer:\String \cup \Nil\}_{closed} =$}\\
& \multicolumn{1}{l}{$\{ [1] = ``foo", [10] = \mathbf{true}, {...}_{m} \}$}\\
\multicolumn{2}{l}{$\mathbf{in} \; a[``x"] = ``bar"$}
\end{tabular}
\end{center}

Making the inference rules of the table constructor infer an \emph{unique}
table type also allow us to use them for initializing record types that
have optional fields.
We can include the optional field in the initialization
\begin{center}
\begin{tabular}{ll}
\multicolumn{2}{l}{$\mathbf{local} \; a:\{``x":\String, ``y":\String \cup \Nil, ``z":\String\}_{closed} =$}\\
& \multicolumn{1}{l}{$\{ [``x"] = ``foo", [``y"] = ``bar", [``z"] = ``baz" \}$}\\
\multicolumn{2}{l}{$\mathbf{in} \; \mathbf{skip}$}
\end{tabular}
\end{center}
and we can also omit it in the initialization
\begin{center}
\begin{tabular}{ll}
\multicolumn{2}{l}{$\mathbf{local} \; a:\{``x":\String, ``y":\String \cup \Nil, ``z":\String\}_{closed} =$}\\
& \multicolumn{1}{l}{$\{ [``x"] = ``foo", [``z"] = ``baz" \}$}\\
\multicolumn{2}{l}{$\mathbf{in} \; \mathbf{skip}$}
\end{tabular}
\end{center}

After discussing the inference rules of the table constructor,
we can get back to the discussion about the refinement of table types.

Besides allowing the refinement of table types through field assignment,
Typed Lua also keeps track whether it is safe or not safe to change the
type of a table.
%To do that, our type system always promotes an \emph{open}
%table to \emph{closed} and an \emph{unique} table to \emph{open}
%when they appear as expressions, or when they enter in a scope that is
%not its declaration scope.
%The order matters, as going from \emph{unique} to \emph{open}
%before going from \emph{open} to \emph{closed} closes \emph{unique} table types.
In the last section we mentioned that aliasing \emph{unique} table
types is unsafe.
To avoid unsafe aliasing of \emph{unique} table types, the rule \textsc{T-IDREAD1}
always promotes them to \emph{closed}, and updates the variable in the
environment to \emph{open}:
\[
\begin{array}{c}
\mylabel{T-IDREAD1}\\
\dfrac{\env_{1}(n) = \{k_{1}{:}v_{1}, ..., k_{n}{:}v_{n}\}_{unique}}
      {\env_{1} \vdash n_{e}:\{k_{1}{:}v_{1}, ..., k_{n}{:}v_{n}\}_{closed}, \env_{1}[n \mapsto \{k_{1}{:}v_{1}, ..., k_{n}{:}v_{n}\}_{open}]}
\end{array}
\]

This is the rule that makes the unsafe example from last section to
not type check:
\begin{center}
\begin{tabular}{lll}
\multicolumn{3}{l}{$\mathbf{local} \; a:\{``foo":\Integer\}_{unique} = \{ [``foo"] = 5 \} \; \mathbf{in}$}\\
& \multicolumn{2}{l}{$\mathbf{local} \; b:\{``foo":\Integer \cup \Nil \}_{closed} = a \; \mathbf{in}$}\\
& & \multicolumn{1}{l}{$b[``foo"] = \mathbf{nil}$}
\end{tabular}
\end{center}

This examples does not type check because aliasing $a$ produces the
type $\{``foo":\Integer\}_{closed}$ that is not subtype of
$\{``foo":\Integer \cup \Nil\}_{closed}$, the type of $b$.

However, sometimes we want to use \emph{unique} table types to initialize
\emph{closed} table types, as they allow us to initialize optional fields.
To do that, Typed Lua includes a cast expression that allow us
to use an \emph{unique} table type before it becomes \emph{open}.
The rule \textsc{T-CAST} describes this behavior:
\[
\begin{array}{c}
\mylabel{T-CAST}\\
\dfrac{t \subtype \env_{1}(n)}
      {\env_{1} \vdash {<}t{>} \; n:t, \env_{1}[n \mapsto t]}
\end{array}
\]

The rule \textsc{T-CAST} allow us to type check the following example:
\begin{center}
\begin{tabular}{lll}
\multicolumn{3}{l}{$\mathbf{local} \; a:\{\}_{unique} = \{ \} \; \mathbf{in}$}\\
& \multicolumn{2}{l}{$a[``x"] \; {<}\String{>} = ``foo";$}\\
& \multicolumn{2}{l}{$a[``y"] \; {<}\String{>} = ``bar";$}\\
& \multicolumn{2}{l}{$\mathbf{local} \; b:\{``x":\String, ``y":\String \cup \Nil \}_{closed} =$}\\
& & \multicolumn{1}{l}{${<}\{``x":\String, ``y":\String \cup \Nil\}_{open}{>} \; a \; \mathbf{in} \; a[``z"] \; {<}\Integer{>} = 1$}
\end{tabular}
\end{center}

We can use $a$ to initialize $b$ because the cast expression converts
the type of $a$ from $\{``x":\String, ``y":\String\}_{unique}$ to
$\{``x":\String, ``y":\String \cup \Nil\}_{open}$, which is subtype of
$\{``x":\String, ``y":\String \cup \Nil\}_{closed}$.
We can continue to refine the type of $a$ after the aliasing,
as it still holds an \emph{open} table.
At the end of this example, $a$ has type
$\{``x":\String, ``y":\String \cup \Nil, ``z":\Integer\}_{open}$.

We should also keep tracking of \emph{open} locals, as they can
also be unsafe like in the next example:
\begin{center}
\begin{tabular}{lll}
\multicolumn{3}{l}{$\mathbf{local} \; a:\{\}_{unique} = \{\} \; \mathbf{in}$}\\
& \multicolumn{2}{l}{$\mathbf{local} \; b:\{\}_{open} = a \; \mathbf{in}$}\\
& & \multicolumn{1}{l}{$a[``x"] \; {<}\String{>} = ``foo";$}\\
& & \multicolumn{1}{l}{$b[``x"] \; {<}\Integer{>} = 1$}\\
\end{tabular}
\end{center}

We cannot add the field $``x"$ to $b$ because its type is \emph{closed},
and thus do not allow changing the value that is stored in the field $``x"$
of local $a$.

We also need to make sure to close all \emph{open} table types before we
type check another scope.
The rule \textsc{T-FUNCTION3} shows this case:
\[
\begin{array}{c}
\mylabel{T-FUNCTION3}\\
\dfrac{closeall(\env_{1}[\vec{n} \mapsto \vec{t}, \ret \mapsto r]) \vdash s:\env_{2}}
      {\env_{1} \vdash \mathbf{fun} \; (\vec{n{:}t}){:}r \; s:\vec{t} \rightarrow r, closeset(\env_{1}, fav(\mathbf{fun} \; (\vec{n{:}t}){:}r \; s))}
\end{array}
\]

This rule prevents the following unsafe example to type check:
\begin{center}
\begin{tabular}{llll}
\multicolumn{4}{l}{$\mathbf{local} \; a:\{\}_{unique} = \{\} \; \mathbf{in}$}\\
& \multicolumn{3}{l}{$\mathbf{local} \; b:\Integer \times \Integer \rightarrow \Integer =$}\\
& & \multicolumn{2}{l}{$\mathbf{fun} \; (x:\Integer, y:\Integer):\Integer$}\\
& & & \multicolumn{1}{l}{$a[``z"] \; {<}\Integer{>} = 1; \; \mathbf{return} \; x + y$}\\
& \multicolumn{3}{l}{$\mathbf{in} \; a[``z"] \; {<}\String{>} = ``foo"$}
\end{tabular}
\end{center}

Typed Lua uses the refinement of table types to allow programmers to
build modules and objects.
It also allows Typed Lua to type check an object-oriented idiom that
programmers often use it.
\[
\begin{array}{c}
\mylabel{T-METHOD3}\\
\dfrac{\begin{array}{c}
       \env_{1}(n_{1}) = \{k_{i}{:}v_{i}, ..., k_{n}{:}v_{n}\}_{unique|open} \;\;\;
       \env_{1} \vdash n_{2} : l \;\;\;
       \not \exists i \in 1..n \; l \lesssim k_{i}\\
       closeall(\env_{1}[self \mapsto \{k_{i}{:}v_{i}, ..., k_{n}{:}v_{n}\}_{closed}, \vec{n} \mapsto \vec{t}, \ret \mapsto r]) \vdash s:\env_{2}\\
       t_{m} = \{k_{i}{:}v_{i}, ..., k_{n}{:}v_{n}, l{:}\Const \; \Self \times \vec{t} \rightarrow r\}_{unique|open}
       \end{array}}
      {\begin{array}{c}
       \env_{1} \vdash \mathbf{fun} \; n_{1}{:}n_{2} \; (\vec{n{:}t}){:}r \; s:\\
       closeset(\env_{1}[n_{1} \mapsto t_{m}, fav(\mathbf{fun} \; (\vec{n{:}t}){:}r \; s))
       \end{array}}
\end{array}
\]

The rule \textsc{T-METHOD3} allow our type system to type check the
following example:
\begin{center}
\begin{tabular}{llll}
\multicolumn{4}{l}{$\mathbf{local} \; s:\{``x":\Number, ``y":\Number\}_{open} = \{ [``x"] = 0.0, [``y"] = 0.0 \}$}\\
\multicolumn{4}{l}{$\mathbf{in}$}\\
& \multicolumn{3}{l}{$\mathbf{fun} \; s{:}new (x:\Number, y:\Number):\Self$}\\
& & \multicolumn{2}{l}{$\mathbf{local} \; s:\{``x":\Number, ``y":\Number\}_{closed} =$}\\
& & & \multicolumn{1}{l}{$setmetatable(\{\}, \{ [``\string_\string_index"] = self \})$}\\
& & \multicolumn{2}{l}{$\mathbf{in} \; s[``x"] = x; \; s[``y"] = y; \; \mathbf{return} \; s$}\\
; & \multicolumn{3}{l}{$\mathbf{fun} \; s{:}move (x:\Number, y:\Number):\Void$}\\
& & \multicolumn{2}{l}{$self[``x"] = self[``x"] + x;$}\\
& & \multicolumn{2}{l}{$self[``y"] = self[``y"] + y$}
\end{tabular}
\end{center}

This example uses the type $\{``x":\Number, ``y":\Number\}_{open}$ to
initialize the local variable $s$, which represents the type of the class
that we are defining.
Then, we use two method definitions to include $new$ and $move$ into
our class that now has type
\begin{align*}
\{ & ``x":\Number, ``y":\Number,\\
   & ``new":\Self \times \Number \times \Number \rightarrow \Self,\\
   & ``move":\Self \times \Number \times \Number \rightarrow \Void \}_{open}
\end{align*}

Inside the definition of the method \emph{new} we use \emph{setmetatable}
to initialize the local $s$ with the type of $\Self$.
The rule \textsc{T-SETMETATABLE} express this idea:
\[
\begin{array}{c}
\mylabel{T-SETMETATABLE}\\
\dfrac{\env_{1} \vdash e:t, \env_{2} \;\;\;
       t \subtype \{\}_{open}}
      {\env_{1} \vdash setmetatable(\{\}, \{[``\string_\string_index"] = e\}):close(t), \env_{2}}
\end{array}
\]

After we define our class, we can use it to create object instances
of this class and call its methods.
The next example assumes the object $o$ is in the environment as
has the type of the class we just defined:
\begin{center}
\begin{tabular}{l}
$o{:}move(10, 10)_{s}$
\end{tabular}
\end{center}

This method call type checks through the rule \textsc{T-INVOKESTM1}.
This rule uses the inference rule \textsc{T-INVOKE1} and discards the results.
We defined the rule \textsc{T-INVOKE1} as follows:
\[
\begin{array}{c}
\mylabel{T-INVOKE1}\\
\dfrac{\env_{1} \vdash e:t, \env_{2} \;\;\;
       \env_{2} \vdash e[n]:p_{1} \rightarrow r, \env_{3} \;\;\;
       \env_{3} \vdash el:p_{2}, \env_{4} \;\;\;
       \Self \times p_{2} \lesssim p_{1}}
      {\env_{1} \vdash e{:}n(el)_{m}:[\Self \mapsto t]r, \env_{4}}
\end{array}
\]

Multiple assignments and multiple return values can also appear in
function applications, as the following example shows:
\begin{center}
\begin{tabular}{llll}
\multicolumn{4}{l}{$\mathbf{local} \; m:\Void \rightarrow \Integer \times \String =$} \\
& \multicolumn{3}{l}{$\mathbf{fun} \; ():\Integer \times \String$} \\
& & \multicolumn{2}{l}{$\mathbf{return} \; 2, ``foo"$} \\
\multicolumn{4}{l}{$\mathbf{in}$} \\
& \multicolumn{3}{l}{$\mathbf{local} \; s:\Integer \times \Integer \rightarrow \Integer =$} \\
& & \multicolumn{2}{l}{$\mathbf{fun} \; (x:\Integer, y:\Integer):\Integer$} \\
& & & \multicolumn{1}{l}{$\mathbf{return} \; x + y$} \\
& \multicolumn{3}{l}{$\mathbf{in} \; s(m()_{e}, m()_{m})_{s}$}
\end{tabular}
\end{center}

This example does not type check because we are trying to call
$s$ with $\Integer \times \Integer \times \String$,
but its input parameter has type $\Integer \times \Integer$.
Lua drops extra values in function calls, and Typed Lua can also
have this behavior.
If we change the input type of $s$ to $\Integer \times \Integer \times \Value{*}$,
this example type checks.
Typed Lua also has to adjust function calls to use $\Nil$ in the
place of missing parameters, otherwise it would not be able to
type check function calls that have optional parameters.

\textit{/* I guess we should have different typing rules for strict and default. */}

Typed Lua includes four typing rules to handle the $\mathbf{or}$
logical operator and its common idioms:
\[
\begin{array}{c}
\begin{array}{c}
\mylabel{T-OR1}\\
\dfrac{\env_{1} \vdash e_{1}:t, \env_{2} \;\;\;
       \Nil \not\lesssim t \;\;\;
       \False \not\lesssim t}
      {\env_{1} \vdash e_{1} \; \mathbf{or} \; e_{2}:t, \env_{2}}
\end{array}
\\ \\
\begin{array}{c}
\mylabel{T-OR2}\\
\dfrac{\env_{1} \vdash e_{1}:\Nil, \env_{2} \;\;\;
       \env_{2} \vdash e_{2}:t, \env_{3}}
      {\env_{1} \vdash e_{1} \; \mathbf{or} \; e_{2}:t, \env_{3}}
\end{array}
\;
\begin{array}{c}
\mylabel{T-OR3}\\
\dfrac{\env_{1} \vdash e_{1}:\False, \env_{2} \;\;\;
       \env_{2} \vdash e_{2}:t, \env_{3}}
      {\env_{1} \vdash e_{1} \; \mathbf{or} \; e_{2}:t, \env_{3}}
\end{array}
\\ \\
\begin{array}{c}
\mylabel{T-OR4}\\
\dfrac{\env_{1} \vdash e_{1}:t_{1}, \env_{2} \;\;\;
       \env_{2} \vdash e_{2}:t_{2}, \env_{3}}
      {\env_{1} \vdash e_{1} \; \mathbf{or} \; e_{2}:filter(filter(t_{1}, \Nil), \False) \cup t_{2}, \env_{3}}
\end{array}
\end{array}
\]

The rule \textsc{T-OR1} is the rule that implements the short circuit.
We use the consistent-subtyping relation in this rule to guarantee that
the type system checks the second expression when the first one
has the dynamic type, as it can be hiding a false value.

The rules \textsc{T-OR2} and \textsc{T-OR3} guarantee that the final
result is the type of the second expression, because the first one
is certainly a false value.

The rule \textsc{T-OR4} is the most general rule, but it is also
the rule that handles the common $\mathbf{or}$ idioms.
It uses the predicate \emph{filter} to filter possible false values
that might be part of the type of the first expression.
We can use pattern matching to the define the recursive predicate
\emph{filter} as follows:
\begin{align*}
filter(t_{1} \cup t_{2}, t_{1}) & = filter(t_{2}, t_{1})\\
filter(t_{1} \cup t_{2}, t_{2}) & = filter(t_{1}, t_{2})\\
filter(t_{1} \cup t_{2}, t_{3}) & = filter(t_{1}, t_{3}) \cup filter(t_{2}, t_{3})\\
filter(t_{1}, t_{2}) & = t_{1}
\end{align*}

Using these typing rules our type system can type check the following
example:
\begin{center}
\begin{tabular}{ll}
\multicolumn{2}{l}{$\mathbf{local} \; x:\String \cup \Nil = \mathbf{nothing} \; \mathbf{in}$}\\
& \multicolumn{1}{l}{$\mathbf{local} \; y:\String = x \; \mathbf{or} \; ``Hello"$}
\end{tabular}
\end{center}

Without the \emph{filter} predicate,
the expression $x \; \mathbf{or} \; ``Hello"$ would have type
$\String \cup \Nil \cup ``Hello"$.
The \emph{filter} predicate removes the type $\Nil$ from the result,
leaving type $\String \cup ``Hello"$.
At the end of the evaluation, the expression has type $\String$
because unions are disjoint and $``Hello" \subtype \String$.

\textit{/* The following example does not type check */}
\begin{center}
\begin{tabular}{ll}
\multicolumn{2}{l}{$\mathbf{local} \; x:\Nil \cup \False = \mathbf{false} \; \mathbf{in}$}\\
& \multicolumn{1}{l}{$\mathbf{local} \; y:\String = x \; \mathbf{or} \; ``Hello"$}
\end{tabular}
\end{center}

\textit{/* Should we define the following rules? */}
\[
\begin{array}{c}
\begin{array}{c}
\mylabel{T-OR4}\\
\dfrac{\env_{1} \vdash e_{1}:t_{1}, \env_{2} \;\;\;
       \env_{2} \vdash e_{2}:t_{2}, \env_{3} \;\;\;
       \Nil \cup \False \subtype t_{1} \;\;\;
       t_{1} \subtype \Nil \cup \False}
      {\env_{1} \vdash e_{1} \; \mathbf{or} \; e_{2}:t_{2}, \env_{3}}
\end{array}
\\ \\
\begin{array}{c}
\mylabel{T-OR5}\\
\dfrac{\env_{1} \vdash e_{1}:t_{1}, \env_{2} \;\;\;
       \env_{2} \vdash e_{2}:t_{2}, \env_{3} \;\;\;
       \Nil \subtype t_{1} \;\;\;
       \False \subtype t_{1}}
      {\env_{1} \vdash e_{1} \; \mathbf{or} \; e_{2}:filter(filter(t_{1}, \Nil), \False) \cup t_{2}, \env_{3}}
\end{array}
\\ \\
\begin{array}{c}
\mylabel{T-OR6}\\
\dfrac{\env_{1} \vdash e_{1}:t_{1}, \env_{2} \;\;\;
       \env_{2} \vdash e_{2}:t_{2}, \env_{3} \;\;\;
       \Nil \subtype t_{1} \;\;\;
       \False \not\subtype t_{1}}
      {\env_{1} \vdash e_{1} \; \mathbf{or} \; e_{2}:filter(t_{1}, \Nil) \cup t_{2}, \env_{3}}
\end{array}
\\ \\
\begin{array}{c}
\mylabel{T-OR7}\\
\dfrac{\env_{1} \vdash e_{1}:t_{1}, \env_{2} \;\;\;
       \env_{2} \vdash e_{2}:t_{2}, \env_{3} \;\;\;
       \Nil \not\subtype t_{1} \;\;\;
       \False \subtype t_{1}}
      {\env_{1} \vdash e_{1} \; \mathbf{or} \; e_{2}:filter(t_{1}, \False) \cup t_{2}, \env_{3}}
\end{array}
\\ \\
\begin{array}{c}
\mylabel{T-OR8}\\
\dfrac{\env_{1} \vdash e_{1}:t_{1}, \env_{2} \;\;\;
       \env_{2} \vdash e_{2}:t_{2}, \env_{3} \;\;\;
       \Nil \not\subtype t_{1} \;\;\;
       \False \not\subtype t_{1}}
      {\env_{1} \vdash e_{1} \; \mathbf{or} \; e_{2}:t_{1} \cup t_{2}, \env_{3}}
\end{array}
\end{array}
\]

\textit{/* Or who writes this kind of code deserves a type error? */}

Another common idiom that programmers use in Lua is to overload
the input parameter of functions, and use the function \texttt{type}
to execute different actions according to their types.
\[
\begin{array}{c}
\mylabel{T-IF2}\\
\dfrac{\begin{array}{c}
       \env_{1}(n) = t\\
       closeall(\env_{1}[n \mapsto \String]) \vdash s_{1}:\env_{2} \\
       closeall(\env_{1}[n \mapsto filter(t, \String)) \vdash s_{2}:\env_{3}\\
       \env_{4} = closeset(\env_{1}[n \mapsto t], fav(s_{1}) \cup fav(s_{2}))
      \end{array}}
      {\env_{1} \vdash \mathbf{if} \; type(n) == ``string" \; \mathbf{then} \; s_{1} \; \mathbf{else} \; s_{2}:\env_{4}}

\end{array}
\]

Using the rule \textsc{T-IF2}, Typed Lua can type check the following example:
\begin{center}
\begin{tabular}{llll}
\multicolumn{4}{l}{$\mathbf{local} \; o:\String \times \String \cup \Integer \rightarrow \String =$}\\
& \multicolumn{3}{l}{$\mathbf{fun} \; (a:\String, b:\String \cup \Integer):\String$}\\
& & \multicolumn{2}{l}{$\mathbf{local} \; r:\String = ``" \; \mathbf{in}$}\\
& & & \multicolumn{1}{l}{$\mathbf{if} \; type(b) == ``string" \; \mathbf{then} \; r = a \;{..}\;b \; \mathbf{else} \; r = rep(a, b)_{e}\; ;$}\\
& & & \multicolumn{1}{l}{$\mathbf{return} \; r$}\\
\multicolumn{4}{l}{$\mathbf{in} \; o(``foo", 2)_{s}$}
\end{tabular}
\end{center}

We are assuming that the function \emph{rep} is in the environment and has type
$\String \times \Integer \times \String \cup \Nil \rightarrow \String$.

Typed Lua also includes similar rules to handle the tags \texttt{nil},
\texttt{boolean}, and \texttt{number}.
There is also a similar rule for handling the type $\Integer$, but
it works only with Lua 5.3, as it depends on the function \texttt{math.type}.
This function appears only in Lua 5.3 and it returns the string $``integer"$
when its input parameter is a number that has an integer representation,
the string $``float"$ when its input parameter is a number that has a
floating point representation, or $\Nil$ otherwise.

Lua programmers also overload the return type of functions to denote
errors, and Typed Lua also handles this case with the rule \textsc{T-IF3}:
\[
\begin{array}{c}
\mylabel{T-IF3}\\
\dfrac{\begin{array}{c}
       \env_{1}(n) = \pi_{i}\\
       closeall(\env_{1}[\pi \mapsto fpt(\env_{1}(\pi), \Nil, i)]) \vdash s_{1}:\env_{2} \\
       closeall(\env_{1}[\pi \mapsto gpt(\env_{1}(\pi), \Nil, i)) \vdash s_{2}:\env_{3}
      \end{array}}
      {\env_{1} \vdash \mathbf{if} \; n \; \mathbf{then} \; s_{1} \; \mathbf{else} \; s_{2}:closeset(\env_{1}, fav(s_{1}) \cup fav(s_{2}))}

\end{array}
\]

\begin{center}
\begin{tabular}{ll}
\multicolumn{2}{l}{$\mathbf{local} \; q:\pi_{1}, r:\pi_{2} = idiv(1, 2)_{m} \; \mathbf{in}$}\\
& \multicolumn{1}{l}{$\mathbf{if} \; q \; \mathbf{then} \; print(q + r)_{s} \; \mathbf{else} \; print(``ERROR: " \; .. \; r)_{s}$}
\end{tabular}
\end{center}

\textit{/* How should we write the typing rule of the local declaration? */}

We are assuming that functions \emph{idiv} and \emph{print} are in
the environment with the respective types
$\Integer \times \Integer \rightarrow \Integer \times \Integer \sqcup \Nil \times \String$
and
$\Value{*} \rightarrow \Void$.

Lua has operator overloading, and allows the programmers to redefine
the behavior of some operations.
For instance, programmers can use metatables to redefine the
behavior of arithmetic operations.
Even though Typed Lua does not support operator overloading yet,
it includes typing rules that allow programmers to use the
dynamic type when they are using overloaded operations.
The following typing rules show how Typed Lua uses the dynamic type
to handle the overloading of arithmetic operations:
\[
\begin{array}{c}
\begin{array}{c}
\mylabel{T-ARITH5}\\
\dfrac{\env_{1} \vdash e_{1}:\Any, \env_{2} \;\;\;
       \env_{2} \vdash e_{2}:t, \env_{3}}
      {\env_{1} \vdash e_{1} + e_{2}:\Any, \env_{3}}
\end{array}
\;
\begin{array}{c}
\mylabel{T-ARITH6}\\
\dfrac{\env_{1} \vdash e_{1}:t, \env_{2} \;\;\;
       \env_{2} \vdash e_{2}:\Any, \env_{3}}
      {\env_{1} \vdash e_{1} + e_{2}:\Any, \env_{3}}
\end{array}
\end{array}
\]

The rule \textsc{T-ARITH5} allows type checking the following
example:
\begin{center}
\begin{tabular}{l}
$\mathbf{local} \; x{:}\Any = 1 \; \mathbf{in} \; x = x + 1$
\end{tabular}
\end{center}

This example is safe, but the following it is not:
\begin{center}
\begin{tabular}{l}
$\mathbf{local} \; x{:}\Integer, \;y{:}\Any = 1 \; \mathbf{in} \; x = x + y$
\end{tabular}
\end{center}

Although this last example is not safe, it shows that optional
type systems still preserve the flexibility of dynamically
typed languages along with the benefits of static type checking.


\chapter{Evaluation}
\label{chap:evaluation}

We performed some case studies on existing Lua libraries
to evaluate the design of our type system.
For each library, we used Typed Lua to either rewrite its modules
or to write statically typed interfaces to its modules through
Typed Lua's description files.
In this chapter we present our evaluation results, discuss
some interesting cases, and compare our type system to
related work.

\begin{table}[!ht]
\begin{center}
\begin{tabular}{|l|c|c|c|c|c|c|}
\hline
\textbf{Case study} & \textbf{easy} & \textbf{poly} & \textbf{hard} & \textbf{Total} & \textbf{\%} \\
\hline
Lua Standard Libraries & 94 & 5 & 34 & 133 & 71\% \\
\hline
MD5 & 13 & 0 & 0 & 13 & 100\% \\
\hline
LuaSocket & 109 & 1 & 13 & 123 & 89\% \\
\hline
HTTP Digest & 0 & 0 & 1 & 1 & 0\% \\
\hline
Typical & 1 & 0 & 0 & 1 & 100\% \\
\hline
Modulo 11 & 7 & 0 & 2 & 9 & 78\% \\
\hline
\end{tabular}
\end{center}
\caption{Evaluation results for each case study}
\label{tab:evalbycase}
\end{table}

Table \ref{tab:evalbycase} sumarizes our evaluation results for each
case study that we used Typed Lua for typing their members.
We split the members that we typed into three categories:
\emph{easy}, \emph{poly}, and \emph{hard}.
The \emph{easy} category represents the members that we could give
a precise static type for them.
The \emph{poly} category represents the members that we believe that
parametric polymorphism would help give a precise static type for them,
and we had to rely on the dynamic type to type them because our
type system does not support parametric polymorphism.
The \emph{hard} category represents the members that even parametric
polymorphism would not help give a precise static type for them,
and we also had to rely on the dynamic type to type them;
for instance, this category includes functions that require
intersection types to describe their precise static type.
The last column of the table shows the percentage of members that
are under the \emph{easy} category for each case study.
This percentage is our evaluation of Typed Lua, as it
represents how much static typing we could introduce to each one of
our case studies.

Before comparing our type system to related work, we will discuss
each case study in more detail.
For each case study, we will split the evaluation results according
to the modules that each one of them include.
This shall allow us to better discuss the contributions and limitations
of our type system.

\section{Lua Standard Libraries}

The Lua Standard Libraries \citep{luamanual} were our first case study.
We started to think about how we would type them at the same time that
we started to design our type system, as they could give us some hints
on our type system.
And they did: optional parameters and overloading on the return type
are two Lua features that our type system should handle to allow us
typing some of the functions that the standard libraries provide.

All libraries are separated C modules, and we used Typed Lua's description
files to give a statically typed interface to each module.
The \texttt{debug} module is the only one that we did not include in our
evaluation results, because it provides several functions that violate
basic assumptions about Lua code \citep{luamanual}.
For instance, we can use its function \texttt{setlocal} to change the value
of a local variable that is outside of the scope.
Table \ref{tab:evallsl} sumarizes the evaluation results for the Lua Standard Libraries.

\begin{table}[!ht]
\begin{center}
\begin{tabular}{|l|c|c|c|c|c|c|}
\hline
\textbf{Case study} & \textbf{Module} & \textbf{easy} & \textbf{poly} & \textbf{hard} & \textbf{Total} & \textbf{\%} \\
\hline
\multirow{9}{*}{Lua Standard Libraries}
& base & 9 & 1 & 16 & 26 & 35\% \\
\cline{2-7}
& coroutine & 0 & 0 & 6 & 6 & 0\% \\
\cline{2-7}
& package & 5 & 0 & 3 & 8 & 62\% \\
\cline{2-7}
& string & 14 & 0 & 0 & 14 & 100\% \\
\cline{2-7}
& table & 1 & 4 & 1 & 6 & 17\% \\
\cline{2-7}
& math & 28 & 0 & 1 & 29 & 97\% \\
\cline{2-7}
& bit32 & 12 & 0 & 0 & 12 & 100\% \\
\cline{2-7}
& io & 15 & 0 & 6 & 21 & 71\% \\
\cline{2-7}
& os & 10 & 0 & 1 & 11 & 91\% \\
\hline
\end{tabular}
\end{center}
\caption{Evaluation results for Lua Standard Libraries}
\label{tab:evallsl}
\end{table}

We could give precise static types to most of the modules,
but some of them are still hard to type with the current type
system.
Now, we will discuss why some modules are difficult to type,
and what kind of limitations they represent on our type system.

The \texttt{base} module was quite hard to type because it
includes several functions that have dynamic behavior.
For instance, the function \texttt{next} traverses all the
fields of a table, but the order that the indices are
enumerated is not specified.
For this reason, it is also difficult to type the function
\texttt{pairs}.
The functions \texttt{getmetatable},  \texttt{rawget}, and
\texttt{rawset} are also examples of functions that their
behavior depend on the C implementation.

There are also some functions that have a dynamic behavior and
their return type depend on their input type.
For instance, the functions \texttt{assert} and \texttt{select}
can return all the values that they received as their arguments.

We could give a more precise type to \texttt{ipairs} if our
type system had parametric polymorphism, as this function
returns an iterator function that operates over a list of
elements.

Furthermore, the \texttt{base} module also includes two very
special cases: the functions \texttt{tonumber} and \texttt{collectgarbage}.

The function \texttt{tonumber} is a special case because the
type of one input parameter depends on the type of another.
For instance, there are two static types that we can
assign to the function \texttt{tonumber}:
\texttt{(value) -> (number?)} and
\texttt{(string, number) -> (number?)}.
More precisely, the first argument of \texttt{tonumber} can be
a value of any type if it is the only argument, but it must
be a value of type string if there is a second argument,
which must be a value of type number.

The function \texttt{collectgarbage} is a special case because
its return type depends on the value of the first parameter,
which should be a string.
This function can return one number, one boolean, or two numbers
depending on the literal string that was given as its first argument.

We could not type the \texttt{coroutine} module because our
type system does not include the type \emph{thread}.
Lua has one-shot delimited continuations \citep{james2011yield}
in the form of \emph{coroutines} \citep{moura2009rc}, and
effect systems \citep{nielson1999type} are an approach that we
could use to describe control transfers with continuations.
However, for now, coroutines are out of the scope of our type
system, and we use an empty \emph{userdata} declaration
to represent the type \emph{thread}.

We could type most of the members of the \texttt{package} module,
but we could not type the following tables: \texttt{package.loaded},
\texttt{package.preload}, and \texttt{package.loaders}.
The first stores loaded modules, while the others store module \emph{loaders}.
They are difficult to type because their types are dynamic and
depend on the modules a program loads.

We could type all the members of the \texttt{string} module.
Even though we can use the \texttt{string} functions in object-oriented style,
we did not use an \emph{userdata} declaration to handle this feature,
as it would complicate the subtyping rules because the type \texttt{string}
would become a table type instead of a base type.
It was simpler and more starightforward to handle this feature as a
special case in the implementation.
For this reason, we did not include the typing of the \texttt{string} methods
in our evaluation results.

The \texttt{table} module is specially difficult to type because
all of its members require parametric polymorphism.
All the functions of this module either receive or return a list
of elements, and parametric polymorphism would help us to describe
them with a generic type.

However, the lack of parametric polymorphism did not prevent us from
giving a precise type to \texttt{table.concat}, as it operates over lists
where all elements are strings or numbers, and we can use the table type
\texttt{\{string|number\}} to express this type on our type system.

Even if our type system had parametric polymorphism, it would
be difficult to type \texttt{table.insert}, as its type depends on
the calling arity.
We can call \texttt{table.insert} passing two or three parameters.
The first parameter is always a list.
If we call it with two arguments, then the second parameter
is the value to be inserted at the end of the list.
If we call it with three arguments, then the third parameter
is the value to be inserted in the list, and the second
parameter is the position where it should be inserted.
This function also does not follow the semantics of Lua on
discarding extra arguments, and generates a run-time error whenever
we pass more than three arguments, even if the first three arguments
match its signature.

Even though the \texttt{math} module is straightforward to type,
it includes a special case that is hard to type: the function \texttt{math.random}.
This function is difficult to type because its type also depends
on the calling arity.
We can call \texttt{math.random} passing zero, one, or two parameters.
If we pass no parameter, then it returns a random floating point
number between \texttt{0} and \texttt{1}.
If we pass one or two parameters, then all the parameters should be
integers and it returns an integer number.
Another problem with this function is that it also does not follow
the semantics of Lua on discarding extra arguments, and generates
a run-time error whenever we pass more than three arguments.
There is also a problem on its documentation, as it suggests that
the two integers are optional parameters, but, for \texttt{math.random},
optional parameters behave in a different way.
Usually, Lua functions replace optional parameters with a default value
when they are \texttt{nil}, but \texttt{math.random} generates a run-time
error instead.

The \texttt{bit32} module was quite straightforward to type,
as its members are only bitwise operations over numbers.

The \texttt{io} module provides operations for manipulating files,
and these operations can use implicit or explicit file descriptors.
The implicit operations are functions in the \texttt{io} table,
while the explicit operations are methods of a file descriptor.
We used an \emph{userdata} declaration to introduce the type
\texttt{file} for representing the type of a file descriptor
and its methods.
For this reason, we included the typing of the \texttt{file}
methods in our evaluation results.

We could type most of members of the \texttt{io} module,
but we could not precisely type the functions \texttt{io.close},
\texttt{io.read}, and \texttt{io.lines}.
We also could not type the methods \texttt{file:close},
\texttt{file:read}, and \texttt{file:lines}.

The function \texttt{io.close} is difficult to type
because its return type depends on whether the file handle that is
being closed was created with \texttt{popen} or not.
If it was created with \texttt{popen}, then \texttt{io.close}
has type \texttt{(file?) -> (boolean?, string, number)};
otwerwise \texttt{io.close} has type \texttt{(file?) -> (boolean)?}.
The method \texttt{file:close} is difficult to type for the same reason.

The functions \texttt{io.read} and \texttt{io.lines} are difficult
to type because their return types depend on their input types.
The type of \texttt{io.read} can be either \texttt{() -> (string*)}
or \texttt{(string|number*) -> (string|number*)}, and
the type of \texttt{io.lines} can be either \texttt{(string?) -> (string*)}
or \texttt{(string, string|number*) -> (string|number*)}.
Returning \texttt{(string|number*)} is also painful to programmers, as
they need to constrain the value of the return type among three
different types before cocatenating it, for example.
The methods \texttt{file:read} and \texttt{file:lines} are difficult
to type for the same reason.

The function \texttt{os.execute} is the only member of the \texttt{os}
module that we could not give a precise type.
This function is difficult to type because its return type depend on
the input type.
Its type can be either \texttt{() -> (boolean)} or
\texttt{(string) -> (boolean?, string, number)}.

The evaluation results show that our type system should include effect
types, parametric polymorphism, and intersection types, as these
features would help to increase the static typing of the Lua Standard Libraries.
Effect types would allow us to type coroutines.
Parametric polymorphism would allow us to define generic function and table types.
Intersection types would allow us to define function types that have
different return types according to their input types.

\section{MD5}

We chose the MD5 library \citep{lmd5} as a case study because it is
simple and contains just one module.
We needed a simple case study to introduce Typed Lua's description
files and \texttt{userdata} declarations.
These Typed Lua's mechanims allow programmers to give statically typed
interfaces to Lua libraries that are written in C.
Table \ref{tab:evalmd5} sumarizes the evaluation results for MD5.

\begin{table}[!ht]
\begin{center}
\begin{tabular}{|l|c|c|c|c|c|c|}
\hline
\textbf{Case study} & \textbf{Module} & \textbf{easy} & \textbf{poly} & \textbf{hard} & \textbf{Total} & \textbf{\%} \\
\hline
\multirow{1}{*}{MD5}
& md5 & 13 & 0 & 0 & 13 & 100\% \\
\hline
\end{tabular}
\end{center}
\caption{Evaluation results for MD5}
\label{tab:evalmd5}
\end{table}

Even tough it was quite straightforward to type the MD5 library,
we found a little difference between its documentation and its static typing.
The documentation suggested that the typing of the function \texttt{update}
should be \texttt{(md5\string_context, string) -> (md5\string_context)},
while a comment in the source code suggested that it should be
\texttt{(md5\string_context, string, string*) -> (md5\string_context)}.
However, while testing it and reading its source code, we noticed that
its actual type is \texttt{(md5\string_context, string*) -> (md5\string_context)},
that is, we can pass zero or more strings to \texttt{update}.

This case study shows that it might be not so obvious to identify which
functions discard extra arguments when we are typing an external library.
In this case study, reading its source code and test script was essential
to confirm the typing of \texttt{update}.

\section{LuaSocket}

LuaSocket \citep{luasocket} is a library that adds network support to Lua,
and it is split into two parts: a core that is written in C and a set of
Lua modules.
The C core provides TCP and UDP support, while the Lua modules provide
support for SMTP, HTTP, and FTP client protocols, MIME encoding,
URL manipulation, and LTN12 filters \citep{nehab2008ltn012}.
We chose LuaSocket as a case study because it is the most popular Lua library.
We used Typed Lua's description files to type both parts.
This is the only case study that we test description files to statically type
a Lua module instead of rewriting it.
Table \ref{tab:evalsocket} sumarizes the evaluation results for LuaSocket.

\begin{table}[!ht]
\begin{center}
\begin{tabular}{|l|c|c|c|c|c|c|}
\hline
\textbf{Case study} & \textbf{Module} & \textbf{easy} & \textbf{poly} & \textbf{hard} & \textbf{Total} & \textbf{\%} \\
\hline
\multirow{7}{*}{LuaSocket}
& socket & 50 & 0 & 10 & 60 & 83\% \\
\cline{2-7}
& ftp & 5 & 0 & 1 & 6 & 83\% \\
\cline{2-7}
& http & 4 & 0 & 1 & 5 & 80\% \\
\cline{2-7}
& smtp & 6 & 0 & 1 & 7 & 86\% \\
\cline{2-7}
& mime & 17 & 0 & 0 & 17 & 100\% \\
\cline{2-7}
& ltn12 & 19 & 1 & 0 & 20 & 95\% \\
\cline{2-7}
& url & 8 & 0 & 0 & 8 & 100\% \\
\hline
\end{tabular}
\end{center}
\caption{Evaluation results for LuaSocket}
\label{tab:evalsocket}
\end{table}

We could type most of the members in the \texttt{socket} module,
which is the C core.
However, this module includes some functions such as
\texttt{socket.try}, \texttt{socket.protect}, and \texttt{socket.skip}
that have a dynamic behavior.
This means that the return type of these functions depend on the
type of the values that we use to call them.

We could type most of the members of the modules \texttt{ftp},
\texttt{http}, and \texttt{smtp}, but we could not precisely type
the functions \texttt{ftp.get}, \texttt{http.request}, and
\texttt{smtp.message}.

The function \texttt{ftp.get} downloads data from a given URL,
and it is difficult to type because its return type depend on
the type of its parameters.
Its type is either \texttt{(string) -> (string)?} or
\texttt{(url\string_argument|url\string_path) -> (number)?}.
The types \texttt{url\string_argument} and \texttt{url\string_path}
are table types that describe the URL that should be used to
download data.
Both table types include the same fields, except one:
while the first includes the mandatory field \texttt{argument},
of type \texttt{string}, the latter includes the mandatory field
\texttt{path}, also of type \texttt{string}.

The function \texttt{http.request} downloads data from a given URL,
and it is difficult to type because its return type depend on
the type of its parameters.
Its type is either
\texttt{(string, string?) -> (string, number, \{string : string\}, number)?} or
\texttt{(url\string_request) -> (number, number, \{string : string\}, number)?}.
The type \texttt{url\string_request} is a table type containing
details about the downloading URL.

The function \texttt{smtp.message} creates a function that sends
an SMTP message body, and it is difficult to type because of the
type of the message body, which is recursive and makes harder
its initialization.

The modules \texttt{mime} and \texttt{ltn12} have a strong connection.
The \texttt{mime} module offers low-level and high-level filters
that apply and remove some text encodings.
The low-level filters are written in C, while the high-level filters
use the function \texttt{ltn12.filter.cycle} along with the low level
filters to create standard filters.
Even though we could type all the members of the \texttt{mime} module,
the function \texttt{ltn12.filter.cycle} is the only member of the
\texttt{ltn12} module that we could not give a precise type.
This function is difficult to type because the \texttt{mime} low-level
filters do not have a standard API.

The \texttt{url} module provides functions that manipulate URLs.
It is straightforward to type, although it defines a table type for
parsed URLs that is not precise enough.
This type is problematic because it is a record type where all
fields are optional, and makes any table type that does not include
the optional fields statically type check.

This case study shows that abusing of dynamic typing usually impacts
the users of a module, as they have to keep on checking each return
value before using it to prevent API misuses.

\section{HTTP Digest}

The HTTP Digest library \citep{luahttpdigest} implements client side
HTTP digest authentication for Lua.
We randonly chose this library as a case study to evaluate Typed Lua
for rewriting an existing Lua module.
Table \ref{tab:evalhttpdigest} summarizes the evaluation results for HTTP Digest.

\begin{table}[!ht]
\begin{center}
\begin{tabular}{|l|c|c|c|c|c|c|}
\hline
\textbf{Case study} & \textbf{Module} & \textbf{easy} & \textbf{poly} & \textbf{hard} & \textbf{Total} & \textbf{\%} \\
\hline
\multirow{1}{*}{HTTP Digest}
& http-digest & 0 & 0 & 1 & 1 & 0\% \\
\hline
\end{tabular}
\end{center}
\caption{Evaluation results for HTTP Digest}
\label{tab:evalhttpdigest}
\end{table}

It is difficul to type the interface of the \texttt{http-digest} module
because it is an extension to the \texttt{http} module from LuaSocket.
The \texttt{http-digest} module only exports the function
\texttt{http-digest.request}, which extends the function
\texttt{http.request} with MD5 authentication.
Like \texttt{http.request}, the return type of \texttt{http-digest.request}
also depends on the type of its parameters.
Its type is either
\texttt{(string) -> (string, number, \{string : string\})?} or
\texttt{(url\string_request) -> (number, number, \{string : string\})?}.

Even though we could not precisely type the interface that \texttt{http-digest}
exports, we could use only static types to rewrite it, and they pointed a bug
in the code.
The problem was related to the way the library was loading the MD5 library
that should be used. 
This part of the code checks the existence of three different MD5 libraries
in the system, and uses the first one that is available, or generates an
error when none is available.
The code that loads the first option was fine, but the code that loads the
second and third options were trying to access an undefined global variable.

\section{Typical}

Typical \citep{luatypical} is a library that extends the behavior of the
function \texttt{type}. 
We randonly chose this library as a case study to evaluate Typed Lua
for rewriting an existing Lua module.
Table \ref{tab:evaltypical} summarizes the evaluation results for Typical.

\begin{table}[!ht]
\begin{center}
\begin{tabular}{|l|c|c|c|c|c|c|}
\hline
\textbf{Case study} & \textbf{Module} & \textbf{easy} & \textbf{poly} & \textbf{hard} & \textbf{Total} & \textbf{\%} \\
\hline
\multirow{1}{*}{Typical}
& typical & 1 & 0 & 0 & 1 & 100\% \\
\hline
\end{tabular}
\end{center}
\caption{Evaluation results for Typical}
\label{tab:evaltypical}
\end{table}

The interface of the \texttt{typical} module is straightforward to type,
as it contains only the function \texttt{typical.type},
which has the same type of the function \texttt{type}: \texttt{(value) -> (string)}.

However, we hit some limitations of our type system to rewrite this module.
First, it uses the function \texttt{getmetatable} to get a table and
checks whether this table has the field \texttt{\string_\string_type} or not.
We could not give a precise type to \texttt{getmetatable}, so we used the dynamic
type \texttt{any} as its return type, and this generates a warning that we could not
remove.
Second, the module returns a metatable that extends \texttt{\string_\string_call}
with \texttt{typical.type}, that is, we can use the module itself as a function,
though it is a table.
Our type system still does not support metatables, so we did not extend our version
of the \texttt{typical} module to support \texttt{\string_\string_call}.
Third, the module uses \texttt{ipairs} to iterate over an array of functions,
but our type system also has limited support to \texttt{ipairs}, and generates
a warning when we try to use the indexed value inside the \texttt{for} body.
We removed this warning using the numeric \texttt{for} to perform the same loop.

\section{Modulo 11}

Modulo 11 \citep{luamod11} is a library that generates and verifies
modulo 11 numbers.
We randonly chose this library as a case study to evaluate Typed Lua
for rewriting an existing Lua module.
Table \ref{tab:evalmod11} summarizes the evaluation results for Typical.

\begin{table}[!ht]
\begin{center}
\begin{tabular}{|l|c|c|c|c|c|c|}
\hline
\textbf{Case study} & \textbf{Module} & \textbf{easy} & \textbf{poly} & \textbf{hard} & \textbf{Total} & \textbf{\%} \\
\hline
\multirow{1}{*}{Modulo 11}
& mod11 & 7 & 0 & 2 & 9 & 78\% \\
\hline
\end{tabular}
\end{center}
\caption{Evaluation results for Modulo 11}
\label{tab:evalmod11}
\end{table}

The \texttt{mod11} module was written using an object-oriented idiom that
our type system does not support, and that is the reason why we could not
type all the members of its interface.
More precisely, the original code uses \texttt{setmetatable} to hide
two attributes, which our type system cannot hide.

In addition, it returns a metatable that extends \texttt{\string_\string_call}
with the class constructor.
This allows us to use the module itself to create new instances of a
Modulo 11 number.
However, our type system does not support this feature, and we need to
make explicit calls to the constructor whenever we want to create a
new instance.

Even though we had these two issues to rewrite the \texttt{mod11} module,
we could use only static types to rewrite it, and found some interesting
points.
The code relies a lot on implicit conversions between strings and numbers,
and some parts of the code keep on changing the type of local variables.
These are two practices that may hide bugs.

\section{Related Work}

\begin{itemize}
\item \citep{tidallock}
\item \citep{bonnaire-sergeant2012typed-clojure}
\item \citep{vitousek2014deg}
\item \citep{allende2013gts}
\item \citep{tobin-hochstadt2008ts} 
\item \citep{dart}
\item \citep{typescript}
\item \citep{politz2012semantics}
\end{itemize}


\chapter{Related Work}
\label{chap:related}
In this chapter we review related work as a way to identify the novel
features of our type system, what are the features that it shares with
other type systems, and whether we can incorporate some interesting
features from other type systems.
We split this chapter into two sections: in the first section we review
other Lua projects, while in the second section we review other
projects that are not related to Lua.

\section{Other Lua projects}

Metalua \citep{metalua} is a Lua compiler that supports compile-time
metaprogramming (CTMP).
CTMP is a kind of macro system that allows the programmers to interact
with the compiler \citep{fleutot2007contrasting}.
Metalua extends Lua 5.1 syntax to include its macro system,
and allows programmers to define their own syntax.
Metalua can provide syntactical support for several object-oriented
styles, and can also provide syntax for turning simple type
annotations into run-time assertions.

MoonScript \citep{moonscript} is a programming language that supports
class-based object-oriented programming.
MoonScript compiles to idiomatic Lua code, but
it does not perform compile-time type checking.

LuaInspect \citep{luainspect} is a tool that uses MetaLua to perform
some code analysis.
For instance, it flags unknown global variables and table fields,
it checks the number of function arguments against signatures, and
it infers function return values.
However, it does not try to analyze object-oriented code and
it does not perform compile-time type checking.

Tidal Lock \citep{tidallock} is a prototype of another optional type
system for Lua, which is written in Metalua.
Tidal Lock covers a little subset of Lua.
Statements include declaration of local variables, multiple assignment,
function application, and the return statement.
This means that Tidal Lock does not include any control-flow statement.
Expressions include primitive literals, table indexing, function application,
function declaration, and the table constructor, but they do not include
binary operations.

A remarkable feature of Tidal Lock is the refinement of table types.
This feature inspired us to also include it in Typed Lua,
but in a simpler way and with different formalization.

The table type from Tidal Lock can only represent records, that is,
it cannot describe hash tables and arrays yet, though we can refine them.
Tidal Lock also includes field types to describe the type of the fields
of a table type.
The field types describe if a table field is mutable or immutable
in a table type.
Field types are the feature that allow the refinement of table types in
Tidal Lock.

Tidal Lock is also a structural type system that relies on subtyping and
local type inference.
However, it does not support union types, recursive types, and variadic types.
It also does not type any object-oriented idiom.

Sol \citep{sol} is an experimental optional type system for Lua.
Its type system is similar to ours, as it includes literal types,
union types, and function types that handle variadic functions.
However, it does not handle the refinement of tables and it
includes different types for tables.
Sol types tables as lists, maps, and objects.
Its object types handle a specific object-oriented idiom that
Sol introduces.

Lua Analyzer \citep{luaanalyzer} is an optional type system for Lua
that is specially designed to work in the Löve Studio,
an IDE for game developing using the Löve framework.
It works in Lua 5.1 only, and uses type annotations inside comments.
It is unsound by design because its dynamic type is both
top and bottom in the subtyping relation.

Lua Analyzer shares some features with Typed Lua, and also
has some interesting features that we do not have in Typed Lua.
It has similar rules for handling the \texttt{or} idiom and
discriminating union types inside conditions.
However, these rules are limited to the \texttt{nil} tag only.
It also includes different types for typing tables.
It includes regular record types that maps names to types,
array types, and map types.
Even though it does not support the refinement of tables,
it allows the definition of nominal table types that simulate classes.
This system allows it to type check custom class systems,
which are common in Lua.
Function types also support multiple return values and
variadic functions, but they do not support overloading the
return type.
Recently, it included experimental support for type aliases and generics.

Luacheck \citep{luacheck} is a tool that performs static analysis on Lua code.
It can flag access to undeclared globals and unused local variables,
but it does not perform static type checking.

Ravi \citep{ravi} is an experimental Lua dialect.
Ravi introduces optional static typing for Lua to improve run-time performance.
To do that, Ravi extends the Lua Virtual Machine to include new
operations that take into account static type information.
Currently, Ravi extends the Lua Virtual Machine to support few types:
\texttt{integer}, \texttt{number}, arrays of integers, and arrays of numbers.

\section{Other projects}

Typed Racket \citep{tobin-hochstadt2008ts} is a statically typed version
of the Racket language, which is a Scheme dialect.
The main purpose of Typed Racket is to allow programmers to combine
untyped modules, which are written in Racket, with typed modules, which are
written in Typed Racket.
It also uses local type inference to deduce the type of unannotated expressions.

The main feature of Typed Racket's type system is \emph{occurrence typing}
\citep{tobin-hochstadt2010ltu}.
It is a novel way to use type predicates in control flow statements
to refine union types.
Occurrence typing is not sound in the presence of mutation.
As these kinds of checks are common in other languages, related systems
have appeared \citep{guha2011tlc,winther2011gtp,pearce2013ccf}.

The type system of Typed Racket also includes function types, recursive
types, and structure types.
Its function types also handle multiple return values, and there is
also a way to describe function types that have optional arguments.
Its structure types are similar to our interfaces, as they describe record types.
The type system is also structural and based on subtyping.
It also includes the dynamic type \texttt{Any}, which is the top type in the system.
Typed Racket also supports polymorphic functions and data structures.

Typed Clojure \citep{bonnaire-sergeant2012typed-clojure} is an
optional type system for Clojure.
Although Clojure is a Lisp dialect that runs on the Java Virtual Machine,
Common Language Runtime, and JavaScript, Typed Clojure runs only on
the Java Virtual Machine.
Perhaps, this restriction pushed Typed Clojure to support Java classes
and some Java types such as \texttt{Long}, \texttt{Double}, and \texttt{String}.
Typed Clojure also provides optional type annotations and uses
local type inference to deduce the type of unannotated expressions.
It also assigns the type \texttt{Any} to unannotated function parameters,
which is the top type in the type system.

The type system of Typed Clojure includes polymorphic function types,
union types, intersection types, lists, vectors, maps, sets, and recursive types.
Function types can also have rest parameters, which are similar
to our variadic types, but can only appear on the input parameter
of function types.
In fact, its function types cannot return multiple results.
It also uses occurrence typing to allow control flow statements to
refine union types.
The type system is also structural and based on subtyping.

Dart \citep{dart} is a new class-based object-oriented programming
language.
It includes optional type annotations and compiles to JavaScript.
The type system of Dart is nominal and includes base types,
function types, lists, and maps.
It also supports generics, and the programmer can define
generic functions, lists, and maps.
Unlike Typed Lua, Dart is unsound by design.

Even though Dart has optional typing and static types by
default do not affect run-time semantics, it has an
execution mode that affects run-time.
The \emph{checked mode} inserts run-time assertions that
verifies whether static types match run-time tags.
The \emph{production mode} is the default execution mode
that does not include any assertions.

TypeScript \citep{typescript} is a JavaScript extension
that includes optional type annotations and class-based
object-oriented programming.
It also uses local type inference to deduce the type
of unannotated expressions.
The type system of TypeScript is structural, based
on subtyping, and supports generics.
It includes the dynamic type, primitive types, union types,
function types, array types, tuple types, recursive types, and
object types.
Unlike Typed Lua, TypeScript uses arrays to represent variadic
functions and multiple return values.

Even though TypeScript is unsound by design,
\citet{bierman2014typescript} shows how to make TypeScript sound.
They use a reduced core of TypeScript to formalize a
sound type system for TypeScript, but also to formalize
its current unsound type system.

TeJaS \citep{lerner2013tejas} is a framework for the construction of
different type systems for JavaScript.
The authors created a base type system for JavaScript with
extensible typing rules that allow the experimentation of
different static analysis.
They used TeJaS to create a type system that simulates the
type system of TypeScript.

\citet{politz2012semantics} proposes semantics and types for objects
with first-class member names, a well-known feature from scripting languages.
Their type system uses string patterns to describe the members of
an object, and define a complex subtyping relation to validate
these patterns.
They also provide an implementation of their system to JavaScript.

Gradualtalk \citep{allende2013gts} is a Smalltalk dialect that
supports gradual typing.
The type system combines nominal and structural typing.
It includes function types, union types, structural types,
nominal types, a self type, and parametric polymorphism.
The type system also relies on subtyping and consistent-subtyping.

Gradualtalk inserts run-time checks that ensure dynamically
typed code does not violate statically typed code.
\citet{allende2013cis} perform a careful evaluation about
cast insertion in Gradualtalk.
They report that usually cast insertions impact on execution
performance, so Gradualtalk also has an option that allows
programmers to turn them off, downgrading Gradualtalk
to an optional type system.

Reticulated Python \citep{vitousek2014deg} is a Python compiler
that supports gradual typing.
The type system is structural and based on subtyping.
It includes base types, the dynamic type, list types,
dictionary types, tuple types, function types, set types,
object types, class types, and recursive types.
It includes class and object types to differentiate the
type of class declarations and instances, respectively.
It also uses local type inference.
Besides static type checking, Reticulated Python also introduces
three different approaches for inserting run-time assertions.

Mypy \citep{mypy} is an optional type system for Python.
The type system of mypy is similar to the type system of
Reticulated Python, but mypy does not insert run-time checks
and it has parametric polymorphism.
In contrast, Reticulated Python can type variadic functions,
but mypy cannot.
Recently, Guido van Rossum, Python's author, proposed a
standard syntax for type annotations in Python \citep{PEP483}
that is extremely inspired by mypy \citep{PEP484}.
The main goal of this proposal is to make easier building
static analysis tools for Python.
Typing \citep{typing} is a tool that is being developed to
implement this proposal.

Hack \citep{hack} is a new programming language that runs on the
Hip Hop Virtual Machine (HHVM).
The HHVM is a virtual machine that executes Hack and PHP programs.
We can view Hack as an extension to PHP that combines static and
dynamic typing.
The type system of Hack includes generics, nullable types, collections,
and function types.

The Ruby Type Checker \citep{ren2013rtc} is a library that
performs type checking during run-time.
The library provides type annotations that the programmer
can use on classes and methods.
Its type system includes nominal types, union types,
intersection types, method types, parametric polymorphism,
and type casts.

Grace \citep{black2013sg} is an object-oriented language
with optional typing.
Grace is not a dynamically typed language that has been
extended with an optional type system, but a language
that has been designed from scratch to have both
static and dynamic typing.
\citet{homer2013modules} explores some useful patterns
that derive from Grace's use of objects as modules
and its brand of optional structural typing, which
can also be expressed with Typed Lua's modules as tables.


\chapter{Conclusions}
\label{chap:conc}
In this work we presented Typed Lua, an optional type system for Lua.
We implemented Typed Lua as a Lua extension that allows programmers to
combine static and dynamic typing in Lua code, making easier the evolution
of simple scripts into large programs.

Our main contribution is the formalization of a complete optional type
system that introduces several novel type system features to statically
type check Lua programs.
Even though Lua shares several features with other dynamically
typed languages such as JavaScript, Lua also has several unusual features.
These unusual features include tables (or associative arrays) as the sole
mechanism for structured data, besides functions with multiple return values
and flexible arity that interact with multiple assignment.
We highlight the following novel features of our type system:
\begin{itemize}
\item type refinement allows the incremental evolution of record and
object types, playing an important role in statically type checking
the idiomatic way in which Lua programmers use tables to define modules
and objects;
\item projection types handle functions that are overloaded on the
number and types of return values, allowing programmers to narrow the
types of a set of variables by narrowing the type of a single component
of this set;
\item union types and variadic types help our type system handle
functions with flexible arity, that is, union types are helpful in
describing optional parameters while variadic types are helpful in
describing the type of the vararg expression and the type of functions
that can receive or return any number of values.
\end{itemize}

A key feature in optional type systems is usability.
This means that optional type systems should not change the idioms
that programmers are already familiar with.
Instead, optional type systems should fit existing idioms to
statically type check them.
Designing a too simple type system can overload programmers by forcing
them to change the way they program in the language to fit the type system,
while designing a too complex type system can overload programmers with
types and error messages that are hard to understand, even if type inference
removes the necessity of annotating the program with these complex types.
The most challenging aspect of designing optional type systems is to find
the right amount of complexity for a type system that feels natural to the programmers.

Usability has been a concern in the design of Typed Lua since the beginning.
We realized that we should not rely on the semantics of Lua only,
as this could lead to a cumbersome type system that would not support
several Lua idioms.
For this reason, we performed a mostly automated survey of Lua idioms
and features to inform our design choices.

After designing and implementing Typed Lua, we performed several
case studies to evaluate how successful we were in our goal of
providing an usable type system.
We evaluated 29 modules from 8 different case studies,
and we could give precise static types to 83\% of the 449
members that these modules export.
In the median, we could give precise static types to 89\%
of the members from each module.
Our evaluation results showed that our type system can statically
type check several Lua idioms and features, though the evaluation
results also exposed several limitations of our type system.
We found that the three main limitations of our type system are
the lack of intersection types, parametric polymorphism, and operator overloading.
Overcoming these limitations is our major target for future work,
as it will allow us to statically type check more programs.

Unlike other optional type systems, we designed Typed Lua without
deliberate unsound parts.
However, we still do not have proofs that the novel features of
our type system are sound.
We see a soundness proof as another major future work, as it is
necessary to use static types for code optimization.

Finally, we believe that Typed Lua is a major contribution to the Lua community,
because it offers a framework that programmers can use to document,
test, and better structure their applications.
For libraries where a full conversion to static type checking should
prove unfeasible or too much work, the community can use Typed Lua
just to document the external interfaces of the libraries,
giving the benefits of static type checking to the users of these
libraries.
In fact, we already have user feedback from Lua programmers that are
using Typed Lua in their projects.
For instance, ZeroBrane Studio is an IDE for Lua development that is
starting to use Typed Lua to perform static analysis in Lua code.



\bibliography{thesis_andre}

\appendix

\chapter{Glossary}
\label{app:glossary}
\begin{description}
\item[bottom type] A type that is subtype of all types.

\item[closed table type] A table type that does not provide any guarantees
about keys with types not listed in the table type.
See complete definition in page \pageref{def:tabletype}.

\item[coercion] A relation that allows converting values from one type to
values of another type without error.

\item[consistency] A relation used by gradual typing to check the interaction
between the dynamic type and other types.
See complete definition in page \pageref{def:consistency}.

\item[consistent-subtyping] A relation that combines consistency and subtyping.
See complete definition in page \pageref{def:consistent-subtyping}.

\item[contravariant] A part of the type constructor is contravariant when
it reverses the subtyping order.

\item[covariant] A part of the type constructor is covariant when
it preserves the subtyping order.

\item[depth subtyping] An operation that allows variance in the type of
record fields.

\item[dynamic type] A type used by gradual typing to denote unknown values.
See complete definition in page \pageref{def:dynamictype}.

\item[filter type] A type used by Typed Lua to discriminate the type of
local variables inside control flow statements.
See complete definition in page \pageref{def:filtertype}.

\item[fixed table type] A table type which guarantees that there are no
keys with a type that is not one of its key types, and that can have
any number of \emph{fixed} or \emph{closed} references point to it.
See complete definition in page \pageref{def:tabletype}.

\item[flow typing] An approach that combines static typing and flow analysis to
allow variables to have different types at different parts of the program.

\item[free assigned variable] A free variable that appears in an assignment.

\item[gradual type system] A type system that uses the consistency relation
instead of type equality to perform static type checking.
See complete definition in page \pageref{sec:gradual}.

\item[gradual typing] An approach that uses a gradual type system to allow
static and dynamic typing in the same code, but inserting run-time checks
between statically typed and dynamically typed code.
See complete definition in page \pageref{sec:gradual}.

\item[invariant] A part of the type constructor is invariant when it forbids variance.
It is also a way to define type equality through subtyping.

\item[metatable] A Lua table that allows changing the behavior of other tables
it is attached to.

\item[nominal type system] A type system that uses the type names to check the
compatibility among them.

\item[open table type] A table type which guarantees that there are no
keys with a type that is not one of its key types, and that only have
\emph{closed} references point to it.
See complete definition in page \pageref{def:tabletype}.

\item[optional type system] A type system that allows combining static and
dynamic typing in the same language, but without affecting the run-time semantics.
See complete definition in page \pageref{sec:optional}.

\item[projection environment] An environment used by Typed Lua to handle unions of
second-level types that are bound to projection types.

\item[projection type] A type used by Typed Lua to discriminate the type of local
variables that have a dependency relation.
See complete definition in page \pageref{def:projectiontype}.

\item[prototype object] An object that works like a class, that is, it is an object from
which other objects inherit its attributes.

\item[self-like delegation] A technique to implement inheritance in dynamically typed
languages through prototype objects.

\item[sound type system] A type system that does not type check all programs that contain a type error.

\item[structural type system] A type system that uses type structures to check the compatibility among them.

\item[table refinement] An operation from Typed Lua that allows programmers to change a table type
to include new fields or to specialize existing fields.
See complete definition in page \pageref{sec:refinement}.

\item[top type] A type that is supertype of all types.

\item[type environment] An environment used by Typed Lua to assign variable names to first-level types.

\item[type tag] A tag that describes the type of a value during run-time in dynamically
typed languages.

\item[unique table type] A table type which guarantees that there are no
keys with a type that is not one of its key types, and that do not have
any references point to it.
See complete definition in page \pageref{def:tabletype}.

\item[unsound type system] A type system that type checks certain programs that contain type errors.

\item[userdata] A Lua data type that allows Lua to hold values from applications
or libraries that are written in C.

\item[vararg expression] A Lua expression that can result in an arbitrary number of values.

\item[variadic function] A function that can receive an arbitrary number of arguments.

\item[variance] A way to define the subtyping order between the components
of a type constructor.

\item[width subtyping] An operation that allows adding fields to a record.

\end{description}


\chapter{The syntax of Typed Lua}
\label{app:syntax}
\allowdisplaybreaks
\begin{align*}
\textit{chunk} & ::= \; \textit{block}\\
\textit{block} & ::= \; \{\textit{stat}\} \; [\textit{retstat}]\\
\textit{stat} & ::= \; \texttt{`;'}\\
& | \; \textit{varlist} \; \texttt{`='} \; \textit{explist}\\
& | \; \textit{functioncall}\\
& | \; \textit{label}\\
& | \; \textbf{break}\\ 
& | \; \textbf{goto} \; \textit{Name}\\
& | \; \textbf{do} \; \textit{block} \; \textbf{end}\\
& | \; \textbf{while} \; \textit{exp} \; \textbf{do} \; \textit{block} \; \textbf{end}\\
& | \; \textbf{repeat} \; \textit{block} \; \textbf{until} \; \textit{exp}\\
& | \; \textbf{if} \; \textit{exp} \; \textbf{then} \; \textit{block} \;
  \{\textbf{elseif} \; \textit{exp} \; \textbf{then} \; \textit{block}\} \;
  [\textbf{else} \; \textit{block}] \; \textbf{end}\\ 
& | \; \textbf{for} \; \textit{Name} \; \texttt{`='} \; \textit{exp} \;
  \texttt{`,'} \; \textit{exp} \; [\texttt{`,'} \; \textit{exp}] \;
  \textbf{do} \; \textit{block} \; \textbf{end}\\
& | \; \textbf{for} \; \textit{namelist} \; \textbf{in} \; \textit{explist} \;
  \textbf{do} \; \textit{block} \; \textbf{end}\\
& | \; \textcolor{blue}{[\textbf{const}]} \; \textbf{function} \; \textit{funcname} \; \textit{funcbody}\\
& | \; \textbf{local} \; \textbf{function} \; \textit{Name} \; \textit{funcbody}\\
& | \; \textbf{local} \; \textit{namelist} \; [\texttt{`='} \; \textit{explist}]\\
& | \; \textcolor{blue}{[\textbf{local}] \; \textbf{interface} \; \textit{Name} \; \textit{interfacedec} \;
  \{\textit{interfacedec}\} \; \textbf{end}}\\
\textit{retstat} & ::= \; \textbf{return} \; [\textit{explist}] \; [\texttt{`;'}]\\
\textit{label} & ::= \; \texttt{`::'} \; \textit{Name} \; \texttt{`::'}\\
\textit{funcname} & ::= \; \textit{Name} \; \{\texttt{`.'} \; \textit{Name}\} \; [\texttt{`:'} \; \textit{Name}]\\
\textit{varlist} & ::= \; \textcolor{blue}{[\textbf{const}]} \; \textit{var} \;
  \{\texttt{`,'} \; \textcolor{blue}{[\textbf{const}]} \; \textit{var}\}\\
\textit{var} & ::= \; \textit{Name} \; | \;
  \textit{prefixexp} \; \texttt{`['} \; \textit{exp} \; \texttt{`]'} \; | \;
  \textit{prefixexp} \; \texttt{`.'} \; \textit{Name}\\
\textit{namelist} & ::= \; \textit{Name} \; \textcolor{blue}{[\texttt{`:'} \; \textit{type}]} \;
  \{\texttt{`,'} \; \textit{Name} \; \textcolor{blue}{[\texttt{`:'} \; \textit{type}]}\}\\
\textit{explist} & ::= \; \textit{exp} \; \{\texttt{`,'} \; \textit{exp}\}\\
\textit{exp} & ::= \; \textbf{nil} \; | \;
  \textbf{false} \; | \;
  \textbf{true} \; | \;
  \textit{Number} \; | \;
  \textit{String} \; | \;
  \texttt{`...'} \; | \;
  \textit{functiondef}\\
& | \; \textit{prefixexp} \; | \;
  \textit{tableconstructor} \; | \;
  \textit{exp} \; \textit{binop} \; \textit{exp} \; | \;
  \textit{unop} \; \textit{exp}\\
\textit{prefixexp} & ::= \; \textit{var} \; | \;
  \textit{functioncall} \; | \;
  \texttt{`('} \; \textit{exp} \; \texttt{`)'}\\
\textit{functioncall} & ::= \; \textit{prefixexp} \; \textit{args} \; | \;
  \textit{prefixexp} \; \texttt{`:'} \; \textit{Name} \; \textit{args}\\
\textit{args} & ::= \; \texttt{`('} \; [\textit{explist}] \; \texttt{`)'} \; | \;
  \textit{tableconstructor} \; | \;
  \textit{String}\\
\textit{functiondef} & ::= \; \textbf{function} \; \textit{funcbody}\\
\textit{funcbody} & ::= \; \texttt{`('} \; [\textit{parlist}] \; \texttt{`)'} \;
  \textcolor{blue}{[\texttt{`:'} \; \textit{rettype}]} \; \textit{block} \; \textbf{end}\\
\textit{parlist} & ::= \; \textit{namelist} \; [\texttt{`,'} \; \texttt{`...'} \;
  \textcolor{blue}{[\texttt{`:'} \; \textit{type}]}] \; | \;
  \texttt{`...'} \; \textcolor{blue}{[\texttt{`:'} \; \textit{type}]}\\
\textit{tableconstructor} & ::= \; \texttt{`\{'} \; [\textit{fieldlist}] \; \texttt{`\}'}\\
\textit{fieldlist} & ::= \; \textcolor{blue}{[\textbf{const}]} \; \textit{field} \;
  \{\textit{fieldsep} \; \textcolor{blue}{[\textbf{const}]} \; \textit{field}\} \; [\textit{fieldsep}]\\
\textit{field} & ::= \; \texttt{`['} \; \textit{exp} \; \texttt{`]'} \; \texttt{`='} \; \textit{exp} \; | \;
  \textit{Name} \; \texttt{`='} \; \textit{exp} \; | \;
  \textit{exp}\\
\textit{fieldsep} & ::= \; \texttt{`,'} \; | \; \texttt{`;'}\\
\textit{binop} & ::= \; \texttt{`+'} \; | \; \texttt{`-'} \; | \; \texttt{`*'} \; | \; \texttt{`/'} \; | \;
  \texttt{`\textasciicircum'} \; | \; \texttt{`\%'} \; | \; \texttt{`..'}\\
& | \; \texttt{`<'} \; | \; \texttt{`<='} \; | \; \texttt{`>'} \; | \; \texttt{`>='} \; | \;
  \texttt{`=='} \; | \; \texttt{`\textasciitilde='}\\
& | \; \textbf{and} \; | \; \textbf{or}\\
\textit{unop} & ::= \; \texttt{`-'} \; | \; \textbf{not} \; | \; \texttt{`\#'}\\
\textcolor{blue}{\textit{interfacedec}} & ::= \; [\textbf{const}] \; \textit{Name} \;
  \{\texttt{`,'} \; [\textbf{const}] \; \textit{Name}\} \; \texttt{`:'} \; \textit{dectype}\\
\textcolor{blue}{\textit{dectype}} & ::= \; \textit{type} \; | \; \textit{methodtype}\\
\textcolor{blue}{\textit{type}} & ::= \; \textit{primarytype} \; [\texttt{`?'}]\\
\textcolor{blue}{\textit{primarytype}} & ::= \; \textit{literaltype} \; | \;
  \textit{basetype} \; | \;
  \textbf{nil} \; | \;
  \textbf{value} \; | \;
  \textbf{any} \; | \;
  \textbf{self} \; | \;
  \textit{Name}\\
& | \; \textit{functiontype} \; | \;
  \textit{tabletype} \; | \;
  \textit{primarytype} \; \texttt{`|'} \; \textit{primarytype}\\
\textcolor{blue}{\textit{literaltype}} & ::= \; \textbf{false} \; | \;
  \textbf{true} \; | \;
  \textit{Number} \; | \;
  \textit{String}\\
\textcolor{blue}{\textit{basetype}} & ::= \; \textbf{boolean} \; | \;
  \textbf{number} \; | \;
  \textbf{string}\\
\textcolor{blue}{\textit{functiontype}} & ::= \; \textit{tupletype} \; \texttt{`->'} \; \textit{rettypelist}\\
\textcolor{blue}{\textit{methodtype}} & ::= \; \textit{tupletype} \; \texttt{`=>'} \; \textit{rettypelist}\\
\textcolor{blue}{\textit{tabletype}} & ::= \; \texttt{`\{'} \; [\textit{tabletypebody}] \; \texttt{`\}'}\\
\textcolor{blue}{\textit{tupletype}} & ::= \; \texttt{`('} \; [typelist] \; \texttt{`)'}\\
\textcolor{blue}{\textit{typelist}} & ::= \; \textit{type} \; \{\texttt{`,'} \; \textit{type}\} \; [\texttt{`*'}]\\
\textcolor{blue}{\textit{rettypelist}} & ::= \; \textit{unionlist} \; [\texttt{`?'}]\\
\textcolor{blue}{\textit{unionlist}} & ::= \; \textit{tupletype} \; | \;
  \textit{unionlist} \; \texttt{`|'} \; \textit{unionlist}\\
\textcolor{blue}{\textit{tabletypebody}} & ::= \; \textit{type} \; | \; \textit{fieldtypelist}\\
\textcolor{blue}{\textit{fieldtypelist}} & ::= \; [\textbf{const}] \; \textit{fieldtype} \; \{\texttt{`,'} \; [\textbf{const}] \; \textit{fieldtype}\}\\ 
\textcolor{blue}{\textit{fieldtype}} & ::= \; \textit{keytype} \; \texttt{`:'} \; \textit{type}\\
\textcolor{blue}{\textit{keytype}} & ::= \; \textit{literaltype} \; | \;
  \textit{basetype} \; | \;
  \textbf{value} \; | \;
  \textbf{any}\\
\textcolor{blue}{\textit{rettype}} & ::= \; \textit{type} \; | \; \textit{rettypelist}\\
\end{align*}


\chapter{The type system of Typed Lua}
\label{app:rules}
This appendix presents the complete type system of Typed Lua.

\section{Subtyping rules}

\noindent

\mylabel{S-LITERAL}
\[
\senv \vdash l \subtype l
\]

\mylabel{S-FALSE}
\[
\senv \vdash \False \subtype \Boolean
\]

\mylabel{S-TRUE}
\[
\senv \vdash \True \subtype \Boolean
\]

\mylabel{S-INT1}
\[
\senv \vdash {\it int} \subtype \Integer
\]

\mylabel{S-INT2}
\[
\senv \vdash {\it int} \subtype \Number
\]

\mylabel{S-FLOAT}
\[
\senv \vdash {\it float} \subtype \Number
\]

\mylabel{S-STRING}
\[
\senv \vdash {\it string} \subtype \String
\]

\mylabel{S-BASE}
\[
\senv \vdash b \subtype b
\]

\mylabel{S-NUMBER}
\[
\senv \vdash \Integer \subtype \Number
\]

\mylabel{S-NIL}
\[
\senv \vdash \Nil \subtype \Nil
\]

\mylabel{S-VALUE}
\[
\senv \vdash t \subtype \Value
\]

\mylabel{S-ANY}
\[
\senv \vdash \Any \subtype \Any
\]

\mylabel{S-SELF}
\[
\senv \vdash \Self \subtype \Self
\]

\mylabel{S-UNION1}
\[
\dfrac{\senv \vdash t_{1} \subtype t \;\;\;
       \senv \vdash t_{2} \subtype t}
      {\senv \vdash t_{1} \cup t_{2} \subtype t}
\]

\mylabel{S-UNION2}
\[
\dfrac{\senv \vdash t \subtype t_{1}}
      {\senv \vdash t \subtype t_{1} \cup t_{2}}
\]

\mylabel{S-UNION3}
\[
\dfrac{\senv \vdash t \subtype t_{2}}
      {\senv \vdash t \subtype t_{1} \cup t_{2}}
\]

\mylabel{S-FUNCTION}
\[
\dfrac{\senv \vdash p_{2} \subtype p_{1} \;\;\;
       \senv \vdash r_{1} \subtype r_{2}}
      {\senv \vdash p_{1} \rightarrow r_{1} \subtype p_{2} \rightarrow r_{2}}
\]

\mylabel{S-TABLE1}
\[
\dfrac{\forall i \in 1..n \; \exists j \in 1..m \;\;\;
       \senv \vdash k_{j} \subtype k_{i}' \;\;\;
       \senv \vdash k_{i}' \subtype k_{j} \;\;\;
       \senv \vdash v_{j} \subtype_{c} v_{i}'}
      {\senv \vdash \{k_{1}{:}v_{1}, ..., k_{m}{:}v_{m}\}_{closed} \subtype \{k_{1}'{:}v_{1}', ..., k_{n}'{:}v_{n}'\}_{closed}}
\]

\mylabel{S-TABLE2}
\[
\dfrac{\forall i \in 1..m \; \senv \vdash k_{i} \subtype k_{i}' \;\;\;
       \senv \vdash v_{i} \subtype_{c} v_{i}' \;\;\;
       \forall j \in m+1..n \; \senv \vdash \Nil \subtype_{o} v_{j}'}
      {\senv \vdash \{k_{1}{:}v_{1}, ..., k_{m}{:}v_{m}\}_{open} \subtype
                    \{k_{1}'{:}v_{1}', ..., k_{m}'{:}v_{m}', ..., k_{n}'{:}v_{n}'\}_{closed|open}}
\]

\mylabel{S-TABLE3}
\[
\dfrac{\forall i \in 1..m \; \senv \vdash k_{i} \subtype k_{i}' \;\;\;
       \senv \vdash v_{i} \subtype_{u} v_{i}' \;\;\;
       \forall j \in m+1..n \; \senv \vdash \Nil \subtype_{o} v_{j}'}
      {\senv \vdash \{k_{1}{:}v_{1}, ..., k_{m}{:}v_{m}\}_{unique} \subtype
                    \{k_{1}'{:}v_{1}', ..., k_{m}'{:}v_{m}', ..., k_{n}'{:}v_{n}'\}_{closed|open|unique}}
\]

\mylabel{S-FIELD1}
\[
\dfrac{\senv \vdash v_{1} \subtype v_{2} \;\;\;
       \senv \vdash v_{2} \subtype v_{1}}
      {\senv \vdash v_{1} \subtype_{c} v_{2}}
\]

\mylabel{S-FIELD2}
\[
\dfrac{\senv \vdash v_{1} \subtype v_{2}}
      {\senv \vdash \Const \; v_{1} \subtype_{c} \Const \; v_{2}}
\]

\mylabel{S-FIELD3}
\[
\dfrac{\senv \vdash v_{1} \subtype v_{2}}
      {\senv \vdash v_{1} \subtype_{c} \Const \; v_{2}}
\]

\mylabel{S-FIELD4}
\[
\dfrac{\senv \vdash \Nil \subtype v}
      {\senv \vdash \Nil \subtype_{o} v}
\]

\mylabel{S-FIELD5}
\[
\dfrac{\senv \vdash \Nil \subtype v}
      {\senv \vdash \Nil \subtype_{o} \Const \; v}
\]

\mylabel{S-FIELD6}
\[
\dfrac{\senv \vdash v_{1} \subtype v_{2}}
      {\senv \vdash v_{1} \subtype_{u} v_{2}}
\]

\mylabel{S-FIELD7}
\[
\dfrac{\senv \vdash v_{1} \subtype v_{2}}
      {\senv \vdash \Const \; v_{1} \subtype_{u} \Const \; v_{2}}
\]

\mylabel{S-ASSUMPTION}
\[
\dfrac{x_{1} \subtype x_{2} \in \senv}
      {\senv \vdash x_{1} \subtype x_{2}}
\]

\mylabel{S-AMBER}
\[
\dfrac{\senv[x_{1} \subtype x_{2}] \vdash t_{1} \subtype t_{2}}
      {\senv \vdash \mu x_{1}.t_{1} \subtype \mu x_{2}.t_{2}}
\]

\mylabel{S-AMBERL}
\[
\dfrac{\senv \vdash [x \mapsto \mu x.t_{1}]t_{1} \subtype t_{2}}
      {\senv \vdash \mu x.t_{1} \subtype t_{2}}
\]

\mylabel{S-AMBERR}
\[
\dfrac{\senv \vdash t_{1} \subtype [x \mapsto \mu x.t_{2}]t_{2}}
      {\senv \vdash t_{1} \subtype \mu x.t_{2}}
\]

\mylabel{S-VOID}
\[
\senv \vdash \Void \subtype \Void
\]

\mylabel{S-VARARG}
\[
\dfrac{\senv \vdash t_{1} \cup \Nil \subtype t_{2} \cup \Nil}
      {\senv \vdash t_{1}* \subtype t_{2}*}
\]

\mylabel{S-TUPLE1}
\[
\dfrac{\senv \vdash t_{1} \subtype t_{2} \;\;\;
       \senv \vdash s_{1} \subtype s_{2}}
      {\senv \vdash t_{1} \times s_{1} \subtype t_{2} \times s_{2}}
\]

\mylabel{S-TUPLE2}
\[
\dfrac{\senv \vdash t_{1} \cup \Nil \subtype t_{2} \;\;\;
       \senv \vdash t_{1}* \subtype s_{2}}
      {\senv \vdash t_{1}* \subtype t_{2} \times s_{2}}
\]

\mylabel{S-TUPLE3}
\[
\dfrac{\senv \vdash t_{1} \subtype t_{2} \cup \Nil \;\;\;
       \senv \vdash s_{1} \subtype t_{2}*}
      {\senv \vdash t_{1} \times s_{1} \subtype t_{2}*}
\]

\mylabel{S-UNION4}
\[
\dfrac{\senv \vdash s_{1} \subtype s \;\;\;
       \senv \vdash s_{2} \subtype s}
      {\senv \vdash s_{1} \sqcup s_{2} \subtype s}
\]

\mylabel{S-UNION5}
\[
\dfrac{\senv \vdash s \subtype s_{1}}
      {\senv \vdash s \subtype s_{1} \sqcup s_{2}}
\]

\mylabel{S-UNION6}
\[
\dfrac{\senv \vdash s \subtype s_{2}}
      {\senv \vdash s \subtype s_{1} \sqcup s_{2}}
\]

\section{Consistent-subtyping rules}

\noindent

\mylabel{C-ANY1}
\[
\senv \vdash t \lesssim \Any
\]

\mylabel{C-ANY2}
\[
\senv \vdash \Any \lesssim t
\]

\section{Typing rules}

\noindent

\mylabel{T-SKIP}
\[
\env_{1} \vdash \mathbf{skip}:\env_{1}
\]

\mylabel{T-SEQ}
\[
\dfrac{\env_{1} \vdash s_{1}:\env_{2} \;\;\;
       \env_{2} \vdash s_{2}:\env_{3}}
      {\env_{1} \vdash s_{1} \; ; \; s_{2}:\env_{3}}
\]

\mylabel{T-ASSIGNMENT}
\[
\dfrac{\env_{1} \vdash el:r_{1}, \env_{2} \;\;\;
       \env_{2} \vdash \vec{l}:r_{2}, \env_{3} \;\;\;
       r_{1} \lesssim r_{2}}
      {\env_{1} \vdash \vec{l} = el:\env_{3}}
\]

\mylabel{T-WHILE}
\[
\dfrac{\env_{1} \vdash e:t, \env_{2} \;\;\;
       closeall(\env_{2}) \vdash s:\env_{3}}
      {\env_{1} \vdash \mathbf{while} \; e \; \mathbf{do} \; s:closeset(\env_{2}, fav(s))}
\]

\mylabel{T-IF1}
\[
\dfrac{\env_{1} \vdash e:t, \env_{2} \;\;\;
       closeall(\env_{2}) \vdash s_{1}:\env_{3} \;\;\;
       closeall(\env_{2}) \vdash s_{2}:\env_{4}}
      {\env_{1} \vdash \mathbf{if} \; e \; \mathbf{then} \; s_{1} \; \mathbf{else} \; s_{2}:closeset(\env_{2}, fav(s_{1}) \cup fav(s_{2}))}
\]

\mylabel{T-IF2}
\[
\dfrac{\begin{array}{c}
       \env_{1} \vdash type(n) == ``string":\Boolean, \env_{2} \\
       closeall(\env_{2}[n \mapsto \String]) \vdash s_{1}:\env_{3} \\
       closeall(\env_{2}[n \mapsto filter(\env_{2}(n), \String)) \vdash s_{2}:\env_{4}
      \end{array}}
      {\env_{1} \vdash \mathbf{if} \; type(n) == ``string" \; \mathbf{then} \; s_{1} \; \mathbf{else} \; s_{2}:closeset(\env_{2}, fav(s_{1}) \cup fav(s_{2}))}
\]

\mylabel{T-LOCAL}
\[
\dfrac{\env_{1} \vdash el:r_{1}, \env_{2} \;\;\;
       r_{1} \lesssim \vec{t} \;\;\;
       \env_{2}[\vec{n} \mapsto \vec{t}] \vdash s:\env_{3}}
      {\env_{1} \vdash \mathbf{local} \; \vec{n{:}t} = el \; \mathbf{in} \; s:\env_{3} - \{\vec{n} \mapsto \vec{t}\}}
\]

\mylabel{T-LOCALREC}
\[
\dfrac{\env_{1}[n \mapsto t] \vdash f:t_{1}, \env_{2} \;\;\;
       t_{1} \lesssim t \;\;\;
       \env_{2} \vdash s:\env_{3}}
      {\env_{1} \vdash \mathbf{rec} \; n{:}t = f \; \mathbf{in} \; s:\env_{3} - \{n \mapsto t\}}
\]

\mylabel{T-RETURN}
\[
\dfrac{\env_{1} \vdash el:r_{1}, \env_{2} \;\;\;
       \env_{2}(\ret) = r_{2} \;\;\;
       r_{1} \lesssim r_{2}}
      {\env_{1} \vdash \mathbf{return} \; el:\env_{2}}
\]

\mylabel{T-STMCALL1}
\[
\dfrac{\env_{1} \vdash e:p_{1} \rightarrow r_{1}, \env_{2} \;\;\;
       \env_{2} \vdash el:p_{2}, \env_{3} \;\;\;
       p_{2} \lesssim p_{1}}
      {\env_{1} \vdash e(el)_{s}:\env_{3}}
\]

\mylabel{T-STMCALL2}
\[
\dfrac{\env_{1} \vdash e:\Any, \env_{2} \;\;\;
       \env_{2} \vdash el:p, \env_{3}}
      {\env_{1} \vdash e(el)_{s}:\env_{3}}
\]

\mylabel{T-STMINVOKE1}
\[
\dfrac{\env_{1} \vdash e[n]:p_{1} \rightarrow r_{1}, \env_{2} \;\;\;
       \env_{2} \vdash el:p_{2}, \env_{3} \;\;\;
       p_{2} \lesssim p_{1}}
      {\env_{1} \vdash e{:}n(el)_{s}:\env_{3}}
\]

\mylabel{T-STMINVOKE2}
\[
\dfrac{\env_{1} \vdash e[n]:\Any, \env_{2} \;\;\;
       \env_{2} \vdash el:p, \env_{3}}
      {\env_{1} \vdash e{:}n(el)_{s}:\env_{3}}
\]

\mylabel{T-NIL}
\[
\env_{1} \vdash \mathbf{nil}:\Nil, \env_{1}
\]

\mylabel{T-FALSE}
\[
\env_{1} \vdash \mathbf{false}:\False, \env_{1}
\]

\mylabel{T-TRUE}
\[
\env_{1} \vdash \mathbf{true}:\True, \env_{1}
\]

\mylabel{T-INT}
\[
\env_{1} \vdash {\it int}:{\it int}, \env_{1}
\]

\mylabel{T-FLOAT}
\[
\env_{1} \vdash {\it float}:{\it float}, \env_{1}
\]

\mylabel{T-STR}
\[
\env_{1} \vdash {\it string}:{\it string}, \env_{1}
\]

\mylabel{T-EXPDOTS}
\[
\dfrac{\env_{1}({...}) = t}
      {\env_{1} \vdash {...}_{e}:t, \env_{1}}
\]

\mylabel{T-FUNCTION1}
\[
\dfrac{closeall(\env_{1}[\ret \mapsto r]) \vdash s:\env_{2}}
      {\env_{1} \vdash \mathbf{fun} \; (){:}r \; s:\Void \rightarrow r, closeset(\env_{1}, fav(\mathbf{fun} \; (){:}r \; s))}
\]

\mylabel{T-FUNCTION2}
\[
\dfrac{closeall(\env_{1}[{...} \mapsto t, \ret \mapsto r]) \vdash s:\env_{2}}
      {\env_{1} \vdash \mathbf{fun} \; ({...}{:}t){:}r \; s:t{*} \rightarrow r, closeset(\env_{1}, fav(\mathbf{fun} \; ({...}{:}t){:}r \; s))}
\]

\mylabel{T-FUNCTION3}
\[
\dfrac{closeall(\env_{1}[\vec{n} \mapsto \vec{t}, \ret \mapsto r]) \vdash s:\env_{2}}
      {\env_{1} \vdash \mathbf{fun} \; (\vec{n{:}t}){:}r \; s:\vec{t} \rightarrow r, closeset(\env_{1}, fav(\mathbf{fun} \; (\vec{n{:}t}){:}r \; s))}
\]

\mylabel{T-FUNCTION4}
\[
\dfrac{closeall(\env_{1}[\vec{n} \mapsto \vec{t}, {...} \mapsto t, \ret \mapsto r]) \vdash s:\env_{2}}
      {\env_{1} \vdash \mathbf{fun} \; (\vec{n{:}t},{...}{:}t){:}r \; s: \vec{t} \times t{*} \rightarrow r, closeset(\env_{1}, fav(\mathbf{fun} \; (\vec{n{:}t},{...}{:}t){:}r \; s))}
\]

\mylabel{T-CONSTRUCTOR1}
\[
\env \vdash \{ \mathbf{nothing} \}:\{ \}_{u}, \env
\]

\mylabel{T-CONSTRUCTOR2}
\[
\dfrac{\env_{1} \vdash cl:k_{i}{:}v_{i}, ..., k_{n}{:}v_{n}, \env_{2}}
      {\env_{1} \vdash \{ \; cl \; \}:\{ k_{i}{:}v_{i}, ..., k_{n}{:}v_{n} \}_{u}, \env_{2}}
\]

\mylabel{T-ARITH1}
\[
\dfrac{\env_{1} \vdash e_{1}:t_{1}, \env_{2} \;\;\;
       \env_{2} \vdash e_{2}:t_{2}, \env_{3} \;\;\;
       t_{1} \subtype \Integer \;\;\;
       t_{2} \subtype \Integer}
      {\env_{1} \vdash e_{1} + e_{2}:\Integer, \env_{3}}
\]

\mylabel{T-ARITH2}
\[
\dfrac{\env_{1} \vdash e_{1}:t_{1}, \env_{2} \;\;\;
       \env_{2} \vdash e_{2}:t_{2}, \env_{3} \;\;\;
       t_{1} \subtype \Integer \;\;\;
       t_{2} \subtype \Number}
      {\env_{1} \vdash e_{1} + e_{2}:\Number, \env_{3}}
\]

\mylabel{T-ARITH3}
\[
\dfrac{\env_{1} \vdash e_{1}:t_{1}, \env_{2} \;\;\;
       \env_{2} \vdash e_{2}:t_{2}, \env_{3} \;\;\;
       t_{1} \subtype \Number \;\;\;
       t_{2} \subtype \Integer}
      {\env_{1} \vdash e_{1} + e_{2}:\Number, \env_{3}}
\]

\mylabel{T-ARITH4}
\[
\dfrac{\env_{1} \vdash e_{1}:t_{1}, \env_{2} \;\;\;
       \env_{2} \vdash e_{2}:t_{2}, \env_{3} \;\;\;
       t_{1} \subtype \Number \;\;\;
       t_{2} \subtype \Number}
      {\env_{1} \vdash e_{1} + e_{2}:\Number, \env_{3}}
\]

\mylabel{T-ARITH5}
\[
\dfrac{\env_{1} \vdash e_{1}:\Any, \env_{2} \;\;\;
       \env_{2} \vdash e_{2}:t, \env_{3}}
      {\env_{1} \vdash e_{1} + e_{2}:\Any, \env_{3}}
\]

\mylabel{T-ARITH6}
\[
\dfrac{\env_{1} \vdash e_{1}:t, \env_{2} \;\;\;
       \env_{2} \vdash e_{2}:\Any, \env_{3}}
      {\env_{1} \vdash e_{1} + e_{2}:\Any, \env_{3}}
\]

\mylabel{T-CONCAT1}
\[
\dfrac{\env_{1} \vdash e_{1}:t_{1}, \env_{2} \;\;\;
       \env_{2} \vdash e_{2}:t_{2}, \env_{3} \;\;\;
       t_{1} \subtype \String \;\;\;
       t_{2} \subtype \String}
      {\env_{1} \vdash e_{1} \; {..} \; e_{2}:\String, \env_{3}}
\]

\mylabel{T-CONCAT2}
\[
\dfrac{\env_{1} \vdash e_{1}:\Any, \env_{2} \;\;\;
       \env_{2} \vdash e_{2}:t, \env_{3}}
      {\env_{1} \vdash e_{1} \; {..} \; e_{2}:\Any, \env_{3}}
\]

\mylabel{T-CONCAT3}
\[
\dfrac{\env_{1} \vdash e_{1}:t, \env_{2} \;\;\;
       \env_{2} \vdash e_{2}:\Any, \env_{3}}
      {\env_{1} \vdash e_{1} \; {..} \; e_{2}:\Any, \env_{3}}
\]

\mylabel{T-EQUAL}
\[
\dfrac{\env_{1} \vdash e_{1}:t_{1}, \env_{2} \;\;\;
       \env_{2} \vdash e_{2}:t_{2}, \env_{3}}
      {\env_{1} \vdash e_{1} == e_{2}:\Boolean, \env_{3}}
\]

\mylabel{T-ORDER1}
\[
\dfrac{\env_{1} \vdash e_{1}:t_{1}, \env_{2} \;\;\;
       \env_{2} \vdash e_{2}:t_{2}, \env_{3} \;\;\;
       t_{1} \subtype \Number \;\;\;
       t_{2} \subtype \Number}
      {\env \vdash e_{1} < e_{2}:\Boolean, \env_{3}}
\]

\mylabel{T-ORDER2}
\[
\dfrac{\env_{1} \vdash e_{1}:t_{1}, \env_{2} \;\;\;
       \env_{2} \vdash e_{2}:t_{2}, \env_{3} \;\;\;
       t_{1} \subtype \String \;\;\;
       t_{2} \subtype \String}
      {\env_{1} \vdash e_{1} < e_{2}:\Boolean}
\]

\mylabel{T-ORDER3}
\[
\dfrac{\env_{1} \vdash e_{1}:\Any, \env_{2} \;\;\;
       \env_{2} \vdash e_{2}:t, \env_{3}}
      {\env_{1} \vdash e_{1} < e_{2}:\Any, \env_{3}}
\]

\mylabel{T-ORDER4}
\[
\dfrac{\env_{1} \vdash e_{1}:t, \env_{2} \;\;\;
       \env_{2} \vdash e_{2}:\Any, \env_{3}}
      {\env_{1} \vdash e_{1} < e_{2}:\Any, \env_{3}}
\]

\mylabel{T-AND1}
\[
\dfrac{\env_{1} \vdash e_{1}:\Nil, \env_{2} \;\;\;
       \env_{2} \vdash e_{2}:t, \env_{3}}
      {\env_{1} \vdash e_{1} \; \mathbf{and} \; e_{2}:\Nil, \env_{3}}
\]

\mylabel{T-AND2}
\[
\dfrac{\env_{1} \vdash e_{1}:\False, \env_{2} \;\;\;
       \env_{2} \vdash e_{2}:t, \env_{3}}
      {\env_{1} \vdash e_{1} \; \mathbf{and} \; e_{2}:\False, \env_{3}}
\]

\mylabel{T-AND3}
\[
\dfrac{\env_{1} \vdash e_{1}:t_{1} \cup \Nil, \env_{2} \;\;\;
       \env_{2} \vdash e_{2}:t_{2}, \env_{3}}
      {\env_{1} \vdash e_{1} \; \mathbf{and} \; e_{2}:\Nil \cup t_{2}, \env_{3}}
\]

\mylabel{T-AND4}
\[
\dfrac{\env_{1} \vdash e_{1}:t_{1} \cup \False, \env_{2} \;\;\;
       \env_{2} \vdash e_{2}:t_{2}, \env_{3}}
      {\env_{1} \vdash e_{1} \; \mathbf{and} \; e_{2}:\False \cup t_{2}, \env_{3}}
\]

\mylabel{T-AND5}
\[
\dfrac{\env_{1} \vdash e_{1}:t_{1}, \env_{2} \;\;\;
       \env_{2} \vdash e_{2}:t_{2}, \env_{3}}
      {\env_{1} \vdash e_{1} \; \mathbf{and} \; e_{2}:t_{1} \cup t_{2}, \env_{3}}
\]

\mylabel{T-OR1}
\[
\dfrac{\env_{1} \vdash e_{1}:\Nil, \env_{2} \;\;\;
       \env_{2} \vdash e_{2}:t, \env_{3}}
      {\env_{1} \vdash e_{1} \; \mathbf{or} \; e_{2}:t, \env_{3}}
\]

\mylabel{T-OR2}
\[
\dfrac{\env_{1} \vdash e_{1}:\False, \env_{2} \;\;\;
       \env_{2} \vdash e_{2}:t, \env_{3}}
      {\env_{1} \vdash e_{1} \; \mathbf{or} \; e_{2}:t, \env_{3}}
\]

\mylabel{T-OR3}
\[
\dfrac{\env_{1} \vdash e_{1}:t_{1} \cup \Nil, \env_{2} \;\;\;
       \env_{2} \vdash e_{2}:t_{2}, \env_{3}}
      {\env_{1} \vdash e_{1} \; \mathbf{or} \; e_{2}:t_{1} \cup t_{2}, \env_{3}}
\]

\mylabel{T-OR4}
\[
\dfrac{\env_{1} \vdash e_{1}:t_{1} \cup \False, \env_{2} \;\;\;
       \env_{2} \vdash e_{2}:t_{2}, \env_{3}}
      {\env_{1} \vdash e_{1} \; \mathbf{or} \; e_{2}:t_{1} \cup t_{2}, \env_{3}}
\]

\mylabel{T-OR5}
\[
\dfrac{\env_{1} \vdash e_{1}:t_{1}, \env_{2} \;\;\;
       \env_{2} \vdash e_{2}:t_{2}, \env_{3}}
      {\env_{1} \vdash e_{1} \; \mathbf{or} \; e_{2}:t_{1} \cup t_{2}, \env_{3}}
\]

\mylabel{T-NOT1}
\[
\dfrac{\env_{1} \vdash e:\Nil, \env_{2}}
      {\env_{1} \vdash \mathbf{not} \; e:\True, \env_{2}}
\]

\mylabel{T-NOT2}
\[
\dfrac{\env_{1} \vdash e:\False, \env_{2}}
      {\env_{1} \vdash \mathbf{not} \; e:\True, \env_{2}}
\]

\mylabel{T-NOT3}
\[
\dfrac{\env_{1} \vdash e:t, \env_{2}}
      {\env_{1} \vdash \mathbf{not} \; e:\Boolean, \env_{2}}
\]

\mylabel{T-MINUS1}
\[
\dfrac{\env_{1} \vdash e:t, \env_{2} \;\;\;
       t \subtype \Integer}
      {\env_{1} \vdash - e:\Integer, \env_{2}}
\]

\mylabel{T-MINUS2}
\[
\dfrac{\env_{1} \vdash e:t, \env_{2} \;\;\;
       t \subtype \Number}
      {\env_{1} \vdash - e:\Number, \env_{2}}
\]

\mylabel{T-MINUS3}
\[
\dfrac{\env_{1} \vdash e:\Any, \env_{2}}
      {\env_{1} \vdash - e:\Any, \env_{2}}
\]

\mylabel{T-LEN1}
\[
\dfrac{\env_{1} \vdash e:t, \env_{2} \;\;\;
       t \subtype \String}
      {\env_{1} \vdash \# \; e:\Integer, \env_{2}}
\]

\mylabel{T-LEN2}
\[
\dfrac{\env_{1} \vdash e:t, \env_{2} \;\;\;
       t \subtype \{\}_{c}}
      {\env_{1} \vdash \# \; e:\Integer, \env_{2}}
\]

\mylabel{T-LEN3}
\[
\dfrac{\env_{1} \vdash e:\Any, \env_{2}}
      {\env_{1} \vdash \# \; e:\Any, \env_{2}}
\]

\mylabel{T-EXPAPPLY1}
\[
\dfrac{\env_{1} \vdash e:p_{1} \rightarrow r, \env_{2} \;\;\;
       \env_{2} \vdash el:p_{2}, \env_{3} \;\;\;
       p_{2} \lesssim p_{1}}
      {\env_{1} \vdash e(el)_{e}:first(r), \env_{3}}
\]

\mylabel{T-EXPAPPLY2}
\[
\dfrac{\env_{1} \vdash e:\Any, \env_{2} \;\;\;
       \env_{2} \vdash el:p, \env_{3}}
      {\env_{1} \vdash e_{1}(\vec{e_{2}})_{e}:\Any, \env_{3}}
\]

\mylabel{T-EXPINVOKE1}
\[
\dfrac{\env_{1} \vdash e[n]:p_{1} \rightarrow r, \env_{2} \;\;\;
       \env_{2} \vdash el:p_{2}, \env_{3} \;\;\;
       p_{2} \lesssim p_{1}}
      {\env_{1} \vdash e{:}n(el)_{e}:first(r), \env_{3}}
\]

\mylabel{T-EXPINVOKE2}
\[
\dfrac{\env_{1} \vdash e[n]:\Any, \env_{2} \;\;\;
       \env_{2} \vdash el:p, \env_{3}}
      {\env_{1} \vdash e{:}n(el)_{e}:\Any, \env_{3}}
\]

\mylabel{T-CAST}
\[
\dfrac{t \subtype \env_{1}(n)}
      {\env_{1} \vdash {<}t{>} \; n:t, \env_{1}[n \mapsto t]}
\]

\mylabel{T-SELF}
\[
\dfrac{\env_{1} \vdash e:\Self, \env_{2} \;\;\;
       \env_{2}(\Self) = t}
      {\env_{1} \vdash e:t, \env_{2}}
\]

\mylabel{T-UNFOLD}
\[
\dfrac{\env_{1} \vdash e:\mu x.t, \env_{2}}
      {\env_{1} \vdash e:[x \mapsto \mu x.t]t, \env_{2}}
\]

\mylabel{T-FOLD}
\[
\dfrac{\env_{1} \vdash e:[x \mapsto \mu x.t]t, \env_{2}}
      {\env_{1} \vdash e:\mu x.t, \env_{2}}
\]

\mylabel{T-TERNARY}
\[
\dfrac{\env_{1} \vdash e_{1}:t_{1}, \env_{2} \;\;\;
       \env_{2} \vdash e_{2}:t_{2}, \env_{3} \;\;\;
       \env_{3} \vdash e_{3}:t_{2}, \env_{4}}
      {\env_{1} \vdash e_{1} \; \mathbf{and} \; e_{2} \; \mathbf{or} \; e_{3}:t_{2}, \env_{4}}
\]

\mylabel{T-ID}
\[
\dfrac{\env_{1}(n) = t}
      {\env_{1} \vdash n:t, \env_{1}}
\]

\mylabel{T-INDEX1}
\[
\dfrac{\env_{1} \vdash e_{1}:\{k_{1}{:}v_{1}, ..., k_{n}{:}v_{n}\}, \env_{1} \;\;\;
       \env_{2} \vdash e_{2}:t, \env_{3} \;\;\;
       \exists i \in 1{..}n \; t \lesssim k_{i}}
      {\env_{1} \vdash e_{1}[e_{2}]:v_{i}, \env_{3}}
\]

\mylabel{T-INDEX2}
\[
\dfrac{\env_{1} \vdash e_{1}:\Any, \env_{2} \;\;\;
       \env_{2} \vdash e_{2}:t, \env_{3}}
      {\env_{1} \vdash e_{1}[e_{2}]:\Any, \env_{3}}
\]

\mylabel{T-REFINE}
\[
\dfrac{\env_{1}(n) = \{ k_{1}{:}v_{1}, ..., k_{n}{:}v_{n} \}_{o|u} \;\;\;
       \env_{1} \vdash e:t_{1}, \env_{2} \;\;\;
       \not \exists i \in 1..n \; t_{1} \lesssim k_{i}}
      {\env_{1} \vdash n[e] {<}t{>}:t, \env_{2}[n \mapsto \{ k_{1}{:}v_{1}, ..., k_{n}{:}v_{n}, t_{1}{:}t\}_{o|u}]}
\]

\mylabel{T-EXPLIST1}
\[
\env \vdash \mathbf{nothing}:\Nil{*}, \env
\]

\mylabel{T-EXPLIST2}
\[
\dfrac{\env \vdash e_{k}:t_{k}, \env_{k} \;\;\;
       n = |\vec{e}|}
      {\env \vdash \vec{e}:t_{1} \times ... \times t_{n}, merge(\env_{1}, ..., \env_{n})}
\]

\mylabel{T-EXPLIST3}
\[
\dfrac{\env \vdash me:r, \env_{1}}
      {\env \vdash me:r, \env_{1}}
\]

\mylabel{T-EXPLIST4}
\[
\dfrac{\env \vdash e_{k}:t_{k}, \env_{k} \;\;\;
       \env \vdash me:r, \env_{n + 1} \;\;\;
       n = |\vec{e}|}
      {\env \vdash \vec{e}, me:t_{1} \times ... \times t_{n} \times r, merge(\env_{1}, ..., \env_{n+1})}
\]

\mylabel{T-APPLY1}
\[
\dfrac{\env_{1} \vdash e:p_{1} \rightarrow r, \env_{2} \;\;\;
       \env_{2} \vdash el:p_{2}, \env_{3} \;\;\;
       p_{2} \lesssim p_{1}}
      {\env_{1} \vdash e(el):r, \env_{3}}
\]

\mylabel{T-APPLY2}
\[
\dfrac{\env_{1} \vdash e:\Any, \env_{2} \;\;\;
       \env_{2} \vdash el:p, \env_{3}}
      {\env_{1} \vdash e_{1}(\vec{e_{2}}):\Any{*}, \env_{3}}
\]

\mylabel{T-INVOKE1}
\[
\dfrac{\env_{1} \vdash e[n]:p_{1} \rightarrow r, \env_{2} \;\;\;
       \env_{2} \vdash el:p_{2}, \env_{3} \;\;\;
       p_{2} \lesssim p_{1}}
      {\env_{1} \vdash e{:}n(el):r, \env_{3}}
\]

\mylabel{T-INVOKE2}
\[
\dfrac{\env_{1} \vdash e[n]:\Any, \env_{2} \;\;\;
       \env_{2} \vdash el:p, \env_{3}}
      {\env_{1} \vdash e{:}n(el):\Any{*}, \env_{3}}
\]

\mylabel{T-DOTS}
\[
\dfrac{\env_{1}({...}) = t}
      {\env_{1} \vdash {...}:t{*}, \env_{1}}
\]



\end{document}
