\documentclass{beamer}

\usepackage[english]{babel}
\usepackage[utf8]{inputenc}
\usepackage{beamerthemesplit}

\newcommand{\mylabel}[1]{\; (\textsc{#1})}
\newcommand{\subtype}{<:}
\newcommand{\pipe}{|\;}
\newcommand{\kw}[1]{\mathbf{#1} \;}
\newcommand{\env}{\Gamma}

\begin{document}

\title{Typed Lua}
\subtitle{An Optional Type System for Lua}
\author{André Murbach Maidl}
\institute{LabLua\\PUC-Rio}
\date{September 13th 2013}

\frame{\titlepage}

\begin{frame}
\frametitle{What is Typed Lua?}
\begin{itemize}
\item A typed superset of Lua that compiles to plain Lua.
\item \textcolor{blue}{Type annotations}.
\item \textcolor{blue}{Compile-time type checking}.
\item \textcolor{gray}{Classes}.
\item \textcolor{gray}{Interfaces}.
\item \textcolor{gray}{Modules}.
\end{itemize}
\end{frame}

\begin{frame}
\frametitle{Why Optional and not Gradual?}
\begin{itemize}
\item Because, first of all, gradual is optional!
\end{itemize}
\end{frame}

\begin{frame}
\frametitle{Optional versus Gradual}
\begin{center}
\begin{tabular}{|r|c|c|}
\hline
& Optional & Gradual\\
\hline
Optional type annotations & Yes & Yes \\ 
\hline
Compile-time type checking & Yes & Yes \\
\hline
Influence the run-time semantics & No & Yes \\
\hline
\end{tabular}
\end{center}
\end{frame}

\begin{frame}
\frametitle{The levels of Gradual Typing according to Jeremy Siek}
\begin{center}
\begin{tabular}{|r|c|c|c|}
\hline
& Level 1 & Level 2 & Level 3\\
\hline
Optional type annotations & Yes & Yes & Yes \\ 
\hline
Compile-time type checking & Yes & Yes & Yes \\
\hline
Run-time checking$^{1}$ & No & Yes & Yes \\
\hline
Blame tracking$^{2}$ & No & No & Yes\\
\hline
\end{tabular}
\end{center}
$1$ -- A compiler for a gradually typed language (level $>$ 1) infers
where dynamic checks are needed and inserts casts into the intermediate
language to performe these checks.\\
$2$ -- Blame tracking solves the problem of tracing a run-time cast
failure back to the source of the error.
\end{frame}

\begin{frame}
\frametitle{Examples of Gradually Typed Languages}
\begin{enumerate}
\item Strongtalk, TypeScript.
\item ActionScript.
\item Typed Scheme.
\end{enumerate}
\end{frame}

\begin{frame}
\frametitle{Overview of what comes next in the presentation}
\begin{itemize}
\item Optional type annotations.
\begin{itemize}
\item Changes in the syntax of Lua.
\item Type language.
\end{itemize}
\item Compile-time type checking.
\begin{itemize}
\item Type system and typing rules.
\end{itemize}
\end{itemize}
\end{frame}

\begin{frame}
\frametitle{Changes in the syntax of Lua}
\begin{align*}
stat ::= & \; ... \; |\\
& \textcolor{blue}{varlist} \; \texttt{`='} \; explist \; |\\
& \; ... \; |\\
& \textbf{function} \; funcname \; \textcolor{blue}{funcbody} \; |\\
& \textbf{local} \; \textbf{function} \; Name \; \textcolor{blue}{funcbody} \; |\\
& \textbf{local} \; \textcolor{blue}{namelist} \; [\texttt{`='} \; explist]\\
exp ::= & \; ... \; |\\
& functiondef \; |\\
& \; ...\\
functiondef ::= & \; \textbf{function} \; \textcolor{blue}{funcbody}
\end{align*}
\end{frame}

\begin{frame}
\frametitle{Declaration of variables}
\begin{align*}
stat ::= & \; ... \; |\\
& \textcolor{blue}{varlist} \; \texttt{`='} \; explist \; |\\
& \; ... \; |\\
& \textbf{local} \; \textcolor{blue}{namelist} \; [\texttt{`='} \; explist]\\
varlist ::= & \; \textcolor{blue}{var} \; \{\texttt{`,'} \; \textcolor{blue}{var}\}\\
var ::= & \; Name \; \textcolor{blue}{[\texttt{`:'} \; type]} \; |\\
& prefixexp \; \texttt{`['} \; exp \; \texttt{`]'} \; |\\
& prefixexp \; \texttt{`.'} \; Name\\
namelist ::= & \; Name \; \textcolor{blue}{[\texttt{`:'} \; type]} \;
\{\texttt{`,'} \; Name \; \textcolor{blue}{[\texttt{`:'} \; type]}\}
\end{align*}
\end{frame}

\begin{frame}
\frametitle{Declaration of functions}
\begin{align*}
stat ::= & \; ... \; |\\
& \textbf{function} \; funcname \; \textcolor{blue}{funcbody} \; |\\
& \textbf{local} \; \textbf{function} \; Name \; \textcolor{blue}{funcbody} \; |\\
& \; ... \\
exp ::= & \; ... \; |\\
& functiondef \; |\\
& \; ...\\
functiondef ::= & \; \textbf{function} \; \textcolor{blue}{funcbody}\\
funcbody ::= & \; \texttt{`('} \; [\textcolor{blue}{parlist}] \; \texttt{`)'} \;
\textcolor{blue}{[\texttt{`:'} \; type]} \; block \; \textbf{end}\\
parlist ::= & \; \textcolor{blue}{namelist} \; [\texttt{`,'} \; \texttt{`...'} \;
\textcolor{blue}{[\texttt{`:'} \; type]}] \; | \;
\texttt{`...'} \; \textcolor{blue}{[\texttt{`:'} \; type]}\\
namelist ::= & \; Name \; \textcolor{blue}{[\texttt{`:'} \; type]} \;
\{\texttt{`,'} \; Name \; \textcolor{blue}{[\texttt{`:'} \; type]}\}
\end{align*}
\end{frame}

\begin{frame}
\frametitle{Type language}
\begin{align*}
T ::= \; & C \; \pipe B \; \pipe Object \; \pipe Any \; \pipe
S \rightarrow S \; \pipe T \cup T\\
C ::= \; & \mathbf{nil} \; \pipe \mathbf{false} \; \pipe \mathbf{true} \;
\pipe {<}double{>} \; \pipe {<}integer{>} \; \pipe {<}string{>}\\
B ::= \; & Boolean \; \pipe Number \; \pipe String\\
S ::= \; & T \; \pipe {T*} \; \pipe T \times S\\ 
\end{align*}
\end{frame}

\end{document}
