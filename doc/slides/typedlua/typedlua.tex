\documentclass{beamer}

\usepackage[english]{babel}
\usepackage[utf8]{inputenc}
\usepackage{beamerthemesplit}

\begin{document}

\title{Typed Lua}
\subtitle{An Optional Type System for Lua}
\author{André Murbach Maidl}
\institute{LabLua\\PUC-Rio}
\date{September 13th 2013}

\frame{\titlepage}

\begin{frame}
\frametitle{What is Typed Lua?}
\begin{itemize}
\item A typed superset of Lua that compiles to plain Lua.
\item \textcolor{blue}{Type annotations}.
\item \textcolor{blue}{Compile-time type checking}.
\item \textcolor{gray}{Classes}.
\item \textcolor{gray}{Interfaces}.
\item \textcolor{gray}{Modules}.
\end{itemize}
\end{frame}

\begin{frame}
\frametitle{Why Optional and not Gradual?}
\begin{itemize}
\item Because, first of all, gradual is optional!
\end{itemize}
\end{frame}

\begin{frame}
\frametitle{Optional versus Gradual}
\begin{center}
\begin{tabular}{|r|c|c|}
\hline
& Optional & Gradual\\
\hline
Optional type annotations & Yes & Yes \\ 
\hline
Compile-time type checking & Yes & Yes \\
\hline
Influence the run-time semantics & No & Yes \\
\hline
\end{tabular}
\end{center}
\end{frame}

\begin{frame}
\frametitle{The levels of Gradual Typing according to Jeremy Siek}
\begin{center}
\begin{tabular}{|r|c|c|c|}
\hline
& Level 1 & Level 2 & Level 3\\
\hline
Optional type annotations & Yes & Yes & Yes \\ 
\hline
Compile-time type checking & Yes & Yes & Yes \\
\hline
Run-time checking$^{1}$ & No & Yes & Yes \\
\hline
Blame tracking$^{2}$ & No & No & Yes\\
\hline
\end{tabular}
\end{center}
$1$ -- A compiler for a gradually typed language (level $>$ 1) infers
where dynamic checks are needed and inserts casts into the intermediate
language to performe these checks.\\
$2$ -- Blame tracking solves the problem of tracing a run-time cast
failure back to the source of the error.
\end{frame}

\begin{frame}
\frametitle{Examples of Gradually Typed Languages}
\begin{enumerate}
\item Strongtalk, TypeScript.
\item ActionScript.
\item Typed Scheme.
\end{enumerate}
\end{frame}

\begin{frame}
\frametitle{Overview of what comes next in the presentation}
\begin{itemize}
\item Optional type annotations.
\begin{itemize}
\item Changes in the syntax of Lua.
\item Type language.
\end{itemize}
\item Compile-time type checking.
\begin{itemize}
\item Type system and typing rules.
\end{itemize}
\end{itemize}
\end{frame}

\end{document}
