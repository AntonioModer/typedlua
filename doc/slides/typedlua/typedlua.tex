%\documentclass[mathserif]{beamer}
\documentclass{beamer}

\usepackage[english]{babel}
\usepackage[utf8]{inputenc}
\usepackage{amsmath}
\usepackage{amssymb}
\usepackage{beamerthemesplit}
%\usecolortheme{dove}

\newcommand{\mylabel}[1]{\; (\textsc{#1})}
\newcommand{\subtype}{<:}
\newcommand{\pipe}{|\;}
\newcommand{\kw}[1]{\mathbf{#1} \;}
\newcommand{\env}{\Gamma}

\begin{document}

\title{Typed Lua}
\subtitle{An Optional Type System for Lua}
\author{André Murbach Maidl}
\institute{Colóquios do LabLua}
\date{September 13th 2013}

\frame{\titlepage}

\begin{frame}
\frametitle{What is Typed Lua?}
\begin{itemize}
\item A typed superset of Lua that compiles to plain Lua.
\item \textcolor{blue}{Type annotations}.
\item \textcolor{blue}{Compile-time type checking}.
\item \textcolor{gray}{Classes}.
\item \textcolor{gray}{Interfaces}.
\item \textcolor{gray}{Modules}.
\end{itemize}
\end{frame}

\begin{frame}
\frametitle{Why Optional and not Gradual?}
\begin{itemize}
\item Because, first of all, gradual is optional!
\end{itemize}
\end{frame}

\begin{frame}
\frametitle{Optional versus Gradual}
\begin{center}
\begin{tabular}{|r|c|c|}
\hline
& Optional & Gradual\\
\hline
Optional type annotations & Yes & Yes \\ 
\hline
Compile-time type checking & Yes & Yes \\
\hline
Influence the run-time semantics & No & Yes \\
\hline
\end{tabular}
\end{center}
\end{frame}

\begin{frame}
\frametitle{Gradual Typing}
Gradual type systems use consistency ($\sim$) instead of equality ($=$).
\begin{Large}
\[
\tau ::= \gamma \;|\; ? \;|\; \tau \rightarrow \tau
\]
\[
\tau \sim \tau
\]
\[
\tau \sim \;?
\]
\[
?\; \sim \tau
\]

\[
\frac{\sigma_{1} \sim \tau_{1} \;\;\; \sigma_{2} \sim \tau_{2}}
     {\sigma_{1} \rightarrow \sigma_{2} \sim \tau_{1} \rightarrow \tau_{2}}
\]
\end{Large}
\end{frame}

\begin{frame}
\frametitle{Gradual Typing}
Gradual type systems use consistency (\textcolor{blue}{$\sim$}) instead of equality ($=$).
\begin{Large}
\[
e ::= c \;|\; x \;|\; \lambda x:\tau.e \;|\; e\;e \;\;\;
(\lambda x.e \equiv \lambda x:?.e)
\]
\[
\frac{\Gamma x = \lfloor\tau\rfloor}
     {\Gamma \vdash_{G} x:\tau} \;\;\;\;\;
\frac{\Delta c = \tau}
     {\Gamma \vdash_{G} c:\tau}
\]

\[
\frac{\Gamma(x \mapsto \sigma) \vdash_{G} e:\tau}
     {\Gamma \vdash_{G} \lambda x:\sigma.e:\sigma \rightarrow \tau} \;\;\;\;\;
\frac{\Gamma \vdash_{G} e_{1}:\;? \;\;\; \Gamma \vdash_{G} e_{2}:\tau_{2}}
     {\Gamma \vdash_{G} e_{1}\;e_{2}:\;?}
\]

\[
\frac{\Gamma \vdash_{G} e_{1}:\tau \rightarrow \tau' \;\;\;
      \Gamma \vdash_{G} e_{2}:\tau_{2} \;\;\; \textcolor{blue}{\tau_{2} \sim \tau}}
     {\Gamma \vdash_{G} e_{1}\;e_{2}:\tau'}
\]
\end{Large}
\end{frame}

\begin{frame}
\frametitle{The levels of Gradual Typing according to Jeremy Siek}
\begin{center}
\begin{tabular}{|r|c|c|c|}
\hline
& Level 1 & Level 2 & Level 3\\
\hline
Optional type annotations & Yes & Yes & Yes \\ 
\hline
Compile-time type checking & Yes & Yes & Yes \\
\hline
Run-time checking$^{1}$ & No & Yes & Yes \\
\hline
Blame tracking$^{2}$ & No & No & Yes\\
\hline
\end{tabular}
\end{center}
$1$ -- A compiler for a gradually typed language (level $>$ 1) infers
where dynamic checks are needed and inserts casts into the intermediate
language to perform these checks.\\
$2$ -- Blame tracking solves the problem of tracing a run-time cast
failure back to the source of the error.
\end{frame}

\begin{frame}
\frametitle{Examples of Gradually Typed Languages}
\begin{enumerate}
\item Strongtalk, TypeScript, and Typed Lua.
\item ActionScript.
\item Typed Racket.
\end{enumerate}
\end{frame}

\begin{frame}
\frametitle{Overview of rest of the presentation}
\begin{itemize}
\item Changes in the syntax of Lua.
\item Typed Lua Type System.
\end{itemize}
\end{frame}

\begin{frame}
\frametitle{Changes in the syntax of Lua}
\begin{align*}
stat ::= & \; ... \; |\\
& \textcolor{blue}{varlist} \; \texttt{`='} \; explist \; |\\
& \; ... \; |\\
& \textbf{function} \; funcname \; \textcolor{blue}{funcbody} \; |\\
& \textbf{local} \; \textbf{function} \; Name \; \textcolor{blue}{funcbody} \; |\\
& \textbf{local} \; \textcolor{blue}{namelist} \; [\texttt{`='} \; explist]\\
exp ::= & \; ... \; |\\
& functiondef \; |\\
& \; ...\\
functiondef ::= & \; \textbf{function} \; \textcolor{blue}{funcbody}
\end{align*}
\end{frame}

\begin{frame}
\frametitle{Declaration of variables}
\begin{align*}
stat ::= & \; ... \; |\\
& \textcolor{blue}{varlist} \; \texttt{`='} \; explist \; |\\
& \; ... \; |\\
& \textbf{local} \; \textcolor{blue}{namelist} \; [\texttt{`='} \; explist]\\
varlist ::= & \; \textcolor{blue}{var} \; \{\texttt{`,'} \; \textcolor{blue}{var}\}\\
var ::= & \; Name \; \textcolor{blue}{[\texttt{`:'} \; type]} \; |\\
& prefixexp \; \texttt{`['} \; exp \; \texttt{`]'} \; |\\
& prefixexp \; \texttt{`.'} \; Name\\
namelist ::= & \; Name \; \textcolor{blue}{[\texttt{`:'} \; type]} \;
\{\texttt{`,'} \; Name \; \textcolor{blue}{[\texttt{`:'} \; type]}\}
\end{align*}
\end{frame}

\begin{frame}
\frametitle{Declaration of functions}
\begin{align*}
stat ::= & \; ... \; |\\
& \textbf{function} \; funcname \; \textcolor{blue}{funcbody} \; |\\
& \textbf{local} \; \textbf{function} \; Name \; \textcolor{blue}{funcbody} \; |\\
& \; ... \\
exp ::= & \; ... \; |\\
& functiondef \; |\\
& \; ...\\
functiondef ::= & \; \textbf{function} \; \textcolor{blue}{funcbody}\\
funcbody ::= & \; \texttt{`('} \; [\textcolor{blue}{parlist}] \; \texttt{`)'} \;
\textcolor{blue}{[\texttt{`:'} \; type]} \; block \; \textbf{end}\\
parlist ::= & \; \textcolor{blue}{namelist} \; [\texttt{`,'} \; \texttt{`...'} \;
\textcolor{blue}{[\texttt{`:'} \; type]}] \; | \;
\texttt{`...'} \; \textcolor{blue}{[\texttt{`:'} \; type]}\\
namelist ::= & \; Name \; \textcolor{blue}{[\texttt{`:'} \; type]} \;
\{\texttt{`,'} \; Name \; \textcolor{blue}{[\texttt{`:'} \; type]}\}
\end{align*}
\end{frame}

\begin{frame}
\frametitle{Type annotations}
\begin{align*}
type ::= & \; \textbf{object} \;|\; \textbf{any} \;|\; \textbf{nil} \;|\\
& basetype \;|\; uniontype \;|\; functiontype\\
basetype ::= & \; \textbf{boolean} \;|\; \textbf{number} \;|\; \textbf{string}\\
uniontype ::= & \; type \;\texttt{`|'}\; type\\
functiontype ::= & \; \texttt{`('} \; [2ndclasstype] \; \texttt{`)'} \;
\texttt{`->'} \; 2ndclasstype\\
2ndclasstype ::= & \; type \; \{\texttt{`,'} \; type\} \; [\texttt{`*'}]
\end{align*}
\end{frame}

\begin{frame}
\frametitle{Example}
\textit{Pensar em um exemplo legal para colocar aqui.}
\end{frame}

\begin{frame}
\frametitle{Typed Lua Type System}
\begin{itemize}
\item Our aim is to design an object oriented language.
\item Most OO languages use the following subsumption rule:
\[
\frac{\env \vdash e:T_{1} \;\;\; T_{1} \subtype T_{2}}
     {\env \vdash e:T_{2}}
\]
\item We also aim to have gradual typing, so we also
need the consistency relation.
\end{itemize}
\end{frame}

\begin{frame}
\frametitle{Typed Lua Type System}
We do not use the subsumption rule, but subtyping
instead of equality.
\[
\frac{\env \vdash e:S_{1} \rightarrow S_{2} \;\;\;
      \env \vdash el:S_{3} \;\;\;
      \textcolor{red}{S_{3} \subtype S_{1}}}
     {\env \vdash e(el):S_{2}}
\]
and also the consistency rule
\[
\frac{\env \vdash e:S_{1} \rightarrow S_{2} \;\;\;
      \env \vdash el:S_{3} \;\;\;
      \textcolor{red}{S_{3} \subtype S_{1}'} \;\;\;
      \textcolor{blue}{S_{1}' \sim S_{1}}}
     {\env \vdash e(el):S_{2}}
\]
so we need to compose the two relations:
\[
\frac{\env \vdash e:S_{1} \rightarrow S_{2} \;\;\;
      \env \vdash el:S_{3} \;\;\;
      \textcolor{red!50!blue}{S_{3} \lesssim S_{1}}}
     {\env \vdash e(el):S_{2}}
\]
which we call consistent-subtyping.
\end{frame}

\begin{frame}
\frametitle{Type language}
\begin{align*}
T ::= \; & C \; \pipe B \; \pipe Object \; \pipe Any \; \pipe
S \rightarrow S \; \pipe T \cup T\\
C ::= \; & \mathbf{nil} \; \pipe \mathbf{false} \; \pipe \mathbf{true} \;
\pipe {<}double{>} \; \pipe {<}integer{>} \; \pipe {<}string{>}\\
B ::= \; & Boolean \; \pipe Number \; \pipe String\\
S ::= \; & T \; \pipe {T*} \; \pipe T \times S\\ 
\end{align*}
\end{frame}

\begin{frame}
\frametitle{Typing rules}
\begin{itemize}
\item Type consistency.
\item Subtyping.
\item Typed Lua typing rules.
\end{itemize}
\end{frame}

\begin{frame}
\textit{Colocar as regras do sistema de tipos.}
\end{frame}

\end{document}
