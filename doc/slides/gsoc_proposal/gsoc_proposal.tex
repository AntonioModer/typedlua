\documentclass{beamer}

\usepackage[english]{babel}
\usepackage[utf8]{inputenc}
\usepackage{beamerthemesplit}

\begin{document}

\title{Typed Lua}
\subtitle{Google Summer of Code Proposal}
\author{André Murbach Maidl}
\institute[PUC-Rio]{LabLua\\PUC-Rio}
\date{June 27th 2013}

\frame{\titlepage}

\section{Static Typing versus Dynamic Typing}
\begin{frame}
\frametitle{Static Typing versus Dynamic Typing}
\begin{itemize}
\item Static Typing
\begin{itemize}
\item Better documentation
\item Error detection at compile time
\item More efficient program execution
\end{itemize}
\item Dynamic Typing
\begin{itemize}
\item Faster adaptation to changing requirements
\item Quicker development
\end{itemize}
\end{itemize}
\end{frame}

\subsection{Multiple languages to build large softwares}
\begin{frame}
\frametitle{Multiple languages to build large softwares}
\begin{itemize}
\item Prototype using a dynamically typed language
\item Rewrite it in a statically typed language
\end{itemize}
\end{frame}

\section{Gradual Typing}
\begin{frame}
\frametitle{Gradual Typing}
\begin{itemize}
\item The choice between static and dynamic typing in the same language
\item Use type annotations where static typing is needed
\item Avoid type annotations where dynamic typing is sufficient
\end{itemize}
\end{frame}

\begin{frame}[fragile]
\frametitle{Gradual Typing Example}
\begin{verbatim}
local function div (x, y)
  if y == 0 then
    return nil,"division by zero"
  end
  return x / y
end

local function div (x:Number, y:Number) -> Any
  if y == 0 then
    return nil,"division by zero"
  end
  return x / y
end
\end{verbatim}
\end{frame}

\subsection{The levels of Gradual Typing}
\begin{frame}
\frametitle{The levels of Gradual Typing}
\begin{itemize}
\item The programmer can gradually migrate to static typing
\item Specially in module boundaries
\begin{enumerate}
\item The programmer write modules that use only dynamic typing.
\item The programmer creates an interface to type a module that
is written using only dynamic typing. This module is not
type checked, but the code that uses this module is.
\item The programmer write modules that use only static typing.
\end{enumerate}
\end{itemize}
\end{frame}

\subsection{Gradual Typing and Dynamically Typed Languages}
\begin{frame}
\frametitle{Gradual Typing and Dynamically Typed Languages}
\begin{itemize}
\item Dart and TypeScript
\begin{itemize}
\item New languages derived from JavaScript
\item Both target JavaScript
\item Both are designed with a gradual type system
\end{itemize}
\item Typed Racket and core.typed
\begin{itemize}
\item Gradual typing extensions to Racket and Clojure
\end{itemize}
\item Perl 6
\begin{itemize}
\item Not compatible with Perl 5
\end{itemize}
\end{itemize}
\end{frame}

\begin{frame}
\frametitle{Gradual Typing and Dynamically Typed Languages}
\begin{itemize}
\item Python
\begin{itemize}
\item py-gradual
\begin{itemize}
\item A package that provides runtime semantics of gradual typing
\item Works only on Python 3
\end{itemize}
\item reticulated
\begin{itemize}
\item A tool that implements gradual typing
\item Works on Python 2.7, 3.2, and 3.3
\end{itemize}
\end{itemize}
\item Jython
\begin{itemize}
\item jython-gradual
\begin{itemize}
\item A fork that includes gradual typing
\end{itemize}
\end{itemize}
\end{itemize}
\end{frame}

\section{Typed Lua}
\begin{frame}
\frametitle{Typed Lua}
\begin{itemize}
\item A proposal to design a gradual type system for Lua
\begin{itemize}
\item Should be simple enough to preserve existing Lua idioms
\end{itemize}
\item A superset of Lua
\begin{itemize}
\item Type annotations
\item Compile-time type checking
\item Classes
\item Interfaces
\item Modules
\end{itemize}
\end{itemize}
\end{frame}

\begin{frame}
\frametitle{Typed Lua}
\begin{itemize}
\item Lua versus modules and OO programming
\begin{itemize}
\item It is possible to split Lua code into modules
\item It is possible to write OO code in Lua
\item Lua does not set policies on how these features should behave
\item Probably due to its use as an embedded and extension language
\end{itemize}
\item Typed Lua intended use is as an application language
\item So the policies for modules and OO programming should be
part of the language
\item The gradual type system helps to enforce these policies
\end{itemize}
\end{frame}

\subsection{Related Work}
\begin{frame}
\frametitle{Related Work}
\begin{itemize}
\item Some Lua projects implement parts of what is being proposed
\item None implement compile-time type checking along with OO programming
\end{itemize}
\end{frame}

\begin{frame}
\frametitle{MetaLua and MoonScript}
\begin{itemize}
\item Both support some sort of OO programming
\item Both do not perform compile-time type checking
\end{itemize}
\end{frame}

\begin{frame}
\frametitle{LuaInspect}
\begin{itemize}
\item A tool that uses MetaLua to perform code analysis
\begin{itemize}
\item Flags unknown global variables and table fields
\item Checks number of function arguments against signatures
\item Infers function return values
\end{itemize}
\item It does not try to analyze OO code
\item It does not perform compile-time type checking
\end{itemize}
\end{frame}

\begin{frame}
\frametitle{Tidal Lock}
\begin{itemize}
\item A prototype of a gradual type system for Lua
\item It is not typing the common OO idioms used by Lua programmers
\end{itemize}
\end{frame}

\section{Statistics}
\begin{frame}[fragile]
\frametitle{Statistics}
\begin{itemize}
\item Data from LuaRocks repository as of June 22nd
\item LuaRocks repository contains 3000 \verb'.lua' files
\item \textbf{luac} (version 5.2.2) compiles 2913 of the 3000 files
\item The statistics include 1966 of the 2913 files
\begin{itemize}
\item I did not include files that are generated by MoonScript
\item I did not include documentation subdirectories
\item I did not include test subdirectories
\item I did not include test files in general
\end{itemize}
\end{itemize}
\end{frame}

\subsection{Function declaration}
\begin{frame}
\frametitle{Function declaration}
\begin{itemize}
\item 17390 functions are being declared in the 1966 files
\begin{itemize}
\item 3961 (22.8\%) are actually methods
\item 1512 (8.7\%) use \textbf{type} to inspect the tags of the arguments
\item 773 (4.5\%) return \textbf{nil} plus something else
\item 118 (0.7\%) return \textbf{false} plus something else
\end{itemize}
\end{itemize}
\end{frame}

\subsection{Metatables}
\begin{frame}
\frametitle{Metatables}
\begin{itemize}
\item 294 (15.0\%) of the 1966 files use \textbf{setmetatable}
\item 99 (5.1\%) of the 196 files use \textbf{getmetatable}
\end{itemize}
\end{frame}

\subsection{Table constructor}
\begin{frame}
\frametitle{Table constructor}
\begin{itemize}
\item The table constructor is being used 17369 times in the 1966 files
\begin{itemize}
\item 4391 (25.28\%) times is the empty construct
\item 12975 (74.70\%) times is a constructor with only static keys
\item 3 (0.02\%) times is a constructor with only dynamic keys
\item There is no constructor with static and dynamic keys
\end{itemize}
\end{itemize}
\end{frame}

\subsection{Module}
\begin{frame}
\frametitle{Module}
\begin{itemize}
\item 655 (33.3\%) of the 1966 files use \textbf{return} at the end of the file
\item 602 (30.7\%) of the 1966 files use the function \textbf{module}
\item The rest of the files are plain scripts
\end{itemize}
\end{frame}

\end{document}
