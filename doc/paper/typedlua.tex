\documentclass[preprint]{sig-alternate}

\usepackage[utf8]{inputenc}
\usepackage[numbers]{natbib}
\usepackage{amsmath}
\usepackage{amssymb}
\usepackage{url}

\newcommand{\Any}{\mathbf{any}}
\newcommand{\Top}{\mathbf{value}}
\newcommand{\Nil}{\mathbf{nil}}
\newcommand{\False}{\mathbf{false}}
\newcommand{\True}{\mathbf{true}}
\newcommand{\Boolean}{\mathbf{boolean}}
\newcommand{\Number}{\mathbf{number}}
\newcommand{\String}{\mathbf{string}}
\newcommand{\Void}{\mathbf{void}}
\newcommand{\Const}{\mathbf{const}}

\def\dstart{\hbox to \hsize{\vrule depth 4pt\hrulefill\vrule depth 4pt}}
\def\dend{\hbox to \hsize{\vrule height 4pt\hrulefill\vrule height 4pt}}

\begin{document}

\conferenceinfo{Dyla}{'14, Edinburgh, UK} 

\title{Typed Lua: An Optional Type System for Lua}

\numberofauthors{3}

\author{
\alignauthor
André Murbach Maidl\\
  \affaddr{PUC-Rio}\\
  \affaddr{Rio de Janeiro, Brazil}\\
  \email{amaidl@inf.puc-rio.br}
\alignauthor
Fabio Mascarenhas\\
  \affaddr{UFRJ}\\
  \affaddr{Rio de Janeiro, Brazil}\\
  \email{fabiom@dcc.ufrj.br}
\alignauthor
Roberto Ierusalimschy\\
  \affaddr{PUC-Rio}\\
  \affaddr{Rio de Janeiro, Brazil}\\
  \email{roberto@inf.puc-rio.br}
}

\date{June 12th 2014}

\maketitle

\begin{abstract}
\end{abstract}

\category{D.3.3}
         {Programming Languages}
         {Language Constructs and Features}
\category{D.2.3}
         {Programming Languages}
         {Coding Tools and Techniques}
         [Object-oriented programming]

\terms{Languages,Verification}

\keywords{Type Systems, Lua}

\section{Introduction} \label{sec:intro}

Dynamically typed languages such as Lua avoid static types in favor of
run-time {\em type tags} that classify the values they compute, and
their implementations use these tags to perform run-time (or dynamic)
checking and guarantee that only valid operations are
performed~\citep{pierce2002tpl}.

The absence of static types means that programmers do not need to
bother about abstracting types that might require a complex type
system and type checker to validate, leading to simpler and more flexible
languages and implementations. But this absence may also hide bugs
that will be caught only after deployment if programmers do not properly
test their code.

In contrast, static type checking helps programmers detect many 
bugs during the development phase. Static types also provide a
conceptual framework that helps programmers define modules
and interfaces that can be combined to structure the development
of large programs.

The early error detection and better program structure afforded by
static type checking can lead programmers to migrate their code from
a dynamically typed to a statically typed language, once their simple
scripts become complex programs~\citep{tobin-hochstadt2006ims}.
As this migration involves languages with different syntax and
semantics, it requires a complete rewrite of existing programs intead
of incremental evolution from dynamic to static types.

Ideally, programming languages should offer programmers the
option to choose between static and dynamic typing:
\textit{optional type systems}~\citep{bracha2004pluggable} and
\textit{gradual typing}~\citep{siek2006gradual} are two approaches
that offer programmers the option to use type annotations where static
typing is needed, incrementally migrating a system from dynamic
to static types. The difference between these two approaches is the
way they treat run-time semantics: while optional type systems
do not affect the run-time semantics,
gradual typing uses run-time checks to ensure that dynamically typed
code does not violate the invariants of statically typed code.

In this work we outline the design of Typed Lua:
an optional type system for Lua that is complex enough to
preserve some of the idioms that Lua programmers are already used to,
while adding new constructs that help programmers structure Lua
programs.


Lua is a small imperative scripting language with first-class
functions (with proper lexical scoping) where the main data
structure is the {\em table}, an associative array that can
play the part of arrays, records, maps, objects, etc.
with syntactic sugar and metaprogramming through operator overloading built into
the language. Unlike other scripting languages, Lua has very
limited coercion among different data types.

The primary use of Lua has always been as an embedded language
for configuration and extension of other applications.
Lua prefers to provide mechanisms instead of fixed policies
for structuring programs, and even features
such as a module system and object orientation are a matter of 
convention instead of built into the language.
The result is a fragmented ecosystem of libraries, and
different ideas among Lua programmers on how they should use
the language features and how they should structure programs.

The lack of standard policies
is a challenge for the design of a static type system for the Lua
language. The design of Typed Lua is informed by a (mostly
automated) survey of Lua idioms used in a large corpus of Lua
libraries, instead of relying just on the semantics of the
language.

Typed Lua allows statically typed Lua code to coexist and
interact with  ynamically typed code. The Typed Lua compiler
warns the programmer about type errors, but always generates
Lua code that runs in unmodified Lua implementations. The
programmer can enjoy some of the benefits of static types even
without converting existing Lua modules to typed Lua:
a dynamically typed module can export a statically typed
interface, and statically typed users of the module will have
their use of the module checked by the compiler.

Unlike gradual type systems, Typed Lua does not insert run-time
checks between dynamically and statically typed parts of the
program. Unlike some optional type systems, the statically
typed subset of Typed Lua is sound by design, so a later
version of the Typed Lua compiler can swicth to gradual
instead of just optional typing.

We cover the main parts of the design, along with
how they relate to Lua, in Sections~\ref{sec:atomic}
through~\ref{sec:classes}. Section~\ref{sec:review} reviews
related work on mixing static and dynamic typing in the same
language. Finally, Section~\ref{sec:con} gives some
concluding remarks and future extensions for Typed Lua.

\section{Atomic Types and Functions}
\label{sec:atomic}

Lua values can have one of eight {\em tags}: {\em nil}, {\em boolean},
{\em number}, {\em string}, {\em function}, {\em table}, {\em userdata},
and {\em thread}. We will see how Typed Lua assigns types to
values of the first five in this section.

Figure~\ref{fig:typelang} gives the abstract syntax of Typed Lua
types. Only {\em first-class types} correspond to actual Lua
values; {\em second-class types} correspond to expression lists,
and Typed Lua uses them to type multiple assignment and function
application. Types are ordered by a subtype relationship, where
any first-class type is a subtype of $\Top$.

The {\em dynamic type} $\Any$ allows dynamically typed code to
interoperate with statically typed code; it is a subtype of $\Top$, but
neither a supertype nor a subtype of any other type. We relate $\Any$
to other types with the {\em consistency} and {\em consistent-subtype}
relationships used by gradual type systems. In practice, we can
pass a value of the dynamic type anytime we want a value of some
other type, and can pass any value where a value of the dynamic type
is expected, but these operations are tracked by the type system,
and the programmer can choose to be warned about them.

\begin{figure}[!ht]
\textbf{Type Language}\\
\dstart
$$
\begin{array}{rlr}
\multicolumn{3}{c}{\textbf{First-class types}} \\
T ::= & \;\; L & \textit{literal types}\\
& | \; B & \textit{base types}\\
& | \; \Top & \textit{top type}\\
& | \; \Any & \textit{dynamic type}\\
& | \; T \cup T & \textit{disjoint union types}\\
& | \; S \rightarrow S & \textit{function types}\\
& | \; \{F, ..., F\} & \textit{table types}\\
& | \; X & \textit{type variables}\\
& | \; \mu X.T & \textit{recursive types}\\
& | \; X_i & \textit{projection types}\\
L ::= & \multicolumn{2}{l}{\False \; | \; \True \; | \; {<}{\it number}{>} \; | \; {<}{\it string}{>} } \\
B ::= & \multicolumn{2}{l}{\Nil \; | \; \Boolean \; | \; \Number \; | \; \String} \\
F ::= & T:T \; | \; \Const \; T:T & \textit{field types} \\
\multicolumn{3}{c}{} \\
\multicolumn{3}{c}{\textbf{Second-class types}} \\
S ::= &  \multicolumn{2}{l}{\Void \; |\; V \; | \; S \cup S} \\
V ::= & \;\; T & \\
& | \; T* & \textit{vararg types}\\
& | \; T \times V & \textit{tuple types}
\end{array}
$$
\dend
\caption{Abstract syntax of Typed Lua types}
\label{fig:typelang}
\end{figure}

Typed Lua allows optional type annotations in variable function
declarations. It assigns the dynamic type to any parameter that
does not have an type annotation, but assigns more precise
types to unnanotated variables, based on the type of the
expression that gives the initial value of the variable.

In the following example, we use type annotations in a function
declaration but do not use type annotations in the declaration
of a local variable:
\begin{verbatim}
    local function factorial(n: number): number
      if n == 0 then
        return 1
      else
        return n * factorial(n - 1)
      end
    end
    local x = 5
    print(factorial(x))
\end{verbatim}

The compiler assigns the type $\Number$ to the local variable \texttt{x},
and this example compiles without warnings. Typed Lua allows
programmers to combine annotated code with
unannotated code, as we show in the following example:
\begin{verbatim}
    local function abs(n: number)
      if n < 0 then
        return -n
      else
        return n
      end
    end

    local function distance(x, y)
      return abs(x - y)
    end
\end{verbatim}

The compiler assigns the dynamic type $\Any$ to the input
parameters of \texttt{distance} because they do not have type annotations.
Subtracting a value of type $\Any$ from another also yields a value of
type $\Any$ (Lua has operator overloading, so the minus operation
is not guaranteed to return a number in this case), but
consistent subtyping lets us pass a value of type $\Any$ to a
function that expects a $\Number$. 

Even though the return types
of both {\tt abs} and {\tt distance} are not given, the compiler is able to
infer a return type of $\Number$ to both functions, as they are
local and not recursive.

Lua has first-class functions, but they have some peculiarities. First,
the number of arguments passed to a function does not need to
match the function's arity; Lua silently drops extra arguments after
evaluating them, or passes {\tt nil} in place of any missing arguments.
Second, functions can return any number of values, and the number
of values returned may not be statically known. Third, Lua also has
multiple assignment, and the semantics of argument passing match
those of multiple assignment (or vice-versa); calling a function is like
doing a multiple assigment where the left side is the parameter list
and the right side is the argument list. 

Typed Lua uses {\em second-class types} to encode the peculiarities
of argument passing, multiple returns, and multiple assignment. We call
them second-class because these types do not correspond to actual
values and cannot be assigned to variables or parameters: they are an
artifact of the interaction between the type system and the semantics of Lua.

As we saw in Figure \ref{fig:typelang}, a second-class type in
Typed Lua can be the type $\Void$ (the empty tuple),
a tuple of first-class types optionally ending in a variadic type,
or a union of these tuples. A variadic type $T*$ is a generator for a
sequence of values of type $T \cup \Nil$. Unions of tuples play
an important part in functions that are overloaded on the return type,
together with {\em projection types}. Both are explained in the next
section.

Typed Lua always adds a
variadic tail to the parts of a function type if none is specified, to match
the semantics of Lua function calls. In the examples above, the types
of {\tt factorial} and {\tt abs} are actually $\Number \times \Top *
 \rightarrow \Number \times \Nil *$, and the type of {\tt distance} is
$\Any \times \Any \times \Top * \rightarrow \Number \times \Nil *$.

If we call {\tt abs} with extra arguments, Typed Lua silently ignores
them, as the type signature lets {\tt abs} receive any number of extra
arguments. If we call {\tt abs} in the right side of an assignment to
more than one lvalue, Typed Lua checks if the first lvalue has a type
consistent with $\Number$, and any other lvalues need to have a type
consistent with $\Nil$.

A variadic type can only appear in the tail position of a tuple,
because Lua takes only the first value of any expression that appears
in an expression list that is not in tail position. The following example
shows the interaction between multiple returns and expression lists:

\begin{verbatim}
    local function multiple()
      return 2, "foo"
    end

    local function sum(x: number, y: number)
      return x + y
    end

    local x, y, z = multiple(), multiple()
    print(sum(multiple(), multiple())
\end{verbatim}

Function {\tt multiple} is
$\Top * \rightarrow \Number \times \String \times \Nil *$,
and {\tt sum} is $\Number \times \Number \times \Top * \rightarrow
\Number \times \Nil *$. In the right side of the multiple assignment,
only the first value produced by the first call to multiple gets used, so
the type of the right side is $\Number \times \Number \times \String \times \Nil*$,
and the types assigned to {\tt x}, {\tt y}, and {\tt z} are respectively $\Number$,
$\Number$, and $\String$. This also means that the call to {\tt sum} compiles
without errors, as the first two components of the tuple are consistent with
the types of the parameters, and the other components are consistent with
$\Top$.

\section{Unions}
\label{sec:unions}

Typed Lua uses union types to encode some common Lua idioms:
optional values, overloading based on the tags of input parameters,
and overloading on the return type of the functions.

Optional values are unions of some type and $\Nil$, and are so
common that Typed Lua uses the {\tt t?} syntax for these unions.
They appear any time a function has optional parameters, and
any time the program reads a value from an array or map.

\begin{verbatim}
    local function message(name: string,
                           greeting: string?)
      local greeting = greeting or "Hello "
      return greeting .. name
    end
    
    print(message("Lua"))
    print(message("Lua", "Hi"))
\end{verbatim}

In this example, the second parameter is optional but, in the first
line of the function, we declare a new variable that is guaranteed to
have type $\String$ instead of $\String \cup \Nil$. In Lua, any value
except {\tt nil} and {\tt false} are "truthy", so the short-circuiting
{\tt or} operator is a common way of giving a default value to an
optional parameter. Typed Lua encodes this idiom with a typing rule: if
the left side of {\tt or} has type $T \cup Nil$ and the right side
has type $T$ then the {\tt or} expression has type $T$.

Declaring a new {\tt greeting} variable that shadows the parameter
is not necessary:

\begin{verbatim}
    local function message(name: string, 
                           greeting: string?)
      greeting = greeting or "Hello "
      return greeting .. name
    end
\end{verbatim}

Typed Lua lets assignment change the type of a local variable in
cases that respect a simple heuristic: the new type must be a
subtype of the previous one, and the variable must be local
to the current function. The new type only applies for the
remainder of the current scope.

Overloaded functions use the {\tt type} function to inspect
the tag of their parameters, and perform different actions
depending on what those tags are. The simplest case overloads
on just a single parameter:

\begin{verbatim}
    local function overload(s1: string, 
                            s2: string|number)
      if type(s2) == "string" then
        return s1 .. s2
      else
        -- string.rep: (string, number) -> string
        return string.rep(s1, s2)
      end
    end
\end{verbatim}

Typed Lua has a small set of type predicates that, when used
over a local variable in a condition, constrain the type of that
variable. The function above uses the {\tt type(X) == "string"}
predicate which constrains the type of $X$ from $T \cup \String$ to
$\String$ when the predicate is true and $T$ otherwise. This is
a simplified form of {\em flow typing}~\cite{guha:flow}.

The type predicates can only discriminate based on tags, so they
are limited on the kinds of unions that they can discriminate. It
is possible to discriminate a union that combines a table type with
a base type, or a table type with a function type, or a two base types,
but it is not possible to discriminate between two different function
types.

Functions that overload their return types to signal the ocurrence
of errors are another common Lua idiom. In this idiom, a function
returns its normal set of return values in case of success but,
if anything fails, returns {\tt nil} as the first value, followed
by an error message or other data describing the error, as in the
following example:

\begin{verbatim}
    local function idiv(d1: number, d2: number):
          (number, number)|(nil, string)
      if d2 == 0 then
        return nil, "division by zero"
      else
        local r = d1 % d2
        local q = (d1 - r)/d2
        return q, r
      end
    end
\end{verbatim}

There is also special syntax for this idiom: we could
annotate the return type of {\tt idiv} with {\tt (number, number)?}
to denote the same union\footnote{The parentheses are always
necessary here: {\tt number?} is {\tt number|nil}, while {\tt (number)}
is {\tt (number)|(nil, string)}.}. 

The full type of {\tt idiv} is $\Number \times \Number \times
\Top * \rightarrow (\Number \times \Number \times \Nil *) \cup
(\Nil \times \String \times \Nil *)$. A typical client of this
function would use it as follows:

\begin{verbatim}
    local q, r = idiv(n1, n2)
    -- q is number|nil, r is number|string
    if q then
      -- q and r are numbers
    else
      -- r is a string
    end
\end{verbatim}

When Typed Lua encounters a union of tuples in the right side
of an declaration, it stores the the union in a special type
environment with a fresh name and assigns {\em projection types} 
to the variables in the left side of the declaration. If the type
variable is $X$, variable {\tt q} gets type $X_1$ and variable
{\tt r} gets type $X_2$.

If we need to check a projection type
against some other type, we take the union of the corresponding
component in each tuple. But if we a variable with a projection
type appears in a type predicate, the predicate discriminates
against all tuples in the union. In the example above,
$X$ is $(\Number \times \Number \times \Nil *) \cup
(\Nil \times \String \times \Nil *)$ outside of the {\tt if}
statement, but $\Number \times \Number \times \Nil *$ in the
{\tt then} block and $\Nil \times \String \times \Nil *$ in
the {\tt else} block.

Notice that we could also discriminate using {\tt r}, using
{\tt type(r) == "number"} as our predicate, with the same
result. The first form is more succint, and more idiomatic.
We can also use projection types to write overloaded functions
where the type of a parameter depends on the type of another
parameter.

Assigning to a variable with a projection type is forbidden,
unless the union has been discriminated down to a single tuple,
Unrestricted assignment to these variables would be unsound,
as it could break the dependency relation between the types
in each tuple that is part of the union.

Currently, a limitation of our overloading mechanisms is that
the return type cannot depend on the input types; we cannot
write a function that is guaranteed to return a number if
passed a number and guaranteed to return a string if passed
a string, for example.

\section{Tables and Interfaces}
\label{sec:tables}

Tables are the main mechanism that Lua has to build data
structures. They are associative arrays where any value
(except {\tt nil}) can be a key, but with language support
for efficiently using tables as tuples, arrays (dense or sparse),
records, modules, and objects. In this section,
we show how Typed Lua encodes tables as arrays, records, tuples,
and plain maps in its type system.

Typed Lua uses the same framework to represent the different
uses that a Lua table has: {\em table types}. A table type
$\{ t_{1}:u_{1}, \ldots, t_{n}:u_{n}\}$ represents a map
from values of type $t_i$ to values of type $u_i$.

The concrete syntax of Typed Lua has syntax for
common table types. One syntax defines table types for
maps: it is written \texttt{\{ t: u \}},
and maps to the table type $\{t:u\}$.
This table type represents a map that maps values of type
$t$ to values of type $u$.
Another syntax defines table types for arrays:
it is written \texttt{\{ t \}}, and maps to the table type
$\{\Number:t\}$. A third syntax sugar defines
table types for records:
it is written \texttt{\{ s1: t1, ..., sn: tn \}}, where
each $s_i$ is a literal number, string, or boolean, 
and maps to the table type $\{s_{1}:t_{1}, ..., s_{n}:t_{n}\}$,
where each $s_i$ is the corresponding literal type.

The example below shows how we can define a map from
strings to numbers; the dot syntax for field access
in Lua is actually syntactic sugar for indexing a table
with a string literal:

\begin{verbatim}
    local t: { string: number } = { foo = 1 }
    local x: number = t.foo         -- x gets 1
    local y: number = t["bar"]      -- runtime error
\end{verbatim}

When accessing a map, there is always the possibility that
the key is not there. In Lua, accessing a non-existing key
returns {\tt nil}. Typed Lua is stricter, and raises a runtime
error in this case. To get a map with the behavior of standard
Lua tables, the programmer can use an union:

\begin{verbatim}
    local t: { string: number? } = { foo = 1 }
    local x: number = t.foo         -- error
    local y: number = t.bar or 0    -- y gets 0
    local z: number? = t["bar"]     -- z gets nil
\end{verbatim}

Now the Typed Lua compiler will complain about the assignment
on line two. The following example shows how we can declare
an array:
\begin{verbatim}
    local days: { string } = { "Sunday", "Monday",
      "Tuesday", "Wednesday", "Thursday",
      "Friday", "Saturday" }
    local x = days[1]          -- x gets "Sunday"
    local y = days[8]          -- runtime error
\end{verbatim}

Notice that we have the same strictness with missing
elements, unless the type of the elements has {\tt nil}
as a possible value.

If we want to declare a tuple, we can leave the
variable declaration unannotated and let Typed Lua assign
a more specific table type to the variable.
If we remove the annotation in the previous example, 
the compiler assigns the following table type to \texttt{days}:
\begin{align*}
\{{1:\String},\;{2:\String},\;{3:\String},\;{4:\String},\;\\
{5:\String},\;{6:\String},\;{7:\String}\}
\end{align*}

This type is not a subtype of $\{\Number:\String\}$, nor
is $\{\Number:\String\}$ a subtype of $\{\Number:
\String\cup\Nil\}$, because in both cases the subtype
relationship would be unsound. In the first case,
a table type with the same fields as the type above,
plus $8: \Number$, is a subtype of the table type
above, so would also be a subtype of $\{\Number:\String\}$,
which is clearly unsound. In the second case, covariance
in the type of mutable fields is also unsound, for the
same reason as the unsoundness of array covariance.

While the record type above, $\{\Number:\String\}$, and
$\{\Number: \String\cup\Nil\}$ are disjoint, all three
types are valid for the table constructed in the example.
Typed Lua actually assigns different types to a table constructor
expression depending on the context where it is used.

Finally, the next example shows how we can declare a record:

\begin{verbatim}
    local person: { "firstname": string,
                    "lastname": string } =
      { firstname = "Lou", lastname = "Reed" } 
\end{verbatim}

We could leave the type annotation out, and Typed Lua would
assign the same type to {\tt person}.

As records get bigger, and types of record fields get more
complicated, writing table types can be unwieldy, so Typed
Lua has {\em interfaces} as syntactic sugar for record types:

\begin{verbatim}
    local interface Person
      firstname: string
      lastname: string
    end
\end{verbatim}

The declaration above declares {\tt Person} as an alias to
the record type $\{$``firstname": $\String$,
 ``lastname": $\String\}$ in
the remainder of the current scope. We can now use {\tt Person} in
type declarations:

\begin{verbatim}
    local function greet(person: Person)
      return "Hello, " .. person.firstname ..
             " " .. person.lastname
    end

    local user1 = { firstname = "Lewis",
                    middlename = "Allan",
                    lastname = "Reed" }
    local user2 = { firstname = "Lou" }
    local user3 = { lastname = "Reed",
                    firstname = "Lou" }
    local user4 = { "Lou", "Reed" }

    print(greeter(user1)) -- Hello, Lewis Reed
    print(greeter(user2)) -- Error
    print(greeter(user3)) -- Hello, Lou Reed
    print(greeter(user4)) -- Error
\end{verbatim}

If our record type has fields that can be {\tt nil}, we need
to use an explicit type declaration when declaring a
variable of this record type, as the following example shows:

\begin{verbatim}
    local interface Person
      firstname: string
      middlename: string?
      lastname: string
    end

    local user1: Person = { firstname = "Lewis",
                            middlename = "Allan",
                            lastname = "Reed" }
    local user2: Person = { lastname = "Reed",
                            firstname = "Lou" }
\end{verbatim}

We need an explicit type declaration because neither of
the types that the Typed Lua compiler assigns to the
table constructors above is a subtype of $\{$``firstname":
$\String$, ``middlename": $\String \cup \Nil$, ``lastname":
$\String\}$, the type that {\tt Person} describes.

We can also use interfaces to define recursive types:

\begin{verbatim}
    local interface Element
      info: number
      next: Element?
    end
\end{verbatim}

It is common in Lua programs to build a record incrementally,
starting with an empty table, as in the following example:

\begin{verbatim}
    local person = {}
    person.firstname = "Lou"
    person.lastname = "Reed"
\end{verbatim}

Ideally, we want the type of {\tt person} to change as the
table gets built, from $\{\}$ to $\{$``firstname": $\String\}$
and finally to $\{$``firstname": $\String$, ``lastname":
 $\String\}$. This is tricker than the type change introduced
by assignment that we saw in Section 2.2, as what is changing
is not just the type of the variable {\tt person} but the
type of the value that {\tt person} points to. This is safe
in the example above, but not in the example below:

\begin{verbatim}
    local bogus = { firstname = 1 }
    local person: {} = bogus
    person.firstname = "Lou"
    person.lastname = "Reed"
\end{verbatim}

The assignment on line two is perfectly legal, as the type
of {\tt bogus} is a subtype of $\{\}$. But changing the type
of {\tt person} would be unsound: {\tt person.firstname} is
now a $\String$, but {\tt bogus.firstname} is still typed
as a $\Number$. We do not even need to declare a type for
aliasing to be a problem: 

\begin{verbatim}
    local person = {}
    local bogus = person
    bogus.firstname = 1
    person.firstname = "Lou"
    person.lastname = "Reed"
\end{verbatim}

Taken individually, the changes to the type of the two
variables look ok, but aliasing makes one of them unsound.
The location of the change also matters, as the next example
shows:

\begin{verbatim}
    local person = {}
    local bogus = { firstname = 1 }
    do
      person.firstname = 1
      bogus = person
    end
    do
      person.firstname = "Lou"
    end
    -- bogus.firstname is now "Lou"
\end{verbatim}

The initial type of {\tt person} in all of these examples is
$\{\}$, but the {\em origin} of this type judgment matters
on whether it is sound to allow a change to the type of
{\tt person} or not, even if the change is always towards
a subtype of the current type.

Typed Lua tags a variable with a table type as either
{\em open} or {\em closed}. If a variable gets its type
from a table constructor then it is open, otherwise it is
always closed. The type of an open variable may change
by field assignment, subject to two restrictions: the
variable must be local to the current block, and the new type
must be a subtype of the old type.

Using a variable with an open type can also trigger a
type change, if the type of the missing fields has
{\tt nil} as a possible value. This lets the programmer
incrementally create an instance of an interface with
an optional type:

\begin{verbatim}
    local interface Person
      firstname: string
      middlename: string?
      lastname: string
    end

    local user = {}
    user.firstname = "Lou"
    user.lastname = "Reed"
    local person: Person = user
\end{verbatim}

Table types are the foundation for modules and objects in
Typed Lua. Type changes triggered by field assignment are
also an important part of Typed Lua's support for the
idiomatic definition of Lua modules, which are the
subject of the next section.

\section{Modules}
\label{sec:modules}

Lua's module system, like other parts of the language, is
a matter of convention. When Lua first needs to load a
module, it executes the module's source file as a function;
the value that this function returns is the module, and Lua
caches it for future loads. While a module can be any
Lua value, most modules are tables where the fields
of the table are functions and other values that the
module exports.

The modules that we surveyed build this table using
three distinct styles. In the first style, the module's
source file ends
with a {\tt return} statement that uses a table constructor
with the exported members. In the second style, the
module declares an empty table in the beginning of its
source file, adds exported members to this table
throughout the module, and returns this table at the end.

These two styles are straightforward for Typed Lua, which
can just take the type of the first value that the module
returns and use it as the type of the module.

In the third style, which has been deprecated in the current
version of Lua, a module begins with a call to the {\tt module}
function. This function installs a fresh table as the
global environment for the rest of the module, so any
assignments to global variables are field assignments to
this table. The {\tt module} function also sets an {\tt \_M}
field in this table as a circular reference to the table
itself, so the module can end with {\tt return \_M}, but
this explicit return is not necessary.

While this style has been deprecated, our survey indicated
that around a third of Lua modules in a popular module
repository still use this style, so Typed Lua also
supports this style: it treats accesses to global
variables as field accesses to an open table
in the top-level scope.

\section{Objects and Classes}
\label{sec:classes}

Lua's built-in support for object oriented programming is
minimal. The basic mechanism is the {\tt :} syntactic
sugar for method calls and method declarations.
The Lua compiler translates {\tt obj:method(args)}
to an operation that evaluates {\tt obj}, looks
for a field named ``method" in the result, then calls
it with the result of evaluating {\tt obj} as the
first argument, followed by the result of evaluating
the argument list in the original expression.

We can use recursive table types to represent objects,
and Typed Lua has syntactic sugar to make defining these
types easier:

\begin{verbatim}
    interface Shape
      x, y: number
      const move: (dx: number, dy: number) => ()
    end
\end{verbatim}

The double arrow in the type of the two methods is
syntactic sugar for having a first parameter named
{\tt self} with the same type as the interface. The
{\tt const} qualifier is necessary for covariance
in the types of the methods, and to make subtyping
among object types work.

Type-checking a method call is straightforward, but
there is still the matter of how to construct a value with
the object type above. The following example shows one
way:

\begin{verbatim}
    local shape = { x = 0, y = 0 }
    const function shape:move(dx: number, 
                              dy: number)
      self.x = self.x + dx
      self.y = self.y + dy
    end
\end{verbatim}

The {\tt :} syntactic sugar that the example uses also comes
from Lua, and assigns a function to the field with a first
parameter named {\tt self}, plus any other parameters.

Lua has a mechanism for Self-like delegation of missing
fields in a table. After we do {\tt setmetatable(t1,
{ \_\_index = t2 })}, Lua looks up in {\tt t2} any missing
fields of {\tt t1}. Lua programmers often use this mechanism
to simulate classes, as in the following example:

\begin{verbatim}
    local Shape = { x = 0, y = 0 }
    const function Shape:new(x: number, y: number)
      local s = setmetatable({},
                             { __index = self })
      s.x = x
      s.y = y
      return shape
    end
    const function Shape:move(dx: number,
                              dy: number)
      self.x = self.x + dx
      self.y = self.y + dy
    end
    local shape1 = Shape:new(0, 5)
    local shape2: Shape = Shape:new(10, 10)
\end{verbatim} 

In the last line of the example, notice how we can refer
to {\tt Shape} in the type annotation, as a shortcut
to the recursive table type that Typed Lua has assigned
to this variable.

Typed Lua assigns the type
$\mathbf{Shape} \times \Number \times \Number \times
\Top* \rightarrow \mathbf{Shape} \times \Nil*$ to
{\tt new}. In a {\tt setmetatable} expression,
if the type $t_1$ of the first operand is open and a supertype
of the type $t_2$ of the second operand's {\_\_index} field,
it changes the type of the first operand to $t_1$, and
it remains open.

As the result of {\tt setmetatable} can be an open
table type, we can simulate single inheritance, and
override methods:

\begin{verbatim}
    local Circle = setmetatable({},
                                { __index = Shape })
    Circle.radius = 0
    const function Circle:new(x: number,
                              y: number,
                              radius: number)
      local c = setmetatable(Shape:new(x, y),
                             { __index = self })
      c.radius = radius
      return c
    end
    const function Circle:area()
      return math.pi * self.radius * self.radius
    end
\end{verbatim} 

In the first line of the redefinition of {\tt new},
notice how we can call {\tt Shape}'s constructor inside
the overriden constructor. A limitation of this class
system is that the overriden constructors must be a
subtype of the original constructor.

If we erase all type and {\tt const} annotations, the
two examples above are valid Lua code, with the same
semantics as the Typed Lua Code.

The current version of Typed Lua does not have a polymorphic
type system, so programmers currently cannot hide the
calls to {\tt setmetatable} behind nicer abstractions, as
some Lua libraries do. A few Lua programs also use
other features of {\tt setmetatable} which are currently
not typeable, such as operator overloading.

\section{Related Work}
\label{sec:review}

Common LISP introduced optional type annotations in the early
eighties~\citep{steele1982ocl}, but they were optimization
hints to the compiler instead of types for static checking.
These annotations were unsafe, and could crash the program
when wrong.

\citet{abadi1989dts} used tagged pairs and explicit
injection and projection operations (coercions) to embed dynamic
typing in the simply-typed lambda calculus. Dynamically-typed
values had a {\tt Dynamic} static type. \citet{thatte1990qst}
removes the necessity of explicit coercions with a system
that automatically inserts coercions and checks them for
correctness.

\textit{Soft typing}~\citep{cartwright1991soft} starts
with a dynamically-typed language, and layers a static
type system with a sophisticated global type inference
algorithm on top, to try to find errors in programs
without needing to rewrite them. In cases where an error
may or may not be present, it warns the programmer and
inserts a runtime check. One problem with the soft
typing approach was the complexity of the inferred
types, leading to errors that were difficult to understand
and fix.

\textit{Dynamic typing}~\citep{henglein1994dts} is
another approach for optimizing dynamically-typed programs.
First the program is translated to a program that uses
a {\tt Dynamic} type and explicit coercions and runtime checks,
then a static analysis removes some  of these coercions and
checks.

Instead of trying to add static checking to a dynamic
language, \citet{findler2002chf} enhances
the dynamic checks with the possibility of {\em contracts}
that give assertions about the input and output of (possibly
higher-order) functions. In case of higher-order functions,
the actual failing check can be far away from the actual
source of the error, so contracts can also add {\em blame
annotations} to values as a way to trace failures back to
the source.

Strongtalk \citep{bracha1993strongtalk,bracha1996strongtalk} is
an optionally-typed version of Smalltalk. It has a
polymorphic structural type system that programmers can use
to annotate Smalltalk programs, but type annotations can be
left out; unnanotated terms are dynamically typed, and can
be cast to any static type. The interaction of the dynamic
type with the rest of the type system is unsound, so
Strongtalk uses the dynamically-checked semantics of
Smalltalk when executing programs, even if the programs
are statically typed.

Pluggable type systems~\citep{bracha2004pluggable} generalize
the idea of Strongtalk, to have type systems that can be
layered on top of a dynamic language without influencing
its runtime semantics. These systems can be unsound in
themselves, or in their interaction with the dynamically
typed part of the language, without sacrificing runtime
safety, as the semantics of the language catch any
runtime errors caused by an unsound type system.

Dart~\citep{dart} and TypeScript~\citep{typescript} are
two recent examples of languages with optional type systems
in this style. Dart is an object-oriented language with
a semantics that is similar to Smalltalk's, while TypeScript
is an object-oriented extension of JavaScript. Dart has a
nominal type system, while TypeScript has a structural one,
but both type systems are designed on purpose with unsound
parts (such as covariant arrays in case of Dart, and
covariant function return types in case of TypeScript) to
increase programmer convenience. The interaction of
statically and dynamically typed code is also unsound.

\citep{tobin-hochstadt2006ims} shows how programs
in the untyped lambda calculus can be incrementally
translated to the simply typed lambda calculus, using
contracts to guarantee that the untyped part cannot
cause errors in the typed part, as a model on how
scripts in a dynamically typed language can be
incrementally translated to a statically typed
language. This approach has been realized in the
Typed Scheme (later Typed Racket) language~\citet{tobin-hochstadt2008ts}.

Gradual typing~\citep{siek2006gradual} combines the
optional type annotations of optional and pluggable
type systems with higher-order contracts. The gradual
type system is the simply typed lambda calculus enriched
with a {\em dynamic type} $?$, where a value with the
dynamic type can assume any type, and vice-versa.
A dynamic value that assumes a static type generates
a runtime check for a first-order value, or a wrapper
for a higher-order value. A static value that assumes
the dynamic type is tagged. The whole system is sound:
any runtime errors in a well-typed program must happen
in the dynamic parts. 

While both gradual typing and the approach of
\citep{tobin-hochstadt2006ims} have the goal of
having a sound interaction of dynamically and statically
typed code, they differ in the granularity, with the
gradual typing providing a finer-grained transition from
dynamically typed to statically typed code.

Gradual typing has been combined with subtyping in
a simple object calculus~\citet{siek2007objects}, with
a nominal, polymorphic type system~\citet{ina:gradual},
with a row polymorphism~\citet{tobin:gradual},
and with the polymorphic lambda calculus~\citet{ahmed2011bfa}.
Gradual typing also adopted blame tracking from
higher-order
contracts~\citet{siek2010blame,ahmed2001bfa,wadler2009wpc}.

While Typed Racket is the first fully-featured programming
language with a sound mixture of dynamic and static typing,
Gradualtalk~\citep{allende2013gts} is the first 
fully-featured language with a fine-grained gradual type
system. In Gradualtalk's case, the extra runtime checks
needed by gradual typing impose a big runtime
cost~\citep{allende2013cis}, and the programmer has the
option of turning off these checks, and downgrading
Gradualtalk to an optional type system.

Tidal Lock \citep{tidallock} is a prototype of another optional
type system for Lua. Compared with Typed Lua, Tidal Lock
has richer record types, and a more robust system of
incremental evolution of these record types, but it lacks
unions and support for expressing objects and classes.

\section{Conclusion} \label{sec:con}

\bibliographystyle{abbrvnat}
\bibliography{typedlua}

\end{document}
