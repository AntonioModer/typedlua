\documentclass[preprint]{sig-alternate}

\usepackage[utf8]{inputenc}
\usepackage{url}

\newcommand{\Any}{\mathbf{any}}
\newcommand{\Nil}{\mathbf{nil}}
\newcommand{\False}{\mathbf{false}}
\newcommand{\True}{\mathbf{true}}
\newcommand{\Boolean}{\mathbf{boolean}}
\newcommand{\Number}{\mathbf{number}}
\newcommand{\String}{\mathbf{string}}
\newcommand{\Void}{\mathbf{void}}

\def\dstart{\hbox to \hsize{\vrule depth 4pt\hrulefill\vrule depth 4pt}}
\def\dend{\hbox to \hsize{\vrule height 4pt\hrulefill\vrule height 4pt}}

\begin{document}

\conferenceinfo{Dyla}{'14, Edinburgh, UK} 

\title{Typed Lua: An Optional Type System for Lua}

\numberofauthors{3}

\author{
\alignauthor
André Murbach Maidl\\
  \affaddr{PUC-Rio}\\
  \affaddr{Rio de Janeiro, Brazil}\\
  \email{amaidl@inf.puc-rio.br}
\alignauthor
Fabio Mascarenhas\\
  \affaddr{UFRJ}\\
  \affaddr{Rio de Janeiro, Brazil}\\
  \email{fabiom@dcc.ufrj.br}
\alignauthor
Roberto Ierusalimschy\\
  \affaddr{PUC-Rio}\\
  \affaddr{Rio de Janeiro, Brazil}\\
  \email{roberto@inf.puc-rio.br}
}

\date{June 12th 2014}

\maketitle

\begin{abstract}
\end{abstract}

\category{D.3.3}
         {Programming Languages}
         {Language Constructs and Features}
\category{D.2.3}
         {Programming Languages}
         {Coding Tools and Techniques}
         [Object-oriented programming]

\terms{Languages,Verification}

\keywords{Type Systems, Lua}

\section{Introduction} \label{sec:intro}

Dynamically typed languages such as Lua avoid static types in favor of
simplicity and flexibility, because the absence of static types means
that programmers do not need to bother about abstracting types that
should be validated by a type checker.
Instead, dynamically typed languages use run-time \textit{type tags}
to classify the values they compute, so their implementation can use
these tags to perform run-time (or dynamic) type checking
\cite{pierce2002tpl}.

This simplicity and flexibility allow programmers to write code that
might require a quite complex type system to statically type check,
though it may also hide bugs that will be caught only after deployment
if programmers do not properly test their code.
In contrast, static type checking helps programmers detect many 
bugs during the development phase.
Static types also provide a conceptual framework that helps
programmers define modules and interfaces that can be combined to
structure the development of programs.

Thus, early error detection and better program structure are two
advantages of static type checking that can lead programmers to
migrate their code from a dynamically typed to a statically
typed language, when their simple scripts become complex programs
\cite{tobin-hochstadt2006ims}.
Dynamically typed languages certainly help programmers during the
beginning of a project, because their simplicity and flexibility
allow quick development and make easier to change code according to
changing requirements.
However, programmers tend to migrate from dynamically typed to
statically typed code as soon as the project has consolidated its
requirements, because the robustness of static types helps
programmers link requirements to abstractions.
This migration usually involves different languages that have
different syntax and semantics, which usually require a complete
rewrite of existing programs instead of incremental evolution from
dynamic to static types.

Ideally, programming languages should offer programmers the
option to choose between static and dynamic typing:
\textit{optional type systems} \cite{bracha2004pluggable} and
\textit{gradual typing} \cite{siek2006gradual} are two similar
approaches for blending static and dynamic typing in the same
language.
The aim of both approaches is to offer programmers the option
to use type annotations where static typing is needed and avoid them
where dynamic typing is sufficient.
The difference between these two approaches is the way they treat
run-time semantics.
While optional type systems do not affect the run-time semantics,
gradual typing uses run-time checks to ensure that dynamically typed
code does not violate the invariants of statically typed code.

Programmers and researchers sometimes use the term gradual typing for
meaning incremental evolution of dynamically typed code to statically
typed code.
For this reason, gradual typing may also refer to optional type
systems and other approaches that blend static and dynamic typing to
help programmers incrementally migrate from dynamic to static typing
without having to switch to a different language, though all these
approaches differ in the way they handle static and dynamic typing
together.
We use the term gradual typing for referring to the work of Siek and
associates \cite{siek2006gradual}.

In this work we propose the design and implementation of Typed Lua:
an optional type system for Lua that should be complex enough to
preserve some of the Lua idioms that programmers are already used to,
but that should also include new constructs that help programmers
structure Lua programs.

Lua is a small scripting language, so designing an optional type
system for Lua should shed some light on optional type systems for
scripting languages in general.
Lua provides mechanisms for organizing a program in modules and
writing object-oriented programs, but it does not set policy on how
these features should behave, due to its use as an embedded and
extension language.
Thus, implementing an optional type system for Lua should offer Lua
programmers one way to obtain most of the advantages of static typing
without compromising the simplicity and flexibility of dynamic typing.

So far, Typed Lua should be a strict superset of Lua that
provides optional type annotations, compile-time type checking, and
class-based object oriented programming through the definition of
classes, interfaces, and modules.
In practice, we should implement Typed Lua as a programming
language that extends Lua syntax to add optional type annotations
and standard constructions for structuring Lua code.
The compiler should use static types to perform compile-time type
checking, but should also allow Lua code to coexist with Typed Lua
code, and generate Lua code that should run in unmodified Lua
implementations.
We have an ongoing implementation of the Typed Lua compiler that is
available online\footnote{https://github.com/andremm/typedlua}.

Typed Lua intended use is as an application language, and we view
that policies for organizing a program in modules and writing
object-oriented programs should be part of the language and
enforced by its optional type system.
An application language is a programming language that helps
programmers develop applications from scratch until these
applications evolve to complex systems rather than just scripts.

We also believe that Typed Lua will help programmers give a more
formal documentation to already existing Lua code, as static types
are also a useful source of documentation in languages that provide
type annotations, because type annotations are always validated by
the type checker and therefore never get outdated.
Thus, programmers can use Typed Lua to define axioms about the
interfaces and types of dynamically typed modules.
We will enforce this point by using Typed Lua to statically type
the interface of the Lua standard libraries and other commonly used
Lua libraries, so our compiler can check their usage by Typed Lua
code.

Typed Lua will perform a very limited form of local type inference
\cite{pierce2000lti}, as static typing does not necessarily mean
that programmers need to insert type annotations in the code.
Several statically typed languages such as Haskell provide some
amount of type inference that automatically deduces the types of
expressions.
Still, Typed Lua will need a small amount of type annotations due
to the nature of its optional type system.

Typed Lua will not deal with code optimization, although another
important advantage of static types is that they help the compiler
perform optimizations and generate more efficient code.
However, we believe that the formalization of our optional type
system will be precise enough to aid optimization in some Lua
implementations.

We will use some of the ideas of gradual typing to formalize Typed Lua.
Even though Typed Lua is an optional type system and thus will not
include run-time checks between dynamic and static regions of the
code, we believe that using the foundations of gradual typing to
formalize our optional type system will allow us to include run-time
checks in the future.

In Section \ref{sec:review} we review the literature about static and
dynamic typing in the same language, which includes the discussion
about the particularities between optional type systems and gradual
typing.
In Section \ref{sec:proposal} we describe the reasons why Typed Lua
includes features of object-oriented programming and the status of
its design and implementation.
In Section \ref{sec:con} we outline our expected contributions.

\section{Blending static and dynamic typing} \label{sec:review}

We begin this section presenting a little bit of the history about
combining static and dynamic typing in the same language.
After that, we introduce optional type systems and gradual typing.
We end this section discussing why optional type systems and two
other approaches are often called gradual typing.

\subsection{A little bit of history}

Common LISP \cite{steele1982ocl} introduced optional type annotations
in the early eighties, but not for static type checking.
Instead, programmers may choose to declare types of variables as
optimization hints to the compiler, that is, type declarations are
just one way to help the compiler to optimize code.
These annotations are unsafe because they can crash the program
when they are wrong.

\cite{abadi1989dts} extended the simply typed lambda-calculus with the
\verb'Dynamic' type and the \verb'dynamic' and \verb'typecase'
constructs, with the aim to safely integrate dynamic code in
statically typed languages.
The \verb'Dynamic' type is a pair \verb'(v,T)' where \verb'v' is a
value and \verb'T' is the tag that represents the type of \verb'v'.
The constructs \verb'dynamic' and \verb'typecase' are explicit
injection and projection operations, respectively.
That is, \verb'dynamic' builds values of type \verb'Dynamic' and
\verb'typecase' safely inspects the type of a \verb'Dynamic' value.
Thus, migrating code between dynamic and static type checking requires
changing type annotations and adding or removing \verb'dynamic' and
\verb'typecase' constructs throughout the code.

The \textit{quasi-static} type system proposed by \cite{thatte1990qst}
performs implicit coercions and run-time checks to replace the
\verb'dynamic' and \verb'typecase' constructs that were proposed by
\cite{abadi1989dts}.
To do that, quasi-static typing relies on subtyping with a top type
$\Omega$ that represents the dynamic type, and splits type checking
into two phases.
The first phase inserts implicit coercions from the dynamic type to
the expected type, while the second phase performs what Thatte calls
\textit{plausibility checking}, that is, it rewrites the program to
guarantee that sequences of up-casts and down-casts always have a
common subtype.

\textit{Soft typing} \cite{cartwright1991soft} is another approach
to combine static and dynamic typing in the same language.
The main goal of soft typing is to add static type checking to
dynamically typed languages without compromising their flexibility.
To do that, soft typing relies on type inference for
translating dynamically typed code to statically typed code.
The type checker inserts run-time checks around inconsistent code and
warns the programmer about the insertion of these run-time checks,
as they mean the existence of potential type errors.
However, the programmer is free to choose between inspecting the
run-time checks and simply running the code.
Soft typing gives this choice to the programmer because it is one of
its key ideas, that is, type inference and static type checking should
not prevent the programmer from running inconsistent code.
One advantage of soft typing is the fact that the compiler for
softly typed languages can use the translated code to generate
more efficient code, as the translated code statically type checks.
One disadvantage of soft typing is that it can be cumbersome when
the inferred types are meaningless large types that just confuse the
programmer.

\textit{Dynamic typing} \cite{henglein1994dts} is an approach
that optimizes code from dynamically typed languages by eliminating
unnecessary checks of tags.
This work describes how to translate dynamically typed code to
statically typed code that uses a \verb'Dynamic' type.
The translation is done through a coercion calculus that uses type
inference to insert the operations that are necessary to type check
the \verb'Dynamic' type during run-time.
Although soft typing and dynamic typing may seem similar, they are not.
They are not similar because soft typing targets statically type
checking of dynamically typed languages for detecting programming
errors while the dynamic typing targets the optimization of dynamically
typed code through the elimination of unnecessary run-time checks.
In other words, soft typing sees code optimization as a side effect
that comes with static type checking.

\cite{findler2002chf} proposed contracts for higher-order functions
and blame annotations for run-time checks.
Contracts perform dynamic checking instead of static checking,
but deferring all verifications to run-time can lead to defects
that are difficult to fix, because run-time errors can show a
stack trace where it is not clear to programmers if the cause of a
certain run-time error is in application code or library code.
Even if programmers identify that the source of a certain run-time
error is in library code, they still may have problems to identify
if this run-time error is due to a violation of library's contract,
or due to a bug, when the library is poorly documented.
In this approach, programmers can insert assertions in the form of
contracts that check the input and output of higher-order functions.
Thus, \cite{findler2002chf} show how programmers can insert
assertions in the form of contracts that check the input and output
of higher-order functions; and they also show how to add blame
annotations in the generated code as a way to track assertion
failures back to the source of the error.

BabyJ \cite{anderson2002babyj} is an object-oriented language
without inheritance that allows programmers to incrementally annotate
the code with more specific types.
Programmers can choose between using the dynamically typed version
of BabyJ, when they do not need types at all, and the statically
typed version of BabyJ, when they need to annotate the code.
In the statically typed BabyJ, programmers can use the
\textit{permissive type} $*$ to annotate the parts of the code that
still do not have a specific type or the parts of the code that should
have dynamic behavior.
The type system of BabyJ is nominal, so types are either class names
or the permissive type $*$.
However, the type system does not use type equality or subtyping,
but the relation $\approx$ between two types.
The relation $\approx$ holds when both types have the same name or
any of them is the permissive type $*$.
Even though the permissive type $*$ is similar to the dynamic type
from previous approaches, BabyJ does not provide any way to add
implicit or explicit run-time checks.

\cite{ou2004dtd} specified a language that combines static types
with dependent types.
To ensure safety, the compiler automatically inserts coercions
between dependent code and static code.
The coercions are run-time checks that ensure static code does not
crash dependent code during run-time.

\subsection{Optional Type Systems}

Optional type systems \cite{bracha2004pluggable} are an approach for
plugging static typing in dynamically typed languages through
optional type annotations that are used to perform compile-time type
checking, though not influencing the original run-time semantics of
the language.
This means that optional type systems can be unsound without affecting
language safety, because the run-time semantics should still catch
type errors independently of the static type checking.
More precisely, optional type systems are unsound if they do not catch
some type errors, but we say that languages with unsound optional type
systems are safe because the semantics of these languages catch during
run-time the type errors that the optional type system did not catch.
Although optional type systems maintain the simplicity and
flexibility of dynamically typed languages, their static type
checking cannot prove the absence of all type errors when the optional
type system is unsound.

Optional type systems have an interesting relation to the untyped
lambda calculus and the typed lambda calculus.
The semantic rules of the unytped lambda calculus are the same of
typed lambda calculus, and the type system is actually optional and
serves only for discarding programs that may have undesired behaviors
\cite{bracha2004pluggable}.

Optional type systems increase program evolution and
allow language designers to experiment with
\textit{pluggable type systems} \cite{bracha2004pluggable}.
Pluggable type systems are different optional type systems for
checking different aspects of the program.

According to \cite{bracha2004pluggable}, optional type systems should
not include mandatory type inference too, because mandatory type
inference may decrease the expressiveness of the type system.
Instead, optional type systems should implement type inference as
an independent mechanism that can be turned on and off according to
the programmers' needs.

Unfortunately, there is no formalization on optional type systems,
so each language ends up implementing its optional type system on
its own way.
Now we will briefly discuss some languages that provide an informal
description of their optional type systems rather than a formal one.

Strongtalk \cite{bracha1993strongtalk,bracha1996strongtalk} is
a version of Smalltalk that comes with an optional type system that
does not include any type inference.

Dart \cite{dart} and TypeScript \cite{typescript} are new
languages that are designed with an optional type system, and both
use JavaScript as their code generation target because their main
purpose is web development.
In fact, Dart is a new class-based object-oriented language with
optional type annotations while TypeScript is a strict superset of
JavaScript that provides optional type annotations and class-based
object-oriented programming.
Both Dart and TypeScript perform type inference to deduce static
types for unannotated code.

\subsection{Gradual Typing}

The main goal of gradual typing \cite{siek2006gradual} is to allow
programmers to choose between static and dynamic typing in the same
language.
To do that, \cite{siek2006gradual} extended the simply typed
lambda-calculus with the dynamic type $?$.
In gradual typing, type annotations are optional, so an untyped
variable is syntactic sugar for a variable whose declared type is
the dynamic type $?$.
Under these circumstances, we view gradual typing as a way to add
a dynamic type to statically typed languages.

The central idea of gradual typing is the \textit{consistency}
relation, written $\tau_{1} \sim \tau_{2}$, that allows implicit
conversions to and from the dynamic type, and disallows conversions
between inconsistent types \cite{siek2006gradual}.
For instance, $\mathbf{int} \sim \,?$ and $?\, \sim \mathbf{int}$,
but $\mathbf{int} \not\sim \mathbf{string}$ and
$\mathbf{string} \not\sim \mathbf{int}$. 
The consistency relation is neither commutative nor transitive.

Gradual typing is similar to the approaches proposed by
\cite{abadi1989dts} and \cite{thatte1990qst} by including a
dynamic type to a statically typed language.
However, these three approaches differ in the way they handle the
dynamic type.
While \cite{siek2006gradual} rely on the consistency relation,
\cite{abadi1989dts} rely on type equality with explicit projections
and injections, and \cite{thatte1990qst} relies on subtyping.
The subtyping relation actually is a pitfall on Thatte's quasi-static
typing, because the dynamic type is put as the top and the bottom of
the subtyping relation.
Therefore, down-casts combined with the transitivity of subtyping
accepts programs that should be rejected.

Later, \cite{siek2007objects} discovered that the consistency relation
is orthogonal to subtyping, so we can combine the two relations to
achieve both consistency and subtyping.
The consistent-subtyping relation,
written $\tau_{1} \lesssim \tau_{2}$,
is essential to design object oriented languages that have a
gradual type system.
Like the consistency relation, and unlike the subtyping relation,
the consistent-subtyping relation is not transitive.

Another important feature of gradual typing is the theoretic
foundation that it provides for inserting run-time checks that
prove dynamically typed code does not violate the invariants of
statically typed code, thus preserving type safety.
To do that, \cite{siek2006gradual} defined the run-time semantics
of gradual typing as a translation to an intermediate language with
explicit casts at the frontiers between statically and dynamically
typed code.
The semantics of these casts is based on the higher-order contracts
proposed by \cite{findler2002chf}.

However, \cite{herman2007sgt} showed that there is an efficiency
concern regarding the run-time checks, because there are two
ways that casts can lead to unbounded space consumption.
The first affects tail recursion while the second appears when
first-class functions or objects cross the border between
static code and dynamic code, that is, some programs can apply
repeated casts to the same function or object.
\cite{herman2007sgt} use the coercion calculus of
\cite{henglein1994dts} to solve the problem of space efficiency,
that is, they use coercions to express casts.
Their approach normalizes an arbitrary sequence of coercions to a
coercion of bounded size.

Another concern about casts is how to improve debugging support,
because a cast application can be delayed and the error related
to that cast application can appear far way from the real error.
\cite{wadler2009wpc} developed the \textit{blame calculus} as a way
to attack this issue, and \cite{ahmed2011bfa} extended the
blame calculus with polymorphism.
The blame calculus is an intermediate language to integrate
static and dynamic typing that uses the blame tracking approach
proposed by \cite{findler2002chf}.

On the one hand, the blame calculus solves the issue regarding
error reporting;
on the other hand, it has the space efficiency problem reported
by \cite{herman2007sgt}.
Thus, \cite{siek2009casts} extended the coercion calculus of
\cite{herman2007sgt} with blame tracking to achieve an
implementation of the blame calculus that is space efficient.
After that, \cite{siek2010blame} proposed a new solution that also
attacks both problems.
This new solution is based on a concept called \textit{threesome},
which is a way to split a cast between two parties into two casts
among three parties.
A cast has a source and a target type (a \textit{twosome}),
so we can split any cast into a downcast from the source to an
intermediate type that is followed by an upcast from the intermediate
type to the target type (a \textit{threesome}).

ActionScript \cite{moock2007as3} is the first language that
incorporated a gradual type system to its implementation and
Perl 6 \cite{tang2007pri} is also being designed with a
gradual type system, though there is little documentation about
the gradual type systems of these languages.
Gradual Python \cite{reticulated} is an attempt to develop a
gradual type system for Python that deals with mutable objects
\cite{siek2013mutable}.
Gradualtalk \cite{allende2013gts} is a gradually-typed Smalltalk
that introduces a new cast insertion strategy for gradually-typed
objects \cite{allende2013cis}.

\subsection{Approaches that are often called Gradual Typing}

Gradual typing is similar to optional type systems in that type
annotations are optional, and unannotated code is dynamically
typed, but unlike optional type systems, gradual typing affects
the run-time semantics to preserve type safety, and it is a way to
add a dynamic type to statically typed languages. 
More precisely, programming languages that include a gradual type
system implement the semantics of statically typed languages, so
the gradual type system inserts casts in the translated code to
guarantee that types are consistent before execution, while
programming languages that include an optional type system still
implement the semantics of dynamically typed languages, so all
the type checking also belongs to the semantics of each operation.

Still, we can view gradual typing as a way to formalize an optional
type system when the gradual type system does not insert run-time
checks.
BabyJ \cite{anderson2002babyj} is an example of object-oriented
language that uses the relation $\approx$ to implement its optional
type system, and this relation is similar to the consistency relation
of gradual typing.
Alore \cite{lehtosalo2011alore} is an example of object-oriented
language that uses subtyping along with the consistency relation to
implement its optional type system.
The authors of Alore defined a \textit{consistent-or-subtype}
relation that is different to the \textit{consistent-subtyping}
relation of \cite{siek2007objects}, but it is also written
$\tau_{1} \lesssim \tau_{2}$.
Their relation holds when $\tau_{1} \sim \tau_{2}$ or $\tau_{1} <: \tau_{2}$,
so $<:$ is transitive and $\sim$ is not.

Hence, optional type annotations for program evolution probably are
the reasons why optional type systems are commonly called
gradual type systems.
Typed Clojure \cite{bonnaire-sergeant2012typed-clojure} is an
optional type system for Clojure that now is adopting the
gradual typing slogan.

\cite{flanagan2006htc} introduced \textit{hybrid type checking},
an approach that combines static types and refinement types.
For instance, programmers can specify the refinement type
$\{x:Int \;|\; x \ge 0\}$ when they need a type for natural numbers.
The programmer can also choose between explicit or implicit casts.
When casts are not explicit, the type checker uses a theorem prover
to insert casts.
In our example of natural numbers, a cast would be inserted to check
whether an integer is greater than or equal to zero.

Sage \cite{gronski2006sage} is a programming language that
extended hybrid type checking with a dynamic type to
support dynamic and static typing in the same language.
Sage also offers optional type annotations in the gradual typing
style, that is, unannotated code is syntactic sugar for
code whose declared type is the dynamic type.

Thus, the inclusion of a dynamic type to hybrid type checking
along with optional type annotations, and the insertion of run-time
checks probably are the reasons why hybrid type checking is
also viewed as a form of gradual typing. 

\cite{tobin-hochstadt2006ims} proposed another approach for gradually
migrating from dynamically typed to statically typed code,
and they coined the term \textit{from scripts to programs} for
referring to this kind of interlanguage migration.
In their approach, the migration from dynamically typed to
statically typed code happens module-by-module, so they designed
and implemented Typed Racket \cite{tobin-hochstadt2008ts} for
this purpose.
Typed Racket is a statically typed version of Racket
(a Scheme dialect) that allows the programmer write typed modules,
so Typed Racket modules can coexist with Racket modules,
which are untyped.

The approach used by \cite{tobin-hochstadt2008ts} to design and
implement Typed Racket probably is also called gradual typing
because it allows the programmer gradually migrate from untyped
scripts to typed programs.
However, note that Typed Racket is a statically typed language
and type annotations are mandatory.
So, what makes it gradual is a type system with a dynamic type
that handles the interaction between Racket and Typed Racket modules.

\section{Typed Lua} \label{sec:proposal}

We begin this section presenting statistics about the usage of Lua
features and idioms.
We collected statistics about how programmers use tables, functions,
dynamic type checking, object-oriented programming, and modules.
We shall see that these statistics informed important design decisions
on our optional type system.
We end this section introducing Typed Lua through examples, and
describing a few Lua projects that are similar to Typed Lua. 

\subsection{Statistics about the usage of Lua}

We used the LuaRocks repository to build our statistics database;
LuaRocks \cite{hisham2013luarocks} is a package manager for Lua
modules.
We downloaded the 3928 \verb'.lua' files that were available in
the LuaRocks repository at February 1st 2014.
However, we ignored files that are not compatible with Lua 5.2,
the latest version of Lua.
We also ignored \textit{machine-generated} files and test files,
because these files may not represent idiomatic Lua code,
and might skew our statistics towards non-typical uses of Lua.
This left 2598 \verb'.lua' files from 262 different projects for
our statistics database;
we parsed these files and processed their abstract syntax tree
to gather the statistics that we show in this section.

To verify how programmers use tables, we measured how they
initialize, index, and iterate tables.
We present these statistics in the next three paragraphs to discuss
their influence on our type system.

The table constructor appears 23185 times.
In 36\% of the occurrences it is a constructor that initializes a
record (e.g., \texttt{\{ x = 120, y = 121 \}});
in 29\% of the occurrences it is a constructor that initializes a
list (e.g., \texttt{\{ "one", "two", "three", "four" \}});
in 8\% of the occurrences it is a constructor that initializes a
record with a list part;
and in less than 1\% of the occurrences (4 times) it is a constructor
that uses only the booleans \verb'true' and \verb'false' as indexes.
At all, in 73\% of the occurrences it is a constructor that uses
only literal keys;
in 26\% of the occurrences it is the empty constructor;
in 1\% of the occurrences it is a constructor with non-literal keys
only, that is, a constructor that uses variables and function calls
to create the indexes of a table;
and in less than 1\% of the occurrences (19 times) it is a constructor
that mixes literal keys and non-literal keys.

The indexing of tables appears 130448 times:
86\% of them are for reading a table field while
14\% of them are for writing into a table field.
About the readings, 89\% of the reads use a literal string key,
4\% of the reads use a literal number key,
and less than 1\% of the reads (10 times) use a literal boolean key.
At all, 93\% of the reads use literals to index a table while
7\% of the reads use non-literal expressions to index a table.
It also worth mentioning that 45\% of the reads are actually
function calls.
More precisely, 25\% of the reads use literals to call a function,
20\% of the reads use literals to call a method, that is,
a function call that uses the colon syntactic sugar, 
and less than 1\% of the reads (195 times) use non-literal expressions
to call a function.
About the writings, 69\% of the writes use a literal string key,
2\% of the writes use a literal number key,
and less than 1\% of the writes (1 time) uses a literal boolean key.
At all, 71\% of the writes use literals to index a table while
29\% of the writes use non-literal expressions to index a table.

We measured how many files have code that iterate over tables to
observe how frequently iteration is used.
We observed that 23\% of the files have code that iterate over keys
of any value, that is, the call to \verb'pairs' appears at least once
in these files (the median is twice per file);
21\% of the files have code that iterate over integer keys, that is,
the call to \verb'ipairs' appears at least once in these files
(the median is also twice per file);
and 10\% of the files have code that use the numeric \verb'for'
along with the length operator (the median is once per file).

The numbers about table initialization, indexing, and iteration
show us that tables are mostly used to represent records, lists,
and associative arrays.
Therefore, Typed Lua should include a table type for handling
these uses of Lua tables.
Even though the statistics show that programmers initialize tables
more often than they use the empty constructor to
dynamically initialize tables, the statistics of the empty
constructor are still expressive and indicate that Typed Lua should
also include a way to handle this style of defining table types.

We found a total of 24858 function declarations in our database
(the median is six per file).
Next, we discuss how frequently programmers use dynamic type
checking and multiple return values inside these functions.

We observed that 9\% of the functions perform dynamic type checking
on their input parameters, that is, these functions use \verb'type'
to inspect the tags of Lua values (the median is once per function).
We randomly selected 20 functions to sample how programmers are
using \verb'type', and we got the following data:
50\% of these functions use \verb'type' for asserting the tags of
their input parameters, that is, they raise an error when the tag of a
certain parameter does not match the expected tag, and
50\% of these functions use \verb'type' for overloading, that is,
they execute different code according to the inspected tag.

These numbers show us that Typed Lua should include union types,
because the use of the \verb'type' idiom shows that disjoint unions
will help programmers define data structures that can hold a value of
several different, but fixed types.
Typed Lua should also use \verb'type' as a mechanism for decomposing
unions, though it may be restricted to base types only.

We observed that 10\% of the functions explicitly return multiple
values.
We also observed that 5\% of the functions return \verb'nil' plus
something else, for signaling an unexpected behavior;
and 1\% of the functions return \verb'false' plus something else,
also for signaling an unexpected behavior.

Typed Lua should include function types to represent Lua functions,
and tuple types to represent the signatures of Lua functions,
multiple return values, and multiple assignments.
Tuple types will require some special attention, because Typed Lua
should be able to adjust tuple types during compile-time, in a
similar way to what Lua does with function calls and multiple
assignments during run-time.
In addition, the numbers show us that overloading on the return
type will be useful to the type system.

We also measured how frequently programmers use the object-oriented
paradigm in Lua.
We observed that 23\% of the function declarations are actually
method declarations.
More precisely, 14\% of them use the colon syntactic sugar while
9\% of them use \verb'self' as their first parameter.
We also observed that 63\% of the projects extend tables with
metatables, that is, they call \verb'setmetatable' at least once,
and 27\% of the projects access the metatable of a given table,
that is, they call \verb'getmetatable' at least once.
In fact, 45\% of the projects extend tables with metatables and
declare methods:
13\% using the colon syntactic sugar, 14\% using \verb'self', and
18\% using both.

Based on these observations, Typed Lua should include interfaces and
classes as syntactic sugar for table types that encode objects.
Proposing an optional type system for Lua that includes classes may
sound unusual at first glance, because Lua does not have standard
policies for object-oriented programming.
However, Lua provides mechanisms that allow programmers to abstract
their code in terms of objects, and our statistics confirm that
an expressive number of programmers are relying on these mechanisms to
use the object-oriented paradigm in Lua.
Typed Lua should include interfaces and classes as merely syntactic
mechanisms that help the compiler type check object-oriented code,
and they should not influence the semantics of Lua.

We also measured how programmers are defining modules.
We observed that 38\% of the files use the current way of defining
modules, that is, these files return a table that contains the
exported members of the module at the end of the file;
22\% of the files still use the deprecated way of defining modules,
that is, these files call the function \verb'module';
and 1\% of the files use both ways.
At all, 61\% of the files are modules while 39\% of the files are
plain scripts.
The number of plain scripts is quite high considering the origin of
our database.
However, we did not ignore sample scripts, which usually serve to
help the users of a given module on how to use this module, and
that is the reason why we have a quite high number of plain scripts.

Based on these observations, Typed Lua should also use interfaces as
syntactic sugar for table types that define modules.
Typed Lua should also support the deprecated style of module definition,
using global names as exported members of the module.

Besides using interfaces for encoding objects and defining modules,
Typed Lua should also use interfaces for defining the types of
userdata and threads.

The last statistics that we collected were about variadic functions
and vararg expressions.
We observed that 8\% of the functions are variadic, that is,
their last parameter is the vararg expression.
We also observed that 5\% of the initialization of lists
(or 2\% of the occurrences of the table constructor) use solely the
vararg expression.
Typed Lua should include a \textit{vararg type} to handle variadic
functions and vararg expressions.

\subsection{Typed Lua through examples}

Our main goal is to allow programmers to combine static and dynamic
typing in Lua, so we propose Typed Lua as an extension to Lua 5.2
syntax that introduces optional type annotations, and class-based
object-oriented programming through the definition of classes,
interfaces, and modules.
An important requirement of Typed Lua is that it should be backwards
compatible with Lua, that is, it should allow programmers to run any
Lua code without modifications.
For this reason, these syntactic extensions include new keywords
that are not reserved words.

Figure \ref{fig:typelang} shows that Typed Lua includes includes
first-class and second-class types.
Typed Lua uses first-class types for representing Lua values and
second-class types for defining the signatures of functions.
First-class types include literal types, base types, the type $\Any$,
union types, function types, table types, and recursive types.
Second-class types include the type $\Void$, first-class types,
vararg types, and tuple types.
Literal types are types that represent a single value
(e.g., $\Nil$, $\False$, $\True$, literal numbers, and
literal strings).
Literal types are subtypes of their respective base types.
More precisely, the literals $\False$ and $\True$ are subtypes of
$\Boolean$, any literal number $n$ is subtype of $\Number$, and any
literal string $s$ is subtype of $\String$.
The type $\Any$ is the dynamic type that allows dynamic code coexist
with static code.
We will discuss union types, function types, table types, recursive
types, and second-class types along this section.

\begin{figure}[!ht]
\textbf{Type Language}\\
\dstart
$$
\begin{array}{rlr}
T ::= & & \textit{first-class types}\\
& \;\; L & \textit{literal types}\\
& | \; B & \textit{base types}\\
& | \; \Any & \textit{dynamic type}\\
& | \; T \cup T & \textit{union type}\\
& | \; S \rightarrow S & \textit{function type}\\
& | \; \{T:T, ..., T:T\} & \textit{table type}\\
& | \; X & \textit{type variable}\\
& | \; \mu X.T & \textit{recursive type}\\
L ::= & \;\; \Nil \; | \; \False \; | \; \True & \\
& | \; {<}{\it numbe}r{>} \; | \; {<}{\it string}{>} & \\
B ::= & \Boolean \; | \; \Number \; | \; \String & \\
S ::= & \Void \; | \; V & \textit{second-class types}\\
V ::= & \;\; T & \\
& | \; T* & \textit{vararg type}\\
& | \; T \times V & \textit{tuple type}
\end{array}
$$
\dend
\caption{Typed Lua types}
\label{fig:typelang}
\end{figure}

Typed Lua allows optional type annotations in variable and function
declarations.
Whenever it is possible to use local type inference, the compiler
assigns a more specific type to an unannotated declaration.
When it is not possible to use local type inference, the compiler
assigns the type $\Any$ to an unannotated declaration.
For instance, in function declarations, the compiler does not infer
a more specific type to unannotated input parameters, because,
in this case, global type inference often is undecidable in a
language with subtyping and overloading \cite{wells1999typability}.

In the following example we use type annotations in a function
declaration, and we do not use type annotations in a declaration of a
local variable:
\begin{verbatim}

    local function fact (n:number):number
      if n == 0 then
        return 1
      else
        return n * fact(n - 1)
      end
    end
    local x = 5
    print(fact(x))
\end{verbatim}

Even though we did not add type annotations in the declaration of
the local variable \verb'x', the compiler uses local type inference to
assign the type $\Number$ to \verb'x'.
The compiler does not use the literal type $5$ instead of the base
type $\Number$ because this would prevent
programmers from assigning other numeric values to the variable
\verb'x'.
The inference that we implement in Typed Lua is quite simple, as it
uses only the type of the local expression.

Typed Lua allows programmers to combine annotated code with
unannotated code, as we show in the following example:
\begin{verbatim}
    local function abs (n:number):number
      if n < 0 then
        return -n
      else
        return n
      end
    end

    local function dist (x, y)
      return abs(x - y)
    end
\end{verbatim}

The function \verb'dist' receives two parameters of type $\Any$
and returns a value of type $\Number$.
The compiler assigns the dynamic type $\Any$ to the input
parameters of \verb'dist' because they do not have type annotations,
and the compiler does not use global type inference, as we mentioned
previously.
The compiler infers the return type of \verb'dist' because it is
not a recursive function, that is, Typed Lua does not infer the return
type of recursive functions.

We are using the consistent-subtyping relation of gradual
typing to check the interaction among the dynamic type and the other
types.
More precisely, we can call \verb'abs' inside \verb'dist' because
\verb'x - y' is of type $\Any$, and the type $\Any$ is
consistent with the type $\Number$.
The type $\Any$ is not the union of all types, but it is a separate
type.

Unlike gradual typing, we do not insert run-time checks that
inspect the interaction between dynamically typed code and statically
typed code during run-time.
In other words, Typed Lua does not guarantee that dynamically typed
code does not violate statically typed code.
Typed Lua does not insert run-time checks because they can decrease
run-time performance \cite{allende2013cis}.
We believe that a careful evaluation of run-time checks should be done
before inserting them on the type system.
In the previous example, we cannot guarantee that \verb'dist' is never
going to call \verb'abs' with a parameter that is not a number,
because in the semantics of Lua the minus operator can result in a
value that is not a number.
Still, we can implement run-time checks in the future because we are
using the consistent-subtyping relation of gradual typing to
formalize our optional type system, and the type system is being
designed to be sound.

Typed Lua includes union types as a way to discriminate data
structures that can hold values of several different, but fixed types.
Next, we show an example:
\begin{verbatim}
    local x:number?            -- OK
    local y:number|string = 1  -- OK
    local z:number = x         -- Error
\end{verbatim}

The annotation \verb'number?' is syntactic sugar for
\verb'number|nil' that means the type $\Number\;\cup\;\Nil$.
Typed Lua includes the syntactic sugar \verb't?' because the
type $t\;\cup\;\Nil$ is very common in Lua code.

The last line of the previous example gives a type error because we
are attempting to assign a value of type $\Number\;\cup\;\Nil$
to a variable that accepts only values of type $\Number$.
Although we can solve this type error changing the annotation of
variable \verb'z' from \verb'number' to \verb'number?', there is
another way that we can use to solve this type error without changing
the annotation:
\begin{verbatim}
    local z:number = x or 0  -- OK
\end{verbatim}

Typed Lua has a typing rule for checking that
$t\;\cup\;\Nil\;\mathbf{or}\;t$ always has the type $t$.

Typed Lua includes table types to represent Lua's tables.
Although Typed Lua has just one table type constructor, it has three
different syntaxes for defining the type of a table.
The first syntax defines table types for hash tables;
it is written \verb'{t1:t2}',
and maps to the table type $\{t_{1}:t_{2}\;\cup\;\Nil\}$.
This table type represents a hash table that maps values of type
$t_{1}$ to values of type $t_{2}\;\cup\;\Nil$, as Lua returns
$\Nil$ when we use a non-existing key to index a table.
The second syntax defines table types for lists;
it is written \verb'{t}',
and maps to the table type $\{\Number:t\;\cup\;\Nil\}$.
This table type is just a syntactic sugar to the definition of a
hash table that maps values of type $\Number$ to values of type
$t\;\cup\;\Nil$.
The third syntax defines table types for records;
it is written \verb'{s1:t1, ..., sn:tn}',
and maps to the table type $\{s_{1}:t_{1}, ..., s_{n}:t_{n}\}$.
This table type represents a record that maps the literals
$s_{i}$\verb', ..., '$s_{n}$ to values of types
$t_{i}$\verb', ..., '$t_{n}$, where $i$ may vary from one to $n$.

Next, we show one example of table type to type a hash table:
\begin{verbatim}
    local t:{string:number} = { foo = 1 }  -- OK
    local x:number = t.foo                 -- Error
    local y:number? = t.bar                -- OK
\end{verbatim}

The second line of this example gives a type error for the same
reason that the last line of the previous example gave a type error
too, that is, we are trying to assign a value that can be of two
different types to a variable that accepts values of only one type.
Although the field \verb'bar' does not exist in \verb't', the last
line of this example does not give a type error because the
annotated type matches the type of the values that can be stored in
\verb't'.

Next, we show one example of table type to type a list:
\begin{verbatim}
    local days:{string} = { "Sunday", "Monday",
      "Tuesday", "Wednesday", "Thursday",
      "Friday", "Saturday" }  -- OK
\end{verbatim}

When we know that a list has fixed elements, we can leave the
variable declaration unannotated and let local type inference assign
a more specific table type to the list.
If we remove the annotation in this example, the compiler uses the
syntax of records to infer the following table type to \verb'days':
\begin{align*}
\{{1:\String},\;{2:\String},\;{3:\String},\;{4:\String},\;\\
{5:\String},\;{6:\String},\;{7:\String}\}
\end{align*}
This inferred table type is a subtype of the type
$\{\Number:\String\;\cup\;\Nil\}$, which is the type that we used to
annotate the variable \verb'days'.

When we use the inferred table type while initializing a table, the
compiler gives an error message when we try to access an index that
is out of bounds.

It also worth mentioning that we can use the style of the inferred
type to define heterogeneous tuples.

Next, we show one example of table type to type a record:
\begin{verbatim}
    local person:{"firstname":string,
                  "lastname":string} =
      { firstname = "Lou", lastname = "Reed" } 
\end{verbatim}

Type inference would infer the very same table type that we used to
annotate the variable \verb'person'.

Typed Lua includes interfaces as syntactic sugar to the table type
that specifies records.
For instance, we can use the following interface as an alias to the
type of the record we showed in the previous example:
\begin{verbatim}
    interface Person
      firstname:string
      lastname:string
    end
\end{verbatim}

Now we can use the type \verb'Person' as an alias to a table type
that certainly has the fields \verb'firstname' and \verb'lastname'.
Next, we show an example of code that uses the type \verb'Person':
\begin{verbatim}
    local function greeter (person:Person):string
      return "Hello " .. person.firstname ..
             " " .. person.lastname
    end

    local user1 = { firstname = "Lewis",
                    middlename = "Allan",
                    lastname = "Reed" }
    local user2 = { firstname = "Lou" }
    local user3 = { lastname = "Reed",
                    firstname = "Lou" }
    local user4 = { "Lou", "Reed" }

    print(greeter(user1)) -- OK
    print(greeter(user2)) -- Error
    print(greeter(user3)) -- OK
    print(greeter(user4)) -- Error
\end{verbatim}

This example shows that our optional type system is structural rather
than nominal, that is, it checks the structure of types instead of
their names, and it uses subtyping and consistent-subtyping for
checking types.

Using interfaces we can also define recursive types such as linked
lists:

\begin{verbatim}
    interface Element
      info:number
      next:Element?
    end
\end{verbatim}

We need to explicitly annotate \verb'next' being of type
${\tt Element}\;\cup\;\Nil$ because in our type system $\Nil$
is not the bottom type, that is, $\Nil$ is subtype of itself only.
Typed Lua has this behavior because in Lua $\Nil$ represents
the absence of an useful value.

Besides interfaces, Typed Lua will also introduce classes.
We want to offer classes as a standard way for writing code that
Lua programmers are already using, but without tying them to a
specific model.
Our classes will not include static attributes and will also not offer
privacy rules.

Typed Lua will also introduce a specific syntax for defining modules.
For now, each file is a module.
We use all global names and their types to build the interface that
represents the type of the module, in case the module is not returned
at the end of the file.

Typed Lua includes function types because functions are first-class
values in Lua.
The syntax for defining a function type is \verb'(s) -> (s)',
where \verb's' is a second-class type.
As we saw in Figure \ref{fig:typelang}, a second-class type in
Typed Lua can be the type $\Void$, a first-class type, a variadic
type, or a tuple type.
We use tuple types to represent multiple parameters, multiple return
values, and multiple assignments, and we use the variadic type to
represent variadic functions and the vararg expression.
Even though we already included the tuple type and the vararg type,
we are still working on these types.

The next example shows how we can annotate a \verb'map' function
that receives another function as its parameter and uses this
function to alter the elements of a list of numbers, which is
also a parameter of \verb'map':
\begin{verbatim}
    local function map (f:(number)->(number),
                        l:{number}):void
      for i=1, #l do
        l[i] = f(l[i] or 0)
      end
    end
\end{verbatim}

To conclude this section, we just started the design and
implementation of Typed Lua, and there is still several design
decisions to be taken on this subject.
For instance, our type system should handle the subtleties that
come with metatables.

\subsection{Similar Lua projects}

Metalua \cite{metalua} is a Lua compiler that supports compile-time
metaprogramming (CTMP).
CTMP is a kind of macro system that allows the programmers to interact
with the compiler \cite{fleutot2007contrasting}. 
Metalua extends Lua 5.1 syntax to include its macro system,
and allows programmers to define their own syntax.
Metalua can provide syntactical support for several object-oriented
styles, and can also provide syntax for turning simple type
annotations into run-time assertions, but without compile-time type
checking.

MoonScript \cite{moonscript} is a programming language that supports
class-based object-oriented programming, and compiles to idiomatic
Lua code.
However, it does not perform compile-time type checking.

LuaInspect \cite{luainspect} is a tool that uses MetaLua to perform
some code analysis, such as flagging unknown global variables and
table fields, checking the number of function arguments against
signatures, and inferring function return values, but it does not
try to analyze object-oriented code and does not perform compile-time
type checking.

Tidal Lock \cite{tidallock} is a prototype of another optional type
system for Lua, which is written in Metalua.
A remarkable feature of Tidal Lock is the way it handles tables,
because it allows the programmer to incrementally build the type of
a table.
More precisely, Tidal Lock allows the programmer to create an empty
table to build its type according to the keys that the programmer
inserts in it, so the programmer do not need to bother about declaring
its type with just one line of code.
We also plan to add this feature to Typed Lua.
Like Typed Lua, Tidal Lock also has local type inference.
Unlike Typed Lua, Tidal Lock is not typing the common object-oriented
idioms used by Lua programmers.

\section{Expected contributions} \label{sec:con}

Our main contribution is to document the process of designing and
implementing an optional type system for Lua, a major scripting
language.
Lua is a small scripting language that shares some features with other
scripting languages, so the thesis should help language designers
that wish to add optional and gradual type systems to other
scripting languages.

We also believe that offering a framework that Lua developers can
use to document, test, and better structure their code is also a
major contribution of our thesis.

Finally, our optional type system could also help some Lua
implementations to perform code optimization.
However, the fact that we are not currently providing guarantees on
the interaction between dynamic and static code is an obstacle for
performing safe optimizations.

\bibliographystyle{abbrv}
\bibliography{typedlua}

\end{document}
